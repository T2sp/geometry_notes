\documentclass[algtopo_main]{subfiles}
\mathchardef\mhyphen="2D
\begin{document}

\setcounter{chapter}{4}

\chapter{積}


この章では $R$ を環とし,$f \in \Hom{\TOP} (X,\ Y)$ および $M \in \Obj{\MOD{R}}$ に対して次のような記法を使う:
\begin{itemize}
    \item 位相空間 $X \in \Obj{\TOP}$ に対して $M$ 係数特異チェイン複体を $S_\bullet (X; M) \coloneqq S_\bullet (X) \otimes_R M$ と書く.
    \item $M$ 係数特異チェイン複体をとる関手 $S_\bullet (\mhyphen; M) \colon \TOP \lto \Chain{\MOD{R}}$ の,$\TOP$ における射 $f$ に対する作用を
    \begin{align}
        \label{not:SingularChain-morphism}
        f_\bullet \coloneqq S_\bullet(f; M) = S_\bullet(f) \otimes_R 1_M \colon S_\bullet (X; M) \lto S_\bullet (Y; M)
    \end{align}
    と書き,そこからさらに関手 $H_q \colon \Chain{\MOD{R}} \lto \MOD{R}$ を作用させるときも
    \begin{align}
        \label{not:SingularHomology-morphism}
        f_q \coloneqq H_q \bigl( S_\bullet(f; M) \bigr) \colon H_q \bigl( S_\bullet (X; M) \bigr) \lto H_q \bigl( S_\bullet (Y; M) \bigr) 
    \end{align}
    と略記する.
    \item 反変関手 $\Hom{R}(\mhyphen,\, M) \colon \Chain{\MOD{R}} \lto \Chain{\MOD{R}}$ の,$\Chain{\MOD{R}}$ における射 $f_\bullet \colon S_\bullet (X; \textcolor{red}{R}) \lto S_\bullet (Y; \textcolor{red}{R})$ に対する作用を
    \begin{align}
        \label{not:SingularChain-Hom-morphism}
        f^\bullet \coloneqq \Hom{R} \bigl(S_\bullet(f; R),\, M\bigr) &\colon \Hom{R} \bigl( S_\bullet(Y; R),\, M\bigr) \lto \Hom{R} \bigl( S_\bullet (X; R),\, M \bigr),\\ 
        &\varphi \lmto \varphi \circ f_\bullet
    \end{align}
    と書く.
    そこからさらに関手 $H^q \colon \Chain{\MOD{R}} \lto \MOD{R}$ を作用させるときも
    \begin{align}
        \label{not:SingularCohomology-morphism}
        f^q \coloneqq H^q\Bigl(\Hom{R} \bigl(S_\bullet(f; R),\, M\bigr)\Bigr) \colon H^q\Bigl(\Hom{R} \bigl( S_\bullet(Y; R),\, M\bigr)\Bigr) \lto H^q \Bigl(\Hom{R} \bigl( S_\bullet (X; R),\, M \bigr)\Bigr)
    \end{align}
    と書く.
    \item 余計な煩雑さを避けるために
    \begin{align}
        S_\bullet X &\coloneqq S_\bullet (X; \textcolor{red}{R}), \label{not:SCC} \\
        H_q X &\coloneqq H_q\bigl(S_\bullet (X; \textcolor{red}{R})\bigr), \label{not:SH} \\
        S^\bullet X &\coloneqq \Hom{R} \bigl( S_\bullet(X; \textcolor{red}{R}),\; \textcolor{red}{R} \bigr), \label{not:ScCC} \\
        H^q X &\coloneqq H^q\Bigl(\Hom{R} \bigl( S_\bullet(X; \textcolor{red}{R}),\; \textcolor{red}{R} \bigr)\Bigr) \label{not:ScH}
    \end{align}
    と略記することがある.
\end{itemize}

\begin{mydef}[label=def:graded-mod, breakable]{次数付き加群}
    \begin{enumerate}
        \item \textbf{次数付き左 $\bm{R}$ 加群} (graded left $R$ module) $A_\bullet$ とは,左 $R$ 加群の族 $\Familyset[\big]{A_n}{n \in \mathbb{Z}}$ であるか,または 左 $R$ 加群の直和分解 $A = \bigoplus_{n \in \mathbb{Z}} A_n$ のこと.
        \item 次数付き左 $R$ 加群の準同型写像とは,集合 $\prod_{n \in \mathbb{Z}} \Hom{R} (A_n,\, B_n)$ の元のこと.
        \item 次数付き左 $R$ 加群 $A_\bullet$ と次数付き右 $R$ 加群 $B_\bullet$ のテンソル積とは,次数付き左 $R$ 加群
        \begin{align}
            (A_\bullet \otimes_R B_\bullet)_n \coloneqq \bigoplus_{p+q=n} A_p \otimes_R B_q
        \end{align}
        のこと.
        \item 次数付き左 $R$ 加群 $\Hom{R}(A_\bullet,\, B_\bullet)$ を
        \begin{align}
            \Hom{R} (A_\bullet,\, B_\bullet)_n \coloneqq \prod_{k \in \mathbb{Z}} \Hom{R} (A_k,\, B_{k+n})
        \end{align}
        と定める.
        \item \textbf{次数付き環} (graded ring) $R_\bullet$  とは,アーベル群の族 $\Familyset[\big]{R_n}{n \mathbb{Z}}$ であって,
        写像
        \begin{align}
            R_\bullet \otimes R_\bullet \lto R_\bullet,\; a \otimes b \lmto ab
        \end{align}
        を持ち,$(ab)c = a(bc)$ が成り立つもののこと.i.e. 環 $R$ であって直和分解 $R = \bigoplus_{n \in \mathbb{Z}} R_n$ を持ち,$R_k \cdot R_l \subset R_{k+l}$ を充たすもののこと.
        \item 次数付き環が\textbf{可換} (commutative) であるとは,$\forall a \in R_{\abs{a}},\; \forall b \in R_{\abs{b}}$ に対して
        \begin{align}
            ab = (-1)^{\abs{a}\abs{b}} ba
        \end{align}
        が成り立つこと.
        \item 次数付き環 $R_\bullet$ 上の次数付き加群 $M_\bullet$ とは,環 $R = \bigoplus_{n \in \mathbb{Z}} R_n$ 上の加群であって $R_k \cdot M_l \subset M_{k+l}$ を充たすもの.
    \end{enumerate}
\end{mydef}

\hyperref[def:double-complx]{二重複体}の\hyperref[def:Tot]{全複体}を複体のテンソル積として定める\footnote{$(d^{-p} \otimes 1_{C'{}^{-q+1}}) \circ (1_{C^{-p}} \otimes d'{}^{-q}) = d^{-p} \otimes d'{}^{-q} = (1_{C^{-p+1}} \otimes d'{}^{-q}) \circ (d^{-p} \otimes 1_{C'{}^{-q+1}})$ なので,$\bigl( C^\bullet \otimes_R C'{}^*,\, d^\bullet \otimes 1_{C'{}^*},\, (-1)^\bullet 1_{C{}^\bullet} \otimes d'{}^* \bigr)$ が\hyperref[def:double-complex]{二重複体}になる.従って\hyperref[def:Tot]{全複体の射の定義}と定義\ref{def:tensor-complex}の $\delta$ の第2項の符号は整合的である.}:
\begin{mydef}[label=def:tensor-complex]{複体のテンソル積}
    複体 $(C^\bullet,\, d^\bullet),\; (C'{}^\bullet,\, d'{}^\bullet)$ の\textbf{テンソル積} $(C^\bullet \otimes C'{}^\bullet,\; \delta^\bullet)$ とは,次数付き加群のテンソル積
    \begin{align}
        (C^\bullet \otimes C'{}^\bullet)^n \coloneqq \bigoplus_{p+q=n} C^p \otimes_R C'{}^q = \Tot (C^\bullet \otimes_R C'{}^*)
    \end{align}
    および射
    \begin{align}
        \delta (z \otimes w) \coloneqq d z \otimes w + (-1)^p z \otimes d' w \quad \WHERE z \in C^p
    \end{align}
    の組のこと.
\end{mydef}

\section{Eilenberg-Zilber写像}

圏 $\TOP^2$ を次のように定める\footnote{要は2つの $\TOP$ を「直積」してできる圏.}:
\begin{itemize}
    \item 位相空間の組 $(X,\, Y)$ を対象とする.空間対ではなく,必ずしも $Y \subset X$ でなくて良い.
    \item 連続写像の組 $(f,\, g)\; \WHERE f \in \Hom{\TOP} (X',\, X),\, g \in \Hom{\TOP} (Y' ,\, Y)$ を射とする.
    つまり,
    \[\Hom{\TOP^2} \bigl( (X,\, Y),\, (X',\, Y') \bigr) \coloneqq \Hom{\TOP} (X,\, X') \times \Hom{\TOP} (Y,\, Y')\]
    とする.
    \item 合成は $(f,\, g) \circ (f',\, g') \coloneqq (f \circ f',\, g \circ g')$ と定める.
\end{itemize}

まず,非輪状モデル定理を述べる.
\begin{mytheo}[label=thm:AM, breakable]{非輪状モデル定理}
    任意の圏 $\Cat{C}$ および関手 $F,\, F' \colon \Cat{C} \lto \Chain{\MOD{R}}$ を与える.$\forall X \in \Obj{\Cat{C}}$ に対して定まる複体 $F(X)$ の第 $-n$ 項を $F^{-n}(X)$ と書く.
    
    与えられた関手 $F,\, F' \colon \Cat{C} \lto \Chain{\MOD{R}}$ は以下の条件を充たすとする:
    \begin{enumerate}
        \item $\forall n < 0,\; F^{-n} = F'{}^{-n} = 0$
        \item $\Cat{C}$ の一部の対象の集まり $\mathcal{M} \subset \Obj{\Cat{C}}$ が存在して以下を充たす:
        \begin{description}
            \item[\textbf{(F)}] $\forall X \in \Obj{\Cat{C}}$ および $\forall n \ge 0$ に対して,左 $R$ 加群 $F^{-n}(X)$ は集合
            \begin{align}
                \bigl\{\, F^{-n}(u) \bigl( F^{-n}(M) \bigr) \bigm| u \in \Hom{\Cat{C}}(M,\, X),\, M \in \mathcal{M} \,\bigr\} 
            \end{align}
            のある部分集合を基底にもつ自由 $R$ 加群となる.このとき関手 $F \colon \Cat{C} \lto \Chain{\MOD{R}}$ は\textbf{自由} (free) であると言われる.
            \item[\textbf{(A)}] $\forall M \in \mathcal{M}$ および $\forall n \ge \textcolor{red}{1}$ に対して $H^{-n}\bigl(F'(M)\bigr) = 0$ が成り立つ\footnote{i.e. $F'$ の\hyperref[def:LD]{左導来関手} $L_nF'$ に対して,$\forall M \in \mathcal{M},\; L_nF'(M) = 0$ が成り立つ.}このとき関手 $F'\colon \Cat{C} \lto \Chain{\MOD{R}}$ は \textbf{非輪状} (acyclic) と呼ばれる. 
        \end{description}
    \end{enumerate}
    このとき,\hyperref[def:nat]{自然変換} $\Phi \colon F \lto F'$ が\underline{自然な}\hyperref[def:chain-homotopy]{チェイン・ホモトピー}を除いて一意に定まる.

    特に,$F,\, F'$ がどちらも自由かつ非輪状ならば $F \simeq F'$ であり,$F,\, F'$ を結ぶどの自然変換も互いに\hyperref[def:chain-homotopy]{チェイン・ホモトピック}である.
\end{mytheo}

\begin{proof}
    ~\cite[p.165 thorem 8]{Spanier}を参照.
\end{proof}

$\bm{R}$ \textbf{係数}特異チェイン複体をとる関手 $S_\bullet (\mhyphen; R) \colon \TOP \lto \Chain{\MOD{R}}$ が自由かつ非輪状であることは,$\mathcal{M} = \Familyset[\big]{\Delta^q}{q \in \mathbb{Z}_{\ge 0}}$ おくと確認できる.


\begin{mytheo}[label=thm:EZ, breakable]{Eilenberg-Zilberの定理}
    2つの関手 $F,\, F' \colon \TOP^2 \lto \Chain{\MOD{R}}$ を
    \begin{align}
        F \bigl( (X,\, Y) \bigr) &\coloneqq S_\bullet (X \times Y), \\
        F' \bigl( (X,\, Y) \bigr) &\coloneqq S_\bullet (X) \otimes_R S_\bullet(Y)
    \end{align}
    と定めると,これらは\hyperref[def:naturallyeq]{自然同値}である.i.e. \hyperref[def:nat]{自然変換} $A \colon F \lto F',\; B \colon F' \lto F$ が存在して,$\forall (X,\, Y) \in \Obj{\TOP^2}$ に対して合成
    \begin{align}
        S_\bullet (X\times Y) &\xrightarrow{A} S_\bullet(X) \otimes_R S_\bullet (Y) \xrightarrow{B} S_\bullet (X\times Y), \\
        S_\bullet (X) \otimes_R S_\bullet(Y) &\xrightarrow{B} S_\bullet(X\times Y) \xrightarrow{B} S_\bullet (X) \otimes_R S_\bullet (Y)
    \end{align}
    が恒等写像に\hyperref[def:chain-homotopic]{チェイン・ホモトピック}で,かつ $A$ (resp. $B$)は\underline{自然な}\hyperref[def:chain-homotopic]{チェイン・ホモトピック}を除いて一意に定まる.
    特に,$\forall n \in \mathbb{Z}$ に対して自然な同型
    \begin{align}
        \label{isom:EZ}
        H_n (X \times Y) \xrightarrow{\cong} H_n \bigl( S_\bullet (X) \otimes_R S_\bullet (Y) \bigr) 
    \end{align}
    がある.
\end{mytheo}

\begin{proof}
    $\mathcal{M} \coloneqq \bigl\{\, (\Delta^p\, \Delta^q)  \bigm| p,\, q \ge 0 \,\bigr\}$ とおく.
    \begin{description}
        \item[\textbf{(A)}]  
         $\Delta^p \times \Delta^q \approx D^{p+q}$ であるから $\forall M \in \mathcal{M}$ は一点に可縮である.よって $\forall n \ge 1$ に対して $H^{-n} \bigl( F(M) \bigr) = 0$ であり $F$ は非輪状.
        
         一方,\hyperref[col:Kunneth]{K\"unneth公式}より
        \begin{align}
            H^{-n} \bigl( S_\bullet(X) \otimes_R S_\bullet(Y) \bigr) &\cong \left( \bigoplus_{p+q = n} H^{-p} \bigl( S_\bullet(X) \bigr) \otimes_R H^{-q} \bigl( S_\bullet (Y) \bigr)   \right) \\
            &\oplus \left( \bigoplus_{p+q=n-1} \Tor^R_1 \bigl( H^{-p} \bigl( S_\bullet(X) \bigr),\, H^{-q} \bigl( S_\bullet (Y) \bigr)   \bigr)  \right) 
        \end{align}
        がわかるが,$\Delta^p \cong D^p$ より $\forall n \ge 1$ に対して右辺は $0$ となり,$F'$ が非輪状であることがわかった.
        \item[\textbf{(F)}]  
         $S_q (X \times Y)$ は $\Hom{\TOP^2} \bigl( (\Delta^q,\, \Delta^q),\, (X,\, Y) \bigr)$ を基底にもち,$S_\bullet (X) \otimes_R S_\bullet (Y) = \bigoplus_{p+q=n} S_p(X) \otimes_R S_q(Y)$ は $\coprod_{p+q = n} \Hom{\TOP^2} \bigl( (\Delta^p,\, \Delta^q),\, (X,\, Y) \bigr) $ を基底に持つ.i.e. $F,\, F'$ はどちらも自由である.
    \end{description}
    以上より\hyperref[thm:AM]{非輪状モデル定理}を使うことができて所望の自然変換 $A \colon F \lto F',\. B \colon F'\lto F$ の存在と,自然なチェイン・ホモトピックを除いた一意性が言えた.

    同型\eqref{isom:EZ}は命題\ref{prop:homotoic-basic}より従う.
\end{proof}

簡単のため,以降では\hyperref[thm:EZ]{Eilenberg-Zilberの定理}における自然変換 $A,\, B$ を一つに固定する.複体をとる段階では積は $A,\, B$ に依存するが,命題\ref{prop:homotoic-basic}よりホモロジー・コホモロジーをとってしまえばこの依存性は消えるので問題ない.

\section{クロス積}

\subsection{ホモロジーのクロス積}

\begin{mydef}[label=def:algcr-homology]{ホモロジーの代数的クロス積}
    チェイン複体 $C_\bullet,\, D_\bullet$ を与える.
    自然な写像
    \begin{align}
        \algcr \colon H_p(C_\bullet) \otimes_R H_q(D_\bullet) \lto H_{p+q} (C_\bullet \otimes D_\bullet)
    \end{align}
    を,well-definedな対応
    \begin{align}
        [z] \otimes [w] \lmto [z \otimes w]
    \end{align}
    を線型に拡張することによって定義し\footnote{$[\mhyphen]$ はホモロジー類をとることを意味する.},
    ホモロジーの\textbf{代数的クロス積}と呼ぶ.$\algcr ([z] \otimes [w]) = [z \otimes w]$ のことを $[z] \algcr [w]$ と書く.
\end{mydef}

\begin{proof}
    複体 $C_\bullet,\, D_\bullet$ の射をそれぞれ $\partial_\bullet,\, \partial'_\bullet$ と書き,\hyperref[def:tensor-complex]{複体のテンソル積}の射を $\delta_\bullet$ と書く.
    $\forall z \in \Ker \partial_p,\; \forall w \in \Ker \partial_q$ および $\forall \bar{z} \in C_{p+1},\; \forall \bar{w} \in D_{q+1}$ に対して,
    \begin{align}
        &(z + \partial_{p+1} \bar{z}) \otimes (w + \partial'_{q+1} \bar{w}) - z \otimes w
        % &=\partial_{p+1} \bar{z}  \otimes (w + \partial_{q+1} \bar{w}) +  z \otimes \partial_{q+1} \bar{w} \\
        = z \otimes \partial_{q+1}' \bar{w} + \partial_{p+1} \bar{z} \otimes w + \partial_{p+1} \bar{z} \otimes \partial'_{q+1} \bar{w} \\
        &= (-1)^p \bigl(\partial_p z \otimes \bar{w} + (-1)^p z \otimes \partial_{q+1}' \bar{w} \bigr) - (-1)^{p}\cancel{ \partial_p z} \otimes \bar{w}\\
        &\quad +  \bigl(\partial_{p+1} \bar{z} \otimes w + (-1)^{p+1} \bar{z} \otimes \partial_{q}' w \bigr) - (-1)^{p+1} \bar{z} \otimes \cancel{\partial_{q}' w}\\
        &\quad +  \bigl(\partial_{p+1} \bar{z} \otimes \partial_{q+1}' \bar{w} + (-1)^{p+1} \bar{z} \otimes \partial'_q \partial_{q+1}' \bar{w} \bigr) \\
        &= \delta_{p+q} \bigl( (-1)^p z \otimes \bar{w} + \bar{z} \otimes w + \bar{z} \otimes \bar{w} \bigr) \in \Im \delta_{p+q}
        % &= \partial \bigl( u \otimes ( w + \partial_{q+1} \bar{w}) + (-1)^p z \otimes \bar{w} \bigr) 
    \end{align}
    より\footnote{$\partial_p z = 0,\; \partial'_q w = 0$ に注意.} $[z] \algcr [w]$ は代表元の取り方によらず,$\algcr$ はwell-definedである.
\end{proof}

% $R$ が\hyperref[def:PID]{単項イデアル整域}のとき,\hyperref[def:algcr-homology]{代数的クロス積}が同型から「どの程度ずれているか」は\hyperref[col:Kunneth-Tor]{K\"unneth公式}によってわかる.

位相空間 $X,\, Y \in \Obj{\TOP}$ を与える.定義\ref{def:algcr-homology}より,特異チェイン複体の上にwell-definedな写像
\begin{align}
    \algcr \colon H_p \bigl( S_\bullet (X) \bigr) \otimes_R H_q \bigl( S_\bullet (X) \bigr) \lto H_{p+q} \bigl( S_\bullet(X) \otimes_R S_\bullet(Y) \bigr) 
\end{align}
がある.一方,\hyperref[thm:EZ]{Eilenberg-Zilberの定理}より,$B$ の取り方によらない同型
\begin{align}
    B_* \colon H_* \bigl( S_\bullet (X) \otimes_R S_\bullet (Y) \bigr)  \xrightarrow{\cong} H_* \bigl( S_\bullet(X \times Y) \bigr) 
\end{align}
がある(記号は記法\eqref{not:SingularChain-Hom-morphism}の通り).

\begin{mydef}[label=def:homology-cross]{ホモロジーのクロス積}
    合成
    \begin{align}
        \times  \coloneqq B_{p+q} \circ \algcr \colon H_p \bigl( S_\bullet (X) \bigr) \otimes_R H_q \bigl( S_\bullet (Y) \bigr) \lto H_{p+q} \bigl( S_\bullet(X \times Y) \bigr) 
    \end{align}
    のことを,ホモロジーの\textbf{クロス積}と呼び,$\forall \alpha \in H_p \bigl( S_\bullet(X) \bigr),\; \forall \beta \in H_p \bigl( S_\bullet(Y) \bigr) $ に対して $\alpha \times \beta \coloneqq \times(\alpha \otimes \beta)$ と書く.
\end{mydef}

\begin{mytheo}[label=thm:Kunneth-1]{K\"unneth公式}
    $R$ を\hyperref[def:PID]{単項イデアル整域}とする.
    このとき $\forall n \in \mathbb{Z}$ に対して分裂する短完全列
    \begin{align}
        0 &\lto \bigoplus_{p+q=n} H_p \bigl( S_\bullet(X) \bigr) \otimes_R H_{q} \bigl( S_\bullet(Y) \bigr) \\
        &\xrightarrow{\times} H_n \bigl( S_\bullet(X \times Y) \bigr) \\
        &\lto \bigoplus_{p+q=n-1} \Tor^R_1 \Bigl( H_p \bigl(S_\bullet (X)\bigr),\, H_{q} \bigl(S_\bullet (Y)\bigr) \Bigr) 
    \end{align}
    が存在する.
\end{mytheo}

\begin{proof}
    自由加群からなる複体 $S_\bullet (X),\, S_\bullet(Y)$ に対して\hyperref[col:Kunneth-Tor]{K\"unneth公式}を適用してから\hyperref[thm:EZ]{Eilenberg-Zilberの定理}を使う.
\end{proof}

\begin{mycol}[]{体係数ホモロジーのクロス積}
    $R$ が体ならば,\hyperref[def:homology-cross]{ホモロジーのクロス積}は次の同型を誘導する:
    \begin{align}
        H_n \bigl(S_\bullet(X \times Y)\bigr) = \bigoplus_{p+q=n} H_p \bigl( S_\bullet (X) \bigr) \otimes_R H_q \bigl( S_\bullet(Y)  \bigr)  
    \end{align}
\end{mycol}

\begin{proof}
    $R$ が体ならば $\Tor_1^R$ は $0$ になる(命題\ref{thm:proj-resol-PID}および\hyperref[def:Tor]{$\Tor$ の定義}を参照). 
\end{proof}

\subsection{コホモロジーのクロス積}

環 $R$ 上のチェイン複体 $C_\bullet$ の双対チェイン複体を
\begin{align}
    C^\bullet  \coloneqq \Hom{R} (C_\bullet,\, R)
\end{align}
で定義する.

\begin{mydef}[label=def:algcr-cohomology]{コホモロジーの代数的クロス積}
    チェイン複体 $C_\bullet,\, D_\bullet$ を与える.
    自然な写像
    \begin{align}
        \cralg \colon H^p(C^\bullet) \otimes_R H^q(D^\bullet) \lto H^{p+q} \bigl( (C_\bullet \otimes_R D_\bullet)^\bullet \bigr)
    \end{align}
    を,well-definedな対応
    \begin{align}
        [\alpha] \otimes [\beta] \lmto \left[ \sum_{i,\, \abs{z_i} + \abs{w_i} = p+q} z_i \otimes w_i \lmto   \sum_{i,\, \abs{z_i} + \abs{w_i} = p+q} \alpha(z_i) \cdot \beta(w_i) \right]
    \end{align}
    を\footnote{$[\mhyphen]$ はコホモロジー類をとることを意味する.}線型に拡張することで定め,コホモロジーの\textbf{代数的クロス積}と呼ぶ.ただし $\alpha$ と $z_i$ の次数が異なる場合は $\alpha(z_i) = 0$ で,$\beta$ と $w_i$ についても同様である.
\end{mydef}

\begin{proof}
    複体 $C^\bullet,\, D^\bullet$ の射をそれぞれ $\partial^\bullet,\, \partial'{}^\bullet$ と書き,\hyperref[def:tensor-complex]{複体のテンソル積}の射を $\delta^\bullet$ と書く.
    $\forall \alpha \in \Ker \partial^p,\; \forall \beta \in \Ker \partial'{}^q$ および $\forall \bar{\alpha} \in C^{p-1},\; \forall \bar{\beta} \in D^{q-1}$ をとると
    \begin{align}
        &\sum_{i,\, \abs{z_i} + \abs{w_i} = p+q} (\alpha + \partial^{p-1} \bar{\alpha})(z_i) \cdot (\beta + \partial'{}^{q-1}\bar{\beta})(w_i) - \sum_{i,\, \abs{z_i} + \abs{w_i} = p+q} \alpha(z_i) \cdot \beta(w_i) \\
        &= \sum_{i,\, \abs{z_i} + \abs{w_i} = p+q} \bigl( \alpha(z_i) \cdot \bar{\beta}(\partial'_{q}w_i) + \bar{\alpha} (\partial_{p} z_i) \cdot \beta(w_i) + \bar{\alpha} (\partial_{p} z_i) \cdot \bar{\beta}(\partial'_{q}w_i)  \bigr) \\
        &= \sum_{i,\, (\abs{z_i},\,\abs{w_i}) = (p,\, q)} \Bigl( (-1)^{\abs{z_i}}\bigl(\alpha(\partial_p z_i) \cdot \bar{\beta}(w_i) + (-1)^{\abs{z_i}} \alpha(z_i) \cdot \bar{\alpha}(\partial'_{q}w_i) \bigr) \\
        &\qquad +\bigl(\bar{\alpha} (\partial_{p} z_i) \cdot \beta(w_i) + (-1)^{\abs{z_i}} \bar{\alpha}(z_i) \cdot \beta(\partial'_q w_i) \bigr) \\
        &\qquad +\bigl(\bar{\alpha} (\partial_{p} z_i) \cdot \bar{\beta}(\partial'_{q}w_i) + (-1)^{\abs{z_i}} \bar{\alpha}(z_i) \cdot \bar{\beta} (\partial'_{q-1}\partial'_q w_i )\bigr)  \Bigr) \\
        &= \sum_{i} \Bigl( (-1)^{\abs{z_i}} (\alpha \otimes \bar{\beta})\bigl(\delta_{p+q} (z_i \otimes w_i) \bigr) \\
        &\qquad +(\bar{\alpha} \otimes \beta)\bigl(\delta_{p+q} (z_i \otimes w_i)\bigr) \\
        &\qquad +(\bar{\alpha} \otimes \bar{\beta})\bigl(\delta_{p+q} (z_i \otimes \partial'_{q}w_i) \bigr) \Bigr) \\
        &= \delta^{p+q-1}\bigl( (-1)^p \alpha \otimes \bar{\beta} + \bar{\alpha} \otimes \beta + \bar{\alpha} \otimes \bar{\beta} \bigr) \left( \sum_{i} z_i \otimes w_i \right)
    \end{align}
    が成り立つ\footnote{$\alpha,\, \beta$ の引数は次数がそれぞれ $p,\, q$ でなければ $0$ になることに注意する.}のでwell-definedである.
\end{proof}

特異チェイン複体に上述の構成を適用することで
\begin{align}
    \cralg \colon H^p \bigl( S_\bullet (X) \bigr) \otimes_R H^q \bigl( S_\bullet (X) \bigr) \lto H^{p+q} \Bigl( \bigl(S_\bullet (X) \otimes S_\bullet(Y) \bigr)^\bullet \Bigr) 
\end{align}
を得る.一方,\hyperref[thm:EZ]{Eilenberg-Zilberの定理}からチェイン・ホモトピー同値写像
\begin{align}
    A \colon S_\bullet (X \times Y) \lto S_\bullet(X) \otimes_R S_\bullet (Y)
\end{align}
があるが,これの双対をとると
\begin{align}
    A^* \colon \bigl(  S_\bullet(X) \otimes_R S_\bullet (Y) \bigr)^* \lto S^* (X \times Y)
\end{align}
になる(記号は\eqref{not:SingularChain-Hom-morphism}).
さらにコホモロジーを取ることで,$A$ の取り方によらない同型
\begin{align}
    A^*\colon H^\bullet \bigl( S_\bullet (X) \otimes_R S_\bullet (Y) \bigr)^\bullet \lto H^\bullet \bigl( S^\bullet (X \times Y) \bigr) 
\end{align}
を得る(記号は\eqref{not:SingularCohomology-morphism}の通り).

\begin{mydef}[label=def:cohomology-cross]{コホモロジーのクロス積}
    合成
    \begin{align}
        \times \coloneqq A^* \circ \cralg \colon H^p \bigl( S^\bullet (X) \bigr) \otimes H^q \bigl( S^\bullet (Y) \bigr) \lto H^{p+q} \bigl( S^\bullet(X \times Y) \bigr) 
    \end{align}
    のことをコホモロジーの\textbf{クロス積}と呼ぶ.
\end{mydef}

\section{カップ積とキャップ積}


この節では,原則として記法\eqref{not:SCC}, \eqref{not:SH}を使う.

対角写像を
\begin{align}
    \Delta \colon X \lto X \times X,\; x \lmto (x,\, x)
\end{align}
とおくと $\Delta \in \Hom{\TOP} (X,\, X\times X)$ である.
このとき記法\eqref{not:SingularCohomology-morphism}に則って
\begin{align}
    \Delta^\bullet \colon H^\bullet (X \times X) \lto H^\bullet (X)
\end{align}
と略記する.

\subsection{カップ積と特異コホモロジーの環構造}

\begin{mydef}[label=def:cup]{カップ積}
    $\forall a \in H^p (X),\; \forall b \in H^q (X)$ の\textbf{カップ積} (cup product) を次のように定義する:
    \begin{align}
        \bm{a \smile b} \coloneqq \Delta^{p+q} \bigl( a \times b \bigr) \in H^{p+q} (X)
    \end{align}
\end{mydef}
つまり,
\begin{align}
    \smile \colon H^p(X) \otimes_R H^q(X) \lto H^{p+q} (X)
\end{align}
となる.バランス写像 $\Phi \colon H^p(X)\times H^q (Y) \lto H^p  (X)\otimes_R H^q (Y)$ を合成することで
\begin{align}
    \smile \colon H^p (X) \times H^q (X)  \lto H^{p+q} (X)
\end{align}
と書くこともできる.

\begin{mylem}[label=lem:cup-basic]{カップ積の基本性質}
    $\forall (f,\, g) \in \Hom{\TOP^2} \bigl( (X',\, Y'),\, (X,\, Y) \bigr) $ および $\forall a,\, b \in H^\bullet (X) ,\; \forall c \in H^\bullet  (Y)$ をとる.
    $\mathrm{pr}_i \colon X\times Y \lto  X_i\; (i=1,\, 2)$ を射影とする.
    \begin{enumerate}
        \item $a \smile b = \Delta^\bullet (a \times b)$
        \item $a \times b = \mathrm{pr}_X {}^\bullet (a) \smile \mathrm{pr}_Y{}^\bullet(b)$
        \item $f^\bullet(a \smile b) = f^\bullet (a) \smile f^\bullet (b)$
        \item $(f \times g)^\bullet (a \times c) = f^\bullet(a) \times g^\bullet(c)$
    \end{enumerate}
    
\end{mylem}

\begin{proof}
    \begin{description}
        \item[(1)] \hyperref[def:cup]{カップ積の定義}.
        \item[(4)] \hyperref[thm:EZ]{Eilenberg-Zilberの定理}において $A,\, B$ が自然変換であることから従う.
        \item[(3)] $(f \times f) \circ \Delta = \Delta \circ f$ であることと (4) から
        \begin{align}
            f^\bullet (a) \smile f^\bullet (b) &= \Delta^\bullet (f^\bullet (a) \times g^\bullet (b)) = \bigl( (f \times f) \circ \Delta \bigr)^\bullet (a \times b) \\
            &= (\Delta \circ f)^\bullet (a \times b) = f^\bullet \bigl( \Delta^\bullet (a \times b) \bigr) = f^\bullet (a \smile b)
        \end{align}
        \item[(2)] $(\mathrm{pr}_X \times \mathrm{pr}_Y) \circ \Delta_{X\times Y} = \mathrm{id}_{X\times Y}$ であることと (4) 
        \begin{align}
            \mathrm{pr}_X{}^\bullet (a) \smile \mathrm{pr}_Y{}^\bullet (b) &= \Delta^\bullet_{X \times Y} \bigl( \mathrm{pr}_X{}^\bullet (a) \times \mathrm{pr}_Y{}^\bullet (c) \bigr) \\
            &= \Delta_{X\times Y}^\bullet \bigl( (\mathrm{pr}_X \times \mathrm{pr}_Y)^\bullet (a \times c) \bigr) \\
            &= \bigl( (\mathrm{pr}_X \times \mathrm{pr}_Y) \circ \Delta_{X\times Y} \bigr)^\bullet (a \times c) \\
            &= \mathrm{id}_{X\times Y}^\bullet (a \times c) = a\times c
        \end{align}
    \end{description}
    
\end{proof}


\begin{mydef}[label=def:d-approx]{対角近似}
    $\forall X \in \Obj{\TOP}$ に対して,チェイン複体の射
    \begin{align}
        \tau \colon S_\bullet (X) \lto S_\bullet(X) \otimes_R S_\bullet (X)
    \end{align}
    は以下の条件を充たすとき\textbf{対角近似} (diagonal approximation) と呼ばれる:
    \begin{enumerate}
        \item 任意の $0$-単体 $\sigma$ に対して $\tau(\sigma) = \sigma \otimes \sigma$
        \item $\tau$ は連続写像に関して自然である.i.e. $\forall f \in \Hom{\TOP}(X,\, Y)$ に対して図式\ref{cmtd:nat-d-approx}が可換になる
    \end{enumerate}
\end{mydef}
\begin{figure}[H]
    \centering
    \begin{tikzcd}[row sep=large, column sep=large]
        &S_\bullet (X) \ar[r, "S_\bullet (f)"] \ar[d, "\tau"] &S_\bullet (Y) \ar[d, "\tau"] \\
        &S_\bullet (X) \otimes_R S_\bullet(X) \ar[r] &S_\bullet(Y) \otimes_R S_\bullet (Y)
    \end{tikzcd}
    \caption{対角近似の自然性}
    \label{cmtd:nat-d-approx}
\end{figure}%

\hyperref[def:d-approx]{対角近似}と\hyperref[thm:EZ]{Eilenberg-Zilber map}は,片方が与えられるともう一方も定まる.従って,状況に応じて便利な方を使えば良い.

\begin{mylem}[label=def:EZ-dapprox, breakable]{Eilenberg-Zilber map と対角近似の関係}
    $A \colon S_\bullet (X \times Y) \xrightarrow{A} S_\bullet (X) \otimes_R S_\bullet (X)$ を与えると,対応する対角近似 $\tau \colon S_\bullet (X) \lto S_\bullet(X) \otimes_R S_\bullet (X)$ が
    \begin{align}
        \tau = A \circ \Delta_\bullet
    \end{align}
    によって定まる.

    逆に,対角近似 $\tau$ が与えられると対応する $A$ が
    \begin{align}
        A = (\mathrm{pr}_X \otimes \mathrm{pr}_Y) \circ \tau
    \end{align}
    によって定まる.
\end{mylem}

\begin{proof}
関手 $S_\bullet (\mhyphen)\colon \TOP \lto \Chain{\MOD{R}}$ は\hyperref[thm:AM]{自由}で,
関手 $S_\bullet (\mhyphen) \otimes_R S_\bullet (\mhyphen)$ は\hyperref[thm:AM]{非輪状}である.
従って\hyperref[thm:AM]{非輪状モデル定理}より\hyperref[def:d-approx]{対角近似} $\tau$ が自然なチェイン・ホモトピーを除いて一意に定まる.
特に,\hyperref[thm:EZ]{Eilenberg-Zilber map} $A \colon S_\bullet (X \times Y) \xrightarrow{A} S_\bullet (X) \otimes_R S_\bullet (X)$ に対して $\tau = A \circ \Delta_\bullet$ は\hyperref[def:d-approx]{対角近似}である.
\footnote{従って\hyperref[def:cup]{カップ積の定義}を
\begin{align}
    a \smile b = \tau^\bullet (a \cralg b)
\end{align}
とすることもできる.}

逆に $(\mathrm{pr}_X \otimes \mathrm{pr}_Y) \circ \tau$ は関手 $F \colon (X,\, Y) \lmto S_\bullet (X \times Y),\; F' \colon (X,\, Y) \lmto S_\bullet(X) \otimes_R S_\bullet(Y)$ の間の\hyperref[def:nat]{自然変換}となる.そして定理\ref{thm:EZ}より,これはEilenberg-Zilber mapと自然にホモトピックである.
\end{proof}

\begin{mytheo}[label=thm:ring-cohomology, breakable]{特異コホモロジーの環構造}
    全ての特異 $0$ 単体を $1 \in R$ に移すコサイクルのコホモロジー類を $1 \in H^0(X)$ と書く.
    $\forall a,\, b,\, c \in H^\bullet (X)$ に対して以下が成り立つ:
    \begin{enumerate}
        \item $1 \smile a = a = a \smile 1$
        \item $(a \smile b) \smile c = a \smile (b \smile c)$
        \item $a \smile b = (-1)^{\abs{a} \abs{b}} b \smile a$
    \end{enumerate}
    従って,組 $\bigl(H^\bullet (X),\, +,\, \smile \bigr)$ は\hyperref[def:graded-mod]{次数付き可換環}になる.
\end{mytheo}

\begin{proof}
    \begin{enumerate}
        \item 
    \end{enumerate}
    
\end{proof}

\subsection{キャップ積と特異ホモロジーの加群構造}

Kroneckerペアリング
\begin{align}
    \expval{\, ,\;} \colon S^\bullet (X) \times S_\bullet (X) \lto R
\end{align}
は,$a \in S^q(X),\; z \in S_p(X)$ に対して
\begin{align}
    \expval{a,\, z} \coloneqq 
    \begin{cases}
        a(z), &p=q \\
        0, &\text{otherwise}
    \end{cases}
\end{align}
として定めた.これを拡張して部分的なevaluation
\begin{align}
    E \colon S^\bullet X \otimes_R S_\bullet X \otimes_R S_\bullet X \lto S_\bullet X
\end{align}
を,次数が合っている時に
\begin{align}
    E(a \otimes z \otimes w) \coloneqq a(w) \otimes z
\end{align}
として定める(定義域の一般の元に対してはこれを線型に拡張する).

\begin{mydef}[label=def:cap-cochain]{コチェインのキャップ積}
    $\forall a \in S^q(X),\;\forall z \in S_{p+q}(X)$ に対して\textbf{キャップ積} (cap product) を,\hyperref[thm:EZ]{Eilenberg-Zilberの定理}の $A$ を用いて
    \begin{align}
        \frown \colon S^q(X) \times S_{p+q} (X) \lto S_q(X),\; (a,\, z) \lmto E \bigl( a \otimes (A \circ \Delta_\bullet)(z) \bigr) 
    \end{align}
\end{mydef}

または,\hyperref[def:d-approx]{対角近似} $\tau$ を用いて
\begin{align}
    a \frown z = E\bigl(a \otimes \tau(z)\bigr)
\end{align}
としてもよい.

\begin{mylem}[label=lem:cap-1]{}
    特異ホモロジー,コホモロジーの境界写像をそれぞれ $\partial_\bullet,\, \delta^\bullet$ と書く.
    $\forall \alpha \in S^q(X),\, \forall z \in S_{p+q} (X)$ に対して以下が成り立つ:
    \begin{align}
        \partial_p(a \frown z) = (-1)^p \delta^q\alpha \frown z + \alpha \frown \partial_{p+q}z
    \end{align}
\end{mylem}

\begin{proof}
    $\tau(z) = \sum_{i,\; \abs{x_i} + \abs{y_i} = p+q} x_i \otimes y_i$ と書ける.このとき
    \begin{align}
        \theta_p (\alpha \frown z) &= \partial_p \left( \sum_{i,\; \abs{y_i} = q} \alpha(y_i) \cdot x_i  \right) = \sum_{i,\; \abs{y_i} = q} \alpha(y_i) \cdot \partial_p x_i, \\
        \delta^q \alpha \frown z &= \sum_i \delta^p \alpha (y_i) \cdot x_i = \sum_{i,\;\abs{y_i} = q+1} \alpha(\partial_{q+1} y_i) \cdot x_i
    \end{align}
    なので,$\tau$ が\hyperref[def:chainmap]{チェイン複体の射}であることに注意すると
    \begin{align}
        \alpha \frown \partial_{p+q} z &= E(\alpha \otimes \tau(\partial_{p+q} z)) = E(\alpha \otimes \partial_{p+q} (\tau(z))) \\
        &= E \left( \alpha \otimes \left( \sum_{i,\; \abs{x_i} + \abs{y_i} = p+q}\partial_{\abs{x_i}} x_i \otimes y_i + \sum_{i,\; \abs{x_i} + \abs{y_i} = p+q} (-1)^{\abs{x_i}} x_i \otimes \partial_{\abs{y_i}} y_i \right)  \right)  \\
        &= \sum_{i,\; \abs{y_i} = q} \alpha(y_i) \cdot \partial_p x_i + \sum_{i,\, \abs{y_i} = q+1} (-1)^{p-1} \alpha(\partial_{q+1} y_i) \cdot x_i \\
        &= \partial_p (a \frown z)  + (-1)^{p-1} \delta^q \alpha \frown z
    \end{align}
    となって示された.
\end{proof}
補題\ref{lem:cap-1}により,次の定義がwell-definedになる.
\begin{mydef}[label=def:cap]{キャップ積}
    \textbf{キャップ積} (cap product) を次のように定義する:
    \begin{align}
        \frown \colon H^q(X) \times H_{p+q} (X) &\lto H_p(X),\\
        ([\alpha],\, [z]) &\lmto [\alpha \frown z]
    \end{align}
\end{mydef}

\begin{mytheo}[label=thm:mod-homology]{特異ホモロジーの加群構造}
    $\forall a,\, b \in H^\bullet (X),\; \forall z \in H_\bullet (X)$ に対して以下が成り立つ:
    \begin{enumerate}
        \item $\expval{a,\, b \frown z} = \expval{a \smile b,\, z}$
        \item $a \frown (b \frown z) = (a \smile b) \frown z$
    \end{enumerate}
    (2) より,組 $\bigl( H_\bullet (X),\, +,\, \frown \bigr)$ は左 $H^\bullet (X)$ 加群になる.
\end{mytheo}

\begin{proof}
    $a = [\alpha],\, b = [\beta]$ とおき,$\tau(z) = \sum_{i,\, \abs{x_i} + \abs{y_i} = \abs{z}} x_i \otimes y_i$ とおく.
    \begin{enumerate}
        \item \hyperref[def:cap-cochain]{コチェインのキャップ積の定義}より
        \begin{align}
            \expval{a,\, b \frown z} &= \alpha \left( E\left( \beta \otimes \sum_{i,\, \abs{x_i} + \abs{y_i} = \abs{z}} x_i \otimes y_i \right) \right) \\
            &= \alpha  \sum_{i,\, \abs{x_i} + \abs{y_i} = \abs{z}} x_i \cdot \beta (y_i) \\
            &= \sum_{i,\, \abs{x_i} + \abs{y_i} = \abs{z}} \alpha (x_i) \beta(y_i).
        \end{align}
        一方,\hyperref[def:algcr-cohomology]{コホモロジーの代数的クロス積の定義}より
        \begin{align}
            \expval{a \smile b,\, z} &= \bigl(\tau^\bullet (a \cralg b)\bigr) (z) \\
            &= (a \cralg b)\bigl(\tau(z)\bigr) \\
            &= \sum_{i,\, \abs{x_i} + \abs{y_i} = \abs{z}} \alpha (x_i) \beta(y_i)
        \end{align}
        が成り立つ.よって $\expval{a,\, b \frown z} = \expval{a \smile b,\, z}$ が言えた.
        \item $\forall c \in H^{\abs{z} - \abs{b} - \abs{a}}$ を与える.このとき $\smile$ の\hyperref[thm:ring-cohomology]{結合則}より
        \begin{align}
            \expval{c,\, a \frown (b \frown z)} = \expval{c \smile a,\, b \frown z} = \expval{c \smile (a\smile b),\, z} = \expval{c,\, (a \smile b) \frown z}
        \end{align}
        が成り立ち,証明が完了する.
    \end{enumerate}
\end{proof}

\subsection{スラント積}

\begin{mydef}[label=def:slant-cochain]{コチェインのスラント積}
    \hyperref[thm:EZ]{Eilenberg-Zilberの定理}における写像 $A \colon S_\bullet (X \times Y) \lto S_\bullet (X) \otimes_R S_\bullet(Y)$ を用いて,コチェインの\textbf{スラント積} (slant product) を次のように定義する:
    \begin{align}
        \backslash \colon S^q(Y) \times S_{p+q} (X \times Y) \lto S_p(X),\; (\alpha,\, z) \lmto E \bigl( \alpha \otimes A(z) \bigr) 
    \end{align}
\end{mydef}

\begin{mydef}[label=def:slant]{スラント積}
    \textbf{スラント積} (slant product) を次のように定義する:
    \begin{align}
        \backslash \colon H^q(Y) \times H_{p+q} (X \times Y) &\lto H_p(X),\\ 
        ([\alpha],\, [z]) &\lmto [\alpha \backslash z]
    \end{align}
\end{mydef}

対応関係は
\begin{itemize}
    \item \hyperref[def:cup]{カップ積} $\leftrightarrow$ \hyperref[def:cohomology-cross]{クロス積}
    \item \hyperref[def:cap]{キャップ積} $\leftrightarrow$ \textbf{スラント積}
\end{itemize}
のようになっている.
例えば定理\ref{thm:mod-homology}-(1) と対応して
\begin{align}
    \expval{a,\, b \backslash z} = \expval{a \times b,\, z}
\end{align}
が成り立つ.

% \section{Alexander-Whitney対角近似}

\section{空間対のカップとキャップ}

$(X,\, A),\; (Y,\, B)$ を空間対とする.標準的射影
\begin{align}
    S_\bullet X \otimes_R S_\bullet Y \twoheadrightarrow \frac{S_\bullet X}{S_\bullet A} \otimes_R \frac{S_\bullet Y}{S_\bullet B}
\end{align}
の核は $S_\bullet A \otimes_R S_\bullet Y + S_\bullet X \otimes_R S_\bullet B$ なので準同型定理から自然な同型
\begin{align}
    \frac{S_\bullet X}{S_\bullet A} \otimes_R \frac{S_\bullet Y}{S_\bullet B} \cong \frac{S_\bullet X \otimes_R S_\bullet Y}{S_\bullet A \otimes_R S_\bullet Y + S_\bullet X \otimes_R S_\bullet B}
\end{align}
が従う.
$X = Y$ とすると,対角近似 $\tau \colon S_\bullet X \lto  S_\bullet (X) \otimes_R S_\bullet (X)$ は $\tau(S_\bullet A) \subset S_\bullet A \otimes_R S_\bullet A,\; \tau (S_\bullet B) \subset S_\bullet B \otimes_R S_\bullet B$ を充たす.従って合成
\begin{align}
    S_\bullet X \xrightarrow{\tau} S_\bullet X \otimes_R S_\bullet X \twoheadrightarrow \frac{S_\bullet X \otimes_R S_\bullet X}{S_\bullet A \otimes_R S_\bullet X + S_\bullet X \otimes_R S_\bullet B}
\end{align}
の核は $S_\bullet A + S_\bullet B$ だから,準同型
\begin{align}
    \overline{\tau} \colon \frac{S_\bullet X}{S_\bullet A + S_\bullet B} \lto \frac{S_\bullet X \otimes_R S_\bullet X}{S_\bullet A \otimes_R S_\bullet X + S_\bullet X \otimes_R S_\bullet B}
\end{align}
が誘導される.

\hyperref[def:algcr-cohomology]{コホモロジーの代数的クロス積}と $\overline{\tau}^\bullet$ の合成
\begin{align}
    \Hom{R} \left( \frac{S_\bullet X}{S_\bullet A},\, R \right) \otimes_R \Hom{R} \left( \frac{S_\bullet X}{S_\bullet B},\, R \right) &\xrightarrow{\cralg} \Hom{R} \left( \frac{S_\bullet X}{S_\bullet A} \otimes_R \frac{S_\bullet Y}{S_\bullet B},\, R \right) \\ 
    &\cong \Hom{R} \left( \frac{S_\bullet X \otimes_R S_\bullet X}{S_\bullet A \otimes_R S_\bullet X + S_\bullet X \otimes_R S_\bullet B},\, R \right) \\
    &\xrightarrow{\overline{\tau}^\bullet} \Hom{R} \left( \frac{S_\bullet X}{S_\bullet A + S_\bullet B},\, R \right) 
\end{align}
のコホモロジーをとることでカップ積
\begin{align}
    \label{eq:rel-cup}
    H^p (X,\, A) \times H^q(X,\, B) \lto H^{p+q} \left( \Hom{R} \left( \frac{S_\bullet X}{S_\bullet A + S_\bullet B},\, R \right)   \right)
\end{align}
を誘導する.

\begin{mydef}[label=def:exc-pair]{切除対}
    位相空間 $X$ と,その部分空間 $A,\, B \subset X$ を与える.標準的包含
    \begin{align}
        S_\bullet (A) + S_\bullet (B) \hookrightarrow S_\bullet (A \cup B)
    \end{align}
    がチェイン・ホモトピー同値写像であるとき,部分空間の対$\{A,\, B\}$ を\textbf{切除対} (excisive pair) と呼ぶ.
\end{mydef}

チェイン・ホモトピーは\hyperref[def:Ab-func]{加法的関手}によって保存されるから,\hyperref[def:exc-pair]{切除対} $\{A,\, B\}$ に対して
\begin{align}
    \Hom{R} \bigl( S_\bullet (A \cup B),\, R \bigr) \lto \Hom{R} \bigl( S_\bullet (A) + S_\bullet (B),\, R \bigr) 
\end{align}
もまたチェイン・ホモトピー同値写像である.従って命題\ref{prop:chain-homotopy}より同型
\begin{align}
    \label{isom:relative-cup}
    H^q (A \cup B) \cong H^q \Bigl( \Hom{R} \bigl( S_\bullet (A) + S_\bullet (B),\, R \bigr)  \Bigr) 
\end{align}
を誘導する.

\hyperref[def:exc-pair]{切除対} $\{A,\, B\}$ に対して,横の2行が完全な図式
\begin{align}
    0 \lto S_\bullet A + S_\bullet B \lto  S_\bullet X \lto \frac{S_\bullet X}{S_\bullet A + S_\bullet B} \lto 0
\end{align}
\begin{center}
    \begin{tikzcd}[row sep=large, column sep=large]
        &0 \ar[r] &S_\bullet (A) + S_\bullet (B) \ar[r] \ar[d, hookrightarrow] & S_\bullet X \ar[r]\ar[d, "1"] &\frac{S_\bullet(X)}{S_\bullet (A) + S_\bullet (B)} \ar[r] \ar[d] &0 &(\text{exact}) \\
        &0 \ar[r] &S_\bullet (A \cup B) \ar[r] & S_\bullet X \ar[r] &\frac{S_\bullet(X)}{S_\bullet (A \cup B)} \ar[r]&0 &(\text{exact})
    \end{tikzcd}
\end{center}
がある.これの\hyperref[prop:LES-cohomology]{コホモロジー長完全列}をとると,同型\eqref{isom:relative-cup}により可換図式

\begin{figure}[H]
    \centering
    \begin{tikzcd}[sep=small]
        &\cdots \ar[r] &H^q(S_\bullet (A) + S_\bullet (B))^\bullet \ar[r] \ar[d, "\cong"] & H^q (X) \ar[r]\ar[d, "="] &H^q \left(\frac{S_\bullet(X)}{S_\bullet (A) + S_\bullet (B)}\right)^\bullet \ar[r, "\delta^q"]\ar[d, red] &H^{q+1}(S_\bullet (A) + S_\bullet (B))^\bullet \ar[r] \ar[d, "\cong"] & H^{q+1} (X) \ar[r]\ar[d, "="] &\cdots \\
        &\cdots \ar[r] &H^q(A \cup B) \ar[r] & H^q(X) \ar[r] &H^q (X,\, A \cup  B) \ar[r, "\delta'{}^q"]&H^{q+1}(A\cup B) \ar[r] &H^{q+1}(X) \ar[r] &\cdots
    \end{tikzcd}
\end{figure}%

が得られる.5項補題により,赤色をつけた写像
\begin{align}
    H^q \left(\Hom{R}\left(\frac{S_\bullet(X)}{S_\bullet (A) + S_\bullet (B)},\, R\right)\right) \lto H^q (X,\, A \cup B) 
\end{align}
が同型であることがわかる.従って式\eqref{eq:rel-cup}から次の定理が言える:

\begin{mytheo}[label=thm:relative-cup]{空間対のカップ積}
    $\{A,\, B\}$ が\hyperref[def:exc-pair]{切除対}ならば,写像
    \begin{align}
        \smile \colon H^p(X,\, A) \times H^q(X,\, B) \lto H^{p+q} (X,\, A\cup B)
    \end{align}
    はwell-definedである.
\end{mytheo}

\begin{marker}
    \begin{itemize}
        \item $\{A,\, A\}$ は常に\hyperref[def:exc-pair]{切除対}である.従って $H^\bullet (X,\, A)$ は\hyperref[thm:relative-cup]{空間対のカップ積}によって環になる.
        \item $\{A,\, \emptyset\}$ も常に\hyperref[def:exc-pair]{切除対}である.従って
        \begin{align}
            \smile \colon H^p(X,\, A) \times H^q(X) \lto H^{p+q} (X,\, A)
        \end{align}
        は常にwell-defined.
    \end{itemize}
\end{marker}

キャップ積に関しても同様の定理が成り立つ:
\begin{mytheo}[label=thm:relative-cap]{空間対のキャップ積}
    $\{A,\, B\}$ が\hyperref[def:exc-pair]{切除対}ならば,写像
    \begin{align}
        \frown \colon H^q(X,\, A) \times H_{p+q}(X,\, A\cup B) \lto H_{p} (X,\, B)
    \end{align}
    はwell-definedである.
\end{mytheo}

\section{Poincar\"e双対}

% \subsection{Poincar\"e双対}

\begin{mytheo}[label=thm:Poincare-dual]{Poincar\"e双対定理}
    $M$ を\underline{コンパクト}な\underline{向き付け可能}な $n$ 次元位相多様体とする.

    このとき,左 $R$ 加群 $\pi$ に対して以下の同型がある:
    \begin{align}
        \frown [M] \colon H^p (M; \pi) \xrightarrow{\cong} H_{n-p} (M; \pi)
    \end{align}
    ただし,$[M]$ は\hyperref[thm:fundamental-class]{$M$ の基本類}である.
\end{mytheo}

\begin{proof}
    ~\cite[p.276]{Milnor}や~\cite[p.241]{Hatcher}などを参照.
\end{proof}


\end{document}