\documentclass[algtopo_main]{subfiles}
\mathchardef\mhyphen="2D
\begin{document}

% \setcounter{chapter}{0}

\chapter{必要最低限の圏論}


\section{諸定義}

ものの集まりを\textbf{クラス} (class) と呼ぶことにする\footnote{集合全体の集合を考えるとRussellのパラドクスに陥る.これを避けるために,集合よりも上位の概念であるクラスを新しく導入する必要がある.なお,考察の対象とする集合の範囲を\textbf{宇宙} (universe) という大きな集合に属するものに限る流儀もある.この場合,圏を扱うときに既知の集合論をそのまま適用できるが,集合論の公理に宇宙の存在の公理を追加する必要がある.}.
3つのもの $\mathcal{C}$ は以下の要素からなるとする:
\begin{enumerate}
    \item クラス $\mathrm{Ob}(\mathcal{C})$.その元は\textbf{対象} (object) と呼ばれる.
    \item $\forall X,\, Y \in \mathrm{Ob}(\mathcal{C})$ に対して定まる\underline{集合} $\Hom{\mathcal{C}}(X,\, Y)$.
    \item $\forall X,\, Y,\, Z \in \mathrm{Ob}(\mathcal{C})$ に対して定まる\underline{写像} $\circ\mathrel{} \colon \Hom{\mathcal{C}}(X,\, Y) \times \Hom{\mathcal{C}}(Y,\, Z) \to \Hom{\mathcal{C}}(X,\, Z)$.
\end{enumerate}

\begin{marker}
    \begin{itemize}
        \item 集合 $\Hom{\mathcal{C}}(X,\, Y)$ は記号として $\mathrm{Hom}(X,\, Y)$ とか $\mathcal{C}(X,\, Y)$ とも書かれる.
        \item 元 $f \in \Hom{\mathcal{C}}(X,\, Y)$ のことを $X$ から $Y$ への\textbf{射} (morphism) と呼び,$\bm{f \colon X \to Y}$ と書く.
        \item 写像 $\circ$ のことを\textbf{合成} (composite) と呼ぶ.
        射 $f \colon X \to Y$ と $g \colon Y \to Z$ の合成は $\bm{g \circ f \colon X \to Z}$ と書かれる.
    \end{itemize}
\end{marker}


\begin{mydef}[label=def:category]{圏}
    $\mathcal{C}$ が\textbf{圏} (category) であるとは,次の2条件を充たすことを言う:
    \begin{enumerate}
        \item $\forall  X \in \mathrm{Ob}(\mathcal{C})$ に対して\textbf{恒等射} (identity) と呼ばれる射 $1_X \in \Hom{\mathcal{C}}(X,\, X)$ が存在して,$\forall W,\ Y \in \mathrm{Ob}(\mathcal{C})$ および $\forall f \in \Hom{\mathcal{C}}(W,\, X),\; \forall g \in \Hom{\mathcal{C}}(X,\, Y)$ について以下が成り立つ:
        \begin{align}
            1_X \circ f = f,\quad g \circ 1_X = g
        \end{align}
        \item $\forall  X,\, Y,\, Z,\, W \in \mathrm{Ob}(\mathcal{C})$ と $\forall f \in \Hom{\mathcal{C}}(X,\, Y),\; \forall g \in \Hom{\mathcal{C}}(Y,\, Z),\; \forall h \in \Hom{\mathcal{C}}(Z,\, W)$ に対して\textbf{結合則} (associativity) が成り立つ.i.e.
        \begin{align}
            (h \circ g) \circ f = h \circ (g \circ f) \in \Hom{\mathcal{C}}(X,\, W)
        \end{align}
    \end{enumerate}
\end{mydef}

\begin{itemize}
    \item $\Obj{\SETS}$ を全ての集合とし,それらの間の全ての写像を射とし,$\circ$ を通常の写像の合成とするものの集まり $\SETS$ は圏を成す.
    \item $\Obj{\TOP}$ を全ての位相空間とし,それらの間の全ての連続写像を射とし,$\circ$ を通常の写像の合成とするものの集まり $\TOP$ は圏を成す.
    \item 可換環 $R$ について,$\Obj{\MOD{R}}$ を全ての $R$ 加群とし,それらの間の全ての $R$ 準同型を射とし,$\circ$ を通常の写像の合成とするものの集まり $\MOD{R}$ は圏を成す.
\end{itemize}

\begin{mydef}[label=def:small-monoid]{小圏・モノイド}
    \begin{itemize}
        \item $\Obj{\mathcal{C}}$ が集合であるような圏 $\mathcal{C}$ は\textbf{小圏} (small category) と呼ばれる.
        \item $\forall X \in \Obj{\mathcal{C}}$ が一点集合であるような圏 $\mathcal{C}$ は\textbf{モノイド} (monoid) と呼ばれる.
    \end{itemize}
\end{mydef}

\begin{mydef}[label=def:mono-epi]{単射,全射}
    $\Cat{C}$ を\hyperref[def:category]{圏},$A,\, B \in \Obj{\Cat{C}},\; f \in \Hom{\Cat{C}}(A,\, B)$ とする.
    \begin{itemize}
        \item $f$ が\textbf{単射} (monomorphism) であるとは,$\forall X \in \Obj{\Cat{C}}$ に対して写像
        \begin{align}
            f^* \colon \Hom{\Cat{C}} (X,\, A) &\lto \Hom{\Cat{C}} (X,\, B), \\
            g \lmto f \circ g
        \end{align}
        が集合の写像として単射であること.
        \item $f$ が\textbf{全射} (epimorphism) であるとは,$\forall  X \in \Obj{\Cat{C}}$ に対して
        に対して写像
        \begin{align}
            {}^* f \colon \Hom{\Cat{C}} (B,\, X) &\lto \Hom{\Cat{C}} (A,\, X), \\
            g \lmto g \circ f
        \end{align}
        が集合の写像として単射であること.
    \end{itemize}
\end{mydef}


\begin{mydef}[label=def:isomorphism]{同型射}
    圏 $\mathcal{C}$ と射 $f \in \Hom{\Cat{C}}(X,\, Y)$ をとる.
    \begin{itemize}
        \item $f$ が\textbf{同型射} (isomorphism) であるとは,
        \begin{align}
            \exists g \in \Hom{\mathcal{C}}(Y,\, X),\quad g \circ f = 1_X \AND f \circ g = 1_Y
        \end{align}
        が成り立つことを言う.このときの射 $g \in \Hom{\mathcal{C}}(Y,\, X)$ のことを $f$ の\textbf{逆} (inverse) と呼ぶ.
        \item 同型射 $f \in \Hom{\mathcal{C}}(X,\, Y)$ が存在するとき,$X$ と $Y$ は\textbf{同型} (isomorphic) であると言う.
        \item 対象 $X \in \Obj{\mathcal{C}}$ について,$\Hom{\mathcal{C}}(X,\, X)$ の元のうち同型射であるもの全体の集合は合成によって群を成す.この部分集合を $\Aut_{\mathcal{C}}(X) \subset \Hom{\mathcal{C}}(X,\, X)$ と書き,$X$ の $\mathcal{C}$ における\textbf{自己同型群} (automorphism group) と呼ぶ.
    \end{itemize}
\end{mydef}

\begin{mydef}[label=def:sub]{部分対象}
    $\Cat{C}$ を圏とする.
    \begin{itemize}
        \item $A \in \Obj{\Cat{C}}$ の \textbf{部分対象} (subobject) とは,$B \in \Obj{\Cat{C}}$ と\hyperref[def:mono-epi]{単射} $f \in \Hom{\Cat{C}}(B,\, A)$ の組 $(B,\, f)$ のことを言う.
        \item $A \in \Obj{\Cat{C}}$ の部分対象 $(B,\, f),\; (C,\, g)$ が\textbf{同値} (equivalent) であるとは,ある\hyperref[def:isomorphism]{同型射} $\varphi \in \Hom{\Cat{C}} (B,\, C)$ であって $g \circ \varphi = f$ を充たすものが存在することを言う\footnote{$B \simeq C$ や $f \simeq g$ と略記することがある.}.
    \end{itemize}
\end{mydef}

\begin{mydef}[label=def:quo]{商対象}
    $\Cat{C}$ を圏とする.
    \begin{itemize}
        \item $A \in \Obj{\Cat{C}}$ の \textbf{商対象} (quotient bject) とは,$B \in \Obj{\Cat{C}}$ と\hyperref[def:mono-epi]{全射} $f \in \Hom{\Cat{C}}(A,\, )$ の組 $(B,\, f)$ のことを言う.
        \item $A \in \Obj{\Cat{C}}$ の部分対象 $(B,\, f),\; (C,\, g)$ が\textbf{同値} (equivalent) であるとは,ある\hyperref[def:isomorphism]{同型射} $\varphi \in \Hom{\Cat{C}} (B,\, C)$ であって $\varphi \circ f = g$ を充たすものが存在することを言う\footnote{$B \simeq C$ や $f \simeq g$ と略記することがある.}.
    \end{itemize}
\end{mydef}

\begin{mydef}[label=def:fullsub]{充満部分圏}
    2つの圏 $\mathcal{C},\, \mathcal{D}$ をとる.圏 $\mathcal{D}$ が $\mathcal{C}$ の\textbf{充満部分圏} (full subcategory) であるとは,次の2条件が充たされることを言う:
    \begin{enumerate}
        \item クラス $\Obj{\mathcal{D}} \subset \Obj{\mathcal{C}},$
        \item $\forall X,\, Y \in \Obj{\mathcal{D}},\quad \Hom{\mathcal{D}}(X,\, Y) = \Hom{\mathcal{C}}(X,\, Y).$
    \end{enumerate}
\end{mydef}

\begin{mydef}[label=def:opcategory]{反対圏}
    圏 $\mathcal{C}$ に対して,その\textbf{反対圏} (opposite category) $\OP{\Cat{C}}$ を次のように定義する:
    \begin{align}
        \Obj{\OP{\Cat{C}}} &\coloneqq \Obj{\mathcal{C}}, \\
        \Hom{\OP{\Cat{C}}}(X,\, Y) &\coloneqq \Hom{\mathcal{C}}(Y,\, X)\quad \forall X,\, Y \in \Obj{\OP{\Cat{C}}}
    \end{align}
\end{mydef}

\section{関手と自然変換}

圏を対象と考えたとき,射にあたるのは\textbf{関手} (functor) である.

$\mathcal{C},\, \mathcal{D}$ を圏とする.$\mathcal{C}$ と $\mathcal{D}$ の間の対応
\begin{align}
    F \colon \mathcal{C} \longrightarrow \mathcal{D}
\end{align}
は次の2種類の対応付けから成るとする:
\begin{enumerate}
    \item $\forall X \in \Obj{\Cat{C}}$ に対して,記号 $F(X)$ で書かれる $\Obj{\Cat{D}}$ の元を\underline{一意に}対応づける\footnote{いわば,「写像」$F \colon \Obj{\Cat{C}} \to \Obj{\Cat{D}}$ とでも言うべきもの.}.
    \item $\forall X,\, Y \in \Obj{\Cat{C}}$ に対して,射を移す\underline{写像}
    \begin{align}
        F \colon \Hom{\mathcal{C}}(X,\, Y) \longrightarrow \Hom{\mathcal{D}} \bigl( F(X),\, F(Y) \bigr),\; f \longmapsto F(f)
    \end{align}
    を対応付ける.
\end{enumerate}

\begin{mydef}[label=def:covariant]{共変関手}
    $F \colon \Cat{C} \longrightarrow \Cat{D}$ が\textbf{共変関手} (covariant functor) であるとは,以下の2条件を充たすことを言う:
    \begin{description}
        \item[\textbf{(恒等射を保つ)}]
        
        $\forall X \in \Obj{\Cat{C}}$ に対して
        \begin{align}
            F(1_X) = 1_{F(X)} \in \Hom{\Cat{D}} \bigl( F(X),\, F(X) \bigr) 
        \end{align}
        が成り立つ.
        \item[\textbf{(合成を保つ)}]

        $\forall f \in \Hom{\Cat{C}}(X,\, Y),\; \forall g \in \Hom{\Cat{D}}(Y,\, Z)$ に対して
        \begin{align}
            F(g \circ f) = F(g) \circ F(f) \in \Hom{\Cat{D}}\bigl( F(X),\, F(Z) \bigr) 
        \end{align}
        が成り立つ.
    \end{description}
\end{mydef}

射の向きを逆にすると\textbf{反変関手}の定義になる.いま $\mathcal{C},\, \mathcal{D}$ を圏とし,$\mathcal{C}$ と $\mathcal{D}$ の間の対応
\begin{align}
    F \colon \mathcal{C} \longrightarrow\mathcal{D}
\end{align}
は次の2種類の対応付けから成るとする:
\begin{enumerate}
    \item $\forall X \in \Obj{\Cat{C}}$ に対して,記号 $F(X)$ で書かれる $\Obj{\Cat{D}}$ の元を\underline{一意に}対応づける.
    \item $\forall X,\, Y \in \Obj{\Cat{C}}$ に対して,射を移す\underline{写像}
    \begin{align}
        F \colon \Hom{\mathcal{C}}(X,\, Y) \longrightarrow \Hom{\mathcal{D}} \bigl( \textcolor{red}{F(Y)},\, \textcolor{red}{F(X)} \bigr),\; f \longmapsto F(f)
    \end{align}
\end{enumerate}

\begin{mydef}[label=def:contrav]{反変関手}
    $F \colon \Cat{C} \to \Cat{D}$ が\textbf{反変関手} (contravariant functor) であるとは,以下の2条件を充たすことを言う:
    \begin{description}
        \item[\textbf{(恒等射を保つ)}]
        
        $\forall X \in \Obj{\Cat{C}}$ に対して
        \begin{align}
            F(1_X) = 1_{F(X)} \in \Hom{\Cat{D}} \bigl( F(X),\, F(X) \bigr) 
        \end{align}
        が成り立つ.
        \item[\textbf{(合成を保つ)}]

        $\forall f \in \Hom{\Cat{C}}(X,\, Y),\; \forall g \in \Hom{\Cat{D}}(Y,\, Z)$ に対して
        \begin{align}
            F(g \circ f) = \textcolor{red}{F(f)} \circ \textcolor{red}{F(g)} \in \Hom{\Cat{D}}\bigl( \textcolor{red}{F(Z)},\, \textcolor{red}{F(X)} \bigr) 
        \end{align}
        が成り立つ.
    \end{description}
\end{mydef}

関手を対象と考えたときに,射にあたるものが次に定義する\textbf{自然変換}である:

\begin{mydef}[label=def:nat]{自然変換}
    2つの共変関手 $F,\, G \colon \Cat{C} \to \Cat{D}$ について,以下のような対応 $\varphi \colon F \to G$ のことを\textbf{自然変換} (natural transformation) と呼ぶ:
    \begin{enumerate}
        \item $\forall X \in \Obj{\Cat{C}}$ に対して,射
        \begin{align}
            \varphi_X \in \Hom{\Cat{D}} \bigl( F(X),\, G(X) \bigr) 
        \end{align}
        を対応付ける.
        \item $\forall f \in \Hom{\Cat{C}} (X,\, Y)$ について図式\ref{fig:nat}が可換になる.
    \end{enumerate}
\end{mydef}

\begin{figure}[H]
    \centering
    \begin{tikzcd}[row sep=large, column sep=large]
		F(X) \ar[d, "\varphi_X"] \ar[r, "F(f)"] &F(Y) \ar[d, "\varphi_Y"] \\
		G(X)                      \ar[r, "G(f)"] &G(Y)
	\end{tikzcd}
    \caption{自然変換}
    \label{fig:nat}
\end{figure}%

共変関手 $F$ から $G$ への自然変換全体が成すクラスを $\NAT(F,\, G)$ と書くことにする.

\begin{mydef}[label=def:naturallyeq]{自然同値}
    2つの共変関手 $F,\, G \colon \Cat{C} \longrightarrow \Cat{D}$ の間の自然変換 $\varphi \colon F \longrightarrow G$ を与える.
    \begin{itemize}
        \item $\varphi$ が\textbf{自然同値} (natural equivalence) であるとは,$\forall X \in \Obj{\Cat{C}}$ に対して $\varphi_X \in \Hom{\Cat{D}} \bigl( F(X),\, G(X) \bigr)$ が\hyperref[def:isomorphism]{同型射}であることを言う.
        \item 共変関手 $F,\, G$ が\textbf{自然同値} (naturally equivalent) であるとは,$F$ から $G$ への自然同値が存在することを言う.
        \item 共変関手 $F$ に対し,$\mathrm{id}{F(X)} \colon F(X) \longrightarrow F(X)$ により定まる $F$ から $F$ 自身への自然変換 $\varphi$ を\textbf{恒等自然変換} (identity natural transformation) と呼ぶ.
    \end{itemize}
\end{mydef}

\begin{mydef}[label=def:faithful]{}
    共変関手 $F \colon \Cat{C} \longrightarrow \Cat{D}$ を与える.
    \begin{itemize}
        \item $F$ が\textbf{忠実} (faithful) であるとは, $\forall X,\,Y \in \Obj{\Cat{C}}$ に対して写像
        \begin{align}
            F \colon \Hom{\mathcal{C}}(X,\, Y) \longrightarrow \Hom{\mathcal{D}} \bigl( F(X),\, F(Y) \bigr),\; f \longmapsto F(f)
        \end{align}
        が単射であること.
        \item $F$ が\textbf{充満} (full) であるとは, $\forall X,\,Y \in \Obj{\Cat{C}}$ に対して写像
        \begin{align}
            F \colon \Hom{\mathcal{C}}(X,\, Y) \longrightarrow \Hom{\mathcal{D}} \bigl( F(X),\, F(Y) \bigr),\; f \longmapsto F(f)
        \end{align}
        が全射であること.
        \item $F$ が\textbf{本質的全射} (essentially surjective) であるとは, $\forall Z \in \Obj{\Cat{D}}$ に対して $X \in \Obj{\Cat{C}}$ が存在して $F(X)$ が $Y$ と同型になること.
    \end{itemize}
    忠実充満関手のことを\textbf{埋め込み}と呼ぶことがある.
\end{mydef}



\end{document}
