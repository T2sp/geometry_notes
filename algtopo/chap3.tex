\documentclass[algtopo_main]{subfiles}
\mathchardef\mhyphen="2D
\begin{document}

\setcounter{chapter}{2}


\chapter{コホモロジーの定義}

まず,純粋に代数的な準備をする.

\begin{mylem}[label=lem:splitting]{分裂補題}
    左 $R$ 加群の短完全列
    \begin{align}
        \label{SES:3-1}
        0 \lto M_1 \xrightarrow{i_1} M \xrightarrow{p_2} M_2 \lto 0
    \end{align}
    が与えられたとする.このとき,以下の二つは同値である:
    \begin{enumerate}
        \item 左 $R$ 加群の準同型 $i_2 \colon M_2 \lto M$ であって $p_2 \circ i_2 = 1_{M_2}$ を充たすものが存在する
        \item 左 $R$ 加群の準同型 $p_1 \colon M \lto M_1$ であって $p_1 \circ i_1 = 1_{M_1}$ を充たすものが存在する
    \end{enumerate}
\end{mylem}

\begin{proof}
    \begin{description}
        \item[\textbf{(1) $\Longrightarrow$ (2)}] 写像
        \begin{align}
            p_1' \colon M \lto M,\; x \lmto x - i_2 \bigl( p_2(x) \bigr) 
        \end{align}
        を定義すると,
        \begin{align}
            p_2 \bigl( p_1'(x) \bigr) = p_2(x) - ((p_2 \circ i_2) \circ p_2)(x) = p_2(x) - p_2(x) = 0
        \end{align}
        が成り立つ.従って,\eqref{SES:3-1}が完全列であることを使うと $p_1'(x) \in \Ker p_2 = \Im i_1$ である.さらに $i_1$ が単射であることから
        \begin{align}
            \exists ! y \in M_1,\; p_1'(x) = i_1(y)
        \end{align}
        が成り立つ.ここで写像
        \begin{align}
            p_1 \colon M \lto M_1,\; x \lmto y
        \end{align}
        を定義するとこれは準同型写像であり,$\forall x \in M_1$ に対して
        \begin{align}
            p_1' \bigl( i_1(x) \bigr) = i_1(x) - (i_2 \circ (p_2 \circ i_1))(x) = i_1(x)
        \end{align}
        が成り立つ\footnote{\eqref{SES:3-1}が完全列であるため,$p_2 \circ i_1 = 0$}ことから
        \begin{align}
            (p_1 \circ i_1)(x) = x
        \end{align}
        とわかる.i.e. $p_1 \circ i_1 = 1_{M_1}$
        \item[\textbf{(1) $\Longleftarrow$ (2)}] \eqref{SES:3-1}は完全列であるから $M_2 = \Ker 0 =  \Im p_2$ である.
        従って $\forall x \in M_2 = \Im p_2$ に対して,$x = p_2(y)$ を充たす $y \in M$ が存在する.ここで写像
        \begin{align}
            i_2 \colon M_2 \lto M,\; x \lmto y - i_1 \bigl( p_1(y) \bigr) 
        \end{align}
        はwell-definedである.$x = p_2(y')$ を充たす勝手な元 $y' \in M$ をとってきたとき,$p_2(y-y') = 0$ より $y-y' \in \Ker p_2 = \Im i_1$ だから,$i_1$ の単射性から
        \begin{align}
            \exists ! z \in M_1,\quad y-y' = i_1(z)
        \end{align}
        が成り立ち,このとき
        \begin{align}
           \Bigl( y - i_1 \bigl( p_1(y) \bigr) \Bigr) - \Bigl( y' - i_1 \bigl( p_1(y') \bigr)  \Bigr) = i_1(z) - (i_1 \circ (p_1 \circ i_1))(z) = i_1(z) - i_1(z) = 0
        \end{align}
        とわかるからである.$i_2$ は準同型写像であり,$\forall x \in M_2$ に対して
        \begin{align}
            (p_2 \circ i_2)(x) = p_2(y) - ((p_2 \circ i_1) \circ p_1)(y) = p_2(y) = x
        \end{align}
        なので $p_2 \circ i_2 = 1_{M_2}$.
    \end{description}
\end{proof}

\begin{mycol}[label=col:split]{}
    左 $R$ 加群の短完全列
    \begin{align}
        0 \lto M_1 \xrightarrow{i_1} M \xrightarrow{p_2} M_2 \lto 0
    \end{align}
    が補題\ref{lem:splitting}の条件を充たすならば
    \begin{align}
        M \cong M_1 \oplus M_2 
    \end{align}
\end{mycol}

\begin{proof}
    補題\ref{lem:splitting}の条件 (1) が満たされているとする.このとき補題\ref{lem:splitting}証明から $\forall x \in M$ に対して
    \begin{align}
        i_1  \bigl( p_1(x) \bigr) = p_1'(x) = x - i_2 \bigl( p_2(x) \bigr) \IFF i_1 \bigl(p_1(x)  \bigr) + i_2 \bigl( p_2(x) \bigr) = x
    \end{align}
    また,完全列の定義から $p_2 \bigl( i_1(x) \bigr) = 0$ であるから $\forall x \in M_2$ に対して
    \begin{align}
        p_1' \bigl( i_2(x) \bigr) = i_2(x) - ((i_2 \circ p_2) \circ i_2)(x) = 0 = i_1(0)
    \end{align}
    であり,結局 $p_1 \bigl( i_2(x) \bigr) = 0$ とわかる.

    ここで準同型写像
    \begin{align}
        &f\colon M_1 \oplus M_2 \lto M,\; (x,\, y) \lmto i_1(x) + i_2(y), \\
        &g\colon M \lto M_1 \oplus M_2,\; x \lmto \bigl( p_1(x),\, p_2(x) \bigr) 
    \end{align}
    を定めると
    \begin{align}
        (g \circ f)(x,\, y) &= \bigl( p_1(i_1(x)) + p_1(i_2(y)) ,\, p_2(i_1(x)) + p_2(i_2(x)) \bigr) = (x,\, y) , \\
        (f\circ g)(x) &= i_1(p_1(x)) + i_2(p_2(x)) = x
    \end{align}
    なので $f,\, g$ は同型写像.
\end{proof}

\begin{mydef}[label=def:split]{分裂}
    左 $R$ 加群の短完全列
    \begin{align}
        0 \lto M_1 \xrightarrow{i_1} M \xrightarrow{p_2} M_2 \lto 0
    \end{align}
    が\textbf{分裂} (split) するとは,補題\ref{lem:splitting}の条件を充たすことをいう.
\end{mydef}

左 $R$ 加群 $N$ および左 $R$ 加群の準同型 $f \colon M_1 \lto M_2$ に対して
\begin{align}
    f_* &\colon \Hom{R}(N,\, M_1) \lto \Hom{R}(N,\, M_2),\; \varphi \lmto f \circ \varphi \\
    f^* &\colon \Hom{R}(M_2,\, N) \lto \Hom{R}(M_1,\, N),\; \varphi \lmto \varphi \circ f
\end{align}
とおく.$f_*,\, f^*$ は $\mathbb{Z}$ 加群の準同型である.

\begin{myprop}[label=prop:Hom-split]{}
    \begin{enumerate}
        \item $0 \lto M_1 \xrightarrow{f} M_2 \xrightarrow{g} M_3$ を左 $R$ 加群の完全列,$N$ を左 $R$ 加群とすると,$\mathbb{Z}$ 加群\footnote{i.e. 和について可換群}の完全列
        \begin{align}
            0 \lto \Hom{R}(N,\, M_1) \xrightarrow{f_*} \Hom{R}(N,\, M_2) \xrightarrow{g_*} \Hom{R}(N,\, M_3)
        \end{align}
        が成り立つ.
        \item $M_1 \xrightarrow{f} M_2 \xrightarrow{g} M_3 \lto 0$ を左 $R$ 加群の完全列,$N$ を左 $R$ 加群とすると,$\mathbb{Z}$ 加群の完全列
        \begin{align}
            0 \lto \Hom{R}(M_3,\, N) \xrightarrow{g^*} \Hom{R}(M_2,\, N) \xrightarrow{f^*} \Hom{R}(M_3,\, N)
        \end{align}
        が成り立つ.
        \item $0 \lto M_1 \xrightarrow{f} M_2 \xrightarrow{g} M_3 \lto 0$ を\textbf{分裂する}左 $R$ 加群の完全列,$N$ を左 $R$ 加群とすると,$\mathbb{Z}$ 加群の完全列
        \begin{align}
            &0 \lto \Hom{R}(N,\, M_1) \xrightarrow{f_*} \Hom{R}(N,\, M_2) \xrightarrow{g_*} \Hom{R}(N,\, M_3) \lto 0\\
            &0 \lto \Hom{R}(M_3,\, N) \xrightarrow{g^*} \Hom{R}(M_2,\, N) \xrightarrow{f^*} \Hom{R}(M_1,\, N) \lto 0
        \end{align}
        が成り立つ.
    \end{enumerate}
\end{myprop}

\begin{proof}
    \begin{enumerate}
        \item  まず,$\varphi \in \Ker f_* \IFF f \circ \varphi = 0$ かつ $f$ は単射なので$\Ker f_* \subset \Im 0$ である\footnote{$f \colon M_1 \to M_2$ は単射だから $\Ker f = \{0\}$.故に $\forall x \in N$ に対して $f \bigl( \varphi(x) \bigr) = 0 \IFF \varphi(x) \in \Ker f = \{0\}$.i.e. $\varphi = 0$} .
        $\Ker f_* \supset \Im 0$ は明らかであり,$\Ker f_* = \Im 0$ が言えた.
        
         次に,$g \circ f = 0$ なので 
        \begin{align}
            \psi \in \Im f_* \IFF \exists \alpha \in \Hom{R}(N,\, M_1),\; \psi = f \circ \alpha\IMP g_* \bigl( \psi \bigr) = g \circ f \circ \alpha = 0
        \end{align}
        が成り立ち,$\Ker g_* \supset \Im f_*$ がわかる.また,
        \begin{align}
            \psi \in \Ker g_* \IFF g \circ \psi = 0 \IMP \Im \psi \subset \Ker g = \Im f
        \end{align}
        が成り立つ.$f$ は単射だから\footnote{したがって $\forall x \in N$ に対して $\exists! y \in \Im f \subset M_2,\; \psi(x) = f(y)$ であり,写像 $\beta \colon N \to M_2,\; x \mapsto y$ はwell-defined.}
        \begin{align}
            \exists \beta \in \Hom{R}(N,\, M_2),\quad \psi = f \circ \beta = f_*(\beta) \in \Im f_*
        \end{align}
        が成り立つ.i.e. $\Ker g_* \subset \Im f_*$ である.
        \item  まず,$\varphi \in \Ker g^* \IFF \varphi \circ g = 0$ かつ $g$ が全射なので $\Ker g_* \subset \Im 0$ である\footnote{$g \colon M_2 \to M_3$ が全射なので $\forall x \in M_3,\; \exists y \in M_2,\; x = g(y)$.故に $\varphi(x) = (\varphi \circ g)(y) = 0$.i.e. $\varphi = 0$.}.
        $\Ker g^* \supset \Im 0$ は自明なので $\Ker g^* = \Im 0$ が言えた.

         次に,$g \circ f = 0$ より
        \begin{align}
            \psi \in \Im g^* &\IFF \exists \alpha \in \Hom{R}(M_3,\, N),\; \psi = \alpha \circ g \IMP f^*(\psi) = \alpha \circ g \circ f = 0
        \end{align}
        が成り立ち,$\Ker f^* \supset \Im g^*$ がわかる.また,
        \begin{align}
            \psi \in \Ker f^* \IFF \psi \circ f = 0 \IMP \psi(\Im f) = 0
        \end{align}
        より $\Im f \subset\Ker \psi$ であるから,\hyperref[lem:fig:quomod-univ]{商加群の普遍性}より次の可換図式が成り立つ\footnote{$M_1 \xrightarrow{f} M_2 \xrightarrow{g} M_3 \to 0$ が完全列なので $\Im f = \Ker g$ かつ $M_3= \Im g$.よって準同型定理から $M_2/\Im f = M_2/\Ker g \cong M_3$.}:
        \begin{figure}[H]
            \centering
            \begin{tikzcd}[row sep=large, column sep=large]
                &M_2 \ar[r, "\psi"]\ar[d, "g"]           &N \\
                &M_2/\Im f \cong M_3 \ar[ur, "\exists! h"'] &
            \end{tikzcd}
        \end{figure}%
        i.e. $\psi = h \circ g = g^*(h) \in \Im g^*$ であり,$\Ker f^* \subset \Im g^*$ がわかった.
        \item  与えられた完全列が分裂するので,
        \begin{align}
            &\exists i_2 \in \Hom{R}(M_3,\,M_2),\; g \circ i_2 = 1_{M_3} \\
            &\exists p_1 \in \Hom{R}(M_2,\,M_1),\; p_1 \circ f = 1_{M_1}
        \end{align}
        である.故に
        \begin{align}
            g_* \circ i_2{}_* &= (g \circ i_2)_* = 1_{M_3}{}_* = 1_{\Hom{R}(N,\, M_3)} \\
            f^* \circ p_1{}^* &= (p_1 \circ f)^* = 1_{M_1}{}^* = 1_{\Hom{R}(M_1,\, N)}
        \end{align}
        なので,$\forall \varphi \in \Hom{R}(N,\, M_3),\; \forall \psi \in \Hom{R}(M_1,\, N)$ に対して
        \begin{align}
            \varphi &= g_* \bigl( i_2{}_*(\varphi) \bigr) \in \Im g_* \\
            \psi &= f^* \bigl( p_2{}^*(\psi) \bigr) \in \Im f^*
        \end{align}
        が成り立つ.i.e. $\Im g_* = \Ker 0,\; \Im f^* = \Ker 0$ である.

        残りは (1), (2) から従う.
    \end{enumerate}
    
\end{proof}

一般のコチェイン複体を定義する.

\begin{mydef}[label=def:CCC]{コチェイン複体}
    $R$ を可換環とする.加群と準同型の族 $C^\bullet = \Familyset[\big]{C^q,\,\delta^q \colon C^q \to C^{q+1}}{q \ge 0}$ が\textbf{$\bm{R}$ コチェイン複体}であるとは,
    $\forall q \ge 0$ について $C^q$ が $R$-加群,$\delta^q$ が $R$ 準同型であって
    \begin{align}
        \delta^{q+1} \delta^q = 0
    \end{align}
    が成り立つことを言う.
\end{mydef}

\begin{itemize}
    \item $\delta^q$ を\textbf{余境界写像} (coboundary map) と呼ぶ.
    \item $C^q$ の部分加群
    \begin{align}
        Z^q(C^\bullet) \coloneqq \Ker \bigl( \delta^q \colon C^q \to C^{q+1} \bigr) 
    \end{align}
    を\textbf{第 $\bm{q}$ コサイクル群},その元を\textbf{$\bm{q}$-コサイクル} ($q$-cocycle),
    \item $C^q$ の部分加群
    \begin{align}
        B^q(C^\bullet) \coloneqq \Im \bigl( \delta^{q-1} \colon C^{q-1} \to C^q \bigr) 
    \end{align}
    を\textbf{第 $\bm{q}$ コバウンダリー群},その元を\textbf{$\bm{q}$-コバウンダリー} ($q$-coboundary) と呼ぶ.
\end{itemize}

\section{コチェイン写像}

$C^\bullet = \Familyset[\big]{C^q,\, \delta^q}{q \ge 0},\;D^\bullet = \Familyset[\big]{D^q,\, \delta'{}^q}{q \ge 0}$ をチェイン複体とする.

\begin{mydef}[label=def:cochainmap]{コチェイン写像}
    準同型 $f_q \colon C^q \to D^q$ の族 $f_\bullet \coloneqq \Familyset[\big]{f_q}{q\ge 0}$ が\textbf{コチェイン写像} (cochain map) であるとは,$\forall q \ge 0$ に対して
    \begin{align}
        \delta'{}^q \circ f_q = f_{q+1} \circ \delta^q
    \end{align}
    が成り立つことを言う.i.e. 図式\ref{fig:cochainmap}が可換になると言うこと.
\end{mydef}

\begin{figure}[H]
    \centering
    \begin{tikzcd}[row sep=large, column sep=large]
		&\cdots \ar[r, "\delta^{q-2}"]  &C^{q-1} \ar[d, "f_{q-1}"]\ar[r, "\delta^{q-1}"]  &C^{q} \ar[d, "f_{q}"]\ar[r, "\delta^{q}"]  &C^{q+1} \ar[d, "f_{q+1}"]\ar[r, "\delta^{q+1}"] &\cdots \\
		&\cdots \ar[r, "\delta'{}^{q-2}"] &D^{q-1}                  \ar[r, "\delta'{}^{q-1}"] &D^{q} 		        \ar[r, "\delta'{}^{q}"] &D^{q+1}                  \ar[r, "\delta'{}^{q+1}"] & \cdots
	\end{tikzcd}
    \caption{コチェイン写像}
    \label{fig:cochainmap}
\end{figure}%



\begin{mylem}[label=lem:chain1]{}
    $\forall q \ge 0$ について以下が成り立つ:
    \begin{align}
        f_q \bigl( Z^q(C^\bullet) \bigr) &\subset Z^q(D^\bullet) \\
        f_q \bigl( B^q(C^\bullet) \bigr) &\subset B^q(D^\bullet)
    \end{align}
\end{mylem}

\begin{proof}
    一つ目は
    \begin{align}
        z \in Z^q(C^\bullet) = \Ker \delta^q &\IFF \delta^q(z) = 0 \\
        &\IMP \delta'{}^q \bigl( f_q(z) \bigr) = f_{q+1} \bigl( \delta^q(z) \bigr) = 0 \\
        &\IFF f_q(z) \in Z^q(D^\bullet) = \Ker \delta'{}^q
    \end{align}
    二つ目は
    \begin{align}
        b \in B^q(C^\bullet) = \Ker \delta^q &\IFF \exists \beta \in C^{q-1},\; b = \delta^{q-1}(\beta)\\
        &\IMP f_q(b) = \delta'{}^{q-1} \bigl( f_{q-1}(\beta) \bigr)  \in B^q(D^\bullet) = \Im \delta'{}^{q-1}
    \end{align}
\end{proof}

いま,コチェイン写像 $f_\bullet \coloneqq \Familyset[\big]{f_q \colon C^q \to D^q}{q \ge 0}$ を与える.
標準射影を
\begin{align}
    \pi \colon Z^q(D^\bullet) \to Z^q(D^\bullet) / B^q(D^\bullet), z \mapsto [z] = z + B^q(D^\bullet)
\end{align}
とおくと,
補題\ref{lem:chain1}から
\begin{align}
    (\pi\circ f_q)\bigl( B^q(C^\bullet)\bigr) \subset \pi \bigl( B^q(D^\bullet) \bigr) = \{0_{Z^q(D^\bullet)/B^q(D^\bullet)} \}
\end{align}
が成り立つ.i.e. $B^q(C^\bullet) \subset \Ker \pi\circ f_q$ である.
よって\hyperref[lem:quomod-univ]{商加群の普遍性}から,
次のような可換図式を書くことができる:
\begin{figure}[H]
    \centering
    \begin{tikzcd}[row sep=large, column sep=large]
        Z^q(C^\bullet) \ar[d, "p"']\ar[r, "f_q"] &Z^q(D^\bullet) \ar[r, "\pi"]  &Z^q(D^\bullet)/B^q(D^\bullet)\\
        Z^q(C^\bullet)/B^q(C^\bullet) \arrow[ur, dashed, blue, "\exists!\overline{f_q}"]\arrow[urr, dashed, red, "\exists!\overline{\pi \circ f_q}"']  & &
    \end{tikzcd}
    \caption{誘導準同型}
    \label{fig:induced-cohom}
\end{figure}%

\begin{mydef}[label=def:induced-cochain]{コチェイン写像による誘導準同型}
    図式\ref{fig:induced-cohom}中の赤字で示した準同型は\textbf{誘導準同型} (induced homomorphism) と呼ばれ,コホモロジー群の間の準同型を定める:
    \begin{align}
        f_* \coloneqq \overline{\pi \circ f_q} \colon H^q(C^\bullet) \to H^q(D^\bullet),\; [z] \mapsto [f_q(z)]
    \end{align}
\end{mydef}

\section{特異コホモロジー}

しばらくの間 $\mathbb{Z}$ 加群 $M$ を一つ取って固定する.
\begin{mydef}[label=def:singularcochain]{特異 $q$ コチェイン}
    位相空間 $X$ および $\forall q \ge 0$ に対して
    \begin{align}
        S^q(X;\, M) \coloneqq \Homo \bigl( S_q(X),\, M \bigr) 
    \end{align}
    と定義される $S^q(X;\, M)$ の元は\textbf{特異 $\bm{q}$ コチェイン} (sigular $q$-cochain) と呼ばれる.
\end{mydef}

双線型写像
\begin{align}
    \label{def:KroneckerPairing}
    \dualip{\;}{\;} \colon S^q(X;\, M) \times S_q(X) \to M,\; (u,\, c) \mapsto u(c)
\end{align}
は\textbf{Kronecker pairing}と呼ばれる.

\begin{mydef}[label=def:SCH-cochainmap]{特異コチェインの余境界写像}
    境界写像 $\partial_{q+1} \colon S_{q+1}(X) \to S_q(X)$ の双対写像を
    \begin{align}
        \delta^q \colon S^q(X;\, M) \to S^{q+1}(X;\, M),\; u \mapsto u \circ \partial_{q+1}
    \end{align}
    と書く.
\end{mydef}
定義より以下が成り立つ:
\begin{align}
    \dualip{\delta^q u}{c} = \dualip{u}{\partial_{q+1} c}
\end{align}

前節の議論により
\begin{align}
    \delta^{q} \delta^{q-1} = 0
\end{align}
は分かっている.これは
\begin{align}
    \Im \delta^{q-1} \subset \Ker \delta^q
\end{align}
を意味するので,
\begin{align}
    Z^q\bigl(S^\bullet(X;\, M)\bigr) &\coloneqq \Ker \delta^q \\
    B^q \bigl( S^\bullet(X;\, M) \bigr) &\coloneqq \Im \delta^{q-1}
\end{align}
と書くと次のような構成ができる:
\begin{mydef}[label=def:SCH]{特異コホモロジー}
    族 $\Familyset[\big]{S^q(X;\, M),\, \delta^q}{q\ge 0}$ は位相空間 $X$ の\textbf{特異コチェイン複体} (singular cochain complex) と呼ばれ,そのコホモロジーは
    \begin{align}
        H^q(X;\, M) \coloneqq Z^q\bigl(S^\bullet(X;\, M)\bigr)/B^q\bigl(S^\bullet(X;\, M)\bigr)
    \end{align}
    と書かれる.$H^q(X;\, M)$ は $\mathbb{Z}$ 加群 $M$ に値を持つ $X$ の\textbf{特異コホモロジー群} (singular cohomology group with coefficients in the $\mathbb{Z}$ module $M$) と呼ばれる.
\end{mydef}

\subsection{Kronecker写像}

\hyperref[def:KroneckerPairing]{Kronecker pairing}は双線型写像
\begin{align}
    \label{eq:Kronecker-induced}
    \dualip{\;}{\;} \colon H^q(X;\, M) \times H_q(X) \to M,\; ([u],\, [z]) \mapsto u(z)
\end{align}
を誘導する.

\begin{mylem}[]{}
    上で定義した写像 $\dualip{\;}{\;} \colon H^q(X;\, M) \times H_q(X) \to M$ はwell-definedである.
\end{mylem}

\begin{proof}
    $\forall f \in S^{q-1}(X;\, M),\; \forall c \in S_{q+1}(X)$ に対して
    \begin{align}
        u + \delta^{q-1} f &\in [u] = u + \Im \delta^{q-1} , \\
        z + \partial_{q+1} c &\in [z] = z + \Im \partial_{q+1} 
    \end{align}
    である.これらのpairingを計算すると
    \begin{align}
        (u + \delta^{q-1} f)(z + \partial_{q+1} c) &= (u + f \partial_q)(z + \partial_{q+1} c) \\
        &= u(z) + f \bigl( \partial_q z \bigr) + u \bigl( \partial_{q+1}c \bigr) + \partial_q \partial_{q+1} c \\
        &= u(z) + f \bigl( \partial_q z \bigr) + (\delta^q u)(c)  
    \end{align}
    $u \in \Ker \delta^q,\; z \in \Ker \partial_q$ なので結局
    \begin{align}
        (u + \delta^{q-1} f)(z + \partial_{q+1} c) = u(z)
    \end{align}
    が従う.
\end{proof}

\begin{mydef}[label=def:KroneckerMap]{Kronecker写像}
    \eqref{eq:Kronecker-induced}の双線型写像が定める準同型
    \begin{align}
        \kappa \colon H^q(X;\, M) \to \Homo \bigl( H_q(X),\, M \bigr),\; [u] \mapsto \dualip{[u]}{\mhyphen}
    \end{align}
    を\textbf{Kronecker写像} (Kronecker map) と呼ぶ.
\end{mydef}

\subsection{関手性}

$f \in \Hom{\TOP}(X,\, Y)$ をとる.補題\ref{lem:SC-chain}より,$f$ はチェイン写像
\begin{align}
    f_* \colon S_\bullet(X) \longrightarrow S_\bullet(Y)
\end{align}
を誘導した.

一方,
\begin{align}
    f^* \colon S^q(\textcolor{red}{Y};\, M) \longrightarrow S^q(\textcolor{red}{X};\, M),\; u \longmapsto u \circ f_*
\end{align}
という準同型も考えられる. $f_*$ がチェイン写像であることから
\begin{align}
    (\delta^q \circ f^*)(u) &= f^*(u) \circ \partial_{q+1} \\
    &= u \circ (f_* \circ \partial_{q+1}) \\
    &= (u \circ \partial'_{q+1}) \circ f_* \\
    &= f^*(u \circ \partial'_{q+1}) \\
    &= (f^* \circ \delta'{}^q)(u)
\end{align}
が成立するので $f^*$ は\hyperref[def:cochainmap]{コチェイン写像}の要件を充たす.
従って\hyperref[def:induced-cochain]{コチェイン写像による誘導準同型}を考えることができる:
\begin{align}
    f^* \colon H^q(\textcolor{red}{Y};\, M) \longrightarrow H^q(\textcolor{red}{X};\, M),\; [u] \mapsto [f^*(u)]
\end{align}
$\forall [u] \in H^q(Y;\, M),\; [z] \in H_q(X)$ に対して
\begin{align}
    \dualip{f^*[u]}{[z]} &= \dualip{[f^* u]}{[z]} \\
    &= (f^* \circ u)(z) \\
    &= (u \circ f_*)(z) \\
    &= u \bigl( f_*(z) \bigr) \\
    &= \dualip{[u]}{f_*[z]}
\end{align}
が成り立つ.この意味で $f^*$ は $f_*$ の一種の双対写像であると言える.

\begin{myprop}[label=prop:co-property1]{コホモロジーの関手性}
    $H^q$ は位相空間の圏 $\TOP$ から $\mathbb{Z}$ 加群の圏 $\MOD{\mathbb{Z}}$ への\textbf{反変}関手となる.
    i.e. $\forall X,\, Y,\, Z \in \Obj{\TOP}$ と $\forall q \ge 0$ に対して以下が成り立つ:
    \begin{enumerate}
        \item 恒等写像 $1_X \in \Hom{\TOP}(X,\, X)$ について
        \begin{align}
            (1_X)^* = 1_{H^q(X;\ M)} \in \Hom{\MOD{\mathbb{Z}}}\bigl( H^q(X;\, M),\, H^q(X;\, M) \bigr) 
        \end{align}
        \item 連続写像 $f \in \Hom{\TOP}(X,\, Y),\; g \in \Hom{\TOP}(Y,\, Z)$ について
        \begin{align}
            (g\circ f)^* = \textcolor{red}{f^* \circ g^*} \in \Hom{\MOD{\mathbb{Z}}}\bigl( H^q(\textcolor{red}{Z};,\, M),\, H_q(\textcolor{red}{X};\, M) \bigr) 
        \end{align}
    \end{enumerate}
\end{myprop}

% \begin{proof}
%     $\forall z \in Z^q(X) = \Ker \bigl(\partial_q \colon S_q(X) \to S_{q-1}(X) \bigr)$ をとる.
%     補題\ref{lem:SC-chain}の対応 $\mhyphen_q \colon \Hom{\TOP}(X,\, Y) \to \Hom{\MOD{\mathbb{Z}}} \bigl( S_q(X),\, S_q(Y) \bigr)$ 
%     により定まる\hyperref[def:chainmap]{チェイン写像}の\hyperref[def:induced-chain]{誘導準同型}を考えることで
%     \begin{align}
%         (1_X)_q{}_*([z]) &= [(1_X)_q(z)] = [z] = 1_{H_q(X)}([z]), \\
%         (g \circ f)_q{}_*([z]) &= [(g\circ f)_q(z)] = [(g_q \circ f_q)(z)] = (g_q{}_* \circ f_q{}_*) ([z]).
%     \end{align}
% \end{proof}

% \section{空間対のコホモロジー}

% \textbf{空間対 $\bm{(X,\, A)}$ の特異コチェイン複体}を次のように定義する:
% \begin{align}
%     S^q(X,\, A;\, M) \coloneqq \Hom{R} \bigl( S_q(X,\, A;,\, M),\, M \bigr) 
% \end{align}

% \subsection{コホモロジー完全列}

% \begin{mylem}[label=lem:split]{}
%     $R$ 加群の短完全列
%     \begin{align}
%         0 \longrightarrow A \longrightarrow B \longrightarrow C \longrightarrow 0
%     \end{align}
%     が与えられたとする.このとき,
%     \begin{align}
%         0 &\longrightarrow \Hom{R}(C,\, M) \longrightarrow \Hom{R}(B,\, M) \longrightarrow \Hom{R}(A,\, M) \longrightarrow 0, \\
%         0 &\longrightarrow \Hom{R}(M,\, A) \longrightarrow \Hom{R}(M,\, B) \longrightarrow \Hom{R}(M,\, C) \longrightarrow 0
%     \end{align}
%     は完全列である.
% \end{mylem}

% \begin{proof}
%     \begin{enumerate}
%         \item 
%     \end{enumerate}
    
% \end{proof}

\section{コホモロジーのEilenberg-Steenrod公理系}

\begin{myaxiom}[label=ax:cohomology]{コホモロジーのEilenberg-Steenrod公理系}
    \textbf{コホモロジー理論}は
    \begin{align}
        H^\bullet \colon \{\text{pairs, ct. maps}\} \longrightarrow \{\text{graded}\; R\; \text{modules},\; \text{homomorphisms}\}
    \end{align}
    なる反変関手であって,以下の公理を充たすものである:
    \begin{description}
        \item[\textbf{(ES-ch1)}] 任意の空間対 $(X,\, A)$ および非負整数 $q \ge 0$ に対して\textbf{自然な}準同型
        \begin{align}
            \delta \colon H^q(A) \longrightarrow H^{q+1}(X,\, A)
        \end{align}
        が存在して,包含写像 $i \colon A \hookrightarrow X,\; j \colon X \hookrightarrow (X,\, A)$ を用いて次のホモロジー長完全列が誘導される:
        \begin{align}
            \cdots \to H^{q-1}(A) \xrightarrow{\delta} H^q(X,\, A) \xrightarrow{j^*} H_q(X) \to \cdots 
        \end{align}
        \item[\textbf{(ES-ch2)}] 2つの連続写像 $f,\, g \colon (X,\, A) \to (Y,\, B)$ がホモトピックならば,誘導準同型 $f^*,\, g^* \colon H^q(Y,\, B) \to H^q(X,\, A)$ は
        \begin{align}
            f^* = g^*
        \end{align}
        となる.
        \item[\textbf{(ES-h3)}]  $U \subset X \AND \overline{U} \subset \Int(A)$ ならば,包含写像 $i \colon (X \setminus U,\, A \setminus U) \longrightarrow (X,\, A)$ が誘導する準同型
        \begin{align}
            i^* \colon H^q(X,\, A) \longrightarrow H^q(X \setminus U,\, A \setminus U)
        \end{align}
        は$\forall q \ge 0$ に対して同型となる.
        \item[\textbf{(ES-h4)}] $q \neq 0$ ならば $H^q(*) = 0$.
    \end{description}
\end{myaxiom}

\end{document}