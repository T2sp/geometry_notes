\documentclass[geometry_main]{subfiles}

\begin{document}

\setcounter{chapter}{3}

\chapter{微分形式}

\section{外積代数}

\begin{mydef}[label=def.ext]{外積代数}
$V$ を体 $\mathbb{K}$ 上のベクトル空間とする.このとき\textbf{外積代数} (exterior algebra) $\left( \extp^\bullet (V),\, +,\, \wedge \right)$ は,以下のように定義される $\mathbb{K}$ 上の\hyperref[ax.alg]{多元環}(定義\ref{ax.alg})である:
	\begin{enumerate}
		\item $\mathbb{K}$ 上 $V$ の元によって生成される
		\item 単位元 $1$ を持つ
		\item 任意の $x,\, y \in V$ に対して以下の関係式が成り立つ:
		\begin{align}
			x \wedge y = - y \wedge x
		\end{align}
	\end{enumerate}
\end{mydef}

$\forall x \in V$の次数を $1$ とおくことで,$\extp^\bullet (V)$ の単項式の次数が定義される.次数が $k$ の単項式の $\mathbb{K}$ 係数線型結合全体の集合を $\extp^k (V)$ と書くと,直和分解
\begin{align}
	\extp^\bullet (V) = \bigoplus_{k=0}^\infty \extp^k (V)
\end{align}
が成立する.

\begin{marker} 
	$\extp^0(V) = \mathbb{K}$ と約束する.また,自然に $\extp^1(V) \cong V$ である.
\end{marker}


$\dim V = n < \infty$ とする.$\{\, e_i \, \}$ を $V$ の基底とすると,$\extp^k (V)$ の基底は $e_{i_1} \wedge \cdots \wedge e_{i_k}$ の形をした単項式のうち,互いに線形独立なものである.定義\ref{def.ext}-(3) より,添字の組 $(i_1,\, \dots ,\, i_k)$ の中に互いに等しいものがあると $0$ になり,また,添字の順番を並べ替えただけの項は線形独立にならない.以上の考察から,次のようになる:

\begin{mydef}[label=basisforext]{外積代数の基底}
	$\extp^k (V)$ の基底として
	\begin{align}
		\bigl\{\, e_{i_1} \wedge \cdots \wedge e_{i_k} \bigm| 1 \le i_1 < \cdots < i_k \le n \, \bigr\}
	\end{align}
	をとることができる.このとき $\displaystyle \dim \extp^k (V) = \binom{n}{k}$ である.	
\end{mydef}
二項定理から,$\dim \extp^\bullet (V) = 2^n$ である.また,$k > n$ のとき $\extp^k (V) = \{ 0 \}$ である.

\section{交代形式}

\begin{mydef}[label=def.anticom]{交代形式}
	$V$ を $\mathbb{K}$ 上のベクトル空間とする.\underline{$(0,\, r)$ 型テンソル} $ \omega \in \mathcal{T}_r^0 (V)$ であって,任意の置換  $ \sigma \in \mathfrak{S}_r$ に対して
	\begin{align}
		\omega \bigl[ X_{ \sigma(1) },\, \dots ,\, X_{ \sigma(r) } \bigr] = \mathrm{sgn}\, \sigma\; \omega \bigl[ X_1,\, \dots ,\, X_r \bigr],\quad X_i \in V
	\end{align}
	となるものを $V$ 上の $r$ 次の\textbf{交代形式}と呼ぶ.
\end{mydef}

$V$ 上の $r$ 次の交代形式全体の集合を $A^r(V)$ と書く.\underline{$A^r(V)$ はテンソル空間 $\mathcal{T}^0_r(V)$ の部分ベクトル空間である}.次数の異なる交代形式全体
\begin{align}
	A^\bullet (V) \coloneqq \bigoplus_{r = 0}^\infty A^r(V)
\end{align}
を考える.ただし $A^0(V) = \mathbb{K}$ と定義する.交代性より $k > n$ のとき $A^k(V) = \{ 0 \}$ になる.

\begin{mytheo}[label=alghom]{外積代数と交代形式の同型}
	写像 $ \iota \colon \extp^\bullet (V^*) \to A^\bullet(V)$ を以下のように定義する:
	
	まず,写像 $ \iota_k \colon \extp^k (V^*) \to A^k(V)$ の $ \omega = \alpha_1 \wedge \cdots \wedge \alpha_k \in \extp^k (V^*) \; ( \alpha_i \in V^* )$ への作用を
	\begin{align}
		\iota_k( \omega )[X_1,\, \dots ,\, X_k] \coloneqq \det \bigl(\, \alpha_i [X_j] \, \bigr)
	\end{align}
	と定義する.$\forall \omega \in \extp^\bullet (V)^*$ に対する $ \iota $ の作用は $ \iota_k $ の作用を線形に拡張する.	

	このとき,$ \iota $ は同型写像である.
\end{mytheo}
\begin{proof}
	各 $ \iota_k $ が同型写像であることを示せば良い\footnote{添字がややこしいのでこの証明ではEinsteinの規約を用いない.}.$V$ の基底 $\{ e_i \}$ と $V^*$ の基底 $\{ e^i \}$ は $e^i[e_j] = \delta^i_j$ を充しているものとする.このとき $\extp^k (V^*)$ の基底を
	\begin{align} 
		\bigl\{\, e^{i_1} \wedge \cdots \wedge e^{i_k} \bigm| 1 \le i_1 < \cdots < i_k \le n \,\bigr\}
	\end{align}
	にとれる.$\bigl\{\, \iota_k \bigl( e^{i_1} \wedge \cdots \wedge e^{i_k} \bigr) \bigm| 1 \le i_1 < \cdots < i_k \le n \,\bigr\} \subset A^k(V)$ が $A^k(V)$ の基底を成すことを示す.
	
	ある $\lambda_{i_1 \dots i_k} \in \mathbb{K}$ に対して
	\begin{align} 
		\sum_{1 \le i_1 < \cdots < i_k \le n}\lambda_{i_1 \dots i_k} \iota_k \bigl( e^{i_1} \wedge \cdots \wedge e^{i_k} \bigr) = 0 \in A^k(V)
	\end{align}
	ならば,
	\begin{align} 
		0 &= \sum_{1 \le i_1 < \cdots < i_k \le n} \lambda_{i_1 \dots i_k} \iota_k \bigl( e^{i_1} \wedge \cdots \wedge e^{i_k} \bigr) [e_{j_1}, \dots ,\, e_{j_k}] \\
		&= \sum_{1 \le i_1 < \cdots < i_k \le n} \lambda_{i_1 \dots i_k} \det \bigl( e^{i_l}[e_{j_m}] \bigr) \\
		&= \sum_{1 \le i_1 < \cdots < i_k \le n} \lambda_{i_1 \dots i_k} \det \bigl( \delta^{i_l}_{j_m} \bigr) \\
		&= \lambda_{j_1 \dots j_k} 
	\end{align}
	なので線形独立である.
	
	次に $\forall \omega \in A^k(V)$ を一つとる.このとき $\omega_{i_1 \dots i_k} \coloneqq \omega[e_{i_1},\, \dots ,\, e_{i_k}]$ とおいて
	\begin{align} 
		\tilde{\omega} \coloneqq \sum_{1 \le i_1 < \cdots < i_k \le n} \omega_{i_1 \dots i_k} e^{i_1} \wedge \cdots \wedge e^{i_k} \in \extp^k(V^*)
	\end{align}
	と定義すると
	\begin{align} 
		\iota_k(\tilde{\omega}) = \sum_{1 \le i_1 < \cdots < i_k \le n} \omega_{i_1 \dots i_k} \iota_k \bigl( e^{i_1} \wedge \cdots \wedge e^{i_k} \bigr) = \omega
	\end{align}
	なので $\iota_k$ は全射である.従って $\iota_k \colon \extp^\bullet(V^*) \xrightarrow{\cong} A^k(V)$ である.
\end{proof}

\begin{marker} 
	定理\ref{alghom}において構成した $\iota_k$ は,定数倍しても同型写像を与える.文献によっては $1/k!$ 倍されていたりするので注意.$1/k!$ 倍する定義は,特性類の一般論の記述に便利である.
\end{marker}

\begin{mycol}[label=lem.op]{交代形式の外積}
	$\tilde{\omega} \in \extp^k(V^*),\; \tilde{\eta} \in \extp^l(V^*)$ を与える.同型写像 $\iota \colon \extp^\bullet(V^*) \xrightarrow{\cong} A^\bullet(V)$ による対応を
	\begin{align} 
		\tilde{\omega} &\mapsto \omega \coloneqq \iota(\tilde{\omega}), \\
		\tilde{\eta} &\mapsto \omega \coloneqq \iota(\tilde{\eta}), \\
		\tilde{\omega}\wedge \tilde{\eta} &\mapsto \omega \wedge \eta \coloneqq \iota(\tilde{\omega} \wedge \tilde{\eta})
	\end{align}
	とおくと, \underline{$A^\bullet(V)$ 上の}\textbf{外積} (exterior product) $\wedge \colon A^k(V) \times A^l(V) \to A^{k+l}(V)$ が次のようにして定まる:
	\begin{align} 
		&(\omega \wedge \eta)[X_1,\, \dots ,\, X_{k+l}] \\
		&= \frac{1}{k !\, l !} \sum_{\sigma \in \mathfrak{S}_{k+l}} \sgn{\sigma} \omega[X_{\sigma(1)},\, \dots ,\, X_{\sigma(k)}]\, \eta[X_{\sigma(k+1)},\dots ,\, X_{\sigma(k+l)}]
	\end{align}
	ただし,$X_i \in V$ は任意とする.
\end{mycol}

\begin{proof} 
	$\iota_k$ の線形性から,
	\begin{align} 
		\tilde{\omega} = e^{i_1} \wedge \cdots \wedge e^{i_k},\quad \tilde{\eta} = e^{j_1} \wedge \cdots \wedge e^{j_{l}}
	\end{align}
	について示せば十分.また,$\extp^\bullet(V^*)$ 上の二項演算 $\wedge$ の交代性から添字 $i_1,\, \dots ,\, i_k,\; j_1, \, \dots ,\, j_l$ は全て異なるとしてよい.
	
	ここで,左辺を計算するために次のような置換 $\tau \in \mathfrak{S}_{k+l}$ を考える:
	\begin{align} 
		\tau \coloneqq \mqty(i_1 & \dots & i_k & j_1 & \dots & j_l \\ m_1 & \dots & m_k & m_{k+1} & \dots & m_{k+l}),\quad m_1 < m_2 < \cdots < m_{k+l}.
	\end{align}
	このとき
	\begin{align} 
		\tilde{\omega} \wedge \tilde{\eta} = \sgn{\tau} e^{m_1} \wedge \cdots \wedge e^{m_{k+l}}
	\end{align}
	である.したがって
	\begin{align} 
		(\omega \wedge \eta)[e_{m_1},\, \dots ,\, e_{m_{k+l}}] = \sgn{\tau} \iota_{k+l}( e^{m_1} \wedge \cdots \wedge e^{m_{k+l}} )[e_{m_1},\, \dots ,\, e_{m_{k+l}}] = \tcbhighmath[]{\sgn{\tau}}.
	\end{align}
	
	次に,右辺を計算する.
	\begin{align} 
		&\sum_{\sigma \in \mathfrak{S}_{k+l}} \sgn{\sigma} \omega[e_{\sigma(m_1)},\, \dots ,\, e_{\sigma(m_{k})}] \, \eta[e_{\sigma(m_{k+1})},\, \dots ,\, e_{\sigma(m_{k+l})}] \\
		&= \sum_{\sigma \in \mathfrak{S}_{k+l}} \sgn{\sigma} \iota_k(e^{\textcolor{red}{i_1}} \wedge \cdots \wedge e^{\textcolor{red}{i_k}})[e_{\sigma\tau(\textcolor{red}{i_1})},\, \dots ,\, e_{\sigma\tau(\textcolor{red}{i_{k}})}]\, \iota_l (e^{\textcolor{blue}{j_1}} \wedge \cdots \wedge e^{\textcolor{blue}{j_{l}}})\eta[e_{\sigma\tau(\textcolor{blue}{j_{1}})},\, \dots ,\, e_{\sigma\tau(\textcolor{blue}{j_{l}})}] \label{eq.33-1}\\
	\end{align}
	式\eqref{eq.33-1}の和において,$\exists \rho \in \textcolor{red}{\mathfrak{S}_k},\; \exists \pi \in \textcolor{blue}{\mathfrak{S}_l},\; \sigma\tau(\textcolor{red}{i_1 \cdots i_k}) = \rho(\textcolor{red}{i_1 \cdots i_k}),\, \sigma\tau(\textcolor{blue}{j_1 \cdots j_l}) = \pi(\textcolor{blue}{j_1 \cdots j_l})$ を充たすような $\sigma \in \mathfrak{S}_{k+l}$ の項のみが非ゼロである.そのような $\sigma$ に対して $\sigma\tau = \rho\pi,\; \rho\pi(i_1 \cdots i_k) = \rho(i_1 \cdots i_k),\; \rho\pi(j_1 \cdots j_l) = \pi(j_1 \cdots j_l)$ と書けるから
	\begin{align} 
		&\sum_{\sigma \in \mathfrak{S}_{k+l}} \sgn{\sigma} \iota_k(e^{i_1} \wedge \cdots \wedge e^{i_k})[e_{\sigma\tau(i_1)},\, \dots ,\, e_{\sigma\tau(i_{k})}]\, \iota_l (e^{j_1} \wedge \cdots \wedge e^{j_{l}})\eta[e_{\sigma\tau(j_{1})},\, \dots ,\, e_{\sigma\tau(j_{l})}] \\
		&= \sum_{\rho \in \mathfrak{S}_{k}} \sum_{\pi \in \mathfrak{S}_{l}} \sgn{\sigma} \iota_k(e^{i_1} \wedge \cdots \wedge e^{i_k})[e_{\rho(i_1)},\, \dots ,\, e_{\rho(i_{k})}]\, \iota_l (e^{j_1} \wedge \cdots \wedge e^{j_{l}})\eta[e_{\pi(j_{1})},\, \dots ,\, e_{\pi(j_{l})}] \\
		&= \sum_{\rho \in \mathfrak{S}_{k}} \sum_{\pi \in \mathfrak{S}_{l}} \sgn{\sigma} \sgn{\rho} \sgn{\pi} \\
		&= \sum_{\rho \in \mathfrak{S}_{k}} \sum_{\pi \in \mathfrak{S}_{l}} \sgn{\sigma} \sgn{\rho\pi} \\
		&= \tcbhighmath[]{k!\, l! \sgn{\tau}}.
	\end{align}
	となる.よって示された.
\end{proof}

\section{\cinfty 多様体上の微分形式}

前節の結果を用いて,局所座標に依存しない微分形式の定義を与えることができる.

\begin{mydef}[label=def.form]{微分形式}
	$M$ を\cinfty 多様体とする.$\omega$ が $M$ 上の\textbf{$k$-形式} ($k$-form) であるとは,各点 $p \in M$ において $\omega_p \in \extp^k\bigl(T^*_pM\bigr)$ を対応させ,$\omega_p$ が $p$ に関して\cinfty 級である,i.e.
	\begin{align} 
		\omega_p = \omega_{i_1 \cdots i_k}(p) \, (\dd{x^{i_1}})_p \wedge \cdots \wedge (\dd{x^{i_k}})_p
	\end{align}
	の各係数 $\omega_{i_1 \cdots i_k}(p)$ が\cinfty 関数であることを言う.
\end{mydef}

ベクトル束の言葉を使うと,
\begin{align} 
	\Omega^k(M) = \bigcup_{p \in M} \extp^k\bigl(T^*_pM\bigr)\; \text{の}\; C^\infty \; \text{級の切断の全体}
\end{align}
となる.

もう一つの解釈は,交代形式の定義\ref{def.anticom}を前面に押し出す方法である.この解釈では\hyperref[ax.alg]{多元環} $\cinftyf{M}$ 上の $(0,\, r)$-階テンソル場(定義\ref{tensorfield})としての側面が明らかになる:

\begin{mytheo}[label=k-form_homo]{$k$ 形式の同型}
	$M$ を\cinfty 多様体とする.$M$ 上の $k$-形式全体の集合 $\Omega^k(M)$ は,
	\begin{align} 
		\bigl\{\, &\tilde{\omega} \colon \mathfrak{X}(M) \times \cdots \times \mathfrak{X}(M) \to \cinftyf{M} \\
		\bigm| \; &\tilde{\omega}\; \text{は}\;\cinftyf{M}\;\text{-加群として多重線型かつ交代的}\; \bigr\} 
	\end{align}
	と自然に同型である.
\end{mytheo}

\begin{proof} 
	$\cinftyf{M}$-加群として多重線型かつ交代的であるような写像 $\tilde{\omega}\colon \vecfield{M} \times \cdots \times \vecfield{M} \to \cinftyf{M}$ が与えられたとする.まず $\forall X_i \in \vecfield{M}$ に対して,$\tilde{\omega}\bigl( X_1,\, \dots ,\, X_k \bigr) \in \cinftyf{M}$ の点 $p \in M$ における値が,各 $X_i$ の点 $p$ における値 $\eval{X_i}_p \in T_p M$ のみによって定まることを確認する.
	$\tilde{\omega}$ の線形性から $\tilde{\omega}\bigl( X_1,\, \dots ,\, X_i - Y_i,\, \dots,\, X_k \bigr) = \tilde{\omega}\bigl( X_1,\, \dots ,\, X_i ,\, \dots,\, X_k \bigr) - \tilde{\omega}\bigl( X_1,\, \dots ,\, Y_i ,\, \dots,\, X_k \bigr)$ なので,ある $i$ について $\eval{X_i}_p = 0$($0$ 写像)ならば $\tilde{\omega}\bigl( X_1,\, \dots ,\, X_k \bigr)(p) = 0$(実数)であることを確認すれば良い.
	$i = 1$ としても一般性を失わない.$(U;\; x^\mu)$ を $p$ の周りのチャートとする.このとき $U$ 上では $X^\mu \in \cinftyf{U}$ を用いて
	\begin{align} 
		\label{eq.thm4-3-1}
		X_1 = X^\mu \pdv{}{x^\mu},\quad X^\mu(p) = 0
	\end{align}
	と書ける.ここで $X_1$ の座標表示\eqref{eq.thm4-3-1}の定義域を補題\ref{gen_cinfty}を用いて $M$ 全域に拡張することを考える.そのために $\overline{V} \subset U$ なる $p$ の開近傍 $V$ と,$V$ 常恒等的に $1$ であり $U$ の外側では $0$ であるような\cinfty 関数 $h \in \cinftyf{M}$ をとることができる.
	このとき 
	\begin{align} 
		Y_i \coloneqq h \pdv{}{x^i}
	\end{align} 
	とおくと $Y_i \in \vecfield{M}$ となり,$\tilde{X}^\mu \coloneqq h X^\mu$ とおけば $\tilde{X}^\mu \in \cinftyf{M}$ となる.このとき
	\begin{align} 
		X_1 = X_1 + h^2(X_1 - X_1) = \tilde{X}^\mu Y_\mu + (1 - h^2) X_1 \in \vecfield{M}
	\end{align}
	の右辺は $V \subset U$ 上至る所で座標表示\eqref{eq.thm4-3-1}を再現することがわかる.従って $\tilde{\omega}$ の$\cinftyf{M}$-線形性から
	\begin{align} 
		&\tilde{\omega}\bigl( X_1,\, \dots ,\, X_k \bigr)(p) \\
		&= \tilde{X}^\mu (p)\, \tilde{\omega}\bigl(Y_\mu,\, X_2,\, \dots ,\, X_k \bigr)(p) + \bigl( 1 - h(p)^2 \bigr)\, \tilde{\omega}\bigl(X_1,\, X_2,\, \dots ,\, X_k \bigr)(p) \\
		&= 0
	\end{align}
	となり,示された.

	故に,次のような $k$-形式 $\omega$ の定義はwell-definedである\footnote{定理\ref{alghom}を使って各点 $p$ において $\omega_p \in \extp^k\bigl(T^*_pM\bigr)$ を $A^k(T_pM)$ の元と見做していることに注意}:任意の $k$ 個の接ベクトル $X_i \in T_pM$ が与えられたとき,$k$ 個のベクトル場 $\tilde{X}_i \in \vecfield{M}$ であって $\eval{\tilde{X}_i}_p = X_i$ を充たすものたちを適当に選ぶ.そして
	\begin{align} 
		\omega_p \bigl[ X_1,\, \dots ,\, X_k \bigr] \coloneqq \tilde{\omega} \bigl( \tilde{X}_1 ,\, \dots ,\, \tilde{X}_k \bigr) (p)
	\end{align}
	と定めると,上述の議論から左辺は $\tilde{X}_i$ の選び方に依らないのである.ベクトル場の\cinfty 性から $\omega_p$ が $p$ に関して\cinfty 級であることは明らかなので,このようにして定義された対応 $\omega \colon p \mapsto \omega_p$ は微分形式である.
\end{proof}


\section{微分形式の演算}

\cinfty 多様体 $M$ 上の $k$-形式全体の集合を $\Omega^k(M)$ と書き,
\begin{align} 
	A^\bullet(M) \coloneqq \bigoplus_{k=0}^n \Omega^k(M)
\end{align}
として $M$ 上の微分形式全体を考える.$A^\bullet(M)$ 上に様々な演算を定義する.
\begin{marker}		 
	しばらくの間,微分形式全体 $\Omega^k(M)$ を定理\ref{k-form_homo}の意味で捉える.i.e. $\omega \in \Omega^k(M)$ は $k$ 個のベクトル場に作用する.作用を受けるベクトル場は $(\;)$ で囲むことにする:
	\begin{align} 
		\omega \colon (X_1,\, \dots ,\, X_k) \mapsto \omega(X_1,\, \dots ,\, X_k)
	\end{align}
\end{marker}		

\subsection{外積}

微分形式全体 $\Omega^k(M)$ を定理\ref{k-form_homo}の意味で捉える.このとき,$k$-形式と $l$-形式の外積 
\begin{align} 
	\wedge \colon \Omega^k(M) \to \Omega^l(M),\; (\omega,\, \eta) \mapsto \omega \wedge \eta
\end{align}
は,各点 $p \in M$ で
\begin{align} 
	(\omega \wedge \eta)_p \coloneqq \omega_p \wedge \eta_p \in A^{k+l}(T_p M)
\end{align}
と定義される双線型写像である.

\begin{myprop}[label=extp_1]{外積の性質}
	外積は以下の性質を持つ:
	\begin{enumerate} 
		\item $\eta \wedge \omega = (-1)^{kl} \omega \wedge \eta$
		\item 任意の\underline{ベクトル場} $X_1,\, \dots ,\, X_{k+l} \in \mathfrak{X}(M)$ に対して
		\begin{align} 
			&(\omega \wedge \eta)(X_1,\, \dots ,\, X_{k+l}) \\
			&= \frac{1}{k !\, l !} \sum_{\sigma \in \mathfrak{S}_{k+l}} \sgn{\sigma} \omega(X_{\sigma(1)},\, \dots ,\, X_{\sigma(k)})\, \eta(X_{\sigma(k+1)},\dots ,\, X_{\sigma(k+l)})
		\end{align}
	\end{enumerate}
\end{myprop}

\begin{proof} 
	定理\ref{k-form_homo}より,各点 $p \in M$ において $(\omega \wedge \eta)_p$ を外積代数 $\extp^{k+l}\bigl(T^*_pM\bigr)$ の元と見做してよい.
	\begin{enumerate} 
		\item 外積代数の基底 $e^{i_1} \wedge \cdots \wedge e^{i_k} \wedge e^{j_1} \wedge \cdots \wedge e^{j_l}$ において $e^{j_1}$ を一番左に持ってくると全体が $(-1)^k$ 倍される.これを $l$ 回繰り返すと全体が $(-1)^{kl}$ 倍される.
		\item $(\omega \wedge \eta)_p$ に対して定理\ref{alghom}を用いればよい.
	\end{enumerate}
\end{proof}


\subsection{外微分}

\begin{mydef}[label=extdiff_1]{外微分(局所表示)} 
	$M$ のチャート $(U;\, x^i)$ を与える.$k$-形式 $\omega \in \Omega^k(M)$ の座標表示が
	\begin{align} 
		\omega = \omega_{i_1 \cdots i_k} \dd{x^{i_1}} \wedge \cdots \wedge \dd{x^{i_k}} 
	\end{align}
	と与えられたとき,\textbf{外微分} (exterior differention) 
	\begin{align} 
		\dd{} \colon \Omega^k(M) \to \Omega^{k+1}(M)
	\end{align}
	は次のように定義される:
	\begin{align} 
		\dd{\omega} \coloneqq \pdv{\omega_{i_1 \cdots i_k}}{x^{\textcolor{red}{j}}} \dd{x^{\textcolor{red}{j}}} \wedge \dd{x^{i_1}} \wedge \cdots \wedge \dd{x^{i_k}} 
	\end{align}
\end{mydef}

\begin{mytheo}[label=extdiff_2]{外微分(内在的)}
	$\omega \in \Omega^k(M)$ を $M$ 上の任意の $k$-形式とする.このとき,任意のベクトル場 $X_1,\, \dots ,\, X_{k+1} \in \mathfrak{X} (M)$ に対して
	\begin{align} 
		\dd{\omega} (X_1,\, \dots ,\, X_{k+1}) = &\sum_{i=1}^{k+1} (-1)^{i+1} X_i \bigl( \omega(X_1,\, \dots ,\, \hat{X_i},\, \dots,\, X_{k+1}) \bigr) \\
		&+ \sum_{i < j} (-1)^{i+j} \omega \bigl( [X_i,\, X_j],\, X_1,\, \dots ,\, \hat{X_i},\, \dots ,\, \hat{X_j},\, \dots ,\, X_{k+1} \bigr) 
	\end{align}
	ただし $\hat{X_i}$ は $X_i$ を省くことを意味する.また,$\comm{X}{Y}$ は\textbf{Lie括弧積}と呼ばれる $\vecfield{M}$ 上の二項演算で,以下のように定義される:
	\begin{align} 
		\comm{X}{Y}f \coloneqq X(Yf) - Y(Xf)
	\end{align}
\end{mytheo}

定理\ref{extdiff_2}の $k=1$ の場合を書くと次の通り:
\begin{align} 
	\dd{\omega}(X,\, Y) = (X\omega)(Y) - (Y\omega)(X) - \omega(\comm{X}{Y}).
\end{align}

\begin{mytheo}[label=extdiff_3]{外微分の性質}
	外微分 $\dd{}$ は以下の性質をみたす:
	\begin{enumerate} 
		\item $\dd{}\circ \dd{} = 0$
		\item $\dd{(\omega \wedge \eta)} = \dd{\omega} \wedge \eta + (-1)^k \omega \wedge \dd{\eta},\quad \omega \in \Omega^k(M)$
	\end{enumerate}
\end{mytheo}

\begin{proof} 
	$\forall \omega \in \Omega^k(M)$ と $M$ のチャート $(U;\, x^i)$ をとる.
	\begin{enumerate} 
		\item \begin{align} 
			\dd[2]{\omega} = \pdv[2]{\omega_{i_1 \dots i_k}}{x^\mu}{x^\nu} \dd{x^\mu} \wedge \dd{x^\nu} \wedge \dd{x^{i_1}} \wedge \cdots \wedge \dd{x^{i_k}}
		\end{align}
		であるが,$\omega$ は \cinfty 級なので偏微分は可換である.従って添字の対 $\mu,\, \nu$ に関して対称かつ反対称な総和をとることになるから $\dd[2]{\omega} = 0$ である.
		\item $\omega = \omega_{i_1 \dots i_k} \dd{x^{i_1}} \wedge \cdots \wedge \dd{x^{i_k}},\; \eta = \eta_{j_1 \dots j_l}\, \dd{x^{j_1}} \wedge \cdots \wedge \dd{x^{j_l}}$ とする.
		\begin{align} 
			\dd{(\omega \wedge \eta)} &= \dd{\bigl( \omega_{i_1 \dots i_k}\eta_{j_1 \dots j_l} \dd{x^{i_1}} \wedge \cdots \wedge \dd{x^{i_k}} \wedge \dd{x^{j_1}} \wedge \cdots \wedge \dd{x^{j_l}} \bigr)}\\
			&=  \left( \pdv{\omega_{i_1 \dots i_k}}{x^\mu} \eta_{j_1 \dots j_l} + \omega_{i_1 \dots i_k} \pdv{\eta_{j_1 \dots j_l}}{x^\mu} \right)\, \dd{x^\mu} \wedge \dd{x^{i_1}} \wedge \cdots \wedge \dd{x^{i_k}} \wedge \dd{x^{j_1}} \wedge \cdots \wedge \dd{x^{j_l}} \\
			&= 	\left(\pdv{\omega_{i_1 \dots i_k}}{x^\mu} \dd{x^\mu} \wedge \dd{x^{i_1}} \wedge \cdots \wedge \dd{x^{i_k}} \right)  \wedge \bigl( \eta_{j_1 \dots j_l}\, \dd{x^{j_1}} \wedge \cdots \wedge \dd{x^{j_l}} \bigr) \\
			&\quad + (-1)^k \bigl( \omega_{i_1 \dots i_k} \, \dd{x^{i_1}} \wedge \cdots \wedge \dd{x^{i_k}} \bigr) \wedge \left( \pdv{\eta_{j_1 \dots j_l}}{x^\mu}\, \dd{x^\mu} \wedge \dd{x^{j_1}} \wedge \cdots \wedge \dd{x^{j_l}} \right) \\
			&= \dd{\omega} \wedge \eta + (-1)^k \omega \wedge \dd{\eta}
		\end{align}
	\end{enumerate}
\end{proof}

\subsection{引き戻し}

二つの\cinfty 多様体 $M,\, N$ と,その上の \cinfty 写像 $f \colon M \to N$ を与える.微分写像
\begin{align} 
	f_* \colon T_p M \to T_{f(p)}N
\end{align}
が $f_* \in \Hom{\mathbb{R}} ( T_p M,\,  T_{f(p)}N)$ であることから,それの\textbf{引き戻し} $f^*$ を,$\forall \alpha \in T_{f(p)}^* N,\, \forall X \in T_p M$ に対して次のように定義できる:
\begin{align} 
	&f^* \colon T_{f(p)}^* N \to T_{p}^*M, \\
	&f^*(\alpha)[X] \coloneqq \alpha\bigl[f_*(X)\bigr]
\end{align}

$\extp^1 \bigl( T^*_{f(p)} N\bigr) \cong T^*_{f(p)} N$ を思い出すと,$f^*$ の定義域,値域は自然に\underline{点 $p$ における} $k$-形式へ拡張される.
具体的には,$\forall \omega \in \extp^k\bigl(T_{f(p)}^* N\bigr),\, \forall X_i \in T_p M$ に対して
\begin{align} 
	&f^* \colon \extp^k\bigl(T_{f(p)}^* N\bigr) \to \extp^k(T_{p}^*M), \\
	&f^*(\omega)[X_1,\, \dots ,\, X_k] \coloneqq \omega\bigl[f_*(X_1),\, \dots ,\, f_*(X_k)\bigr]
\end{align}
と定義する.
さらに,各点 $p \in M$ について和集合をとることで$k$-形式全体に作用するようになる:
\begin{align} 
	&f^* \colon \Omega^k(N) \to \Omega^k(M), \\
	&\tcbhighmath[]{f^*(\omega)(X_1,\, \dots ,\, X_k) \coloneqq \omega\bigl(f_*(X_1),\, \dots ,\, f_*(X_k)\bigr)} \label{def.k-form_pullback}
\end{align}
ただし $\forall \omega \in \Omega^k(N),\, \forall X_i \in \vecfield{M}$ である.$f^*(\omega) \in \Omega^k(M)$ を $f$ による $\omega \in \Omega^k(N)$ の\textbf{引き戻し}と呼ぶ.

\begin{myprop}[label=prop.pullback]{引き戻しの性質} 
	$f^*$ は線型写像であり,以下の性質をみたす:
	\begin{enumerate} 
		\item $f^*(\omega \wedge \eta) = f^*(\omega) \wedge f^*(\eta)$
		\item $\dd{(f^*(\omega))} = f^*(\dd{\omega})$
	\end{enumerate}
	特に,性質(1)から $f^* \colon A^\bullet(M) \to A^\bullet(N)$ は環準同型写像である.
\end{myprop}

\begin{proof} 
	$k+l$ 個のベクトル場 $X_i \in \vecfield{M}$ を任意にとる.
	\begin{enumerate} 
		\item 命題\ref{extp_1}-(2) より
		\begin{align} 
			&\bigl( f^*(\omega) \wedge f^*(\eta) \bigr)(X_1,\, \dots ,\, X_{k+l}) \\
			&= \frac{1}{k!\, l!} \sum_{\sigma \in \mathfrak{S}_{k+l}} \sgn{\sigma} (f^* \omega)(X_{\sigma(1)},\, \dots ,\, X_{\sigma(k)})\, (f^* \eta)(X_{\sigma(k+1)},\, \dots ,\, X_{\sigma(k+l)}) \\
			&= \frac{1}{k!\, l!} \sum_{\sigma \in \mathfrak{S}_{k+l}} \sgn{\sigma} \omega\bigl(f_*(X_{\sigma(1)}),\, \dots ,\, f_*(X_{\sigma(k)}) \bigr)\, \eta\bigl( f_*(X_{\sigma(k+1)}),\, \dots ,\, f_*(X_{\sigma(k+l)})\bigr) \\
			&= (\omega \wedge \eta)\bigl( f_*(X_1),\, \dots ,\, f_*(X_{k+l}) \bigr) \\
			&= f^*(\omega \wedge \eta)(X_1,\, \dots ,\, X_{k+l})
		\end{align}
		\item 微分写像の定義\eqref{def.differential_map}から,$f_* \colon \vecfield{M} \to \vecfield{N}$ の\underline{ベクトル場} $X \in \vecfield{M}$ への作用は
		\begin{align} 
			(f_* X) h = X(h \circ f) \circ f^{-1} \in \cinftyf{N},\quad \forall h \in \cinftyf{N}
		\end{align}
		である.故にLie括弧積との順序は
		\begin{align} 
			\comm{f_*X}{f_*Y} h &= f_* X \bigl((f_* Y) h\bigr) - f_*Y \bigl( (f_* X) h\bigr) \\
			&= X \Bigl( \bigl((f_* Y) h\bigr) \circ f \Bigr) \circ f^{-1} - Y \Bigl( \bigl((f_* X) h\bigr) \circ f \Bigr) \circ f^{-1} \\
			&= X \Bigl( \bigl(Y(h \circ f) \circ f^{-1}\bigr) \circ f \Bigr) \circ f^{-1} - Y \Bigl( \bigl(X(h \circ f) \circ f^{-1}\bigr) \circ f \Bigr) \circ f^{-1} \\
			&= \Bigl( X \bigl( Y(h \circ f)\bigr) - Y \bigl( X(h \circ f) \bigr) \Bigr) \circ f^{-1} \\
			&= \comm{X}{Y}(h \circ f) \circ f^{-1} \\
			&= (f_*\comm{X}{Y})h,\quad \forall h \in \cinftyf{N}
		\end{align}
		となり,可換である.従って定理\ref{extdiff_1}より
		\begin{align} 
			&\dd{\bigl( f^*(\omega) \bigr)} (X_1,\, \dots ,\, X_{k+1}) \\
			&= \sum_{i=1}^{k+1} (-1)^{i+1} X_i \bigl( (f^*\omega)(X_1,\, \dots ,\, \hat{X_i},\, \dots,\, X_{k+1}) \bigr) \\
			&\quad + \sum_{i < j} (-1)^{i+j} (f^*\omega) \bigl( [X_i,\, X_j],\, X_1,\, \dots ,\, \hat{X_i},\, \dots ,\, \hat{X_j},\, \dots ,\, X_{k+1} \bigr) \\
			&= \sum_{i=1}^{k+1} (-1)^{i+1} X_i \Bigl( \omega\bigl(f_*(X_1),\, \dots ,\, \widehat{f_*(X_i)},\, \dots,\, f_*(X_{k+1})\bigr) \Bigr) \\
			&\quad + \sum_{i < j} (-1)^{i+j} \omega \bigl( f_*([X_i,\, X_j]),\, f_*(X_1),\, \dots ,\, \widehat{f_*(X_i)},\, \dots ,\, \widehat{f_*(X_j)},\, \dots ,\, f_*(X_{k+1}) \bigr) \\
			&= \sum_{i=1}^{k+1} (-1)^{i+1} X_i \Bigl( \omega\bigl(f_*(X_1),\, \dots ,\, \widehat{f_*(X_i)},\, \dots,\, f_*(X_{k+1})\bigr) \Bigr) \\
			&\quad + \sum_{i < j} (-1)^{i+j} \omega \bigl( [f_*(X_i),\, f_*(X_j)],\, f_*(X_1),\, \dots ,\, \widehat{f_*(X_i)},\, \dots ,\, \widehat{f_*(X_j)},\, \dots ,\, f_*(X_{k+1}) \bigr) \\
			&= f^*(\dd{\omega}) (X_1,\, \dots ,\, X_{k+1}).
		\end{align}
	\end{enumerate}
\end{proof}

\subsection{内部積とLie微分}

\begin{mydef}[label=int_prod]{内部積}
	$X \in \vecfield{M}$ を $M$ 上の任意のベクトル場とする.このとき $X$ による\textbf{内部積} (interior product) 
	\begin{align} 
		\iunit_X \colon \Omega^k(M) \to \Omega^{k-1}(M)
	\end{align}
	が次のように定義される:
	\begin{align} 
		\iunit_{\textcolor{red}{X}}(\omega)(X_1,\, \dots ,\, X_{k-1}) \coloneqq \omega(\textcolor{red}{X},\, X_1,\, \dots ,\, X_{k-1}),\quad \forall \omega \in \Omega^k(M),\, \forall X_i \in \vecfield{M}
	\end{align}
	ただし,$k=0$ のときは $\iunit_X = 0$ と定義する.
\end{mydef}
	
\begin{myprop}[label=prop.int_prod]{内部積の性質}
	$\forall \omega,\, \omega_1,\, \omega_2 \in \Omega^k(M),\; \forall f \in \cinftyf{M}$ をとる.
	\begin{enumerate} 
		\item $\Omega^k(M)$ を $\cinftyf{M}$ 加群と見たとき,$\iunit_X \in \Hom{\cinftyf{M}}\bigl(\Omega^k(M),\,\Omega^{k-1}(M)\bigr)$ である:
		\begin{align} 
			\iunit_X(\omega_1 + \omega_2) = \iunit_X(\omega_1) + \iunit_X(\omega_2),\quad \iunit_X(f \omega) = f\, \iunit_X(\omega).
		\end{align}
		\item $\iunit_X$ は\textbf{反微分}である:
		\begin{align} 
			\iunit_X(\omega \wedge \eta) = \iunit_X(\omega) \wedge \eta + (-1)^k \omega \wedge \iunit_X(\eta)
		\end{align}
	\end{enumerate}
\end{myprop}

\begin{proof} 
	\begin{enumerate} 
		\item 定義より明らか.
		\item 命題\ref{extp_1}より
		\begin{align} 
			&\iunit_X(\omega \wedge \eta) (X_1,\, \dots,\, X_{k+l-1}) \\
			&= \omega \wedge \eta (X,\, X_1,\, \dots,\, X_{k+l-1}) \\
			&= \frac{1}{k!\, l!} \sum_{i=1}^k \sum_{\sigma \in \mathfrak{S}_{k+l-1}} (-1)^{i+1}\sgn{\sigma} \omega(X_{\sigma(1)},\, \dots ,\, \underbrace{X}_{i} ,\,\dots,\, X_{\sigma(k-1)})\, \eta(X_{\sigma(k)},\, \dots ,\, X_{\sigma(k+l-1)}) \\
			&\quad + (-1)^k \frac{1}{k!\, l!} \sum_{j=1}^l \sum_{\sigma \in \mathfrak{S}_{k+l-1}} (-1)^{j+1}\sgn{\sigma} \omega(X_{\sigma(1)},\, \dots ,\, X_{\sigma(k)})\, \eta(X_{\sigma(k+1)},\, \dots ,\, \underbrace{X}_{j} ,\,\dots,\, X_{\sigma(k+l-1)}) \\
			&= \bigl(\iunit_X(\omega) \wedge \eta + (-1)^k \omega \wedge \iunit_X(\eta)\bigr) (X_1,\, \dots,\, X_{k+l-1}).
		\end{align}
	\end{enumerate}
\end{proof}

\begin{mydef}[label=Lie_diff]{Lie微分}
	$X \in \vecfield{M}$ を $M$ 上の任意のベクトル場とする.このとき $X$ による\textbf{Lie微分} (Lie derivative) が
	\begin{align} 
		\mathcal{L}_X \colon \Omega^k(M) \to \Omega^{k}(M)
	\end{align}
	が次のように定義される:
	\begin{align} 
		\mathcal{L}_{\textcolor{red}{X}}(\omega)(X_1,\, \dots ,\, X_k) \coloneqq \textcolor{red}{X}\bigl(\omega(X_1,\, \dots ,\, X_k)\bigr) - \sum_{i=1}^k \omega(X_1,\, \dots ,\, \comm{\textcolor{red}{X}}{X_i},\, \dots ,\, X_k)
	\end{align}
\end{mydef}

Lie微分は定理\ref{k-form_homo}の条件を充している,i.e. $\mathcal{L}_X \omega$ は $\cinftyf{M}$ 加群として多重線型かつ交代的であるから,微分形式と呼ばれうる.

\begin{mytheo}[label=thm.Cartan]{Cartanの公式}
	$X,\, Y \in \vecfield{M},\; \omega \in \Omega^k(M)$ とする.このとき,以下が成立する:
	\begin{enumerate} 
		\item $\iunit_{\comm{X}{Y}}(\omega) = \comm{\mathcal{L}_X}{\iunit_Y} \omega$
		\item $\mathcal{L}_X = \iunit_X \dd{} + \dd{} \iunit_X$
	\end{enumerate}
\end{mytheo}

\begin{proof} 
	任意の $k-1$ 個のベクトル場 $X_i \in \vecfield{M}$ をとる.
	\begin{enumerate} 
		\item $k=0$ のときは $\Omega^0(M) = \cinftyf{M}$ なので,明らかである.
		
		 $k > 0$ とする.Lie微分の定義\ref{Lie_diff}より
		\begin{align} 
			&\bigl( \Liedv{X} \iunit_Y (\omega) \bigr)(X_1,\, \dots X_{k-1}) \\
			&= X \bigl( (\iunit_Y(\omega))(X_1,\, \dots ,\, X_{k-1}) \bigr)  - \sum_{i=1}^{k-1} \bigl( \iunit_Y(\omega) \bigr) (X_1,\, \dots ,\, \comm{X}{X_i},\, \dots ,\, X_{k-1}) \\
			&= X \bigl( \omega(Y,\, X_1,\, \dots ,\, X_{k-1}) \bigr)  - \sum_{i=1}^{k-1} \omega(Y,\, X_1,\, \dots ,\, \comm{X}{X_i},\, \dots ,\, X_{k-1}) 
		\end{align}
		である.一方
		\begin{align} 
			&\bigl( \iunit_Y(\Liedv{X} \omega) \bigr)(X_1,\, \dots X_{k-1}) \\
			&= (\Liedv{X} \omega)(Y,\, X_1,\, \dots X_{k-1}) \\
			&= X \bigl( \omega(Y,\, X_1,\, \dots ,\, X_{k-1}) \bigr) - \omega(\comm{X}{Y},\, X_1,\, \dots ,\, X_{k-1}) \\
			&\quad - \sum_{i=1}^{k-1} \omega(Y,\, X_1,\, \dots ,\, \comm{X}{X_i},\, \dots ,\, X_{k-1}) 
		\end{align}
		だから,
		\begin{align} 
			\bigl(\comm{\Liedv{X}}{\iunit_Y}\omega\bigr) (X_1,\, \dots X_{k-1}) &= \bigl( \Liedv{X} \iunit_Y (\omega) \bigr)(X_1,\, \dots X_{k-1}) - \bigl( \iunit_Y(\Liedv{X} \omega) \bigr)(X_1,\, \dots X_{k-1}) \\
			&= \iunit_{\comm{X}{Y}}(\omega) (X_1,\, \dots X_{k-1}).
		\end{align}
		
		\item $k=0$ のときは $\Liedv{\omega} = X \omega\quad (\omega \in \cinftyf{M})$ より明らか.$k > 0$ とする.定理\ref{extdiff_1}より
		\begin{align} 
			&\iunit_X(\dd{\omega})(X_1,\, \dots ,\, X_k) \\
			&= \dd{\omega}(X,\, X_1,\, \dots ,\, X_k) \\
			&= X\omega (X_1,\, \dots ,\, X_k) + \sum_{i=1}^{k} (-1)^{i} X_i \bigl( \omega(X,\, X_1,\, \dots ,\, \hat{X_i},\, \dots ,\, X_k) \bigr)  \\
			&\quad + \sum_{j=1}^k (-1)^{j} \omega(\comm{X}{X_j},\, X_1,\, \dots ,\, \hat{X}_j ,\, \dots,\, X_{k}) \\
			&\quad + \sum_{i<j} (-1)^{i+j} \omega(\comm{X_i}{X_j},\, X,\, X_1,\, \dots ,\, \hat{X}_i,\, \dots ,\, \hat{X}_j ,\, \dots ,\, X_k).
		\end{align}
		一方,
		\begin{align} 
			&\dd{ \bigl( \iunit_X (\omega) \bigr) }(X_1,\, \dots ,\, X_k) \\
			&= \sum_{i=1}^k (-1)^{i+1} X_i \bigl( \omega(X,\, X_1,\, \dots ,\, \hat{X}_i ,\,\dots ,\, X_k) \bigr) \\
			&\quad + \sum_{i<j} (-1)^{i+j} \omega(X,\, \comm{X_i}{X_j},\, X_1,\, \dots ,\, \hat{X}_i ,\, \dots ,\, \hat{X}_j,\, \dots ,\, X_k)
		\end{align}
		であるから,
		\begin{align} 
			&\bigl( \iunit_X \dd{} + \dd{} \iunit_X \bigr) \omega (X_1,\, \dots ,\, X_k) \\
			&= X\omega (X_1,\, \dots ,\, X_k) + \sum_{j=1}^k (-1)^{j} \omega(\comm{X}{X_j},\, X_1,\, \dots ,\, \hat{X}_j ,\, \dots,\, X_{k}) \\
			&= (\Liedv{X}\omega)(X_1,\, \dots,\, X_k).
		\end{align}
	\end{enumerate}
\end{proof}

Cartanの公式\ref{thm.Cartan}より,Lie微分の様々な性質が示される:

\begin{mytheo}[label=thm.Lie_1]{Lie微分の性質}
	$X,\, Y \in \vecfield{M},\; \omega \in \Omega^k(M)$ とする.このとき,以下が成立する:
	\begin{enumerate} 
		\item $\Liedv{X} (\omega \wedge \eta) = \Liedv{X}(\omega) \wedge \eta + \omega \wedge \Liedv{X}(\eta)$
		\item $\Liedv{X}(\dd{\omega}) = \dd{\bigl(\Liedv{X}(\omega)\bigr)}$
		\item $\comm{\Liedv{X}}{\Liedv{Y}} = \Liedv{\comm{X}{Y}}$
	\end{enumerate}
\end{mytheo}

\begin{proof} 
	\begin{enumerate} 
		\item Cartanの公式\ref{thm.Cartan}-(2) および内部積と外微分が共に反微分であることを利用すると
		\begin{align} 
			&\Liedv{X}(\omega \wedge \eta) \\
			&= \iunit_X \bigl( \dd{\omega} \wedge \eta + (-1)^k \omega \wedge \dd{\eta} \bigr) + \dd{\bigl( \iunit_X(\omega) \wedge \eta + (-1)^k \omega \wedge \iunit_X(\eta) \bigr) } \\
			&= \iunit_X(\dd{\omega}) \wedge \eta + \textcolor{red}{\cancel{(-1)^{k+1} \dd{\omega} \wedge \iunit_X(\eta)}} + \textcolor{blue}{\cancel{(-1)^k \iunit_X(\omega) \wedge \dd{\eta}}} + (-1)^{2k} \omega \wedge \iunit_X(\dd{\eta}) \\
			&\quad + \dd{\iunit_X(\omega) \wedge \eta} + \textcolor{blue}{\cancel{(-1)^{k-1} \iunit_X(\omega) \wedge \dd{\eta}}} + \textcolor{red}{\cancel{(-1)^k \dd{\omega} \wedge \iunit_X(\eta)}} + (-1)^{2k} \omega \wedge \dd{\iunit_X(\eta)} \\
			&= \Liedv{X}(\omega) \wedge \eta + \omega \wedge \Liedv{X}(\eta).
		\end{align}
		\item Cartanの公式\ref{thm.Cartan}-(2) から
		\begin{align} 
			\Liedv{X}(\dd{\omega}) = \cancel{\iunit_X (\dd[2]{\omega})} + \dd{} \iunit_X(\dd{\omega}) =  \dd{} \iunit_X(\dd{\omega}) + \dd[2]{\iunit_X(\omega)} = \dd{ \bigl( \Liedv{X}(\omega) \bigr) }.
		\end{align}
		\item $k$ に関する数学的帰納法により示す.$k=0$ のとき $\Omega^0(M) = \cinftyf{M}$ なので
		\begin{align} 
			\Liedv{\comm{X}{Y}} \omega = \comm{X}{Y}\omega = X(Y\omega) - Y(X\omega) = \comm{\Liedv{X}}{\Liedv{Y}}\omega
		\end{align}
		であり,成立している.

		 $k \ge 0$ について正しいと仮定する.$\forall \omega \in \Omega^{k+1}(M)$ を一つとる.$\forall Z \in \vecfield{M}$ に対して $\iunit_Z \omega \in \Omega^{k}(M)$ だから,帰納法の仮定より
		\begin{align} 
			\label{eq.prop4-7_ass}
			\Liedv{\comm{X}{Y}} \iunit_Z(\omega) = \comm{\Liedv{X}}{\Liedv{Y}}\iunit_Z(\omega)
		\end{align}
		が成立する.一方Cartanの公式\ref{thm.Cartan}-(1) より
		\begin{align} 
			\Liedv{\comm{X}{Y}} \iunit_Z = \iunit_Z \Liedv{\comm{X}{Y}} + \iunit_{\comm{\comm{X}{Y}}{Z}} \label{eq.prop4-7_3}
		\end{align}
		である.さらに
		\begin{align} 
			&\Liedv{X} \Liedv{Y} \iunit_Z \\
			&= \Liedv{X} \bigl( \iunit_Z \Liedv{Y} + \iunit_{\comm{Y}{Z}} \bigr) \\
			&= \iunit_Z \Liedv{X} \Liedv{Y} + \iunit_{\comm{X}{Z}} \Liedv{Y} + \iunit_{\comm{Y}{Z}} \Liedv{X} + \iunit_{\comm{X}{\comm{Y}{Z}}}, \label{eq.prop4-7_1}\\
			&\Liedv{Y} \Liedv{X} \iunit_Z \\
			&= \iunit_Z \Liedv{Y} \Liedv{X} + \iunit_{\comm{Y}{Z}} \Liedv{X} + \iunit_{\comm{X}{Z}} \Liedv{Y} + \iunit_{\comm{Y}{\comm{X}{Z}}} \label{eq.prop4-7_2}
		\end{align}
		であることもわかる.式\eqref{eq.prop4-7_1}$-$\eqref{eq.prop4-7_2}より
		\begin{align} 
			&\comm{\Liedv{X}}{\Liedv{Y}} \iunit_Z \\
			&= \iunit_Z \comm{\Liedv{X}}{\Liedv{Y}} + \iunit_{\comm{X}{\comm{Y}{Z}}} - \iunit_{\comm{Y}{\comm{X}{Z}}}
		\end{align}
		これを式\eqref{eq.prop4-7_3}から引いてJacobi恒等式 $\comm{\comm{X}{Y}}{Z} + \comm{\comm{Y}{Z}}{X} + \comm{\comm{Z}{X}}{Y} = 0$ を用いると
		\begin{align} 
			\comm{\bigl( \Liedv{\comm{X}{Y}} - \comm{\Liedv{X}}{\Liedv{Y}} \bigr)}{\iunit_Z } = 0.
		\end{align}
		\eqref{eq.prop4-7_ass}に代入すると
		\begin{align} 
			\iunit_Z \bigl( \Liedv{\comm{X}{Y}} - \comm{\Liedv{X}}{\Liedv{Y}}  \bigr) \omega = 0.
		\end{align}
		$Z$ は任意だったから $ \bigl( \Liedv{\comm{X}{Y}} - \comm{\Liedv{X}}{\Liedv{Y}}  \bigr) \omega = 0$ を得て証明が完了する.
	\end{enumerate}
\end{proof}

\section{$C^\infty$ 多様体の向き}

\subsection{$C^\infty$ 多様体の向き付けとその特徴付け}

% $n$ 次元多様体 $M$ をとる.多様体の各点 $p$ における\textbf{向き} (orientation) は,接空間 $T_pM$ の順序付き基底を考えることで定義される.

% $e_1,\, \dots ,\, e_n$ と $f_1,\, \dots ,\, f_n$ を $T_pM$ の二つの順序付き基底とする.これらは正則な線型変換 $T \colon T_p M \to T_pM$ で移り合うが,$\det T > 0$ であるときに $\{ e_i \} \sim \{f_i\}$ であるとして,$T_pM$ の順序付き基底全体の集合 $\mathcal{B}_p$ の上の同値関係 $\sim$ を定義する:
% \begin{align} 
% 	\sim \; \coloneqq \bigl\{ \bigl(\{e_i\},\, \{f_i\}\bigr) \in \mathcal{B}_p \times \mathcal{B}_p \bigm| \exists T = [T^i_j] \in \mathrm{GL}(n,\, \mathbb{R})\, \mathrm{s.t.}\, f_i = T^j_i e_j,\; \det T > 0 \bigr\} 
% \end{align}
% $\sim$ が同値関係であることは明らかである.故に同値類 $[\{ e_i \}] \in \mathcal{B}_p /\mathord{\sim}$ を考えることができ,これを点 $p$ における\textbf{向き}と呼ぶ.

% $M$ が\cinfty 多様体の場合,$T_pM$ における自然基底を代表元にもつ同値類 $[\{(\pdv*{}{x^i})_p\}]$ を定義でき,これを\textbf{正の向き}と呼ぶ.$M$ の座標変換に伴う基底の取り替えはJacobi行列で表現されるから,Jacobianの正負で向きを判定できる.

% \begin{mydef}[label=def:smooth-orientation]{向き付け可能} 
% 	\cinfty 多様体 $M$ が\textbf{向き付け可能} (orientable) であるとは,$M$ のアトラス $\mathcal{S} = \{(U_\lambda,\, \varphi_\lambda)\}_{\lambda \in \Lambda}$ であって,全ての座標変換 $f_{\beta \alpha} = \varphi_{\beta} \circ \varphi_{\alpha}^{-1}$ のJacobianが $\varphi_{\alpha}(U_\alpha \cap U_\beta)$ の全ての点で正になるようなものが存在することを言う.
% \end{mydef}


有限次元ベクトル空間 $V$ ($\dim V > 0$)の順序付き基底全体の集合を $\mathcal{B}_V$ と書く.
\begin{itemize}
	\item $\mathcal{B}_V$ 上の同値関係を
	\begin{align}
		(e_1,\, \dots,\, e_{\dim V}) \sim (f_1,\, \dots,\, f_{\dim V}) \DEF \exists T = [T^{\mu}{}_\nu] \in \LGL(\dim V),\; e_\mu = f_\nu T^{\nu}{}_\mu \AND \det T > 0
	\end{align}
	で定め,この同値関係による同値類のことをベクトル空間 $V$ の\textbf{向き} (orientation) と呼ぶ.
	\item ベクトル空間 $V$ とその向き $\mathcal{O}_V \in \mathcal{B}_V / \sim$ の組 $(V,\, \mathcal{O}_V)$ のことを\textbf{向き付けられたベクトル空間} (oriented vector space) と呼ぶ.
	\item 向き付けられたベクトル空間 $(V,\, \mathcal{O}_V)$ の順序付き基底 $(e_1,\, \dots,\, e_{\dim V}) \in \mathcal{B}_V$ は,$(e_1,\, \dots,\, e_{\dim V}) \in \mathcal{O}_V$ のとき\textbf{正の向き} (positively oriented),$(e_1,\, \dots,\, e_{\dim V}) \notin \mathcal{O}_V$ のとき\textbf{負の向き} (negatively oriented) であるという.
\end{itemize}

\begin{marker}
	$\dim V = 0$ のときは $\mathcal{O}_V \in \{\pm 1\}$ とする.
\end{marker}

$C^\infty$ 多様体 $M$ の各点 $p \in M$ における接空間 $T_p M$ は有限次元ベクトル空間なので,上述の意味で向き $\mathcal{O}_{T_p M}$ を与えることができる.
この各点各点で与えた接空間の向きたち $\Familyset[\big]{\mathcal{O}_{T_p M}}{p \in M}$ を次の意味で「連続に繋げる」ことができたとき,$C^\infty$ 多様体 $M$ に\textbf{向き}が与えられたと言う:

\begin{mydef}[label=def:smooth-orientation]{$C^\infty$ 多様体の向き}
    境界なし/あり $C^\infty$ 多様体 $M$ を与える.
    \begin{itemize}
        \item $M$ の\textbf{各点の向き} (pointwise orientation) とは,族 $\Familyset[\big]{\mathcal{O}_{T_p M}}{p \in M}$ のこと.
        \item $M$ の各点の向き $\Familyset[\big]{\mathcal{O}_{T_p M}}{p \in M}$ を与える.
        開集合 $U \subset M$ 上の局所フレーム $(E_1,\, \dots,\, E_{\dim M})$ が\textbf{正の向き} (positively oriented) であるとは,$\forall p \in U$ において $(E_1|_p,\, \dots,\, E_{\dim M}|_p) \in \mathcal{O}_{T_p M}$ であることを言う.\textbf{負の向き}も同様に定義する.
        \item 与えられた $M$ の各点の向き $\Familyset[\big]{\mathcal{O}_{T_p M}}{p \in M}$ が\textbf{連続} (continuous) であるとは,$\forall p \in M$ に対してある正の向きの局所フレーム $E^{(p)} \coloneqq (E^{(p)}_1,\, \dots,\, E^{^{(p)}}_{\dim M})$ が存在して $p \in \dom E^{(p)}$ を充たすことを言う.
        \item $M$ の\textbf{向き} (orientation) とは,$M$ の各点の向きであって連続であるもののことを言う.
        \item $M$ が\textbf{向き付け可能} (orientable) であるとは,$M$ の向きが存在することを言う.
        \item $M$ が向き付け可能なとき,向き $\Familyset[\big]{\mathcal{O}_{T_p M}}{p \in M}$ と $M$ の組 $(M,\, \Familyset[\big]{\mathcal{O}_{T_p M}}{p \in M})$ のことを\textbf{向きづけられた多様体} (oriented manifold) という.
    \end{itemize}
\end{mydef}

\begin{mylem}[label=lem:orientation]{}
    \begin{itemize}
        \item \hyperref[def:smooth-orientation]{向き付け可能}な境界あり/なし $C^\infty$ 多様体 $M$ の向き $\Familyset[\big]{\mathcal{O}_{T_p M}}{p \in M}$ 
        \item \textbf{連結}な開集合 $U \subset M$
    \end{itemize}
    を任意に与える.このとき,$U$ 上の任意の局所フレーム $(E_1,\, \dots,\, E_{\dim M})$ は正の向きであるか負の向きであるかのどちらかである.
\end{mylem}

\begin{proof}
    背理法により示す.$U$ 上の局所フレーム $E \coloneqq (E_1,\, \dots,\, E_{\dim M})$ であって,
    % ある異なる2点 $p,\, q \in U$ において $E_p \in \mathcal{O}_{T_p M} \AND E_q \notin \mathcal{O}_{T_p M}$ が成り立つようなものが存在するとする.
    \begin{align}
        W \coloneqq \bigl\{\, p \in U \bigm| E_p \in \mathcal{O}_{T_p M} \,\bigr\} 
    \end{align}
    とおいたときに $W \neq \emptyset \AND U \setminus W \neq \emptyset$ が成り立つものが存在するとする.
    
	$\forall p \in W \subset M$ を1つ固定する.仮定より各点の向き $\Familyset[\big]{\mathcal{O}_{T_p M}}{p \in M}$ は連続だから,ある正の向きの局所フレーム $F^{(p)}$ が存在して $p \in \dom F^{(p)} \eqqcolon V_p$ を充たす.
	このとき連続写像 $T \colon U \cap V_p \lto \LGL(\dim M)$ を
	\begin{align}
		E_\mu|_x \eqqcolon F^{(p)}{}_\nu|_x\, \bigl[\, T(x)\,\bigr]^\nu{}_\mu\quad (\forall x \in U \cap V_p)
	\end{align}
	により定めると\footnote{$T$ の連続性は,$E,\, F^{(p)}$ が局所フレームであることによる.},
	\begin{align}
		W \cap V_p = T^{-1}\bigl( \LGL_+ (\dim M) \bigr) 
	\end{align}
	と書けるので $W \cap V_p$ は開集合であり\footnote{$\LGL_+ (\dim M) \coloneqq \bigl\{\, X \in \LGL (\dim M) \bigm| \det X > 0 \,\bigr\}$ である.$\det \colon \LGL(\dim M) \lto \mathbb{R}$ が連続写像なので $\LGL_+ (\dim M) = \det^{-1} \bigl( (0,\, \infty) \bigr)$ は $\LGL (\dim M)$ の開集合である.} 
	,かつ $p \in W \cap V_p \subset W$ を充たす.i.e. $\forall p \in W$ に対して $W$ における $p$ の開近傍 $W \cap U_p$ が存在するので,命題\ref{prop.opdet}より $W$ が開集合だと分かった.
    同じ議論により $U \setminus W$ が開集合であることもわかるが,これは $U$ の連結性に矛盾する.
\end{proof}

定義\ref{def:smooth-orientation}の意味は直感的だが,扱い辛い.
しかし,実は微分形式によって $C^\infty$ 多様体の向きを特徴付けることができて計算上便利である:

\begin{myprop}[label=prop:orientation-form]{微分形式による向きの特徴付け}
    境界あり/なし $C^\infty$ 多様体 $M$ を与える.
    \begin{enumerate}
        \item $\forall p \in M$ において $0$ にならない $\omega \in \Omega^{\dim M}(M)$ が与えられたとする.
        このとき $\forall p \in M$ に対して
		\begin{align}
			\mathcal{O}_{\omega_p} \coloneqq \bigl\{\, (e_1,\, \dots,\, e_{\dim M}) \in \mathcal{B}_{T_p M} \bigm| \omega_p (e_1,\, \dots,\, e_{\dim M}) > 0 \,\bigr\} 
		\end{align}
		とおくと,族 $\Familyset[\big]{\mathcal{O}_{\omega_p}}{p \in M}$ は $M$ の\hyperref[def:smooth-orientation]{向き}である.
		%  $M$ の\hyperref[def:smooth-orientation]{向き}を一意に定める.
        \item $M$ に向き $\Familyset[\big]{\mathcal{O}_{T_p M}}{p \in M}$ が与えられたとする.
		このとき $\forall p \in M$ および $\forall (e_1,\, \dots,\, e_{\dim M}) \in \mathcal{O}_{T_ p M}$ に対して
        \begin{align}
			\omega_p \neq 0 \AND  \omega_p (e_1,\, \dots,\, e_{\dim M}) > 0
		\end{align}
		を充たす $\omega \in \Omega^{\dim M}(M)$ が存在する.
    \end{enumerate}
\end{myprop}

\begin{proof}
    \begin{enumerate}
        \item $\forall p \in M$ において $\omega_p \neq 0$ とする.このとき $\forall (e_1,\, \dots,\, e_{\dim M}),\, (f_1,\, \dots,\, f_{\dim M}) \in \mathcal{B}_{T_p M}$ に対して,$e_\mu = f_\nu T^{\nu}{}_\mu$ ならば,
        \begin{align}
            \omega_p (e_1,\, \dots,\, e_{\dim M}) 
            &= \omega_p (f_{\nu_1} T^{\nu_1}{}_1,\, \dots,\, f_{\nu_{\dim M}} T^{\nu_{\dim M}}{}_{\dim M}) \\
            &= T^{\nu_{1}}{}_{1} \cdots T^{\nu_{\dim M}}{}_{\dim M} \epsilon_{\nu_1 \dots \nu_{\dim M}} \omega_p (f_{1},\, \dots,\, f_{\dim M}) \\
            &= (\det T) \omega_p (f_{1},\, \dots,\, f_{\dim M})
        \end{align}
        で $\det T \neq 0$ なので,
        \begin{align}
            &(e_1,\, \dots,\, e_{\dim M}) \sim (f_1,\, \dots,\, f_{\dim M}) \\
			\DEF &\omega_p (e_1,\, \dots,\, e_{\dim M}),\, \omega_p (f_1,\, \dots,\, f_{\dim M})\; \text{が同符号}
        \end{align}
        によって定めた $\mathcal{B}_{T_p M}$ の同値関係 $\sim$ による同値類はベクトル空間 $T_p M$ の向きである.
		特に2つある同値類のうち $\omega_p$ の符号が正であるものを $\mathcal{O}_{\omega_p}$ とおいたので,族 $\Familyset[\big]{\mathcal{O}_{\omega_p}}{p \in M}$ は\hyperref[def:smooth-orientation]{各点の向き}である.

         次に,各点の向き $\Familyset[\big]{\mathcal{O}_{\omega_p}}{p \in M}$ が連続であることを示す.$\forall p \in M$ を一つ固定し,$p$ の連結な開近傍 $p \in U \subset M$ とその上の任意の局所フレーム $(E_1,\, \dots,\, E_{\dim M})$ をとる\footnote{例えば座標ベクトル場の定義域を十分小さく取れば良い.}.$(E_i)$ の双対フレームを $(\varepsilon_i)$ とする.
        すると $U$ 上で任意の $\omega \in \Omega^n(M)$ はある $f \in C^\infty (U)$ を使って $\omega = f \varepsilon^1 \wedge \cdots \wedge \varepsilon^{\dim M}$ の形で書ける.
        $\omega$ が $0$ にならないと言うことは $f$ が $0$ にならないので,$U$ 上至る所で $\omega(E_1,\, \dots,\, E_{\dim M}) = f \neq 0$ である.i.e. $f(U) \subset \mathbb{R} \setminus \{0\}$ であるが,$U$ の連結性から $f(U) \subset \mathbb{R}_{> 0}$ か $f(U) \subset \mathbb{R}_{< 0}$ のどちらかしかあり得ない.
        \item $\Familyset[\big]{\mathcal{O}_{T_p M}}{p \in M}$ を $M$ の\hyperref[def:smooth-orientation]{向き}とする.
        このとき $\forall p \in M$ に対してある\hyperref[def:smooth-orientation]{正の向き}の局所フレーム $E^{(p)}$ が存在して $p \in \dom E^{(p)}$ を充たす.そのような局所フレームの族 $\Familyset[\big]{E^{(p)}}{p \in M}$ を1つ固定する.
		すると $M$ の部分集合族 $\mathcal{E} \coloneqq \Familyset[\big]{\dom E^{(p)}}{p \in M}$ は $M$ の開被覆であるから,$\mathcal{E}$ に\hyperref[def:PoU]{従属する $C^\infty$ 級の1の分割} $\Familyset[\big]{\psi_p \in M \lto [0,\, 1]}{p \in M}$ をとることができる.

		 ここで $\forall p \in M$ について $E^{(p)}$ の双対フレーム $\varepsilon^{(p)}$ をとり,
		\begin{align}
			\omega^{(p)} \coloneqq \varepsilon^{(p)}{}^1 \wedge \cdots \wedge \varepsilon^{(p)}{}^{\dim M}
		\end{align}
		と定義する.
		このとき $\forall \textcolor{red}{x} \in \dom E^{(p)}$ および $(e_1,\, \dots,\, e_{\dim M}) \in \mathcal{O}_{T_{\textcolor{red}{x}} M}$ に対して,
		向きの定義から $E^{(p)}|_{\textcolor{red}{x}} \in \mathcal{O}_{T_{\textcolor{red}{x}} M}$ であることと (1) の議論から
		$\omega^{(p)}|_{\textcolor{red}{x}} (E^{(p)}{}_1|_{\textcolor{red}{x}},\, \dots,\, E^{(p)}{}_{\dim M}|_{\textcolor{red}{x}}) = 1 > 0$ と $\omega^{(p)}|_{\textcolor{red}{x}} (e_1,\, \dots,\, e_{\dim M})$ は同符号,i.e. 正である.
		したがって $\omega \in \Omega^{\dim M}(M)$ を
		\begin{align}
			\omega \coloneqq \sum_{p \in M} \psi_p\, \omega^{(p)}
		\end{align}
		で定義すると $\forall x \in M$ および $\forall (e_1,\, \dots,\, e_{\dim M}) \in \mathcal{O}_{T_x M}$ に対して
		\begin{align}
			\omega_x (e_1,\, \dots,\, e_{\dim M})
			&= \sum_{\substack{p \in M, \\ x \in \dom E^{(p)}}} \psi_p (x)\, \underbrace{\omega^{(p)}|_x (e_1,\, \dots,\, e_{\dim M})}_{> 0}
		\end{align}
		であり,\hyperref[def:PoU]{1の分割の性質-(4)}から少なくとも $1$ つの $p \in M$ について $\psi_p(x) > 0$ であるので,左辺が正であることが示された.構成から明らかに $\forall p \in M$ について $\omega_p \neq 0$ である.
    \end{enumerate}
\end{proof}

命題\ref{prop:orientation-form}を踏まえて次のように定義する:

\begin{mydef}[label=def:orientation-form]{向き付け形式}
	境界あり/なし $C^\infty$ 多様体 $M$ の至る所で $0$ にならない $\omega \in \Omega^{\dim M}(M)$ のことを\textbf{向き付け形式} (orientation form) と呼ぶ.
\end{mydef}

\hyperref[def:smooth-orientation]{向きの定義}では局所フレームが登場したが,実際には座標フレームを考えれば十分だとわかる:

\begin{mydef}[label=def:atlas-oriented]{向きづけられたアトラス}
	境界あり/なし多様体 $M$ を与え\footnote{この時点では\hyperref[def:smooth-orientation]{向き付け可能}でなくても良い},$M$ の\hyperref[maxatlas]{極大アトラス} $\mathcal{A}^+ \coloneqq \bigl\{\, (U_\alpha,\; \varphi_\alpha)  \,\bigr\}_{\alpha \in \Lambda^+}$ を1つ固定する.

	\begin{itemize}
		\item $M$ が\hyperref[def:smooth-orientation]{向きづけられた多様体}であるとする.
		このとき $M$ のチャート $\bigl(U,\, (x^\mu)\bigr) \in \mathcal{A}^+$ が\textbf{正の向き} (positively oriented) であるとは,座標フレーム $\bigl( \pdv{}{x^1},\, \dots,\, \pdv{}{x^{\dim M}} \bigr)$ が\hyperref[def:smooth-orientation]{正の向き}であることを言う.
		\item (極大とは限らない)\hyperref[diffmani]{$C^\infty$ アトラス} $\mathcal{A} \coloneqq \bigl\{\, (U_\alpha,\, \varphi_\alpha) \,\bigr\}_{\alpha \in \Lambda \subset \Lambda^+}  \subset \mathcal{A}^+$ が\textbf{向き付けれられた $C^\infty$ アトラス} (oriented smooth atlas) であるとは,
		$\forall \alpha,\, \beta \in \Lambda \subset \Lambda^+$ に対して座標変換
		\begin{align}
			\varphi_\beta^{-1} \circ \varphi_\alpha \colon \varphi_\alpha^{-1} (U_\alpha \cap U_\beta) \lto \varphi_\beta^{-1}(U_\alpha \cap U_\beta)
		\end{align}
		のJacobianが正であることを言う.
		% \item $\mathcal{A}^+$ に含まれる全ての向きづけられた $C^\infty$ アトラスの集合を $\irm{\mathscr{A}}{oriented} (\mathcal{A}^+)$ と書く.$\irm{\mathscr{A}}{oriented} (\mathcal{A}^+)$ 上の同値関係 $\sim$ を
		% \begin{align}
		% 	\mathcal{A} \sim \mathcal{B} \DEF \mathcal{A} \cup \mathcal{B} \in \irm{\mathscr{A}}{oriented}(\mathcal{A}^+)
		% \end{align}
		% により定義する.
		% $\mathcal{A} \in \irm{\mathscr{A}}{oriented}(\mathcal{A})$ の同値関係 $\sim$ による同値類を $[\mathcal{A}]$ と書く.
		% このとき
		% \begin{align}
		% 	\bigcup_{\mathcal{B} \in [\mathcal{A}]} \mathcal{B} \in \irm{\mathscr{A}}{oriented} (\mathcal{A}^+)
		% \end{align}
		% のことを\textbf{向きづけられた極大アトラス}と呼ぶ.
	\end{itemize}
\end{mydef}


\begin{myprop}[label=prop:orientation-coordinate]{向きづけられた $C^\infty$ アトラスによる向きの特徴付け}
	境界あり/なし $C^\infty$ 多様体 $M$ を与え,$M$ の\hyperref[maxatlas]{極大アトラス} $\mathcal{A}^+ \coloneqq \bigl\{\, (U_\alpha,\; \varphi_\alpha)  \,\bigr\}_{\alpha \in \Lambda^+}$ を1つ固定する.
	\begin{enumerate}
		\item $M$ に\hyperref[def:atlas-oriented]{向きづけられた $C^\infty$ アトラス} $\mathcal{A} \coloneqq \bigl\{\, (U_\alpha,\, \varphi_\alpha) \,\bigr\}_{\alpha \in \Lambda}$ が与えられたとする.
		このとき $M$ の\hyperref[def:smooth-orientation]{向き} $\Familyset[\big]{\mathcal{O}_{T_p M}}{p \in M}$ であって,$\forall \alpha \in \Lambda$ について $M$ のチャート $(U_\alpha,\, \varphi_\alpha) \in \mathcal{A}$ が\hyperref[def:atlas-oriented]{正の向き}になるようなものが一意的に存在する.
		\item $M$ に\hyperref[def:smooth-orientation]{向き} $\Familyset[\big]{\mathcal{O}_{T_p M}}{p \in M}$ が与えられたとする.
		このとき\underline{$\partial M = \emptyset \OR \dim M > 1$ ならば}, $\mathcal{A}^+$ の部分集合
		\begin{align}
			\mathcal{A} \coloneqq \bigl\{\, (U,\, \varphi) \in \mathcal{A}^+ \bigm| (U,\, \varphi)\; \text{は\hyperref[def:atlas-oriented]{正の向き}} \,\bigr\} 
		\end{align}
		は\hyperref[def:atlas-oriented]{向きづけられた $C^\infty$ アトラス}である.
	\end{enumerate}
	
\end{myprop}

\begin{proof}
	\begin{enumerate}
		\item まず $\forall p \in M$ を1つ固定する.$\mathcal{A}$ は $M$ の $C^\infty$ アトラスなので,ある $\alpha_p \in \Lambda$ が存在して $p \in U_{\alpha_p}$ を充たす.このときチャート $\bigl( U_{\alpha_p},\, \varphi_{\alpha_p} \bigr) = \bigl( U_{\alpha_p},\, (x^\mu) \bigr) \in \mathcal{A}$ について,
		自然基底 $\left( \eval{\pdv{}{x^1}}_{p},\, \dots,\, \eval{\pdv{}{x^{\dim M}}}_p \right) \in \mathcal{B}_{T_p M}$ が属する $T_p M$ の向きを $\mathcal{O}_{T_p M}$ と定義する.
		$p \in U_{\beta_p}$ を充たす別の $\beta_p \in \Lambda$ に関しても,$\mathcal{A}$ が\hyperref[def:atlas-oriented]{向きづけられた $C^\infty$ アトラス}であるという仮定からチャート $\bigl( U_{\beta_p},\, \varphi_{\beta_p} \bigr) = \bigl( U_{\beta_p},\, (y^\mu)\bigr) \in \mathcal{A}$ について 
		\begin{align}
			\left( \eval{\pdv{}{x^1}}_{p},\, \dots,\, \eval{\pdv{}{x^{\dim M}}}_p \right) \sim \left( \eval{\pdv{}{y^1}}_{p},\, \dots,\, \eval{\pdv{}{y^{\dim M}}}_p \right)
		\end{align}
		が成り立つ\footnote{自然基底の間の基底の取り替え行列はJacobi行列である.}ので,$\mathcal{O}_{T_p M}$ はwell-definedである.
		族 $\Familyset[\big]{\mathcal{O}_{T_p M}}{p \in M}$ は $M$ の\hyperref[def:smooth-orientation]{各点の向き}であるが,座標フレームは局所フレームなので $M$ の\hyperref[def:smooth-orientation]{向き}でもある.
		
		$M$ にこの向きを与えたとき,構成から明らかに $\forall (U,\, \varphi) \in \mathcal{A}$ は\hyperref[def:atlas-oriented]{正の向き}のチャートである.

		\item $\mathcal{A}^+$ は $M$ の $C^\infty$ アトラスなので,$\forall p \in M$ に対して $\alpha_p \in \Lambda^+$ が存在して $p \in U_{\alpha_p}$ を充たす.必要なら $U_{\alpha_p}$ を十分小さくとることで $U_{\alpha_p}$ は連結であるとして良い\footnote{$\mathcal{A}^+$ は極大アトラスなのでこのようなチャート $\bigl( U_{\alpha_p},\, \varphi_{\alpha_p} \bigr)$ を必ず含む.}.このときチャート $\bigl( U_{\alpha_p},\, \varphi_{\alpha_p} \bigr) = \bigl( U_{\alpha_p},\, (x^1,\, x^2,\, \dots) \bigr) \in \mathcal{A}^+$ に付随する座標フレームは補題\ref{lem:orientation}により\hyperref[def:smooth-orientation]{正の向き}であるか負の向きであるかのどちらかである.もしも負の向きならば,$x^1$ を $-x^1$ に置き換えた別の $C^\infty$ チャート\footnote{$\mathcal{A}^+$ は極大アトラスなのでこのようなチャートを必ず含む.なお,この構成は $\partial M \neq \emptyset \AND \dim M = 1$ のとき\hyperref[def:int-manifold-with-boundary]{境界チャート}に適用することができない.} $\bigl( U_{\alpha_p}',\, \varphi_{\alpha_p}' \bigr) \coloneqq  \bigl( U_{\alpha_p},\, (-x^1,\, x^2, \, \dots) \bigr) \in \mathcal{A}^+$ は正の向きである.よって $\mathcal{A}$ は $C^\infty$ アトラスを成す.
		$\mathcal{A}$ に属する全てのチャートが正の向きなので,それらの間の変換関数のJacobianもまた正でなくてはいけない.
	\end{enumerate}
	
\end{proof}


\begin{mydef}[label=def:orientation-preserving]{向き付けを保つ $C^\infty$ 写像}
    \hyperref[def:smooth-orientation]{向き付けられた} $C^\infty$ 多様体 $(M,\, \{\mathcal{O}_{T_p M}\}_{p \in M}),\, (N,\, \{\mathcal{O}_{T_q N}\}_{q \in N})$ と,局所微分同相写像 $F \colon M \lto N$ を与える.
    
    \begin{itemize}
        \item $F$ が\textbf{向きを保つ} (orientation-preserving) とは,$\forall p \in M$ において同型写像 $T_p F \colon T_p M \lto T_{F(p)} N$ が $T_p F (\mathcal{O}_{T_p M}) = \mathcal{O}_{T_{F(p)} N}$ を充たすことを言う.
        \item $F$ が\textbf{向きを逆にする} (orientation-reversing) とは,$\forall p \in M$ において同型写像 $T_p F \colon T_p M \lto T_{F(p)} N$ が $T_p F (\mathcal{O}_{T_p M}) = -\mathcal{O}_{T_{F(p)} N}$ を充たすことを言う.
    \end{itemize}

\end{mydef}

\subsection{新しい向きの構成}

素材となる\hyperref[def:smooth-orientation]{向き付けられた $C^\infty$ 多様体}から新しい向きづけられた多様体を作る方法をいくつか紹介する.

\begin{myprop}[label=prop:product-orientation]{積多様体の向き}
	\begin{itemize}
		\item \hyperref[def:smooth-orientation]{向き付けられた $C^\infty$ 多様体} $M_1,\, M_2$
		\item $C^\infty$ 写像
		\begin{align}
			\pi_1 &\colon M_1 \times M_2 \lto M_1,\; (x,\, y) \lmto x, \\
			\pi_2 &\colon M_1 \times M_2 \lto M_2,\; (x,\, y) \lmto y
		\end{align}
	\end{itemize}
	を与える.このとき,積多様体 $M_1 \times M_2$ の\hyperref[def:smooth-orientation]{向き}であって以下を充たすもの(\textbf{product orientation}と言う)が一意的に存在する:

	\begin{description}
		\item[\textbf{(product orientation)}] 
		
		$M_i$ に与えられた向きを命題\ref{prop:orientation-form}-(1) の方法で再現する\hyperref[def:orientation-form]{向き付け形式} $\omega_i \in \Omega^{\dim M_i}(M_i)$ に関して,$\pi_1^* \omega_1 \wedge \pi_2^* \omega_2 \in \Omega^{\dim (M_1 \times M_2)} (M_1 \times M_2)$ はproduct orientationを命題\ref{prop:orientation-form}-(1) の方法で再現する向き付け形式である.
	\end{description}
\end{myprop}

\begin{proof}
	$n_i \coloneqq \dim M_i$ とおく.$\forall (p,\, q) \in M_1 \times M_2$ を1つ固定する.
	$\omega_i$ は $M_i$ 上至る所 $0$ でないので,ある $v_1,\, \dots,\, v_{n_1} \in T_p M_1$ および $w_1,\, \dots,\, w_{n_2} \in T_q M_2$ が存在して
	\begin{align}
		\omega_1|_p (v_1,\, \dots,\, v_{n_1}) &\neq 0, \quad
		\omega_2|_q (w_1,\, \dots,\, w_{n_2}) \neq 0
	\end{align}
	を充たす.ここで $C^\infty$ 写像
	\begin{align}
		\mathrm{inj}_1^{q} &\colon M_1 \lto M_1 \times M_2,\; x \lmto (x,\, q) \\
		\mathrm{inj}_2^{p} &\colon M_2 \lto M_1 \times M_2,\; x \lmto (p,\, x)
	\end{align}
	によって
	\begin{align}
		V_i &\coloneqq T_{p} (\mathrm{inj}_1^{q}) (v_i) \in T_{(p,\, q)} (M_1 \times M_2) \WHERE i = 1,\, \dots,\, n_1, \\ 
		V_{n_1 + j} &\coloneqq T_q(\mathrm{inj}_2^{p})(w_j) \in T_{(p,\, q)} (M_1 \times M_2) \WHERE j = 1,\, \dots,\, n_2
	\end{align}
	を定義すると,命題\ref{prop:product-tangentv}から
	\begin{align}
		T_{(p,\, q)} \pi_1 (V_i) &= v_i, & T_{(p,\, q)} \pi_2 (V_i) = 0, \\
		T_{(p,\, q)} \pi_1 (V_{n_1 + j}) &= 0, & T_{(p,\, q)} \pi_2 (V_{n_1+j}) = w_j
	\end{align}
	が言える.よって
	\begin{align}
		&\pi_1^* \omega_1 \wedge \pi_2^* \omega_2|_{(p,\, q)} (V_1,\, \dots,\, V_{n_1},\, V_{n_1 + 1},\, \dots,\, V_{n_1 + n_2}) \\
		&= \frac{1}{n_1 ! n_2 !} \sum_{\sigma \in \mathfrak{S}_{n_1 + n_2}} \sgn{\sigma}\, \pi_1^* \omega_1|_{(p,\, q)} (V_{\sigma(1)},\, \dots,\, V_{\sigma(n_1)})\, \pi_2\omega_2|_{(p,\, q)} (V_{\sigma(n_1 + 1)},\, \dots,\, V_{\sigma(n_1 + n_2)}) \\
		&= \frac{1}{n_1 ! n_2 !} \sum_{\sigma \in \mathfrak{S}_{n_1 + n_2}} \sgn{\sigma}\, \omega_1|_p \bigl( T_{(p,\, q)} \pi_1(V_{\sigma(1)}),\, \dots,\, T_{(p,\, q)} \pi_1(V_{\sigma(n_1)})\bigr)\\
		&\qquad \times \omega_2|_q \bigl(T_{(p,\, q)} \pi_2 (V_{\sigma(n_1 + 1)}),\, \dots,\, T_{(p,\, q)} \pi_2(V_{\sigma(n_1 + n_2)})\bigr) \\
		&= \frac{1}{n_1 ! n_2 !} \sum_{\sigma \in \mathfrak{S}_{n_1},\, \tau \in \mathfrak{S}_{n_2}} \sgn{\sigma \tau}\, \omega_1|_p \bigl( T_{(p,\, q)} \pi_1(V_{\sigma(1)}),\, \dots,\, T_{(p,\, q)} \pi_1(V_{\sigma(n_1)})\bigr)\\
		&\qquad \times \omega_2|_q \bigl(T_{(p,\, q)} \pi_2 (V_{n_1+\tau(1)}),\, \dots,\, T_{(p,\, q)} \pi_2(V_{n_1 + \tau(n_2)})\bigr) \\
		&= \omega_1|_p (v_1,\, \dots,\, v_{n_1})\,  \omega_2|_p (w_1,\, \dots,\, w_{n_2}) \\
		&\neq 0
	\end{align}
	であり,$\pi_1^* \omega_1 \wedge \pi_2^* \omega_2|_{(p,\, q)} \neq 0$ が言えた.
\end{proof}

\begin{mydef}[label=def:loc-diffeo]{局所微分同相写像}
    境界なし/あり $C^\infty$ 多様体 $M,\, N$ を与える.
    
    $C^\infty$ 写像 $F \colon M \lto N$ が\textbf{局所微分同相写像} (local diffeomorphism) であるとは,
    $\forall p \in M$ が以下の条件を充たす近傍 $p \in U_p \subset M$ を持つことを言う:
    \begin{enumerate}
        \item $F(U_p) \subset N$ が開集合
        \item $F|_{U_p} \colon U_p \lto F(U_p)$ が微分同相写像
    \end{enumerate}
\end{mydef}

\begin{myprop}[label=prop:pullback-orientation]{引き戻しによる向き}
	\begin{itemize}
		\item 境界あり/なし $C^\infty$ 多様体 $M$ 
		\item \hyperref[def:smooth-orientation]{向き付けられた}境界あり/なし $C^\infty$ 多様体 $N$
		\item \hyperref[def:loc-diffeo]{局所微分同相写像} $F \colon M \lto N$
	\end{itemize}
	を与える.このとき $M$ は $F$ が\hyperref[def:orientation-preserving]{向きを保つ}ような向き(\textbf{pullback orientation}と言う)を一意にもつ.
\end{myprop}

\begin{proof}
	$N$ に与えられた向きを $\Familyset[\big]{\mathcal{O}_{T_q(N)}}{q \in N}$ と書く.
	$\forall p \in M$ を1つ固定する.$F$ が局所微分同相写像なので,点 $p$ における $F$ の微分 $T_p F \colon T_p M \lto T_{F(p)} N$ はベクトル空間の同型写像である.
	よって $T_p M$ の向き $\mathcal{O}_{T_p M}$ であって $T_p F (\mathcal{O}_{T_p M}) = \mathcal{O}_{T_{F(p)} N}$ を充たすものが一意的に存在する.
	% こうして\hyperref[def:smooth-orientation]{各点の向き} $\Familyset[\big]{\mathcal{O}_{T_p M}}{p \in M}$ であって $F$ が\hyperref[def:orientation-preserving]{向きを保つ}ようなものが得られる.

	$N$ の向き $\Familyset[\big]{\mathcal{O}_{T_q(N)}}{q \in N}$ を命題\ref{prop:orientation-form}-(1) の方法で再現する\hyperref[def:orientation-form]{向き付け形式} $\omega \in \Omega^{\dim N}(N)$ について,
	$F^* \omega \in \Omega^{\dim M}(M)$ は明らかに $M$ 上至る所 $0$ でなく,$F^* \omega$ が命題\ref{prop:orientation-form}-(1) の方法で作る $M$ の向き $\Familyset[\big]{\mathcal{O}_{F^* \omega|_p}}{p \in M}$ は $\Familyset[\big]{\mathcal{O}_{T_p M}}{p \in M}$ と等しい.
\end{proof}


\subsection{パラコンパクト・1の分割}

ここでいったん1の分割の技巧を説明しよう.

\begin{mydef}[label=def:locally-finite]{局所有限}
	$X$ を位相空間とし, $\mathcal{U} = \{\, U_\lambda \,\}_{\lambda \in \Lambda}$ を $X$ の被覆とする.
	$\forall x \in X$ において,$x$ の近傍 $V$ であって,$V$ と交わる $U_\lambda$ が有限個であるようなものが存在するとき,$\mathcal{U}$ は\textbf{局所有限} (locally finite) な被覆と呼ばれる.
\end{mydef}

$X$ の\textbf{任意の開被覆}が局所有限な細分を持つとき,$X$ は\textbf{パラコンパクト} (paracompact) であるという.この条件は\hyperref[def.compact]{コンパクト}よりも弱い.次の定理は,全ての\underline{多様体}(\cinfty 多様体だけでなく!)がパラコンパクトよりももう少し良い性質を持っていることを保証してくれる:

\begin{mytheo}[]{}
	$M$ を位相多様体とする.$M$ の任意の開被覆に対して,その細分となる高々可算個の元からなる\footnote{添字集合 $I$ の濃度 (cardinality) が $\abs{I} \le \aleph_0.$}局所有限な開被覆 $\mathcal{V} = \bigl\{\, V_i \,\bigr\}_{i \in I}$ であって,$\overline{V_i}$ が全てコンパクトとなるものが存在する.

	必要ならば,さらに強い条件を充たすようにできる.i.e. 開被覆を成す各 $V_i$ 上にチャート $(V_i,\, \psi_i)$ をとることができて,$\psi_i(V_i) = D(3)$ \footnote{半径 $3$ の開円板.記号の使い方は \ref{gen_cinfty} を参照.}かつ $\bigl\{\, \psi_i^{-1}\bigl(D(1)\bigr) \,\bigr\}_{i \in I}$ が既に $M$ の開被覆となっている.
\end{mytheo}

\begin{proof}
	~\cite[p.30, 命題1.29]{Morita}
\end{proof}


位相空間 $X$ の連続関数 $f \colon X \to \mathbb{R}$ に対して,$f$ の値が $0$ にならない点全体の集合を含む最小の閉集合
\begin{align} 
	\mathrm{supp} f \coloneqq \overline{\bigl\{ x \in X \bigm| f(x) \neq 0 \bigr\} }
\end{align}
を $f$ の \textbf{台} (support) と呼ぶ.


\begin{mydef}[label=def:PoU]{1の分割}
    \begin{itemize}
        \item 境界なし/あり $C^\infty$ 多様体 $M$
        \item $M$ の開被覆 $\mathcal{U} \coloneqq \Familyset[\big]{U_\lambda}{\lambda \in \Lambda}$
    \end{itemize}
    を与える.\textbf{$\bm{\mathcal{U}}$ に従属する $\bm{C}^\infty$ 級の $1$の分割}\footnote{本資料では「$C^\infty$ 級の」を省略する} (smooth partition of unity subordinate to $\mathcal{U}$) とは,$C^\infty$ 関数の族 $\Familyset[\big]{\psi_\lambda \colon M \lto \mathbb{R}}{\lambda \in \Lambda}$ であって以下の条件を充たすもののこと:
    \begin{enumerate}
        \item $\forall \lambda \in \Lambda,\; \forall p \in M$ に対して $\psi_\alpha (x) \in [0,\, 1]$
        \item $\forall \lambda \in \Lambda$ に対して $\supp \psi_\lambda \subset U_\lambda$
        \item $M$ の部分集合族 $\Familyset[\big]{\supp \psi_\lambda}{\lambda \in \Lambda}$ は局所有限である.i.e. $\forall p \in M$ に対してある開近傍 $p \in U_p \subset M$ が存在して,$\bigl\{\, \lambda \in \Lambda \bigm| U_p \cap \supp \psi_\lambda \neq \emptyset \,\bigr\} \subset \Lambda$ が有限集合になる.
        \item $\forall p \in M$ に対して\footnote{条件 $(3)$ により左辺の和はwell-definedである.} $\sum_{\lambda \in \Lambda} \psi_\lambda(p) = 1$
    \end{enumerate}
\end{mydef}

\begin{myprop}[label=prop:PoU]{1の分割の存在}
    \begin{itemize}
        \item 境界なし/あり $C^\infty$ 多様体 $M$
        \item $M$ の開被覆 $\mathcal{U} \coloneqq \Familyset[\big]{U_\lambda}{\lambda \in \Lambda}$
    \end{itemize}
    を任意に与える.このとき,\hyperref[def:PoU]{$\mathcal{U}$ に従属する1の分割}が存在する.
\end{myprop}

\begin{proof}
    かなり技術的なので省略する.例えば~\cite[Theorem 2.23]{Lee12}を参照.
\end{proof}

この存在定理のおかげで,ある1つのチャート(したがって $\mathbb{R}^n,\, \mathbb{H}^n$)の上で定義した構造をアトラス全体にわたって「貼り合わせる」ことができる.
したがって,しばしば $\mathbb{R}^n,\, \mathbb{H}^n$ の上でだけ考えれば十分である.


\subsection{部分多様体の $C^\infty$ 構造}

次に,部分多様体に向きを入れる方法を考える.
然るに,そのためには部分多様体の $C^\infty$ 構造を真面目に扱う必要がある.まずいくつかの定義を述べる:

\begin{mydef}[label=def:rank-smooth]{$C^\infty$ 写像のランク}
    境界あり/なし $C^\infty$ 多様体 $M,\, N$ および $C^\infty$ 写像 $F \colon M \lto N$ を与える.
    \begin{itemize}
        \item 点 $p \in M$ における $F$ の\textbf{ランク} (rank) とは,線型写像 $T_p F \colon T_p M \lto T_{F(p)} N$ のランク,i.e. $\dim \bigl(\Im (T_p F)\bigr) \in \mathbb{Z}_{\ge 0}$ のこと.$\forall p \in M$ における $F$ のランクが等しいとき,$F$ は\textbf{定ランク} (constant rank) であると言い,$\bm{\rank F} \coloneqq \dim \bigl(\Im (T_p F)\bigr)$ と書く.
        \item 点 $p \in M$ における $F$ のランクが $\min \bigl\{\dim M,\, \dim N\bigr\}$ に等しいとき,$F$ は\textbf{点 $\bm{p}$ においてフルランク} (full rank at $p$) であると言う.
        $\rank F = \min \bigl\{\dim M,\, \dim N\bigr\}$ ならば $F$ は\textbf{フルランク} (full rank) であると言う.
    \end{itemize}
\end{mydef}

位相空間 $M,\, N$ を与える.連続写像 $F \colon M \lto N$ が\textbf{位相的埋め込み} (topological embedding) であるとは,$F(M) \subset N$ に $N$ からの相対位相を入れたときに写像 $F \colon M \lto F(M)$ が同相写像になることを言う.

\begin{mydef}[label=def:submersion-smooth]{$C^\infty$ 沈めこみ・$C^\infty$ はめ込み・$C^\infty$ 埋め込み}
    境界あり/なし $C^\infty$ 多様体 $M,\, N$ および\hyperref[def:rank-smooth]{定ランク}の $C^\infty$ 写像 $F \colon M \lto N$ を与える.
    \begin{itemize}
        \item $F$ が\textbf{ $\bm{C^\infty}$ 沈め込み} (smooth submersion) であるとは,$\forall p \in M$ において $T_p F \colon T_p M \lto T_{F(p)} N$ が全射である,i.e. $\rank F = \dim N$ であることを言う.
        \item $F$ が\textbf{ $\bm{C^\infty}$ はめ込み} (smooth immersion) であるとは,$\forall p \in M$ において $T_p F \colon T_p M \lto T_{F(p)} N$ が単射である,i.e. $\rank F = \dim M$ であること\footnote{階数・退化次元の定理から $\dim (\Ker T_p F) + \dim (\Im T_p F) = \dim M$ なので,$\rank F = \dim (\Im T_p F) = \dim M \IMP \dim (\Ker T_p F) = 0 \IFF \Ker T_p F = 0$}を言う.
        \item $F$ が\textbf{ $\bm{C^\infty}$ 埋め込み} (smooth embedding) であるとは,$F$ が $C^\infty$ はめ込みであってかつ位相的埋め込みであることを言う.
    \end{itemize}
\end{mydef}

\begin{mydef}[label=def:submanifold,breakable]{部分多様体}
	境界あり/なし $C^\infty$ 多様体 $(M,\, \mathscr{O}_M)$ を与え\footnote{$\mathscr{O}_M$ は $M$ の\hyperref[ax.topo]{位相}},その\hyperref[maxatlas]{極大アトラス} $\mathcal{A}_M^+$ を1つ固定する.
	$M$ の\underline{部分集合}\footnote{この時点では $S$ の位相を指定していない.} $S \subset M$ を与える.
	\begin{itemize}
		\item $S$ が $M$ の境界あり/なし\textbf{$\bm{C^\infty}$ 部分多様体} (smooth submanifold) であるとは,
		$S$ に位相\footnote{位相空間 $(M,\, \mathscr{O}_M)$ からの\hyperref[def.reltopo]{相対位相}\underline{でなくても良い}} $\mathscr{O}_S$ と\hyperref[diffmani]{$C^\infty$ アトラス}\footnote{$\mathcal{A}_S \subset \mathcal{A}_M^+$ \underline{でなくても良い}} $\mathcal{A}_S$ が与えられていて $(S,\, \mathscr{O}_S,\, \mathcal{A}_S)$ が境界あり/なし\hyperref[diffmani]{$C^\infty$ 多様体}になっていることを言う.
		\item 境界あり/なし $C^\infty$ 部分多様体 $(S,\, \mathscr{O}_S,\, \mathcal{A}_S)$ が $M$ に\textbf{はめ込まれた(境界あり/なし)$\bm{C^\infty}$ 部分多様体} (immersed submanifold) であるとは,
		$C^\infty$ アトラス $\mathcal{A}_S$ に関して包含写像 $\iota \colon S \hookrightarrow M$ が\hyperref[def:submersion-smooth]{$C^\infty$ はめ込み}になっていることを言う.
		\item 境界あり/なし $C^\infty$ 部分多様体 $(S,\, \mathscr{O}_S,\, \mathcal{A}_S)$ が $M$ に\textbf{埋め込まれた(境界あり/なし)$\bm{C^\infty}$ 部分多様体} (embedded submanifold) であるとは,
		位相 $\mathscr{O}_S$ が位相空間 $(M,\, \mathscr{O}_M)$ からの相対位相であり,かつ
		$C^\infty$ アトラス $\mathcal{A}_S$ に関して包含写像 $\iota \colon S \hookrightarrow M$ が\hyperref[def:submersion-smooth]{$C^\infty$ 埋め込み}になっていることを言う.
	\end{itemize}
	
	\tcblower

	$M$ にはめ込まれた $C^\infty$ 部分多様体 $(S,\, \mathscr{O}_S,\, \mathcal{A}_S)$ について,$\dim M - \dim S$ を $S$ の\textbf{余次元} (codimension) と呼ぶ.
\end{mydef}

混乱の恐れがない場合,以下では $C^\infty$ 部分多様体の位相と $C^\infty$ アトラスを明示しない.また,特に断らずに部分多様体と言ったら $C^\infty$ 部分多様体のことを指すものとする.

\begin{mydef}[label=def:slice-chart]{スライスチャート}
	境界あり/なし $C^\infty$ 多様体 $M$ を与え,その\hyperref[maxatlas]{極大アトラス} $\mathcal{A}_M^+$ を1つ固定する.
	部分集合 $S \subset M$ を与える.

	\begin{itemize}
		\item \underline{$M$ の}チャート $(U,\, \varphi) = \bigl( U,\, (x^\mu) \bigr) \in \mathcal{A}_M^+$ が\textbf{$\bm{S}$ に関する内部 $\bm{k}$-スライスチャート}であるとは,
		\begin{align}
			\varphi(S \cap U) 
			&= \bigl\{\, (x^1,\, \dots,\, x^k,\, c^{k+1},\, \dots,\, c^{\dim M}) \in \varphi(U) \bigm| c^{k+1},\, \dots,\, c^{\dim M}\; \text{は定数} \,\bigr\} 
		\end{align}
		が成り立つことを言う.
		\item \underline{$M$ の}チャート $(U,\, \varphi) = \bigl( U,\, (x^\mu) \bigr) \in \mathcal{A}_M^+$ が\textbf{$\bm{S}$ に関する境界 $\bm{k}$-スライスチャート}であるとは,
		\begin{align}
			\varphi(S \cap U) 
			&= \bigl\{\, (x^1,\, \dots,\, x^k,\, c^{k+1},\, \dots,\, c^{\dim M}) \in \varphi(U) \bigm| x^k \ge 0 \AND c^{k+1},\, \dots,\, c^{\dim M}\; \text{は定数} \,\bigr\} 
		\end{align}
		が成り立つことを言う.
	\end{itemize}
	
	\tcblower

	$S$ に関する境界 $k$-スライスチャートを考えない場合は,$S$ に関する内部 $k$-スライスチャートのことを単に \textbf{$\bm{S}$ に関する $\bm{k}$-スライスチャート}と呼ぶ.
	% ,i.e. $\varphi(S \cap U) \subset \mathbb{R}^{\dim M}$ が $\varphi(U)$ と $k$ 次元超平面との共通部分になっていることを言う.
\end{mydef}

\begin{mytheo}[label=thm:slice-shart-embedded,breakable]{スライスチャートによる埋め込まれた $C^\infty$ 部分多様体の特徴付け}
	\begin{itemize}
		\item 境界を持たない $C^\infty$ 多様体 $M$ とその\hyperref[maxatlas]{極大アトラス} $\mathcal{A}_M^+$
		\item $C^\infty$ 写像
		\begin{align}
			\pi \colon \mathbb{R}^{\dim M} \lto \mathbb{R}^k,\; (x^1,\, \dots,\, x^k,\, x^{k+1},\, \dots,\, x^{\dim M}) \lmto (x^1,\, \dots,\, x^k)
		\end{align}
		\item 部分集合 $S \subset M$
	\end{itemize}
	を与える.
	このとき,以下の2つが成り立つ.
	\begin{enumerate}
		\item $S$ が\hyperref[def:submersion-smooth]{埋め込まれた境界なし $C^\infty$ 部分多様体} 
		$\IMP$
		$\forall p \in S$ に対して, $p$ を含む\hyperref[def:slice-chart]{$S$ に関する $\dim S$-スライスチャート} $(U,\, \varphi) \in \mathcal{A}^+_M$ が存在する.
		\item $\forall p \in S$ に対して, $p$ を含む\hyperref[def:slice-chart]{$S$ に関する $k$-スライスチャート} $(U,\, \varphi) \in \mathcal{A}^+_M$ が存在する.
		$\IMP$
		$S$ は $M$ からの\hyperref[def.reltopo]{相対位相}によって $k$ 次元の\hyperref[def.topomani]{位相多様体}になり,
		かつ $S$ の $C^\infty$ アトラス
		\begin{align}
			\mathcal{A}_S \coloneqq \bigl\{\, (S \cap U,\, \pi \circ \varphi|_{S \cap U}) \bigm| (U,\, \varphi) \in \mathcal{A}_M^+ \ST S\,\text{に関する}\,k\text{-スライスチャート} \,\bigr\} 
		\end{align}
		が $S$ に与える\hyperref[diffmani]{$C^\infty$ 構造}に関して $S$ は $M$ に\hyperref[def:submanifold]{埋め込まれた境界なし $C^\infty$ 部分多様体}になる.
	\end{enumerate}
	
	\tcblower

	もしくは,$S$ に境界がある場合は次の通り:
	\begin{enumerate}
		\item $S$ が\hyperref[def:submersion-smooth]{埋め込まれた境界付き $C^\infty$ 部分多様体} 
		$\IMP$
		$\forall p \in S$ に対して, $p$ を含む\hyperref[def:slice-chart]{$S$ に関する内部 $\dim S$-スライスチャート} $(U,\, \varphi) \in \mathcal{A}^+_M$ が存在するか,$p$ を含む\hyperref[def:slice-chart]{$S$ に関する境界 $\dim S$-スライスチャート} $(U,\, \varphi) \in \mathcal{A}^+_M$ が存在するかのどちらかである.
		\item $\forall p \in S$ に対して, $p$ を含む\hyperref[def:slice-chart]{$S$ に関する内部 $\dim S$-スライスチャート} $(U,\, \varphi) \in \mathcal{A}^+_M$ が存在するか,$p$ を含む\hyperref[def:slice-chart]{$S$ に関する境界 $\dim S$-スライスチャート} $(U,\, \varphi) \in \mathcal{A}^+_M$ が存在するかのどちらかである.
		$\IMP$
		$S$ は $M$ からの\hyperref[def.reltopo]{相対位相}によって $k$ 次元の\hyperref[def.topomani]{境界付き位相多様体}になり,
		かつ $S$ の $C^\infty$ アトラス
		\begin{align}
			\mathcal{A}_S \coloneqq \bigl\{\, (S \cap U,\, \pi \circ \varphi|_{S \cap U}) \bigm| (U,\, \varphi) \in \mathcal{A}_M^+ \ST S\,\text{に関する}\,k\text{-スライスチャート} \,\bigr\} 
		\end{align}
		が $S$ に与える\hyperref[diffmani]{$C^\infty$ 構造}に関して $S$ は $M$ に\hyperref[def:submanifold]{埋め込まれた境界付き $C^\infty$ 部分多様体}になる.
	\end{enumerate}
\end{mytheo}

\begin{proof}
	$S$ の包含写像を $\iota \colon S \hookrightarrow M$ とおく.まずは $S$ の境界がない場合に示す.
	\begin{enumerate}
		\item $\forall p \in S$ を1つ固定する.
		$(S,\, \mathcal{O}_S,\, \mathcal{A}_S)$ が埋め込まれた $C^\infty$ 部分多様体だとする.このとき $\iota$ は\hyperref[def:submersion-smooth]{$C^\infty$ 埋め込み},従って\hyperref[def:rank-smooth]{定ランク写像}であるから,\hyperref[thm:rank]{局所的ランク定理}により
		$p$ を含む $C^\infty$ チャート $(U,\, \varphi) \in \mathcal{O}_S$,$(V,\, \psi) \in \mathcal{O}_M$ が存在して,
		% $\varphi(p) = \psi(p) = 0$ かつ
		$\forall (x^1,\, \dots,\, x^{\dim S}) \in \varphi(U)$ に対して
		\begin{align}
			\psi \circ \iota|_U \circ \varphi^{-1}(x^1,\, \dots,\, x^{\dim S})
			&= (x^1,\, \dots,\, x^{\dim S},\, 0,\, \dots,\, 0)
		\end{align}
		を充たす.
		ここで $B_\varepsilon^{\dim S} \bigl( \varphi(p) \bigr)  \subset \varphi(U) \AND B_{\varepsilon}^{\dim M} \bigl( \psi(p) \bigr) \subset \psi(V)$ を充たすような十分小さい $\varepsilon > 0$ をとり,$U_0 \coloneqq \varphi^{-1} \bigl( B_\varepsilon^{\dim S} (\varphi(p)) \bigr) \subset U,\; V_0 \coloneqq \psi^{-1} \bigl( B_\varepsilon^{\dim M} (\psi(p)) \bigr) \subset V$ とおく.
		すると $U_0 \in \mathcal{O}_S$ であるが,$\mathcal{O}_S$ は $M$ からの相対位相なので,ある $M$ の開集合 $W \subset M$ が存在して $U_0 = S \cap W$ と書ける.このとき $(V_0 \cap W,\, \psi|_{V_0 \cap W}) \in \mathcal{A}_M^+$ は
		\begin{align}
			\psi|_{V_0 \cap W} \bigl((V_0 \cap W) \cap U\bigr) = \psi(U_0) = \psi \circ \varphi^{-1} \bigl( B_{\varepsilon}^{\dim S}(\varphi(p)) \bigr) = \psi \circ \iota|_U \circ \varphi^{-1} \bigl( B_{\varepsilon}^{\dim S}(\varphi(p)) \bigr) 
		\end{align}
		を充たすので,$S$ に関する $\dim S$-スライスチャートになっている.

		\item 
		% $\forall p \in S$ に対して, $p$ を含む\hyperref[def:slice-chart]{$S$ に関する $\dim S$-スライスチャート} $(U,\, \varphi) \in \mathcal{A}^+_M$ が存在する.
		まず,$S$ に $M$ からの相対位相 $\mathcal{O}_S$ を入れて位相空間にしたときにそれが位相多様体になっていることを示す.
		位相空間 $(S,\, \mathcal{O}_S)$ が第2可算なHausdorff空間であることは明らかなので,あとは $\forall p \in S$ に対して
		\begin{itemize}
			\item $p$ の開近傍 $p \in V \subset S$
			\item $\mathbb{R}^{k}$ の開集合 $\psi(V) \subset \mathbb{R}^{k}$
			\item 同相写像 $\psi \colon V \xrightarrow{\approx} \psi(V)$
		\end{itemize}
		が存在することを言えば良い.実際,点 $p$ を含む $k$-スライスチャート $(U,\, \varphi) \in \mathcal{A}_M^+$ をとり,
		\begin{align}
			V \coloneqq U \cap S,\quad \psi \coloneqq \pi \circ \varphi|_V \colon V \lto \psi(V)
		\end{align}
		とおくと連続写像 $\psi \colon V \lto \psi(V)$ は同相写像である.と言うのも,$C^\infty$ 写像
		\begin{align}
			j \colon \mathbb{R}^k \lto \mathbb{R}^{\dim M},\; (x^1,\, \dots,\, x^k) \lmto (x^1,\, \dots,\, x^k,\, c^{k+1},\, \dots,\, c^{\dim M})
		\end{align}
		を使って定義される連続写像
		\begin{align}
			\varphi^{-1} \circ j|_{\psi(V)} \colon \psi(V) \lto V
		\end{align}
		が $\psi$ の逆写像になっているからである.

		 次に,$\mathcal{A}_S$ が位相多様体 $(S,\, \mathcal{O}_S)$ に $C^\infty$ 構造を定めることを示す.
		仮定より $\mathcal{A}_S$ は $S$ の開被覆を与える.よって座標変換が $C^\infty$ 級であることを示せば良い. 
		実際,$U \cap U' \neq \emptyset$ を充たす2つの $S$ に関する $k$-スライスチャート $(U,\, \varphi),\; (U',\, \varphi') \in \mathcal{A}_M^+$ をとると,対応する
		$\mathcal{A}_S$ の座標変換は
		\begin{align}
			(\pi \circ \varphi'|_{S \cap U'}) \circ (\pi \circ \varphi_{S \cap U})^{-1} = \pi \circ \varphi'|_{S \cap U'} \circ \varphi \circ j|_{\pi (\varphi(S \cap U \cap U'))}
		\end{align}
		であり,$C^\infty$ 写像の合成で書けているので $C^\infty$ 写像である.

		 最後に,$C^\infty$ 多様体 $(S,\, \mathcal{O}_S,\, \mathcal{A}_S)$ が $M$ に埋め込まれた $C^\infty$ 部分多様体であることを示す.$\mathcal{O}_S$ が $M$ からの相対位相なので包含写像 $\iota$ が\hyperref[def:submersion-smooth]{位相的埋め込み}であることは明らか.
		$\forall p \in S$ を一つ固定する.このとき $p$ の近傍における包含写像 $\iota$ の座標表示であって
		\begin{align}
			(x^1,\, \dots,\, x^{k}) \lmto (x^1,\, \dots,\, x^{k},\, c^{k+1},\, \dots,\, c^{\dim M})
		\end{align}
		の形のものが必ず存在し,その微分(i.e. Jacobi行列)は単射なので,$\iota$ は $C^\infty$ はめ込みだと分かった.
	\end{enumerate}
	$S$ の境界がある場合,(1) において $p \in \partial S$ のときは\hyperref[thm:rank-b]{境界がある場合の局所的ランク定理}を使えば良い.(2) は内部スライスチャートの場合と境界スライスチャートの場合とで全く同じ議論ができる.
\end{proof}

\hyperref[def:mani-with-boundary]{境界付き $C^\infty$ 多様体} $M$ の境界 $\partial M$ に $M$ からの相対位相を入れると $\dim M - 1$ 次元の\hyperref[def.topomani]{境界を持たない位相多様体}になることは命題\ref{prop:manifold-boundary-basic}-(2) で見たが,その際は $\partial M$ の $C^\infty$ 構造については何も言及していなかった.
ここでそれが明らかになる:

\begin{mytheo}[label=thm:smooth-structure-of-boundary]{境界の $C^\infty$ 構造}
	\begin{itemize}
		\item 境界付き $C^\infty$ 多様体 $(M,\, \mathscr{O}_M)$ とその\hyperref[maxatlas]{極大アトラス} $\mathcal{A}_M^+$
		\item $C^\infty$ 写像 
		\begin{align}
			\pi \colon \mathbb{R}^{\dim M} \lto \mathbb{R}^{\dim M - 1},\; (x^1,\, \dots,\, x^{\dim M -1},\, x^{\dim M}) \lmto (x^1,\, \dots,\, x^{\dim M -1})
		\end{align}
	\end{itemize}
	を与える.
	このとき,$M$ の境界 $\partial M$ に $M$ からの相対位相 $\mathscr{O}_{\partial M}$ を入れてできる $\dim M -1$ 次元位相多様体 $(\partial M,\, \mathscr{O}_{\partial M})$ について以下が成り立つ:
	\begin{enumerate}
		\item $(\partial M,\, \mathscr{O}_{\partial M})$ は
		\begin{align}
			\mathcal{A}_{\partial M} \coloneqq \bigl\{\, (\partial M \cap U,\, \pi \circ \varphi|_{\partial M \cap U}) \bigm| (U,\, \varphi) \in \mathcal{A}_M^+ \ST \text{\hyperref[def:int-manifold-with-boundary]{境界チャート}} \,\bigr\} 
		\end{align}
		を $C^\infty$ アトラスに持つ.
		\item $C^\infty$ 多様体 $(\partial M,\, \mathscr{O}_{\partial M},\, \mathcal{A}_{\partial M})$ は $M$ にproperに\footnote{つまり,包含写像 $\iota \colon \partial M \lto M$ による $M$ の任意のコンパクト集合 $K \subset M$ の逆像 $\iota^{-1}(K) \subset \partial M$ がコンパクト}
		\hyperref[def:submanifold]{埋め込まれた $C^\infty$ 部分多様体}になる.
	\end{enumerate}
\end{mytheo}

\begin{proof}
	包含写像を $\iota \colon \partial M \hookrightarrow M$ と書く.
	\begin{enumerate}
		\item \hyperref[def:int-manifold-with-boundary]{$\partial M$ の定義}から,$\forall p \in \partial M$ に対して境界チャート $(U,\, \varphi) \in \mathcal{A}_M^+$ が存在して $p \in U$ を充たす.i.e. $\mathcal{A}_{\partial M}$ は $\partial M$ の開被覆を与えるので,あとは $\mathcal{A}_{\partial M}$ の座標変換が $C^\infty$ 級であることを示せば良い.
		実際,任意の境界チャート $(U,\, \varphi) = \bigl(U,\, (x^\mu)\bigr) \in \mathcal{A}_M^+$ に関して
		\begin{align}
			\varphi|_{\partial M \cap U} \colon p \lmto \bigl( x^1(p),\, \dots,\, x^{\dim M - 1}(p),\, 0 \bigr) 
		\end{align}
		であるから,
		$C^\infty$ 写像
		\begin{align}
			j \colon \mathbb{R}^{\dim M - 1} \lto \mathbb{R}^{\dim M},\; (x^1,\, \dots,\, x^{\dim M -1}) \lmto  (x^1,\, \dots,\, x^{\dim M -1},\, 0) 
		\end{align}
		を使って
		\begin{align}
			(\pi \circ \varphi|_{\partial M \cap U})^{-1} = \varphi^{-1} \circ j|_{\pi(\varphi(\partial M \cap U))}
		\end{align}
		と書けるので,$U \cap U' \neq \emptyset$ を充たす2つの境界チャート $(U,\, \varphi),\, (U',\, \varphi') \in \mathcal{A}_M^+$ をとると,対応する $\mathcal{A}_{\partial M}$ の座標変換は
		\begin{align}
			(\pi \circ \varphi'|_{\partial M \cap U'}) \circ (\pi \circ \varphi|_{\partial M \cap U})^{-1} = \pi \circ \varphi'|_{\partial M \cap U'} \circ \varphi^{-1} \circ j|_{\pi(\varphi(\partial M \cap U \cap U'))}
		\end{align}
		と $C^\infty$ 写像の合成で書けるので $C^\infty$ 写像である.

		\item $\forall p \in \partial M$ を1つ固定する.このとき $p$ を含む境界チャート $(U,\, \varphi) \in \mathcal{A}_M^+$ が存在するので,$(V,\, \psi) \coloneqq (\partial M \cap U,\, \pi \circ \varphi|_{\partial M \cap U}) \in \mathcal{A}_{\partial M}$ と $(U,\, \varphi) \in \mathcal{A}_M^+$ による包含写像 $\iota$ の座標表示は
		\begin{align}
			\varphi \circ \iota \circ \psi^{-1} = \varphi \circ \iota \circ \varphi^{-1} \circ j|_{\pi(\varphi(\partial M \cap U))} = j|_{\pi(\varphi(\partial M \cap U))}
		\end{align}
		である.従って $T_{\psi(p)} (\varphi \circ \iota \circ \psi^{-1})$ は明らかに単射なので連鎖率および $\psi,\, \varphi$ が微分同相写像であることから $T_p \iota$ は単射だとわかる.よって $\iota$ は\hyperref[def:submersion-smooth]{$C^\infty$ はめ込み}である.
		$\mathcal{O}_{\partial M}$ が $M$ からの相対位相であることから包含写像 $\iota$ が位相的埋め込みであることは明らか.
	\end{enumerate}
	
\end{proof}

実は,もう少し緻密な考察をすると定理\ref{thm:slice-shart-embedded}, \ref{thm:smooth-structure-of-boundary}で構成した $C^\infty$ 構造は微分同相を除いて一意であることがわかる~\cite[p.114, Theorem5.31]{Lee12}が,議論が若干技術的なのでここでは省略する.

\subsection{$C^\infty$ 部分多様体の向き付け}

ここでは特に,\hyperref[def:submersion-smooth]{はめ込まれた or 埋め込まれた境界あり/なし $C^\infty$ 部分多様体}のみ考える.
\begin{itemize}
	\item \hyperref[def:smooth-orientation]{向きづけられた境界あり/なし $C^\infty$ 多様体} $\bigl(M,\, \Familyset[\big]{\mathcal{O}_{T_p M}}{p \in M}\bigr)$
	\item $M$ に\hyperref[def:submersion-smooth]{はめ込まれた\underline{余次元 $1$ の}境界あり/なし $C^\infty$ 部分多様体} $S \subset M$
\end{itemize}
が与えられたとき,包含写像を $\iota_S \colon S \hookrightarrow M$ と書くことにする.
% このとき位相は相対位相で,$C^\infty$ 構造としては定理\ref{thm:slice-shart-embedded}で定まるものになる.

\begin{mydef}[label=def:vecf-along]{部分多様体に沿ったベクトル場}
	\begin{itemize}
		\item 境界あり/なし $C^\infty$ 多様体 $M$
		\item $M$ に\hyperref[def:submersion-smooth]{はめ込まれた境界あり/なし $C^\infty$ 部分多様体} $S \subset M$
	\end{itemize}
	を与える.\textbf{$\bm{S}$ に沿ったベクトル場}とは,接束
	\begin{align}
		\mathbb{R}^{\dim M} \hookrightarrow TM \xrightarrow{\pi} M
	\end{align}
	の部分束
	\begin{align}
		\mqty{	
			\mathbb{R}^{\dim M} &\hookrightarrow &\displaystyle\underbrace{\coprod_{p \in S} T_p M}_{\eqqcolon \bm{TM|_S}} &\xrightarrow{\pi} &S \\
			&&&& \\
			&&\rotatebox{90}{\subset}&& \\
			&&&& \\
			&&TS&&
		}
	\end{align}
	の $C^\infty$ 切断のことを言う\footnote{$S$ の接束 $TS$ の $C^\infty$ 切断\underline{ではない}.また,$\mathfrak{X}(M)|_S \subsetneq TM|_S$ である.}.
	
	\tcblower

	$X \in \Gamma(TM|_S)$ が\textbf{至る所で $\bm{S}$ に接さない}とは,$\forall p \in S$ において $X_p \in T_p M \setminus T_p \iota_S (T_p S)$ が成り立つことを言う.
\end{mydef}

\begin{myprop}[label=prop:submanifold-orientation]{余次元 $1$ の部分多様体の向き}
	\begin{itemize}
		\item \hyperref[def:smooth-orientation]{向きづけられた境界あり/なし $C^\infty$ 多様体} $\bigl(M,\, \Familyset[\big]{\mathcal{O}_{T_p M}}{p \in M}\bigr)$
		\item $M$ に\hyperref[def:submersion-smooth]{はめ込まれた\underline{余次元 $1$ の}境界あり/なし $C^\infty$ 部分多様体} $S \subset M$
	\end{itemize}
	を与える.
	このとき,\hyperref[def:vecf-along]{$S$ に沿った,至る所で $S$ に接さないベクトル場} $N \in \Gamma(TM|_S)$ が存在するならば,
	以下が成り立つ:
	\begin{enumerate}
		\item 与えられた $M$ の向きを命題\ref{prop:orientation-form}-(1) の方法で再現する $M$ の\hyperref[def:orientation-form]{向き付け形式} $\omega \in \Omega^{\dim M}(M)$ に対して,$\iota_S^* \bigl( \iunit_N (\omega) \bigr) \in \Omega^{\dim M - 1}(S)$ は $S$ の向き付け形式である.
		\item 
		$\forall p \in S$ に対して
		\begin{align}
			\mathcal{O}_{N_p} \coloneqq \Bigl\{(e_1,\, \dots,\, e_{\dim M - 1}) \in \mathcal{B}_{T_p S} \Bigm| \bigl(N_p,\, T_p \iota_S (e_1),\, \dots,\, T_p \iota_S(e_{\dim M - 1})\bigr) \in \mathcal{O}_{T_p M}\Bigr\}
		\end{align}
		と定義すると,族 $\Familyset[\big]{\mathcal{O}_{N_p}}{p \in S}$ は $S$ の\hyperref[def:smooth-orientation]{向き}であり,
		(1) の向き付け形式 $\iota_S^* \bigl( \iunit_N (\omega) \bigr)$ によって\ref{prop:orientation-form}-(1) の方法で再現される.
	\end{enumerate}
\end{myprop}

\begin{proof}
	\begin{enumerate}
		\item $\iota_S^* \bigl( \iunit_N (\omega) \bigr)$ が $\forall p \in S$ において $0$ にならないことを示す.
		
		仮定より $S$ がはめ込まれた $C^\infty$ 部分多様体でかつ $N$ は至る所 $S$ に接さないので,
		$T_p \iota_S \colon T_p S \lto T_p M$ は単射でかつ $\forall (e_1,\, \dots,\, e_{\dim M - 1}) \in \mathcal{B}_{T_p S}$ に対して $\bigl(N_p,\, T_p \iota_S(e_1),\, \dots,\, T_p \iota_S(e_{\dim M - 1})\bigr) \in \mathcal{B}_{T_p M}$ である.よって
		\begin{align}
			\iota_S^* \bigl( \iunit_N (\omega) \bigr)|_p \bigl( e_1,\, \dots,\, e_{\dim M - 1} \bigr) = \omega_p \bigl( N_p,\, T_p \iota_S(e_1),\, \dots,\, T_p \iota_S (e_{\dim M - 1}) \bigr) \neq 0
		\end{align}
		が言えた.
		\item (1) の証明より明らか.
	\end{enumerate}
\end{proof}

\hyperref[def:vecf-along]{$S$ に沿った,至る所で $S$ に接さないベクトル場} $N \in \Gamma(TM|_S)$ はいつでも存在するとは限らないので,$N$ をどのようにして構成するかが問題となる.
$S = \partial M$ の場合はいつでも $N$ を作ることができる:

\begin{mydef}[label=def:inward-pointing]{境界における内向き/外向きの接ベクトル}
	境界付き $C^\infty$ 多様体 $M$ と,$\forall p \in \partial M$ を1つ与える.
	\begin{itemize}
		\item $v \in T_p M$ が\textbf{内向き}であるとは,ある $\varepsilon > 0$ と $C^\infty$ 曲線 $\gamma \colon [0,\, \varepsilon) \lto M$ が存在して $\gamma \bigl( \textcolor{red}{(}0,\, \varepsilon) \bigr) \subset \Int M,\;  \gamma(0) = p,\; \dot{\gamma}(0) = v$ を充たすことを言う.
		\item $v \in T_p M$ が\textbf{外向き}であるとは,ある $\varepsilon > 0$ と $C^\infty$ 曲線 $\gamma \colon (-\varepsilon,\, 0] \lto M$ が存在して $\gamma \bigl( (-\varepsilon,\, 0 \textcolor{red}{)} \bigr) \subset \Int M,\; \gamma(0) = p,\; \dot{\gamma}(0) = v$ を充たすことを言う.
	\end{itemize}
	点 $p \in M$ における内向きの接ベクトル全体の集合を $\bm{T_p^+ M}$,外向きの接ベクトル全体の集合を $\bm{T_p^- M}$ と書く.
\end{mydef}

\begin{mylem}[label=lem:inward-pointing]{内向き/外向きの接ベクトルの特徴付け}
	境界付き $C^\infty$ 多様体 $M$ と,$\forall p \in \partial M$ を1つ与える.
	$p$ を含む境界チャート $(U,\, \varphi) = \bigl( U,\, (x^\mu) \bigr)$ を1つ固定する.
	\begin{enumerate}
		\item $v \in T_p M$ が内向き $\IFF$ $v = v^\mu \eval{\pdv{}{x^\mu}}_p$ と展開したときに $v^{\dim M} > 0$
		\item $v \in T_p M$ が外向き $\IFF$ $v = v^\mu \eval{\pdv{}{x^\mu}}_p$ と展開したときに $v^{\dim M} < 0$
		\item 集合として
		\begin{align}
			T_p M = T_p \iota_{\partial M} \bigl( T_p (\partial M) \bigr) \sqcup T_p^+ M \sqcup T_p^- M
		\end{align}
		が成り立つ.特に $v \in T_p^+ M \IFF -v \in T_p^- M$ である.
	\end{enumerate}
	
\end{mylem}

\begin{proof}
	\begin{enumerate}
		\item 
		\begin{description}
			\item[\textbf{($\bm{\Longrightarrow}$)}] 
			
			$C^\infty$ 関数
			\begin{align}
				\gamma^\mu &\coloneqq x^\mu \circ \gamma \colon [0,\, \varepsilon) \lto \mathbb{R} \WHERE \mu \neq \dim M \\
				\gamma^\mu &\coloneqq x^\mu \circ \gamma \colon [0,\, \varepsilon) \lto \mathbb{H} \WHERE \mu = \dim M
			\end{align}
			を考える.
			このとき
			\begin{align}
				v^{\mu} = \dv{\gamma^{\mu}}{t}()(0)
			\end{align}
			なのでTaylorの定理より $\gamma^\mu(t) = \gamma^\mu (0) + tv^\mu + \order{t^2}$ だとわかるが,$p = \gamma(0)\in \partial M$ なので $\gamma^{\dim M}(0) = 0$.
			よって $v^{\dim M} > 0$ でなくてはいけない.

			\item[\textbf{($\bm{\Longleftarrow}$)}] 
			
			$C^\infty$ 曲線 $\gamma \colon [0,\, \varepsilon) \lto M$ を
			\begin{align}
				\gamma(t) \coloneqq \varphi^{-1} \bigl( x^1(p) + tv^1,\, \dots,\, tv^{\dim M} \bigr) 
			\end{align}
			で定義すると $\gamma \bigl( (0,\, \varepsilon) \bigr) \subset \Int M,\; \gamma(0) = p,\; \dot{\gamma}(0) = v$ を充たす.
		\end{description}

		\item (1) と全く同様
		
		\item 
		命題\ref{prop:tangent-velocity}より,$v \in T_p \iota_{\partial M} \bigl( T_p (\partial M) \bigr) \IFF v = v^\mu \eval{\pdv{x^\mu}}_p$ と展開したときに $v^{\dim M} = 0$ である.
	\end{enumerate}
	
\end{proof}

\begin{myprop}[label=prop:outward-vecf]{境界に沿った外向きベクトル場}
	境界付き $C^\infty$ 多様体 $M$ を与える.このとき,\hyperref[def:vecf-along]{$\partial M$ に沿った}ベクトル場 $X \in \Gamma(TM|_{\partial M})$ であって,$\forall p \in \partial M$ において $X_p$ が\hyperref[def:inward-pointing]{外向き}(内向き)であるものが存在する.
\end{myprop}

\begin{proof}
	$M$ の全ての境界チャートの集合を $\Familyset[\big]{\bigl(U_\alpha,\, (x_\alpha{}^\mu)\bigr)}{\alpha \in A}$ と書くと,開集合族 $\mathcal{U} \coloneqq \Familyset[\big]{U_\alpha \cap \partial M}{\alpha \in A}$ は $\partial M$ の開被覆である.
	命題\ref{thm:smooth-structure-of-boundary}より $\partial M$ は $C^\infty$ 多様体だから,\hyperref[def:PoU]{$\mathcal{U}$ に従属する1の分割} $\Familyset[\big]{\psi_\alpha \colon \partial M \lto [0,\, 1]}{\alpha \in A}$ をとることができる.
	
	ところで,補題\ref{lem:inward-pointing}-(2) より $\forall \alpha \in A$ および $\forall p \in U_\alpha$ について $-\eval{\pdv{}{x_\alpha{}^{\dim M}}}_p \in T_p M$ は外向きであるから,
	\begin{align}
		X \coloneqq -\sum_{\alpha \in A} \psi_\alpha\,\eval{\pdv{}{x_\alpha{}^{\dim M}}}_{\partial M} \in \Gamma(TM|_{\partial M})
	\end{align}
	は $\partial M$ 上至る所外向きである.
\end{proof}

補題\ref{lem:inward-pointing}-(3) より $\partial M$ に沿った至る所\hyperref[def:inward-pointing]{外向き}なベクトル場は\hyperref[def:vecf-along]{至る所 $\partial M$ に接さない}ので,次のことがわかる:

\begin{myprop}[label=prop:boundary-orientation]{境界の向き}
	$1$ 次元以上の\hyperref[def:smooth-orientation]{向きづけられた境界付き $C^\infty$ 多様体} $\bigl(M,\, \Familyset[\big]{\mathcal{O}_{T_p M}}{p \in M}\bigr)$ を与える.
	このとき $\dim M - 1$ 次元 $C^\infty$ 多様体\footnote{$\partial M$ の $C^\infty$ 構造は命題\ref{thm:smooth-structure-of-boundary}によって指定されるものである.} $\partial M$ は\hyperref[def:smooth-orientation]{向き付け可能}であり,
	任意の\hyperref[def:vecf-along]{$\partial M$ に沿った}\hyperref[def:inward-pointing]{外向きベクトル場} $N \in\Gamma(TM|_{\partial M})$ が命題\ref{prop:submanifold-orientation}-(2) の方法で $\partial M$ に与える向き $\Familyset[\big]{\mathcal{O}_{N_p}}{p \in \partial M}$ は,$N$ の取り方によらずに一意に定まる.
\end{myprop}

\begin{proof}
	命題\ref{prop:outward-vecf}より $\partial M$ が向き付け可能だとわかる.

	勝手な2つの $\partial M$ に沿った外向きベクトル場 $N,\, N'$ をとる.$\forall p \in \partial M$ を一つ固定し,$p$ を含む境界チャート $\bigl( U,\, (x^\mu) \bigr)$ を1つとる.
	すると
	\begin{align}
		\left(N_p,\, \eval{\pdv{}{x^1}}_p,\, \dots,\, \eval{\pdv{}{x^{\dim M -1}}}_p \right),\, \left(N'_p,\, \eval{\pdv{}{x^1}}_p,\, \dots,\, \eval{\pdv{}{x^{\dim M -1}}}_p \right) \in \mathcal{B}_{T_p M}
	\end{align}
	であるが,両者の間の変換行列は
	\begin{align}
		\mqty[
			N^{\dim M}(p)/ N'{}^{\dim M}(p) &0 &\cdots &0 \\
			* &1 &\cdots &0 \\
			\vdots &  &\ddots &\vdots \\
			* &0 &\cdots &1 \\
		]
	\end{align}
	の形をしているのでその行列式は $N^{\dim M}(p)/ N'{}^{\dim M}(p)$ に等しく,補題\ref{lem:inward-pointing}-(2) から正だとわかる.よって $\mathcal{O}_{N_p} = \mathcal{O}_{N'_p}$ が言えた.
\end{proof}

\begin{myexample}[label=ex:orientation-of-halfplane]{$\mathbb{H}^n$ の境界の向き}
	$\mathbb{H}^n$ に $\mathbb{R}^n$ からの標準的な向きを入れたとき,$\partial \mathbb{H}^n$ に命題\ref{prop:boundary-orientation}の向きを入れてみよう.
	ベクトル場 $-\eval{\pdv{}{x^n}}_{\partial \mathbb{H}^n}$ は $\partial \mathbb{H}^n$ に沿った外向きベクトル場だから,
	$\mathbb{H}^n$ の向き付け形式 $\omega \in \Omega^n(\mathbb{H}^n)$ に対して
	\begin{align}
		\iunit_{-\pdv{}{x^n}}(\omega) \left( \pdv{}{x^1},\, \dots,\, \pdv{}{x^{n-1}} \right) 
		&= (-1)^{n} \omega \left( \pdv{}{x^1},\, \dots,\, \pdv{}{x^{n-1}},\, \pdv{}{x^n} \right) 
	\end{align}
	が $\partial \mathbb{H}^n$ の向きを定めている.i.e. $n$ が偶数のとき $\mathbb{R}^{n-1}$ の標準的な向きと同一だが,$n$ が奇数のときは逆になる.
\end{myexample}


\section{微分形式の積分}

% $n$ 次元\cinfty 多様体 $M$ が向き付け可能としよう.

% $M$ のチャート $(U,\, \varphi) = (U;\; x^i)$ をとる.$n$-形式 $\omega \in \Omega^k(M)$ が
% \begin{align} 
% 	\omega \coloneqq h(p) \dd{x^1} \wedge \cdots \wedge \dd{x^n}
% \end{align}
% と座標表示されているとする.
% 別のチャート $(V,\, \psi) = (V;\; y^i)$ をとったときの $U \cap V \neq \emptyset$ 上の $\omega$ は
% \begin{align} 
% 	\omega &= h(p) \pdv{x^1}{y^{j_1}}\dd{y^{j_1}} \wedge \cdots \wedge \pdv{x^n}{y^{j_n}}\dd{y^{j_n}} \\
% 	&=  h(p)\, \epsilon^{j_1 \dots j_n} \pdv{x^1}{y^{j_1}} \cdots \pdv{x^n}{y^{j_n}} \dd{y^1} \wedge \cdots \wedge \dd{y^n} \\
% 	&= h(p)\, \det \left( \pdv{x^k}{y^{l}} \right) \dd{y^1} \wedge \cdots \wedge \dd{y^n}
% \end{align}
% である.よって座標変換としてJacobianが正のものだけを考えれば, $\displaystyle \int \omega$ は既知の重積分の変数変換公式と整合的である.ここで\underline{$M$ が向き付け可能であると言う仮定が効いてくる}のである.

% 以上の考察から,$n$-形式 $\omega$ のチャート $(U_i;\, x^\mu)$ 上の積分を
% \begin{align} 
% 	\int_{U_i} \omega \coloneqq \int_{\varphi(U_i)} h\bigl(\varphi^{-1}_i (x)\bigr) \dd{x^1} \cdots \dd{x^n}
% \end{align}
% として定義できる.積分範囲を $M$ に拡張するには\hyperref[def:PoU]{1の分割}を使う:

% \begin{mydef}[label=def.int-n]{$n$-形式の積分} 
% 	$\omega \in \Omega^n(M)$ は台がコンパクトであるとする.また,$M$ の座標近傍からなる開被覆 $\{U_i\}$ と,それに従属する1の分割 $\{ f_i \}$ をとる.このとき $\omega$ の $M$ 上の積分を次のように定義する:
% 	\begin{align} 
% 		\int_M \omega \coloneqq \sum_{i} \int_{U_i} f_i \omega
% 	\end{align}
% \end{mydef}

% \begin{myprop}[]{} 
% 	定義\ref{def.int-n}は座標近傍の開被覆 $\{U_i\}$ やそれに従属する1の分割 $\{f_i\}$ の取り方によらない.
% \end{myprop}

\subsection{$\mathbb{R}^n,\, \mathbb{H}^n$ の場合}

\begin{mydef}[label=def:domain-of-integration]{積分領域}
    $\mathbb{R}^n,\, \mathbb{H}^n$ の\textbf{積分領域} (domain of integration) とは,$\mathbb{R}^n,\, \mathbb{H}^n$ の有界な部分集合であって境界が零集合であるようなもののこと.
\end{mydef}

\hyperref[def:domain-of-integration]{積分領域} $D \subset \mathbb{R}^n,\, \mathbb{H}^n$ を1つ固定する.
$\forall \omega = f \dd{x^1} \wedge \cdots \wedge \dd{x^n} \in \Omega^n (\overline{D})$ の $D$ 上の積分を
\begin{align}
    \int_D \omega  \coloneqq \int_D \dd{x^1} \cdots \dd{x^n} f
\end{align}
で定義する.

\begin{myprop}[label=prop:integral-pullback]{微分同相写像による引き戻しの積分}
    
    \begin{itemize}
        \item 開\hyperref[def:domain-of-integration]{積分領域} $D,\, E \subset \mathbb{R}^n,\, \mathbb{H}^n$
        \item $C^\infty$ 写像 $F \colon \overline{D} \lto \overline{E}$ であって,$F|_D \colon D \lto E$ が\hyperref[def:orientation-preserving]{向きを保つ or 逆にする}微分同相写像であるもの
    \end{itemize}
    を与える.このとき $\forall \omega \in \Omega^n (\overline{E})$ に対して
    \begin{align}
        \int_D F^* \omega = 
        \begin{cases}
            +\int_E \omega, &F\; \text{が向きを保つ} \\
            -\int_E \omega, &F\; \text{が向きを逆にする}
        \end{cases}
    \end{align}
    が成り立つ.
\end{myprop}

\begin{proof}
    $D,\, E$ の標準的な座標をそれぞれ $(x^\mu),\, (y^\mu)$ と書いて区別する.
    $\omega = f \dd{y^1} \wedge \cdots \wedge \dd{y^n}$ と書くと,補題\ref{lem:pullback}より
    \begin{align}
        F^* \omega 
        &= (f \circ F) \dd{(y^1 \circ F)} \wedge \cdots \wedge \dd{(y^n \circ F)} \\
        &= (f \circ F) (\det DF)\dd{x^1} \wedge \cdots \wedge \dd{x^n}
    \end{align}
    が成り立つ.よって変数変換公式から
    \begin{align}
        \int_E \omega
        &= \int_E \dd{y^1} \cdots \dd{y^n} f \\
        &= \int_D \dd{x^1} \cdots \dd{x^n} \abs{\det DF} (f \circ F) \\
        &= \pm \int_D \dd{x^1} \cdots \dd{x^n} (\det DF) (f \circ F) \\
        &= \pm \int_D (f \circ F) (\det DF) \dd{x^1} \wedge \cdots \wedge \dd{x^n} \\
        &= \pm \int_D F^* \omega
    \end{align}
\end{proof}

一般の開集合 $U \subset \mathbb{R}^n,\, \mathbb{H}^n$ 上の,コンパクト台を持つ $\omega \in \Omega^n(U)$ を積分するには,$\supp \omega \subset D$ を充たす\hyperref[def:domain-of-integration]{積分領域} $D$ に $\omega$ の定義域を拡張して
\begin{align}
    \int_U \omega \coloneqq \int_D \omega
\end{align}
とすれば良い.このとき命題\ref{prop:integral-pullback}は任意の $\mathbb{R}^n,\, \mathbb{H}^n$ の開集合に対して成り立つ\footnote{証明には若干の技術的な注意が必要}.

\subsection{一般の $C^\infty$ 多様体の場合}

\hyperref[def:smooth-orientation]{向き付けられた} $C^\infty$ 多様体 $M$ と,そのチャート $(U,\, \varphi)$ をとる.
コンパクト台を持つ $\omega \in \Omega^n(U)$ を与えたとき,$M$ 上の $\omega$ の積分を
\begin{align}
    \int_M \omega \coloneqq 
    \begin{cases}
        + \int_{\varphi(U)} (\varphi^{-1})^* \omega, & (U,\, \varphi)\; \text{は\hyperref[def:atlas-oriented]{正の向き}} \\
        - \int_{\varphi(U)} (\varphi^{-1})^* \omega, & (U,\, \varphi)\; \text{は負の向き}
    \end{cases}
\end{align}
% ただし,チャート $(U,\, \varphi) = \bigl( U,\, (x^\mu) \bigr) $ が正の向きであると言うのは,座標フレーム $\bigl( \pdv{}{x^1},\, \dots,\, \pdv{}{x^{\dim M}} \bigr)$ が\hyperref[def:smooth-orientation]{正の向き}であることを言う.
この定義がチャートの取り方によらないことを示そう.$(U,\, \varphi),\, (\tilde{U},\, \tilde{\varphi})$ を,$\supp \omega \subset U \cap \tilde{U}$ を充たす $M$ のチャートとする.
どちらのチャートも向きが同じならば,座標変換 $\tilde{\varphi} \circ \varphi^{-1} \colon \varphi(U \cap \tilde{U}) \lto \tilde{\varphi}(U \cap \tilde{U})$ は\hyperref[def:orientation-preserving]{向きを保つ}微分同相写像なので,命題\ref{prop:integral-pullback}から
\begin{align}
    \int_{\tilde{\varphi}(\tilde{U})} (\tilde{\varphi}^{-1})^* \omega
    &= \int_{\tilde{\varphi}(U \cap \tilde{U})} (\varphi^{-1} \circ \varphi \circ \tilde{\varphi}^{-1})^* \omega \\
    &= \int_{\tilde{\varphi}(U \cap \tilde{U})} (\varphi \circ \tilde{\varphi}^{-1})^* \bigl((\varphi^{-1})^*\omega\bigr) \\
    &= \int_{\varphi(U \cap \tilde{U})} (\varphi^{-1})^*\omega \\
    &= \int_{\varphi(U)} (\varphi^{-1})^*\omega
\end{align}
が成り立つ.違う向きならば,座標変換 $\tilde{\varphi} \circ \varphi^{-1} \colon \varphi(U \cap \tilde{U}) \lto \tilde{\varphi}(U \cap \tilde{U})$ が\hyperref[def:orientation-preserving]{向きを逆にする}微分同相写像なので命題\ref{prop:integral-pullback}から $-$ が出て,定義の $-$ を相殺する.

$M$ 上の積分にするには,正 or 負の向きの座標近傍からなる $\supp \omega$ の有限開被覆 $\{U_i\}_{i \in I}$ と,それに\hyperref[def:PoU]{従属する1の分割}$\{\psi_i \colon M \lto [0,\, 1]\}_{i \in I}$ をとってきて
\begin{align}
    \int_M \omega \coloneqq \sum_{i \in I} \int_M \psi_i \omega
\end{align}
と定義すれば良い.
この定義は1の分割の取り方によらない.



\begin{myprop}[label=prop:integral-pullback-general]{微分同相写像による引き戻しの積分}
    \begin{itemize}
        \item 空でない\hyperref[def:smooth-orientation]{向き付けられた} $C^\infty$ 多様体 $M,\, N$
        \item \hyperref[def:orientation-preserving]{向きを保つ or 逆にする} 微分同相写像写像 $F \colon M \lto N$
    \end{itemize}
    を与える.このときコンパクト台を持つ任意の $\omega \in \Omega^n (N)$ に対して
    \begin{align}
        \int_M F^* \omega = 
        \begin{cases}
            +\int_N \omega, &F\; \text{が向きを保つ} \\
            -\int_N \omega, &F\; \text{が向きを逆にする}
        \end{cases}
    \end{align}
    が成り立つ.
\end{myprop}

\begin{proof}
    1の分割を使って貼り合わせれば良いので,単一のチャートについてのみ示す.
    $(U,\, \varphi)$ を正の向きの $N$ のチャートであって $\supp \omega \subset U$ を充たすものとする.このとき $\bigl( F^{-1}(U),\, \varphi \circ F\bigr)$ は,$F$ が向きを保つなら正の向きの $M$ のチャートであって $\supp F^* \omega \subset F^{-1}(U)$ を充たし,$F$ が向きを逆にする場合は負の向きのチャートである.命題\ref{prop:integral-pullback}より示された.
\end{proof}

\section{ベクトル空間に値をとる微分形式}

$k$-形式 $\omega \in \Omega^k(M)$ は $\forall p \in M$ において多重線型写像
\begin{align}
	\omega_p \coloneqq T_pM \times \cdots  \times T_pM \to \mathbb{K}
\end{align}
を対応させ,それが $p$ に関して\cinfty 級につながっているものであった.ここで,値域 $\mathbb{K}$ を一般の $\mathbb{K}$-ベクトル空間 $V$ に置き換えてみる:
\begin{align}
	\omega_p \coloneqq T_pM \times \cdots  \times T_pM \to V
\end{align}
このようなもの全体の集合を $\Omega^k(M;\, V)$ と書くことにする.$V$ の基底を $\{\, \hat{e}_i\, \}_{1\le i \le r}$ とおくと
\begin{align}
	\omega(X_1,\, \cdots ,\, X_k) = \sum_{i=1}^r \omega_i(X_1,\, \cdots ,\, X_k)\, \hat{e}_i, \quad \omega_i \in \Omega^k(M)
\end{align}
と展開できる.$\hat{e}_i$ は $A^\bullet(M)$ の演算と無関係である.

\subsection{外微分}

定義\ref{extdiff_1}, \ref{extdiff_2}による外微分を $\{\, \hat{e}_i\, \}_{1\le i \le r}$ による展開係数に適用するだけである:
\begin{align}
	&\dd{} \colon \Omega^k(M;\, V) \to \Omega^{k+1}(M;\, V),\\
	&\dd{\omega} \coloneqq \sum_{i=1}^r \dd{\omega_i} \hat{e}_i
\end{align}

\subsection{外積}

外積をとった後の値域はテンソル積 $V \otimes W$ である:
\begin{align}
	&\wedge \colon \Omega^k(M;\, V) \times \Omega^l(M;\, W) \to \Omega^{k+l}(M;\, V \otimes W),\\
	&(\omega \wedge \eta)(X_1,\, \dots ,\, X_{k+l}) \coloneqq \frac{1}{k!\, l!}\sum_{\sigma \in \mathfrak{S}_{k+l}} \sgn{\sigma} \omega\bigl(X_{\sigma(1)},\, \dots ,\, X_{\sigma(k)}\bigr) \otimes \eta\bigl(X_{\sigma(k+1)},\, \dots ,\, X_{\sigma(k+l)}\bigr)
\end{align}
予め $\omega = \sum_{i=1}^r \omega_i \hat{e}_i,\; \eta = \sum_{i=1}^s \eta_i \hat{f}_i$ と展開しておくと
\begin{align}
	\omega \wedge \eta = \sum_{i,\, j} \omega_i \wedge \eta_j\, \hat{e}_i \otimes \hat{f}_j
\end{align}
と書ける.外微分は基底 $\{\, \hat{e}_i \otimes \hat{f}_j\, \}$ には作用しないので,命題\ref{extdiff_3}-(2) はそのまま成り立つ:
\begin{align}
	\dd{(\omega \wedge \eta)} = \dd{\omega} \wedge \eta + (-1)^k \omega \wedge \dd{\eta}
\end{align}

\subsection{括弧積}

双線型写像 $\comm{\;}{\;} \colon V \times V \to V$ がLie代数の公理を充してしているとする.このとき
\begin{align}
	&\comm{\;}{\;} \colon \textcolor{red}{\Omega^k(M;\, V) \times \Omega^l(M;\, V)} \xrightarrow{\wedge} \Omega^{k+l}(M;\, V \otimes V) \xrightarrow{\comm{\;}{\;}} \textcolor{red}{\Omega^{k+l}(M;\, V)},\\
	&\comm{\omega}{\eta} \coloneqq \sum_{i,\, j} \omega_i \wedge \eta_j \comm{\hat{e}_i}{\hat{e}_j}
\end{align}
と定義する.命題\ref{extp_1}-(1), \ref{extdiff_3}-(2) から
\begin{align}
	\tcbhighmath[]{\comm{\eta}{\omega}} &= \sum_{i,\, j} \eta_j \wedge \omega_i \comm{\hat{e}_j}{\hat{e}_i} = \sum_{i,\, j} (-1)^{kl} \omega_i \wedge \omega_j \cdot -\comm{\hat{e}_i}{\hat{e}_j} = \tcbhighmath[]{(-1)^{kl+1} \comm{\omega}{\eta}} \\
	\tcbhighmath[]{\dd{\comm{\omega}{\eta}}} &= \sum_{i,\, j} \dd{\bigl( \omega_i \wedge \eta_j \bigr)} \comm{\hat{e}_i}{\hat{e}_j} = \tcbhighmath[]{\comm{\dd{\omega}}{\eta} + (-1)^k \comm{\omega}{\dd{\eta}} }
\end{align}
がわかる.


\end{document}
