\documentclass[geometry_main]{subfiles}

\begin{document}

\setcounter{chapter}{8}

\chapter{ファイバー束}

二つの \cinfty 多様体 $B,\, N$ が与えられたとしよう.$B$ を\textbf{底空間},$F$ を\textbf{ファイバー}と呼ぶことにする.
このとき,大雑把に言うと,\emph{局所的に}積多様体\footnote{位相は積位相(定義\ref{def.prodtopo})を入れるのだった.} $B \times F$ と同一視される\cinfty 多様体 $E$ のことを \emph{$\bm{F}$ をファイバーとする $\bm{B}$ 上のファイバー束}と呼ぶ.もう少し真面目に言うと,$M$ のチャート $(U_i,\, \varphi)$ をとってきたときに積多様体
\begin{align}
	\label{eq.chap9-1}
	U_i \times F
\end{align}
と $E$ の開集合との間に微分同相写像が存在することである.

しかし,これだけだと $E$ の\emph{大域的な}幾何構造が見えてこない.
情報の欠落をなんとかするには $M$ の開被覆 $\{ U_i \}$ に関して局所的な積多様体\eqref{eq.chap9-1}の構造を張り合わせる必要がある.そのために,我々は全ての $U_i \cap U_j \neq \emptyset$ 上において,\textbf{変換関数} $\{\Phi_{ij}\}$ を
\begin{align}
	\Phi_{ij} \colon F|_{U_i} \to F|_{U_j}
\end{align}
として用意する.変換関数の構成の如何によっては,ファイバー束 $E$ の大域的な幾何構造は極めて複雑なものになりうる.

これだけだとよくわからないので,まず手始めに $S^1$ を底空間とするファイバー束を具体的に構成してみよう.
$1$次元実多様体 $S^1$ の \cinfty アトラス $\{(U_+,\, \varphi_+),\; (U_-,\, \varphi_-)\}$ を次のようにとる:
\begin{align}
	U_+ &\coloneqq \bigl\{ e^{\iunit \theta} \bigm| \theta \in (-\varepsilon,\, \pi + \varepsilon) \bigr\}, & \varphi_+ &\colon U_+ \to \mathbb{R},\; e^{\iunit \theta} \mapsto \theta \\
	U_- &\coloneqq \bigl\{ e^{\iunit \theta} \bigm| \theta \in (\pi -\varepsilon,\, 2\pi + \varepsilon) \bigr\}, & \varphi_- &\colon U_- \to \mathbb{R},\; e^{\iunit \theta} \mapsto \theta \\
\end{align}
ファイバー $F$ としては $1$ 次元実多様体
\begin{align}
	F \coloneqq [-1,\, 1] \subset \mathbb{R}
\end{align}
を選ぶ.このときファイバー束 $E$ は積多様体 $U_+ \times F$ および $U_- \times F$ の二部分からなり,それぞれチャート
\begin{align}
	\bigl(\, U_+;\; \theta,\, t_+\, \bigr),\quad \bigl(\, U_-;\; \theta,\, t_-\, \bigr)
\end{align}
を持つ(当然だが $t_{\pm} \in [-1,\, 1]$ である).なお,この時点では $U_+ \times F,\, U_- \times F$ の「つながり方」は未定義である.

ところで,$S^1$ の開被覆 $U_+,\, U_-$ は2ヶ所で重なっている:
\begin{align}
	\varphi_{\pm}(U_+ \cap U_-) = \textcolor{red}{(-\varepsilon,\, 0] \cup [2\pi,\, 2\pi + \varepsilon)} \cup \textcolor{blue}{(\pi-\varepsilon,\, \pi + \varepsilon)}
\end{align}
ここで,変換関数を
\begin{align}
	\Phi_{+-} \colon F|_{U_-} \to F|_{U_+},\; 
	\begin{cases}
		t_+ = t_- & \colon \theta \in \textcolor{red}{(-\varepsilon,\, 0] \cup [2\pi,\, 2\pi + \varepsilon)} \\
		t_+ = t_- &	\colon \theta \in \textcolor{blue}{(\pi-\varepsilon,\, \pi + \varepsilon)}
	\end{cases}
\end{align}
と定義することでファイバー束 $E$ は\emph{円筒}と同相に, 
\begin{align}
	\Phi_{+-} \colon F|_{U_-} \to F|_{U_+},\; 
	\begin{cases}
		t_+ = t_- & \colon \theta \in \textcolor{red}{(-\varepsilon,\, 0] \cup [2\pi,\, 2\pi + \varepsilon)} \\
		t_+ = -t_- & \colon \theta \in \textcolor{blue}{(\pi-\varepsilon,\, \pi + \varepsilon)}
	\end{cases}
\end{align}
と定義することでファイバー束 $E$ は\textbf{M\"obiusの帯}と同相になる.前者は特に $E \approx S^2 \times F$ と言うことだが,このような状況を指してファイバー束 $E$ は\textbf{自明束}であると表現する.

\section{定義の精密化}

ファイバー束のイメージが掴めたところで,数学的に厳密な定義を与える.
まずは変換関数を入れる前の段階までの定式化である:

% \begin{mydef}[label=def.fiber-1]{微分可能ファイバー束}
% 	\cinfty 多様体 $F,\, E,\, B$ を与える.\cinfty 写像 $\pi \colon E \to B$ が与えられ,それが次の条件を充たすとき,組 $(E,\, \pi ,\, B,\, F)$ を $F$ をファイバーとする\textbf{微分可能ファイバー束} (differentiable fiber bundle) と呼ぶ:
% 	\begin{description}
% 		\item[\textbf{(局所自明性)}] 
		
% 		$\forall b \in B$ に対して,$b$ のある開近傍 $U$ と微分同相写像 $\varphi \colon \pi^{-1}(U) \xrightarrow{\simeq} U \times F$ が存在して
% 		\begin{align}
% 			\forall u \in \pi^{-1} (U),\; \pi(u) = \mathrm{proj}_1 \circ \varphi(u)
% 		\end{align}
% 		となる,i.e. 図\ref{fig.fiber1}が可換図式となる.
% 		ただし,$\mathrm{proj}_1$ は第一成分への射影である:
% 		\begin{align}
% 			\mathrm{proj}_1 \colon U \times F \to U ,\; (p,\, f) \mapsto p
% 		\end{align}
% 		微分同相写像 $\varphi$ のことを\textbf{局所自明化} (local trivialization) と呼ぶ.
% 	\end{description}
% \end{mydef}

% \begin{figure}[H]
% 	\centering
% 	\begin{tikzcd}
% 		\pi^{-1}(U) \arrow[d, "\pi"'] \arrow{r}{\varphi} & 
% 			U \times F \arrow{dl}{\mathrm{proj}_1} \\
% 		U &
% 	\end{tikzcd}
% 	\caption{局所自明性}
% 	\label{fig.fiber1}
% \end{figure}%

% \begin{marker}
% 	定義\ref{def.fiber-1}において,$\pi$ を連続写像に,$\varphi$ を位相同型写像に置き換えると一般のファイバー束の定義が得られる.
% 	しかし,以降では微分可能ファイバー束しか考えないので定義\ref{def.fiber-1}の条件を充たす $(E,\, \pi ,\, B,\, F)$ のことを\textbf{ファイバー束}と呼ぶことにする.
% \end{marker}

% ファイバー束 $(E,\, \pi ,\, B,\, F)$ に関して,
% \begin{itemize}
% 	\item $E$ を\textbf{全空間} (total space)
% 	\item $B$ を\textbf{底空間} (base space)
% 	\item $F$ を\textbf{ファイバー} (fiber)
% 	\item $\pi$ を\textbf{射影} (projection)
% \end{itemize}
% と呼ぶ\footnote{紛らわしくないとき,ファイバー束 $(E,\, \pi,\, B,\, F)$ のことを $\pi \colon E \to B$ ,または単に $E$ と略記することがある.}.また,射影 $\pi$ による1点集合 $\{b\}$ の逆像 $\pi^{-1}(\{b\}) \subset E$ のことを\textbf{点} $\bm{b}$ \textbf{のファイバー} (fiber) と呼び,$F|_b$ と書く.\label{def:point-fiber}

% \subsection{ファイバー束の同型}

% \cinfty 多様体 $F$ を共通のファイバーに持つ二つのファイバー束 $(E_i,\, \pi_i,\, B_i,\, F)$ を考える.このとき,二つの底空間 $B_i$ の間の\cinfty 写像と同様に,全空間 $E_i$ の間の微分同相写像を考えることができる.これら二つの\cinfty 写像は\textbf{束写像} (bundle map) と呼ばれる.

% \begin{mydef}[label=def.bundlemap]{束写像}
% 	ファイバー $F$ を共有する二つのファイバー束 $\xi_i = (E_i,\, \pi_i,\, B_i,\, F)$ を与える.
% 	このとき $\xi_1$ から $\xi_2$ への\textbf{束写像} (bundle map) とは,二つの\cinfty 写像 $\textcolor{blue}{f} \colon B_1 \to B_2,\; \textcolor{red}{\tilde{f}} \colon E_1 \to E_2$ であって図\ref{fig.bundlemap}が可換図式になり,
% 	かつ底空間 $B_1$ の各点 $b$ において,\textbf{点} $\bm{b}$ \textbf{のファイバー} $\bm{\pi_1^{-1}(\{b\})} \subset E_1$ への $\textcolor{red}{\tilde{f}}$ の制限 
% 	\begin{align}
% 		\tilde{f}|_{\pi_1^{-1}(\{b\})} \colon \pi_1^{-1}(\{b\}) \to \tilde{f} \bigl( \pi_1^{-1}(\{b\}) \bigr) \subset E_2
% 	\end{align}
% 	が微分同相写像になっているもののことを言う.
% \end{mydef}

% \begin{figure}[H]
% 	\centering
% 	\begin{tikzcd}
% 			E_1 \arrow[d, "\pi_1"'] \arrow[red]{r}[red]{\tilde{f}}
% 				& E_2 \arrow{d}{\pi_2} \\
% 			B_1 \arrow[blue]{r}[blue]{f}
% 			&B_2
% 	\end{tikzcd}
% 	\caption{束写像}
% 	\label{fig.bundlemap}
% \end{figure}%

% \begin{mydef}[label=def.bundle_isomorphism]{ファイバー束の同型}
% 	ファイバー $F$ と底空間 $B$ を共有する二つのファイバー束 $\xi_i = (E_i,\, \pi_i,\, B,\, F)$ を与える.
% 	このとき,ファイバー束 $\xi_1$ と $\xi_2$ が\textbf{同型} (isomorphic) であるとは,
% 	$f \colon B \to B$ が恒等写像となるような束写像 $\tilde{f} \colon E_1 \to E_2$ が存在することを言う.記号として $\xi_1 \simeq \xi_2$ とかく.
% \end{mydef}

% \begin{figure}[H]
% 	\centering
% 	\begin{tikzcd}[column sep=small]
% 			E_1 \arrow[dr, "\pi_1"'] \arrow[red]{rr}[red]{\tilde{f}} &	& E_2 \arrow{dl}{\pi_2} \\
% 			& B &
% 	\end{tikzcd}
% 	\caption{ファイバー束の同型}
% 	\label{fig.bundle_homo}
% \end{figure}%

% 積束 $(B \times F,\, \mathrm{proj}_1,\, B,\, F)$ と同型なファイバー束を\textbf{自明束} (trivial bundle) と呼ぶ.

% \subsection{切断}

% % さて,ファイバー束 $\xi = (E,\, \textcolor{red}{\pi},\, B,\, F)$ が与えられたとき,底空間 $B$ の任意の部分多様体 $M \subset B$ は自然にファイバー束の構造を持つ.それはファイバー $F$ と射影 $\pi$ を共有し,元の全空間 $E$ を部分集合 $\pi^{-1}(M) \subset E$ に制限することで構成される:
% % \begin{align}
% % 	\xi|_M \coloneqq \bigl( \textcolor{red}{\pi^{-1}}(M),\, \textcolor{red}{\pi}|_{\pi^{-1}(M)},\, M,\, F \bigr) 
% % \end{align}
% % $\xi|_M$ のことを $\xi$ の $M$ への\textbf{制限} (restriction) と呼ぶ.

% % \begin{mydef}[label=def.trivial]{自明化}
% % 	ファイバー束 $\xi = (E,\, \pi,\, B,\, F)$ を与える.底空間 $B$ の部分多様体 $M$ に対して,$\xi$ の $M$ への制限が $M$ 上の積束 $M \times F$ と同型である,i.e.
% % 	\begin{align}
% % 		\xi|_M \simeq (M \times F,\, \pi_0,\, M,\, F)
% % 	\end{align}
% % 	であるとき,$\xi_M$ は $\xi$ の $M$ 上の\textbf{自明化} (trivialization) と呼ばれる.
% % \end{mydef}

% ファイバー束 $(E,\, \pi,\, B,\, F)$ は,射影 $\pi$ によってファイバー $F$ の情報を失う.$F$ を復元するためにも,$s \colon B \to E$ なる写像の存在が必要であろう.

% \begin{mydef}[label=def.section]{切断}
% 	ファイバー束 $\xi = (E,\, \pi,\, B,\, F)$ の\textbf{切断} (cross section) とは,\cinfty 写像 $s \colon B \to E$ であって $ \pi \circ s = \mathrm{id}_B$ となるもののことを言う.
% \end{mydef}
% 各点 $b \in B$ に対して,明らかに $s(b) \in \pi^{-1}(\{b\})$ である.

% 切断は\textbf{大域的な}対象であり,与えられたファイバー束が切断を持つとは限らない.一方,各点 $b \in B$ の開近傍 $U$ 上であれば,図\ref{fig.fiber1}の示す局所自明性から\textbf{局所切断} $s \colon U \to \pi^{-1}(U)$ が必ず存在する.
% $\mathrm{proj}_1^{-1}(\{	b \}) = \{b\} \times F$ であることを考慮すると $\pi^{-1}(\{b\}) \simeq F$ とわかるので,局所切断 $s \colon U \to \pi^{-1}(U)$ は \cinfty 写像 $\tilde{s} \colon U \to F$ と一対一に対応する.

% $B$ 上の切断全体の集合を $\Gamma(B,\, E)$ と書くことにする.例えば $\Gamma(B,\, TB) \simeq \vecfield{B}$ である.

% \section{変換関数}

% つぎに,変換関数を定式化しよう.
% $\xi = (E,\, \pi,\, B,\, F)$ をファイバー束とする.底空間 $B$ の開被覆 $\{ U_\lambda\}_{\lambda \in \Lambda}$ をとると,定義\ref{def.fiber-1}から,どの $\alpha \in \Lambda$ に対しても局所自明性(図\ref{subfig.fiber-l})
% が成り立つ.ここでもう一つの $\beta \in \Lambda$ をとり,$U_\alpha \cap U_\beta$ に関して局所自明性の図式を横に並べることで,自明束 $\mathrm{proj}_1 \colon (U_\alpha \cap U_\beta) \times F \to U_\alpha \cap U_\beta$ の自己同型(図\ref{subfig.fiber-lm})が得られる.
% \begin{figure}[H]
% 	\centering
% 	\begin{subfigure}{0.4\columnwidth}
% 		\centering
% 		\begin{tikzcd}
% 			U_\alpha \times F \arrow[dr, "\mathrm{proj}_1"']  & 
% 				\pi^{-1}(U_\alpha) \arrow[l, "\varphi_\alpha"'] \arrow[d, "\pi"] \\
% 			& U_\alpha
% 		\end{tikzcd}
% 		\caption{$U_\alpha$ に関する局所自明性}
% 		\label{subfig.fiber-l}
% 	\end{subfigure}
% 	\hspace{5mm}
% 	\begin{subfigure}{0.4\columnwidth}
% 		\centering
% 		\begin{tikzcd}
% 			\pi^{-1}(U_\beta) \arrow[d, "\pi"'] \arrow{r}{\varphi_\beta} & 
% 			U_\beta \times F \arrow{dl}{\mathrm{proj}_1} \\
% 			U_\beta &
% 		\end{tikzcd}
% 		\caption{$U_\beta$ に関する局所自明性}
% 		\label{subfig.fiber-m}
% 	\end{subfigure}
% 	\vspace{5mm}
% 	\begin{subfigure}{0.4\columnwidth}
% 		\centering
% 		\begin{tikzcd}[column sep=small]
% 			(U_\alpha \cap U_\beta) \times F \arrow[dr, "\mathrm{proj}_1"'] \arrow[red]{rr}[red]{\varphi_\beta \circ \varphi_\alpha^{-1}} & &
% 				(U_\alpha \cap U_\beta) \times F \arrow{dl}{\mathrm{proj}_1}  \\
% 				& U_\alpha \cap U_\beta &
% 		\end{tikzcd}
% 		\caption{自明束 $(U_\alpha \cap U_\beta) \times F$ の自己同型}
% 		\label{subfig.fiber-lm}
% 	\end{subfigure}
% 	\caption{局所自明性の結合}
% 	\label{fig.fiber2}
% \end{figure}

% つまり,$F \to F$ の微分同相写像全体のなす群(\textbf{微分同相群})を $\Diff F$ と書くとき写像
% \begin{align}
% 	t_{\beta\alpha} \colon U_\alpha \cap U_\beta \to \Diff F \label{eq.9-2-1}
% \end{align}
% が存在し,$\forall (b,\, f) \in (U_\alpha \cap U_\beta) \times F$ に対して
% \begin{align}
% 	\bigl(\varphi_\beta \circ \varphi_\alpha^{-1}\bigr)(b,\, f) = \bigl( b,\, t_{\beta\alpha}(b)(f) \bigr) 
% \end{align}
% と作用する\footnote{なお $\varphi_\beta \circ \varphi_\alpha^{-1}$ の作用で点 $b$ が動かないのは,図式\ref{subfig.fiber-lm} が可換図式である,i.e. $\mathrm{proj}_1(b,\, f) = b = \bigl(\mathrm{proj}_1 \circ (\varphi_\beta \circ \varphi^{-1}_\alpha) \bigr)(b,\, f)$ であることによる.}.

% \begin{mydef}[label=def.fiber_transform]{変換関数}
% 	上の設定において,式\eqref{eq.9-2-1}の $t_{\alpha\beta}$ をファイバー束 $\xi$ の\textbf{変換関数} (transition function) と呼ぶ.
% \end{mydef}

% 全ての $U_\alpha \cap U_\beta$ に関する変換関数の族 $\{t_{\alpha\beta}\}$ が $\forall b \in U_\alpha \cap U_\beta \cap U_\gamma$ に対して条件
% \begin{align}
% 	\label{eq.cocycle}
% 	t_{\alpha\beta}(b) \circ t_{\beta\gamma}(b) = t_{\alpha\gamma}(b)
% \end{align}
% を充たすことは図式\ref{fig.fiber2}より明かである.
% 次の命題は,ファイバー束 $(E,\, \pi,\, B,\, F)$ を構成する「素材」には
% \begin{itemize}
% 	\item 底空間となる\cinfty 多様体 $B$
% 	\item ファイバーとなる\cinfty 多様体 $F$
% 	\item $B$ の開被覆 $\{ U_\lambda \}$
% 	\item \eqref{eq.cocycle}を充たす\cinfty 関数族 $\{t_{\alpha\beta} \colon U_\beta \cap U_\alpha \to \Diff F\}$
% \end{itemize}
% があれば十分であることを主張する:

% \begin{myprop}[label=prop.cocycle]{ファイバー束の復元}
% 	任意の \cinfty 多様体 $B,\, F$ を与える.

% 	$B$ の開被覆 $\{U_\lambda\}$ と,\textbf{コサイクル条件}\eqref{eq.cocycle} (cocycle condition) を充たす\cinfty 級関数の族 $\{ t_{\alpha\beta} \colon U_\beta \cap U_\alpha \mapsto \Diff F \}$ が与えられたとき,ファイバー束 $\xi = (E,\, \pi,\, B,\, F)$ であって,その変換関数が $\{t_{\alpha\beta}\}$ となるものが存在する.
% \end{myprop}
% \begin{proof}
% 	まず手始めに,cocycle条件\eqref{eq.cocycle}より
% 	\begin{align}
% 		t_{\alpha\alpha}(b) \circ t_{\alpha\alpha} (b ) = t_{\alpha\alpha}(b),\quad \forall b \in U_\alpha
% 	\end{align}
% 	だから $t_{\alpha\alpha}(b) = \mathrm{id}_{F}$ であり,また
% 	\begin{align}
% 		t_{\alpha\beta}(b) \circ t_{\beta\alpha} (b) = t_{\alpha\alpha}(b) = \mathrm{id}_{F},\quad \forall b \in U_\alpha \cap U_\beta
% 	\end{align}
% 	だから $t_{\beta\alpha}(b) = t_{\alpha\beta}(b)^{-1}$ である.

% 	開被覆 $\{U_\lambda\}$ の添字集合を $\Lambda$ とする.このとき $\forall \lambda \in \Lambda$ に対して,$U_\lambda \subset B$ には底空間 $B$ からの\hyperref[def.reltopo]{相対位相}を入れ,$U_\lambda \times F$ にはそれと $F$ の位相との\hyperref[def.prodtopo]{積位相}を入れることで,\hyperref[def.disjoint_topo]{直和位相空間}
% 	\begin{align}
% 	\mathcal{E} \coloneqq \coprod_{\lambda \in \Lambda} U_\lambda \times F
% 	\end{align}
% 	を作ることができる\footnote{$\mathcal{E}$ はいわば,「貼り合わせる前の互いにバラバラな素材(局所自明束 $U_\alpha \times F$)」である.証明の以降の部分では,これらの「素材」を $U_\alpha \cap U_\beta \neq \emptyset$ の部分に関して「良い性質\eqref{eq.cocycle}を持った接着剤 $\{ t_{\alpha\beta} \}$」を用いて「貼り合わせる」操作を,位相を気にしながら行う.}.
% 	$\mathcal{E}$ の任意の元は $(\textcolor{red}{\lambda},\, b,\, f) \in  \textcolor{red}{\Lambda} \times  U_\lambda \times F$ と書かれる.

% 	さて,$\mathcal{E}$ 上の二項関係 $\sim$ を以下のように定める:
% 	\begin{align}
% 		\label{eq.prop9-1_equiv}
% 		\sim\; \coloneqq \Bigl\{ \bigl( \, (\alpha,\, b,\, f),\, (\beta,\, c,\, h)\, \bigr) \in \mathcal{E} \times \mathcal{E} \Bigm| b=c,\; f = t_{\alpha\beta}(b)(h) \Bigr\} 
% 	\end{align}
% 	$\sim$ が同値関係の公理\ref{ax.equiv}を充たすことを確認する:
% 	\begin{description}
% 		\item[\textbf{反射律}] 冒頭の議論から $t_{\alpha\alpha}(b) = \mathrm{id}_F$ なので良い.
% 		\item[\textbf{対称律}] 冒頭の議論から $t_{\beta\alpha}(b) = t_{\alpha\beta}(b)^{-1}$  なので,
% 		\begin{align}
% 			(\alpha,\, b,\, f) \sim (\beta,\, c,\, h) \quad &\Longrightarrow \quad b=c,\; f = t_{\alpha\beta}(b)(h) \\
% 			&\Longrightarrow \quad c=b,\; h = t_{\alpha\beta}(b)^{-1}(f) = t_{\beta\alpha}(b)(f) \\
% 			&\Longrightarrow \quad (\beta,\, c,\, h) \sim (\alpha,\, b,\, f).
% 		\end{align}
% 		\item[\textbf{推移律}] cocycle条件\eqref{eq.cocycle}より
% 		\begin{align}
% 			(\alpha,\, b,\, f) \sim (\beta,\, c,\, h),\; (\beta,\, c,\, h) \sim (\gamma,\, d,\, k) \quad
% 			&\Longrightarrow \quad b=c,\, c=d,\; f = t_{\alpha\beta}(b)(h),\, h = t_{\beta\gamma}(c)(k) \\
% 			&\Longrightarrow \quad b=d,\; f = t_{\alpha\beta}(b) \circ t_{\beta\gamma}(b)(k) = t_{\alpha\gamma}(b)(k)  \\
% 			&\Longrightarrow \quad (\alpha,\, b,\, f) \sim (\gamma,\, d,\, k).
% 		\end{align}
% 	\end{description}
% 	したがって $\sim$ は同値関係である.
% 	$\sim$ による $\mathcal{E}$ の商集合を $E$ と書き,\hyperref[def.quo-proj]{標準射影} (canonical injection) を $\mathrm{pr} \colon \mathcal{E} \to E,\; (\alpha,\, b,\, f)  \mapsto [ (\alpha,\, b,\, f)]$ と書くことにする.

% 	集合 $E$ に\hyperref[def.quotopo]{商位相}を入れて $E$ を位相空間にする.このとき開集合 $\{\alpha\} \times U_\alpha \times F \subset \mathcal{E}$ は $\mathrm{pr}$ によって $E$ の開集合 $\mathrm{pr}(\{\alpha\} \times U_\alpha \times F) \subset{E}$ に移される.ゆえに $E$ は $\bigl\{\, \mathrm{pr}(\{\alpha\} \times U_\alpha \times V_\beta)\, \bigr\}$ を座標近傍にもつ\cinfty 多様体である(ここに $\{ V_\beta \}$ は,\cinfty 多様体 $F$ の座標近傍である).
	
% 	次に\cinfty 写像 $\pi \colon E \to B$ を
% 	\begin{align}
% 		\pi \bigl(\, [(\alpha,\, b,\, f)]\, \bigr) \coloneqq b
% 	\end{align}
% 	と定義すると,これは\hyperref[def.fiber-1]{局所自明化} 
% 	\begin{align}
% 		\varphi_\alpha \colon \pi^{-1}(U_\alpha) \to U_\alpha \times F,\; [(\alpha,\, b,\, f)] \mapsto (b,\, f)
% 	\end{align}
% 	による\hyperref[fig.bundle_homo]{局所自明性}を持つ.
% 	従って組 $\xi = (E,\, \pi,\, B,\, F)$ はファイバー束になり,証明が終わる.
% \end{proof}


% \subsection{構造群}

% 以上の議論から,任意の $F$ をファイバーとするファイバー束が,底空間 $B$ の開被覆 $\{U_\lambda\}$ に対して局所的な自明束 $U_\lambda \times F$ を変換関数 $\{ t_{\alpha\beta}\}$ によって「張り合わせる」ことで構成されることがわかった.しかし,式\eqref{eq.9-2-1}の変換関数の値域として選んだ $\Diff F$ は集合として大きすぎて扱いが難しい.
% そこで,微分同相群 $\Diff F$ の代わりにその部分群 $G \subset \Diff F$ を使うと言う発想に至る.
% 特に $G$ としてLie変換群\footnote{つまり,構造群 $G$ はファイバー $F$ に左から作用する.}を選ぶことが多い,これが\textbf{構造群}である.

% \begin{mydef}[label=def.structure_group]{構造群}
% 	ファイバー束 $\xi = (E,\, \pi,\, B,\, F)$ を与える.底空間 $B$ の開被覆 $\{U_\lambda\}_{\lambda \in \Lambda}$ と,各 $U_\lambda$ に関する\hyperref[def.fiber-1]{局所自明化} $\varphi_\lambda \colon \pi^{-1}(U_\lambda) \to U_\lambda \times F$ が与えられたとする.
% 	このとき,全ての添字の組 $\forall (\alpha,\, \beta) \in \Lambda \times \Lambda$ に対して,
% 	変換関数 $t_{\alpha\beta} \colon U_\alpha \cap U_\beta \to \Diff F$ と $F$ 上のあるLie変換群 $G \subset \Diff F$ が
% 	\begin{align}
% 		\Im t_{\alpha\beta} \subset G
% 	\end{align}
% 	を充し,かつ $t_{\alpha\beta}$ 自身が\cinfty 級ならば,\textbf{開被覆と局所自明化の組} $\bm{\{U_\lambda\} \times \{\varphi_\lambda\}}$ はファイバー束 $\xi$ に $G$ を\textbf{構造群} (structure group) とするファイバー束の構造を定めると言う.

% 	構造群 $G$ が指定されたファイバー束のことを記号として $(E,\, \pi,\, B,\, F,\, G)$ と書く.
% \end{mydef}
% 構造群 $G$ を指定する開被覆 $\{U_\lambda\}$ およびその上の\hyperref[def.fiber-1]{局所自明化} $\{	\varphi_\lambda \}$ を明記するときは\textbf{座標束}と呼び,記号として $\bm{\bigl( E,\, \pi ,\, B,\, F,\, G,\, \{ \varphi_\lambda\},\, \{U_\lambda\} \bigr)}$ と書く.
% 座標束は\hyperref[def.atlas]{多様体のアトラス}と類似の概念である.

% \subsection{構造の類別}

% 座標束 $(E,\, \pi,\, B,\, F,\, G,\, \{\varphi_\lambda\},\, \{U_\lambda\})$ を与える.
% ここで,底空間 $B$ の開集合 $U$ 上に別の局所自明化 $\varphi \colon \pi^{-1} (U) \to U \times F$ が与えられたとしよう.
% 開被覆の添字集合を $\Lambda$ とするとき,$\forall \alpha \in \Lambda$ に対して自明束 $(U \cap U_\alpha) \times F$ の自己\hyperref[def.bundle_isomorphism]{同型}
% \begin{figure}[H]
% 	\centering
% 	\begin{tikzcd}[column sep=small]
% 		(U_\alpha \cap U) \times F \arrow[dr, "\mathrm{proj}_1"'] \arrow[red]{rr}[red]{\varphi \circ \varphi_\alpha^{-1}} & &
% 			(U_\alpha \cap U) \times F \arrow{dl}{\mathrm{proj}_1}  \\
% 			& U_\alpha \cap U &
% 	\end{tikzcd}
% \end{figure}%
% を考えることができる.このとき,ある写像 $t_{\alpha} \colon U_\alpha \cap U \to \Diff F$ が存在して
% \begin{align}
% 	(\varphi \circ \varphi^{-1}_\alpha) (b,\, f) = \bigl(\, b,\, t_{\alpha}(b)(f)\, \bigr)
% \end{align}
% と書けるが,$\bm{\mathrm{Im}}\, \bm{t_\alpha \subset G}$ \emph{とは限らない}!

% \begin{mydef}[label=def.addmissible]{許容}
% 	上述の設定において,\hyperref[def.fiber-1]{局所自明化} $\varphi$ が座標束 $\bigl(E,\, \pi,\, B,\, F,\, G,\, \{\varphi_\lambda\},\, \{U_\lambda\}\bigr)$ の\textbf{許容される} (admissible) \textbf{局所自明化}
% 	であるとは,$\forall \alpha \in \Lambda$ に対して $\Im g_\alpha \subset G$ かつ $g_\alpha$ が\cinfty 級であることを言う.
% \end{mydef}

% 全空間 $E$,射影 $\pi$,底空間 $B$,ファイバー $F$ を持ち,$G$ を構造群とする座標束全体の集合を $\mathscr{F}(E,\, B,\, F,\, G)$ と書こう.
% $(E,\, \pi,\, B,\, F,\, G,\, \{\varphi_\lambda\},\, \{U_\lambda\}) \in \mathscr{F}(E,\, B,\, F,\, G)$ のことを $(\{U_\lambda\},\, \{\varphi_\lambda\})$ と略記する.

% \begin{mydef}[label=def.structure_equiv]{座標束の同値関係}
% 	$\mathscr{F}(E,\, B,\, F,\, G)$ 上の同値関係 $\sim$ を以下のように定める:
% 	\begin{align}
% 		\sim \; \coloneqq \Bigl\{ \bigl(\, (\{U_\lambda\},\, \{\varphi_\lambda\}),\, (\{V_\mu\},\, \{\psi_\mu\})\, \bigr)   \Bigm|
% 		\psi_\mu\;(\forall \mu)\; \text{は座標束}\; (\{U_\lambda\},\, \{\varphi_\lambda\})\;\text{に許容される} \Bigr\} 
% 	\end{align}
% \end{mydef}

% 同値関係\ref{def.structure_equiv}は
% \begin{align}
% 	\sim \; = \Bigl\{ \bigl(\, (\{U_\lambda\},\, \{\varphi_\lambda\}),\, (\{V_\mu\},\, \{\psi_\mu\})\, \bigr)  \Bigm| (\{U_\lambda\} \cup \{V_\mu\},\, \{\varphi_\lambda\} \cup \{\psi_\mu\}) \in \mathscr{F}(E,\, B,\, F,\, G)  \Bigr\} 
% \end{align}
% とも書けて,アトラスの同値関係\ref{manieq}と似ている.

% \begin{mydef}[label=def.G-bundle]{$G$-束}
% 	同値関係\ref{def.structure_equiv}による同値類を $\bm{G}$\emph{-束} ($G$-bundle) と呼び,$\bm{(E,\, \pi,\, B,\, F,\, G)}$ と書く.
% \end{mydef}

% \section{$G$-束}

% ほとんどファイバー束と同じ扱いである.
% \begin{mydef}[label=def.G-bundlemap]{$G$-束の束写像}
% 	ファイバー $F$ を共有する二つの $G$-束 $\xi_i = (E_i,\, \pi_i,\, B_i,\, F,\, G)$ を与える.
% 	このとき $\xi_1$ から $\xi_2$ への\textbf{束写像} (bundle map) とは,ファイバー束の束写像(図式\ref{fig.bundlemap})であって,以下の条件を充たすもののことを言う:
% 	\tcblower
% 	$\xi_1,\, \xi_2$ の任意の\hyperref[def.addmissible]{許容される\hyperref[def.fiber-1]{局所自明化}} $\varphi \colon \pi_1^{-1}(U) \to U \times F,\; \psi \colon \pi_2^{-1}(V) \to V \times F $ に対して,自明束の束写像 $\psi \circ \tilde{f} \circ \varphi^{-1} \colon (U \cap f^{-1}(V)) \times F \to V \times F$ (図式\ref{fig.G-bundlemap}の外周部)がある連続写像 $h \colon U \cap f^{-1}(V) \to \Diff F$ を用いて
% 	\begin{align}
% 		(\psi \circ \tilde{f})(b,\, f) = \bigl( \, f(b),\, h(b)(f)\, \bigr) 
% 	\end{align}
% 	と書かれるとき,$\Im h \subset G$ かつ $h$ が\cinfty 級である.
% \end{mydef}

% \begin{figure}[H]
% 	\centering
% 	\begin{tikzcd}
% 			&U \times F \arrow[dr, "\mathrm{proj}_1"']
% 				&\pi_1^{-1}(U) \arrow[l, "\varphi"'] \arrow[d, "\pi_1"'] \arrow[red]{r}[red]{\tilde{f}}
% 				& \pi_2^{-1}(V) \arrow{d}{\pi_2} \arrow[r, "\psi"]
% 			&V \times F \arrow[dl, "\mathrm{proj}_1"] \\
% 			& &U \arrow[blue]{r}[blue]{f} &V &
% 	\end{tikzcd}
% 	\caption{$G$-束の束写像}
% 	\label{fig.G-bundlemap}
% \end{figure}%

% \begin{mydef}[label=def.Gbundle_isomorphism]{$G$-束の同型}
% 	ファイバー $F$ と底空間 $B$ を共有する二つの$G$-束 $\xi_i = (E_i,\, \pi_i,\, B,\, F,\, G)$ を与える.
% 	このとき,$G$-束 $\xi_1$ と $\xi_2$ が\textbf{同型} (isomorphic) であるとは,
% 	$f \colon B \to B$ が恒等写像となるような $G$-束の束写像 $\tilde{f} \colon E_1 \to E_2$ が存在することを言う.記号として $\xi_1 \simeq \xi_2$ とかく.
% \end{mydef}

% % \begin{figure}[H]
% % 	\centering
% % 	\begin{tikzcd}[column sep=small]
% % 			E_1 \arrow[dr, "\pi_1"'] \arrow[red]{rr}[red]{\tilde{f}} &	& E_2 \arrow{dl}{\pi_2} \\
% % 			& B &
% % 	\end{tikzcd}
% % 	\caption{$G$-束の同型}
% % 	\label{fig.G-bundle_homo}
% % \end{figure}%

% \begin{mydef}[label=reduce]{縮小}
% 	$G$-束 $\xi = (E,\, \pi,\, B,\, F,\, G)$ を与える.
% 	$H \subset G$ を $G$ の部分群とするとき,ある $\xi$ の座標束 $\bigl(E,\, \pi,\, B,\, F,\, G,\, \{U_\lambda\},\, \{\varphi_\lambda\}\bigr)$ 上の変換関数の族 $\{ t_{\alpha\beta} \colon U_\alpha \cap U_\beta \to G\}$ の像が $\Im t_{\alpha\beta} \subset H$ を充たすとき,$\bm{\xi}$ \emph{の構造群が} $\bm{H}$ \emph{に縮小} (reduce) すると言う.
% \end{mydef}

% 命題\ref{prop.cocycle}と全く同様にして以下が示される:

% \begin{myprop}[label=prop.G-cocycle]{$G$-束の復元}
% 	任意の \cinfty 多様体 $B,\, F$ を与える.

% 	$B$ の開被覆 $\{U_\lambda\}$ と,\textbf{コサイクル条件}\eqref{eq.cocycle} (cocycle condition) を充たす\cinfty 級関数の族 $\{ t_{\alpha\beta} \colon U_\beta \cap U_\alpha \mapsto G \}$ が与えられたとき,$G$-束 $\xi = (E,\, \pi,\, B,\, F,\, G)$ であって,その変換関数が $\{t_{\alpha\beta}\}$ となるものが存在する.
% \end{myprop}

% \subsection{同伴束}

% 命題\ref{prop.G-cocycle}より,変換関数 $t_{\alpha\beta}$ はファイバー $F$ の情報を何も持っていない.
% したがってLie群 $G$ が別の\cinfty 多様体 $F'$ にLie変換群として作用するならば,同じ変換関数だが異なるファイバーを持つ $G$-束 $\xi' = (E,\, \pi,\, B,\, F',\, G)$ を構成できる.

% \begin{mydef}[label=associatedbundle]{同伴束}
% 	上記の設定のとき,$\xi$ と $\xi'$ は互いに他の\textbf{同伴束} (associated bundle) であると言う.
% \end{mydef}

% \subsection{誘導束}

% $G$-束 $\xi = (E,\, \pi,\, B,\, F,\, G)$ を与え,$\xi$ の代表元となる座標束 $(E,\, \pi,\, B,\, F,\, G,\, \{U_\lambda\},\, \{\varphi_\lambda\})$ および変換関数 $t_{\alpha\beta} \colon U_\alpha \cap U_\beta \to G$ をとる.

% ここで新しい\cinfty 多様体 $M$ を導入し,底空間 $B$ との間に \cinfty 写像 $f\colon M \to B$ が与えられたとする.命題\ref{prop.G-cocycle}を用いて $M$ を底空間とする $G$-束(座標束)を構成できる.

% \subsubsection*{$M$ の開被覆}

% まず,$M$ の開被覆を構成しよう.
% $f$ は連続写像だから(定義\ref{def.cinfty_mapping})開集合 $U_\alpha \subset B$ の逆像 $f^{-1}(U_\alpha) \subset M$ は開集合である.$f^{-1}(\bigcup_\lambda U_\lambda) = \bigcup_\lambda f^{-1}(U_\lambda)$ なので,$\{f^{-1}(U_\lambda)\}$ が $M$ の開被覆であるとわかる.

% \subsubsection*{$M$ の変換関数}

% 次に,変換関数 $t^*_{\alpha\beta} \colon f^{-1}(U_\alpha) \cap f^{-1}(U_\beta) \to G$ を構成しよう.
% 試しに
% \begin{align}
% 	t^*_{\alpha\beta} \coloneqq t_{\alpha\beta} \circ f
% \end{align}
% とおいてみると,$t^*_{\alpha\beta}$ は明らかに\cinfty 級である.また,$\forall p \in f^{-1}(U_\alpha) \cap f^{-1}(U_\beta) \cap f^{-1}(U_\gamma)$ に対して
% \begin{align}
% 	t^*_{\alpha\gamma}(p) = t_{\alpha\gamma}(f(p)) = t_{\alpha\beta}(f(p)) \circ t_{\beta\gamma}(f(p)) = t^*_{\alpha\beta}(p) \circ t^*_{\beta\gamma}(p)
% \end{align}
% なのでcocycle条件\eqref{eq.cocycle}を充たす.

% \begin{tcolorbox}
% 	以上の考察と命題\ref{prop.G-cocycle}から,$\{t_{\alpha\beta} \circ f\}$ を変換関数とする $M$ 上の $G$-束が存在するとわかる.
% 	これを\textbf{誘導束} (induced bundle) と呼び,$f^*(\xi)$ と書く.
% \end{tcolorbox}

% 具体的には,
% \begin{align}
% 	f^*E \coloneqq \bigl\{ (p,\, u) \in M \times E \bigm| f(p) = \pi(u) \bigr\} 
% \end{align}
% とおけば $G$-束 $f^*(\xi) \coloneqq \bigl( f^*E,\, \mathrm{proj}_1,\, M,\, F,\, G \bigr)$ が誘導束になる.このとき $G$-束の束写像(定義\ref{def.G-bundlemap})は $\mathrm{proj}_2 \colon (p,\, u) \mapsto u$ である:
% \begin{figure}[H]
% 	\centering
% 	\begin{tikzcd}
% 				&f^*E \arrow[d, "\mathrm{proj}_1"'] \arrow[red]{r}[red]{\mathrm{proj}_2}
% 				& E \arrow{d}{\pi} \\
% 			&M \arrow[blue]{r}[blue]{f} &B
% 	\end{tikzcd}
% 	\caption{誘導束の束写像}
% 	\label{fig.induced}
% \end{figure}%

% $\varphi \colon \pi^{-1}(U) \to U \times F$ を $\xi$ の\hyperref[def.fiber-1]{局所自明化}とすると,$f^*(\xi)$ の\hyperref[def.fiber-1]{局所自明化}は
% \begin{align}
% 	\tilde{\varphi} \colon \mathrm{proj}^{-1}\bigl( f^{-1}(U) \bigr) \to f^{-1}(U) \times F,\; \bigl( p,\, u \bigr) \mapsto \bigl(p,\, \varphi(u)\bigr)
% \end{align}
% となる:
% \begin{figure}[H]
% 	\centering
% 	\begin{tikzcd}[row sep=large]
% 			&f^{-1}(U) \times F \arrow[dr, "\mathrm{proj}_1"']
% 				&\mathrm{proj}_1^{-1}\bigl( f^{-1}(U) \bigr) \arrow[l, "\tilde{\varphi}"'] \arrow[d, "\mathrm{proj}_1"'] \arrow[red]{r}[red]{\mathrm{proj}_2}
% 				& \pi^{-1}(U) \arrow{d}{\pi} \arrow[r, "\varphi"]
% 			&U \times F \arrow[dl, "\mathrm{proj}_1"] \\
% 			& &f^{-1}(U) \arrow[blue]{r}[blue]{f} &U &
% 	\end{tikzcd}
% 	\caption{$f^*(\xi)$ の局所自明化}
% 	\label{fig.induced_local}
% \end{figure}%

% \section{主束}

% \begin{mydef}[label=def.PFD]{主束}
% 	$G$ をLie群とする.$G$-束 $\xi = (P,\, \pi,\, M,\, G,\, G)$ は,$G$ の $G$ 自身への作用が自然な\textbf{左作用}であるとき,\textbf{主束} (principal bundle) あるいは\textbf{主 $\bm{G}$-束} (principal $G$-bundle) と呼ばれる.
% \end{mydef}

% 主束 $(P,\, \pi,\, M,\, G,\, G)$ は $(P,\, \pi,\, M,\, G)$ とか $P(M,\, G)$ と書かれることもある.

% \begin{myprop}[label=prop.PFD_right]{全空間への右作用}
% 	$\xi = (P,\, \pi,\, M,\, G)$ を主 $G$-束とする.このとき,$G$ の全空間 $P$ への右作用が自然に定義される.
% 	この作用は任意のファイバーをそれ自身の上にうつす\textbf{推移的かつ自由な作用}(定義\ref{def.orbit})であり,その商空間は底空間 $M$ と一致する.
% \end{myprop}

% \begin{proof}
% 	まず座標束 $(P,\, \pi,\, M,\, G,\, \{U_\lambda \},\, \{\varphi_\lambda \})$ をとり,関数の族
% 	$\{t_{\alpha\beta} \colon U_\alpha \cap U_\beta \to G\}$ を座標束に対応する変換関数族とする.

% 	$\forall u \in P,\, \forall g \in G$ をとる.$\pi(u) \in U_\alpha$ となる $\alpha$ を選び,対応する\hyperref[def.fiber-1]{局所自明化}が $\varphi_\alpha (u) = (p,\, h) \in U_\alpha \times G$ であるとする.このとき写像 $\phi \colon P \times G \to P$ を次のように定義する\footnote{$G$ の $G$ 自身への右作用は,$G$ の右からの積演算を選ぶ.この作用は推移的かつ効果的である.}:
% 	\begin{align}
% 		\label{eq.prop.9-3-1}
% 		\phi(u,\, \textcolor{red}{g}) \coloneqq \varphi_\alpha^{-1}(p,\, h \cdot \textcolor{red}{g})
% 	\end{align}
	
% 	\begin{description}
% 		\item[\textbf{$\bm{\phi}$ のwell-definedness}] 
		
% 			$\beta \neq \alpha$ に対しても $\pi(u) \in U_\beta$ であるとする.このとき $\varphi_\beta(u) = (p,\, h') \in (U_\alpha \cap U_\beta) \times G$ と書けて,また変換関数の定義から
% 			\begin{align}
% 				h' = t_{\alpha\beta}(p) \cdot h \quad \bigl(\, t_{\alpha\beta}(p) \in G\, \bigr)
% 			\end{align}
% 			である.したがって
% 			\begin{align}
% 				\varphi_\beta^{-1}(p,\, h' \cdot g) = \varphi_\beta^{-1}\bigl(p,\, t_{\alpha\beta}(p) \cdot h \cdot g\bigr) = \varphi_\beta^{-1}\bigl(p,\, t_{\alpha\beta}(p) \cdot (h \cdot g) \bigr) = \varphi_\beta^{-1} \circ (\varphi_\beta \circ \varphi_\alpha^{-1})(p,\, h \cdot g) = \varphi_\alpha^{-1}(p,\, h \cdot g)
% 			\end{align}
% 			が分かり,式\eqref{eq.prop.9-3-1}の右辺は座標束の取り方によらない.
% 		\item[\textbf{$\bm{\phi}$ は右作用}] 
		
% 			定義\ref{def.group_action}の2条件を充たしていることを確認する.
% 			\begin{enumerate}
% 				\item $\phi(u,\, 1_G) = \varphi_\alpha^{-1}(p,\, h \cdot 1_G) = \varphi_\alpha^{-1}(p,\, h) = u$ よりよい.
% 				\item $\forall g_1,\, g_2 \in G$ をとる.
% 				\begin{align}
% 					\phi(u,\, g_1g_2) = \varphi_\alpha^{-1}\bigl(p,\, (h \cdot g_1) \cdot g_2 \bigr) = \phi\bigl(\varphi_\alpha^{-1}(p,\, h \cdot g_1),\, g_2\bigr) = \phi \bigl( \phi(u,\, g_1),\, g_2 \bigr) 
% 				\end{align} 
% 				よりよい.
% 			\end{enumerate}
% 		\item[\textbf{$\bm{\phi}$ は推移的かつ自由}] 
		
% 			$G$ の $G$ 自身への右作用が推移的なので $\phi \colon \pi^{-1}(p) \times G \to \pi^{-1}(p)$ も明らかに推移的.
% 			また $\forall u \in \pi^{-1}(p)$ に対して,$\phi(u,\, g) = u$ ならば
% 			\begin{align}
% 				\phi(u,\, g) = \varphi_\alpha^{-1}(p,\, h \cdot g) = u = \varphi_\alpha^{-1}(p,\, h \cdot 1_G)
% 			\end{align}
% 			であり,$\varphi_\alpha$ が全単射であることから $g = 1_G$ である.i.e. 安定化群は $\forall u \in \pi^{-1}(p)$ に対して自明であるからこの作用は自由である.
% 	\end{description}
% \end{proof}

% \begin{myprop}[]{}
% 	主 $G$-束 $\xi = (P,\, \pi,\, M,\, G)$ が自明束になるための必要十分条件は,それが切断(定義\ref{def.section})を持つことである.
% \end{myprop}

% \begin{proof}
% 	\textbf{($\bm{\Longrightarrow}$)} $\xi$ が自明束ならば切断をもつことは明らか.

% 	\textbf{($\bm{\Longleftarrow}$)} 切断 $s \colon M \to P$ が存在するとする.命題\ref{prop.PFD_right}より $G$ は $P$ に右から自由に作用する.従って $p \in M$ のファイバー $\pi^{-1}(p)$ 上の任意の2点 $\forall u,\, v \in \pi^{-1}(p)$ に対して,ただ一つの $g \in G$ が存在して $v = u \cdot g$ となる.
	
% 	ここで,写像 $\tilde{f} \colon P \to M \times G$ を次のように定義する:

% 	$u,\, s(\pi(u)) \in \pi^{-1}(\pi(u))$ だから
% 	\begin{align}
% 		\exists! g \in G,\, u = s(\pi(u)) \cdot g
% 	\end{align}
% 	であり,この $g$ を用いて
% 	\begin{align}
% 		\tilde{f}(u) \coloneqq \bigl( \pi(u),\, g \bigr)
% 	\end{align}
% 	とする.この $\tilde{f}$ が下図を可換図式にすることは明らかであり,$(P,\, \pi,\, M,\, G) \cong (M \times G,\, \mathrm{proj}_1,\, M,\, G)$ が示された.
% 	\begin{figure}[H]
% 		\centering
% 		\begin{tikzcd}[column sep=small]
% 				P \arrow[dr, "\pi_1"'] \arrow[red]{rr}[red]{\tilde{f}} &	& M \times G \arrow{dl}{\mathrm{proj}_1} \\
% 				& M &
% 		\end{tikzcd}
% 		\caption{主 $G$-束の同型}
% 	\end{figure}%
% \end{proof}


% \section{ベクトル束}

% \begin{mydef}[label=def.vectorbandle]{ベクトル束}
% 	$M$ を $n$ 次元\cinfty 多様体とする.ファイバー束(定義\ref{def.fiber-1}) $\xi = (E,\, \pi,\, M,\, F)$ が $k$ 次元\textbf{ベクトル束} (vector bundle) であるとは,$F = \mathbb{R}^k$ であり,かつ $M$ の任意の開集合 $U$ および $\forall p \in U$ に対して $U$ 上の\emph{\hyperref[def.fiber-1]{局所自明化}の $\bm{\pi^{-1}(p)}$ への制限が線型同型写像}になっていることを言う.
	
% 	$F = \mathbb{C}^k$ のときは $\xi$ は $k$ 次元\textbf{複素ベクトル束}と呼ばれる.
% \end{mydef}

% $\xi$ は $E \xrightarrow{\pi} M$ と略記されることがある.
% また,$k$ を\textbf{ファイバー次元}と呼び,記号として $\dim E$ と書く.
% $E_p \coloneqq \pi^{-1}(p)$ を\textbf{点 $\bm{p} \in M$ におけるファイバー}と呼ぶことがある.

% $k$ 次元ベクトル束の変換関数は $\mathrm{GL}(k,\, \mathbb{R})$ の元である.

% \begin{marker}
% 	ベクトル束の束写像および同型の概念はファイバー束の場合(定義\ref{def.bundlemap}, \ref{def.bundle_isomorphism})とほぼ同様に定義されるが,束写像 $\tilde{f} \colon E_1 \to E_2$ が\cinfty 写像であるだけでなく,$\forall p \in M$ におけるファイバー $E_1{}_p$ への制限 $\tilde{f}|_{E_1{}_p} \colon E_1{}_p \to E_2{}_{f(p)}$ が\underline{線型同型写像}であるという点が異なる.
% \end{marker}

% \begin{mydef}[label=def.zero_section]{ゼロ切断}
% 	$\xi = (E,\, \pi,\, M)$ の切断(定義\ref{def.section})のうち以下の条件を充たすものを\textbf{ゼロ切断} (zero section) と呼ぶ:
% 	\begin{align}
% 		\forall p \in M,\; s(p) = \vb*{0} \in E_p
% 	\end{align}
% \end{mydef}

% \begin{marker}
% 	ゼロ切断の定義式を充たすように作った写像 $s_0 \colon M \to E$ は明らかに \cinfty 写像で,かつ $\pi\circ s_0 = \mathrm{id}_M$ を充たす.i.e. 任意のベクトル束には零切断が存在する.
% \end{marker}


% \subsection{局所フレーム}

% \begin{mydef}[label=def.VB_frame]{局所フレーム}
% 	$k$ 次元ベクトル束 $E \xrightarrow{\pi} M$ および $M$ の開集合 $U$ を与える.$U$ 上の $E$ の\textbf{局所フレーム} (local frame) とは,順序付けられた $k$ 個の局所切断の組 $\{\, s_i \colon U \to E\, \}_{1\le i \le k}$ であって,$\forall p \in U$ に対して $\{\, s_i(p)\,\} $ が $\pi^{-1}(p)$ の基底を成すもののことを言う.
% \end{mydef}

% \begin{myprop}[label=prop.localtrivialization-frame]{局所自明化と局所フレーム}
% 	$k$ 次元ベクトル束 $E \xrightarrow{\pi} M$ および $M$ の開集合 $U$ を与える.$U$ 上の\hyperref[def.fiber-1]{局所自明化}(図\ref{fig.fiber1})を与えることと局所フレームを与えることは同値である.
% \end{myprop}

% \begin{proof}
% 	$\bm{(\Longrightarrow)}$  \hyperref[def.fiber-1]{局所自明化} $\varphi \colon \pi^{-1}(U) \to U \times \mathbb{R}^k$ が与えられたとする.$\mathbb{R}^k$ の標準基底を $\hat{e}_i$ (第 $i$ 成分のみ $1$ の $k$ 次元数ベクトル)とおくと,$\forall p \in U$ に対して $\varphi|_{E_p} \colon E_p \to \{p\} \times \mathbb{R}^k$ が線型同型写像であることから $s_i (p) \coloneqq \varphi^{-1}(p,\, \hat{e}_i)$ が $E_p$ の基底を成す.

% 	$\bm{(\Longleftarrow)}$  局所フレーム $\{\, s_i \colon U \to E\, \}_{1\le i \le k}$ が与えられたとする.このとき局所フレームの定義から,$\forall p \in U$ および $v_p \in E_p$ に対して
% 	\begin{align}
% 		\exists \,! (c_1,\, \dots ,\, c_k) \in \mathbb{R}^k,\; v_p = \sum_{i=1}^k c_i\, s_i(p)
% 	\end{align}
% 	が成り立つ.したがって,写像 $\varphi \colon \pi^{-1}(U) \to U \times \mathbb{R}^k$ を
% 	\begin{align}
% 		\varphi(p,\, v_p) \coloneqq (p;\; c_1,\, \dots ,\, c_k)
% 	\end{align}
% 	と定義すれば $\varphi$ は\hyperref[def.fiber-1]{局所自明化}になる.
% \end{proof}

% \begin{mycol}[]{}
% 	ベクトル束 $E \xrightarrow{\pi} M$ が自明束になる必要十分条件は,$M$ 上の大域的なフレームが存在することである.
% \end{mycol}

% \begin{mycol}[]{}
% 	ベクトル束 $E \xrightarrow{\pi} M$ が自明束になる必要十分条件は,$\forall p \in M$ において $s(p) \neq \vb*{0}$ となる $s \in \Gamma(E)$ が存在することである.このような $s$ を\textbf{ゼロにならない切断} (non-zero section) と呼ぶ.
% \end{mycol}

% \subsection{切断のなすベクトル空間}

% ベクトル束 $E \xrightarrow{\pi} M$ の切断全体の集合を $\Gamma (E,\, M)$ または $\Gamma(E)$ と書く.

% \begin{mydef}[label=def.section_vec]{$\Gamma(E)$ の演算}
% 	$\Gamma(E)$ 上の和とスカラー倍を次のように定義すると,$\Gamma(E)$ は $\mathbb{K}$-ベクトル空間になる:
	
% 	$\forall s,\, s_1,\, s_2 \in \Gamma(E),\, \forall \lambda \in \mathbb{K}$ に対して
% 	\begin{enumerate}
% 		\item $(s_1 + s_2)(p) \coloneqq s_1(p) + s_2(p),\quad \forall p \in M$
% 		\item $(\lambda s) \coloneqq \lambda s(p),\quad \forall p \in M$
% 	\end{enumerate}
% 	\tcblower
% 	また,$\forall f \in \cinftyf{M}$ に対して
% 	\begin{description}
% 		\item[\textrm{(2')}] $(f s)(p) \coloneqq f(p) s(p),\quad \forall p \in M$
% 	\end{description}
% 	とおけば $\Gamma(E)$ は $\cinftyf{M}$-\hyperref[ax:module]{加群}になる.
% \end{mydef}

% \subsection{ベクトル束の計量}

% \begin{mydef}[label=def.metric-fiber]{ベクトル束のRiemann計量}
% 	ベクトル束 $E \xrightarrow{\pi} M$ 上のRiemann計量は,$\forall p \in M$ における正定値内積 $g_p \colon E_p \times E_p \to \mathbb{R}$ であって,$p$ に関して\cinfty 級であるものをいう.
	
% 	i.e. $U$ を $M$ の開集合とし,$\{\, s_i \colon U \to E\, \}$ を $U$ 上の\hyperref[def.VB_frame]{局所フレーム}とするとき,$U$ 上の関数
% 	\begin{align}
% 		g_p \bigl( s_i(p),\, s_j(p) \bigr) \quad \forall p \in U
% 	\end{align}
% 	が \cinfty 関数となることである.

% 	複素ベクトル束についてはHermite内積として定義する.
% \end{mydef}

% \begin{myprop}[]{}
% 	任意のベクトル束には計量が存在する.
% \end{myprop}

% \section{ベクトル束の構成法}

% \subsection{部分束}

% \begin{mydef}[label=def.restriction]{制限}
% 	ベクトル束 $E \xrightarrow{\pi} M$ を与える.$M$ の任意の部分多様体(定義\ref{def.submani})$N$ に対して
% 	\begin{align}
% 		E|_N \coloneqq \pi^{-1}(N)
% 	\end{align}
% 	とおき,射影 $\pi_N \colon E|_N \to N$ を $\pi_N \coloneqq \pi|_{E|_N}$ によって定義すれば $E|_N \xrightarrow{\pi_N} N$ はベクトル束になる.これを $E$ の $N$ への\textbf{制限} (restriction) と呼ぶ. 
% \end{mydef}

% \begin{mydef}[label=def.subbundle]{部分束}
% 	ベクトル束 $E \xrightarrow{\pi} M$ を与える.ベクトル束 $F \xrightarrow{\pi} M$ が $E$ の\textbf{部分束} (subbundle) であるとは,全空間 $F$ が $E$ の部分多様体であり,$\forall p \in M$ におけるファイバー $F_p$ が $E_p$ の部分ベクトル空間になっていることを言う.
% \end{mydef}

% $N$ を \cinfty 多様体 $M$ の部分多様体とすると,$TN$ は $TM|_N$ の部分束になる.

% \subsection{商束}

% % \begin{mydef}[label=def.quo-linear]{商ベクトル空間}
% % 	$V$ を $\mathbb{K}$-ベクトル空間, $W$ を $V$ の $\mathbb{K}$-部分ベクトル空間とする.$V$ は $+$ に関して可換群なので $W$ は $+$ に関して\hyperref[def.subgroup_normal]{正規部分群}である.
% % 	このとき,加法\hyperref[def.quotient_group]{剰余群}\footnote{演算が $+$ なので,$x \in V$ の剰余類は $x + W = \bigl\{\, y \in V \bigm| -x + y \in W \,\bigr\} $ と書かれる.} $V/W$ の上にスカラー倍を
% % 	\begin{align}
% % 		\lambda (x + W) \coloneqq (\lambda x) + W,\quad \forall \lambda \in \mathbb{K},\, \forall x \in V
% % 	\end{align}
% % 	で定義すると $V/W$ はベクトル空間になる.これを\textbf{商ベクトル空間} (quotient vector space) と呼ぶ.
% % \end{mydef}

% % \begin{proof}
% % 	スカラー倍のwell-definednessを確認する.$\forall y \in x + W$ はある $w \in W$ を用いて $y = x + w$ と書けるから
% % 	\begin{align}
% % 		\lambda y = \lambda x + \lambda w \in (\lambda x) + W
% % 	\end{align}
% % 	であり,剰余類 $x + W \in V/W$ の代表元の取り方によらない.
% % \end{proof}


% \begin{mydef}[label=def.quo-VB]{商束}
% 	ベクトル束 $E \xrightarrow{\pi} M$ とその部分束 $F \xrightarrow{\pi} M$ が与えられたとする.
% 	このとき $\forall p \in M$ において\hyperref[prop:quotient-vec]{商ベクトル空間} $E_p / F_p$ を考え,
% 	\begin{align}
% 		E/F \coloneqq \bigcup_{p \in M} E_p/F_p
% 	\end{align}
% 	とおくと,自然な射影 $\pi \colon E/F \to M,\; x_p + F_p \mapsto p$ はベクトル束を成す.この $E/F \xrightarrow{\pi} M$ を $E$ の $F$ による\textbf{商束} (quotient buncle) と呼ぶ.
% \end{mydef}

% \begin{proof}
	
% \end{proof}

% $\dim E = n,\, \dim F = m$ とおくと $\dim E/F = n-m$ である.

% \begin{mydef}[label=def.normal-bundle]{法束}
% 	$N$ を $M$ の\hyperref[def.submani]{\cinfty 部分多様体}とする.このとき接束 $TN$ は $TM|_N$ の部分束であるから,商束 $TM|_N / TN$ を定義できる.これを $N$ の $M$ における\textbf{法束} (normal bundle) と呼ぶ.
% \end{mydef}

% \cinfty 多様体 $M$ にRiemann計量を入れると,部分ベクトル空間 $T_pN \subset T_pM$ の直交法空間 $(T_pN)^{\bot}$ が定義できる.このとき
% \begin{align}
% 	\bigcup_{p \in M} (T_pN)^\bot
% \end{align}
% は $TM|_N$ の部分束になる.一方,標準射影 $\mathrm{pr} \colon T_pM \to T_pM/T_pN,\; x \mapsto x + T_pN$ の  $(T_pN)^{\bot}$ への制限 $\mathrm{pr}|_{(T_pN)^{\bot}}$ は線型同型写像だからベクトル束の同型
% \begin{align}
% 	\bigcup_{p \in M} (T_pN)^\bot \cong TM|_N/TN
% \end{align}
% がわかる.

% \subsection{Whitney和}

% \begin{mydef}[label=def.Whitney-sum]{Whitney和}
% 	底空間 $M$ を共有する2つのベクトル束 $\pi_i \colon E_i \to M$ を与える.このとき
% 	\begin{align}
% 		E_1 \oplus E_2 \coloneqq \bigl\{\, (u_1,\, u_2) \in E_1 \times E_2 \bigm| \pi_1(u_1) = \pi_2(u_2)  \,\bigr\} 
% 	\end{align}
% 	とおいて
% 	\begin{align}
% 		\pi \colon E_1 \oplus E_2 \to M,\; (u_1,\, u_2) \mapsto \pi_1(u_1)\; \bigl(= \pi_2(u_2)\, \bigr)
% 	\end{align}
% 	と定義すると $\pi \colon E_1 \oplus E_2 \to M$ はベクトル束になる.これを\textbf{Whitney和} (Whitney sum) と呼ぶ.
% \end{mydef}

% 射影 $\pi_1 \times \pi_2 \colon E_1 \times E_2 \to M\times M ,\; (u_1,\, u_2) \mapsto \bigl( \,\pi_1(u_1),\, \pi_2(u_2)\, \bigr)$ と定義すると,Whitney和は $f(p) \coloneqq (p,\, p)$ で定義される\cinfty 写像 $f \colon M \to M\times M$ による $E_1 \times E_2$ の引き戻し束である(図\ref{fig.Whitney}).
% \begin{figure}[H]
% 	\centering
% 	\begin{tikzcd}
% 			E_1 \oplus E_2 \arrow[d, "\pi_1"'] \arrow[red]{r}[red]{\pi_2}
% 				& E_1 \times E_2 \arrow{d}{\pi_1 \times \pi_2} \\
% 			M \arrow[blue]{r}[blue]{f}
% 			&M \times M
% 	\end{tikzcd}
% 	\caption{Whitney和}
% 	\label{fig.Whitney}
% \end{figure}%

% \subsection{双対束}

% \begin{mydef}[label=def.dual-bundle]{双対束}
% 	ベクトル束 $\pi \colon E \to M$ を与える.このとき $\forall p \in M$ におけるベクトル空間 $E_p$ の双対ベクトル空間 $E^*_p$ を用いて
% 	\begin{align}
% 		E^* \coloneqq \bigcup_{p\in M} E_p^*
% 	\end{align}
% 	とおくと $\pi \colon E^* \to M$ はベクトル束になる.これを\textbf{双対束} (dual bundle) と呼ぶ.
% \end{mydef}

% \begin{marker}
% 	実ベクトル束 $E$ の双対束は $E$ にRiemann計量を入れると自然に $E \cong E^*$ となるが,複素ベクトル束 $E$ の場合は必ずしもこの同型は成り立たない.
% \end{marker}

% 特に $M$ の接束 $TM$ の双対束 $T^*M$ を\textbf{余接束} (cotangent bundle) と呼ぶ.

% \subsection{テンソル積束}

% \begin{mydef}[label=def.tensor-bundle]{テンソル積束}
% 	ベクトル束 $\pi_i \colon E_i \to M$ を与える.このとき $\forall p \in M$ におけるベクトル空間 $E_i{}_p$ のテンソル積空間 $E_1 \otimes E_2$ を用いて
% 	\begin{align}
% 		E_1 \otimes E_2 \coloneqq \bigcup_{p\in M} E_1{}_p \otimes E_2{}_p
% 	\end{align}
% 	とおくと $\pi \colon E_1 \otimes E_2 \to M$ はベクトル束になる.これを\textbf{テンソル積束} (tensor product bundle) と呼ぶ.
% \end{mydef}

% \begin{mydef}[label=def.tensor-bundle]{外積束}
% 	ベクトル束 $\pi \colon E \to M$ を与える.このとき $\forall p \in M$ におけるベクトル空間 $E_i{}_p$ の $r$ 次の\hyperref[def.ext]{外積代数} $\extp^r(E_p)$ を用いて
% 	\begin{align}
% 		\extp^r (E) \coloneqq \bigcup_{p\in M} \extp^r(E_p)
% 	\end{align}
% 	とおくと $\pi \colon \extp^r (E) \to M$ はベクトル束になる.
% \end{mydef}

% 例えば\cinfty 多様体 $M$ 上の $k$ 次の微分形式全体の集合 $\Omega^k(M)$ は
% \begin{align}
% 	\Omega^k(M) = \Gamma \left( \extp^k (T^*M) \right) 
% \end{align}
% と書かれる.

\cinfty 多様体 $M$ の\textbf{微分同相群} (diffeomorphism group) $\bm{\Diff M}$ とは,
\begin{itemize}
    \item 台集合 $\Diff M \coloneqq \bigl\{\, f \colon M \lto M \bigm| \text{微分同相写像} \,\bigr\}$
    \item 単位元を恒等写像
    \item 積を写像の合成
\end{itemize}
として構成される群のことを言う.

\begin{mydef}[label=def:Lie-action]{Lie群の作用}
    \begin{itemize}
        \item Lie群 $G$ の \cinfty 多様体 $M$ への\textbf{左作用}とは,
        群準同型 $\rho \colon G \lto \Diff M$ であって写像
        \begin{align}
            \blacktriangleright \colon G \times M \lto M,\; (g,\, x) \lmto \rho(g)(x)
        \end{align}
        が \cinfty 写像となるようなもののこと.
        $\bm{g \blacktriangleright x} \coloneqq \blacktriangleright(g,\, x)$ と略記する.
        \item Lie群 $G$ の \cinfty 多様体 $M$ への\textbf{右作用}とは,
        群準同型 $\rho \colon G^{\mathrm{op}} \lto \Diff M$ であって写像
        \begin{align}
            \btl \colon M \times G \lto M,\; (x,\, g) \lmto \rho(g)(x)
        \end{align}
        が \cinfty 写像となるようなもののこと.
        $\bm{x \blacktriangleleft g} \coloneqq \blacktriangleleft(g,\, x)$ と略記する.
        \item Lie群の左(resp. 右)作用が\textbf{自由} (free) であるとは,$\forall x \in X,\; \forall g  \in G \setminus \{1_G\},\; g \btr x \neq x\quad (\text{resp.}\quad x \btl g \neq x)$ を充たすことを言う.
        \item Lie群の左(resp. 右)作用が\textbf{効果的} (effective) であるとは,$\rho \colon G \lto \Diff M\quad (\text{resp.}\quad \rho \colon G^{\text{op}} \lto \Diff M)$ が単射であることを言う.
    \end{itemize}
\end{mydef}

\begin{mydef}[label=def.fiber-1,breakable]{\cinfty ファイバー束}
    Lie群 $G$ が \cinfty 多様体 $F$ に\hyperref[def:Lie-action]{効果的に作用}しているとする.
    $\bm{C^\infty}$ \textbf{ファイバー束} (fiber bundle) とは,
    \begin{itemize}
        \item \cinfty 多様体 $E,\, B,\, F$
        \item \cinfty の全射 $\pi \colon E \lto B$
        \item Lie群 $G$ と,$G$ の $F$ への\hyperref[def:Lie-action]{左作用} $\btr \colon G \times F \lto F$
        \item $B$ の開被覆 $\bigl\{\, U_\lambda  \,\bigr\}_{\lambda \in \Lambda}$
        \item 
        微分同相写像の族
        \begin{align}
            \bigl\{\, \varphi_\lambda \colon \pi^{-1}(U_\lambda) \lto U_\lambda \times F  \,\bigr\}_{\lambda \in \Lambda}
        \end{align}
        であって,$\forall \lambda \in \Lambda$ に対して図\ref{fig.fiber1}を可換にするもの.
        \begin{figure}[H]
            \centering
            \begin{tikzcd}
                \pi^{-1}(U_\lambda) \arrow[d, "\pi"'] \arrow{r}{\varphi} & 
                    U_\lambda \times F \arrow{dl}{\mathrm{proj}_1} \\
                    U_\lambda &
            \end{tikzcd}
            \caption{局所自明性}
            \label{fig.fiber1}
        \end{figure}%
        \item \cinfty 写像の族
        \begin{align}
            \bigl\{\, t_{\alpha\beta} \colon B \lto G \bigm| \forall (p,\, f) \in (U_\alpha \cap U_\beta) \times F,\; \varphi_\beta^{-1} (p,\, f) = \varphi_\alpha^{-1} \bigl(p,\, t_{\alpha\beta} (p) \btr f\bigr)  \,\bigr\}_{\alpha,\, \beta \in \Lambda}
        \end{align}
    \end{itemize}
    の6つのデータの組みのこと.記号としては $\bm{(E,\, \pi,\, B,\, F)}$ や $\bm{F \hookrightarrow E \xrightarrow{\pi} B}$ と書く.
	% \cinfty 多様体 $F,\, E,\, B$ と \cinfty 写像 $\pi \colon E \lto B$ を与える.
    % $\pi$ が以下の条件を充たすとき,組 $(E,\, \pi ,\, B,\, F)$ を $F$ をファイバーとする $\bm{C^\infty}$ \textbf{ファイバー束} (smooth fiber bundle) と呼ぶ:
	% \begin{description}
	% 	\item[\textbf{(局所自明性)}] 
		
	% 	$\forall b \in B$ に対して,$b$ のある開近傍 $U$ と微分同相写像 $\varphi \colon \pi^{-1}(U) \xrightarrow{\simeq} U \times F$ が存在して
	% 	\begin{align}
	% 		\forall u \in \pi^{-1} (U),\; \pi(u) = \mathrm{proj}_1 \circ \varphi(u)
	% 	\end{align}
	% 	となる
    %     \footnote{ただし,$\mathrm{proj}_1$ は第一成分への射影である:
	% 	\begin{align}
	% 		\mathrm{proj}_1 \colon U \times F \to U ,\; (p,\, f) \mapsto p
	% 	\end{align}}
    %     ,i.e. 図\ref{fig.fiber1}が可換図式となる.
    %     \begin{figure}[H]
    %         \centering
    %         \begin{tikzcd}
    %             \pi^{-1}(U) \arrow[d, "\pi"'] \arrow{r}{\varphi} & 
    %                 U \times F \arrow{dl}{\mathrm{proj}_1} \\
    %             U &
    %         \end{tikzcd}
    %         \caption{局所自明性}
    %         \label{fig.fiber1}
    %     \end{figure}%
	% \end{description}
    % 微分同相写像 $\varphi$ のことを\textbf{局所自明化} (local trivialization) と呼ぶ.
\end{mydef}

以下ではファイバー束と言ったら \hyperref[def.fiber-1]{\cinfty ファイバー束}のことを指すようにする.
% \begin{marker}
% 	定義\ref{def.fiber-1}において,$\pi$ を連続写像に,$\varphi$ を位相同型写像に置き換えると一般のファイバー束の定義が得られる.
% 	しかし,以降では微分可能ファイバー束しか考えないので定義\ref{def.fiber-1}の条件を充たす $(E,\, \pi ,\, B,\, F)$ のことを\textbf{ファイバー束}と呼ぶことにする.
% \end{marker}
ファイバー束 $(E,\, \pi ,\, B,\, F)$ に関して,
\begin{itemize}
	\item $E$ を\textbf{全空間} (total space)
	\item $B$ を\textbf{底空間} (base space)
	\item $F$ を\textbf{ファイバー} (fiber)
	\item $\pi$ を\textbf{射影} (projection)
	\item $\varphi_\lambda$ を\textbf{局所自明化} (local trivialization)
	\item $t_{\alpha\beta}$ を\textbf{変換関数} (transition map)
\end{itemize}
と呼ぶ\footnote{紛らわしくないとき,ファイバー束 $(E,\, \pi,\, B,\, F)$ のことを $\pi \colon E \to B$ ,または単に $E$ と略記することがある.}.また,射影 $\pi$ による1点集合 $\{b\}$ の逆像 $\pi^{-1}(\{b\}) \subset E$ のことを\textbf{点} $\bm{b}$ \textbf{のファイバー} (fiber) と呼び,$E_b$ と書く.\label{def:point-fiber}

\section{ベクトル束}

\begin{mydef}[label=def:vect]{ベクトル束}
    ファイバーを $n$ 次元 $\mathbb{K}$-ベクトル空間 $V$ とし,構造群を $\gGL{n}{\mathbb{K}}$ とするような\hyperref[def.fiber-1]{ファイバー束} $V \hookrightarrow E \xrightarrow{\pi} M$ であって,
    その局所自明化 $\Familyset[\big]{\varphi_\lambda \colon \pi^{-1}(U_\lambda) \lto U_\lambda \times V}{\lambda \in \Lambda}$ が以下の条件を充たすもののことを\textbf{階数 $\bm{n}$ のベクトル束} (vector bundle of rank $n$) と呼ぶ:
    \begin{description}
        \item[\textbf{(vect-1)}] 
        
        $\forall \lambda \in \Lambda$ および $\forall x \in U_\lambda$ に対して,$\mathrm{proj}_2 \circ \varphi_\alpha|_{\pi^{-1}(\{x\})} \colon \pi^{-1}(\{x\}) \lto V$ は $\mathbb{K}$-ベクトル空間の同型写像である.
    \end{description}
    
\end{mydef}


% ここで,\hyperref[chap3]{第3章}で雑に導入した接束を正確に構成しよう.

\begin{myexample}[label=ex:tangentbundle]{接束}
    $n$ 次元\cinfty 多様体 $M$ の\hyperref[def:tangentbundle]{接束}は,構造群を $\LGL(n,\, \mathbb{R})$ とするベクトル束
    $\mathbb{R}^n \hookrightarrow TM \xrightarrow{\pi} M$ である.
    実際,$M$ のチャート $\bigl( U_\lambda,\, (x^\mu) \bigr)$ に対して局所自明化は
    \begin{align}
        \varphi_\lambda \colon \pi^{-1} (U_\lambda) \lto U_\lambda \times \mathbb{R}^n,\; \left(p,\, v^\mu \eval{\pdv{}{x^\mu}}_p \right) \lmto \bigl(p,\, \mqty[v^1 \\ \vdots \\ v^n] \bigr)
    \end{align}
    となり,チャート $\bigl( U_\alpha,\, (x^\mu) \bigr),\, \bigl( U_\beta,\, (y^\mu) \bigr)$ に対して
    \begin{align}
        \varphi_\beta^{-1} \bigl( p,\, (v^1,\, \dots,\, v^n) \bigr) = \varphi_\alpha^{-1} \bigl( p,\, \mqty[\pdv{x^1}{y^1}()(p) &\cdots &\pdv{x^1}{y^n}()(p) \\ \vdots &\ddots &\vdots \\ \pdv{x^n}{y^1}()(p) &\cdots &\pdv{x^n}{y^n}()(p)]\mqty[v^1 \\ \vdots \\ v^n] \bigr) 
    \end{align}
    となる.故に変換関数は
    \begin{align}
        t_{\alpha\beta} (p) \coloneqq \mqty[\pdv{x^1}{y^1}()(p) &\cdots &\pdv{x^1}{y^n}()(p) \\ \vdots &\ddots &\vdots \\ \pdv{x^n}{y^1}()(p) &\cdots &\pdv{x^n}{y^n}()(p)] \in \gGL{n}{\mathbb{R}}
    \end{align}
    で,ファイバーへの構造群の左作用とはただ単に $n$ 次元の数ベクトルに行列を左から掛けることである.
\end{myexample}

\section{束写像と $C^\infty$ 切断}
% \subsection{束写像}

% \cinfty 多様体 $F$ を共通のファイバーに持つ二つのファイバー束 $(E_i,\, \pi_i,\, B_i,\, F)$ を考える.このとき,二つの底空間 $B_i$ の間の\cinfty 写像と同様に,全空間 $E_i$ の間の微分同相写像を考えることができる.これら二つの\cinfty 写像は\textbf{束写像} (bundle map) と呼ばれる.

\begin{mydef}[label=def.bundlemap, breakable]{束写像}
	ファイバー $F$ と構造群 $G$ を共有する二つのファイバー束 $\xi_i = (E_i,\, \pi_i,\, B_i,\, F)$ を与える.
    \begin{itemize}
        \item $\xi_1$ から $\xi_2$ への\textbf{束写像} (bundle map) とは,二つの\cinfty 写像 $\textcolor{blue}{f} \colon B_1 \to B_2,\; \textcolor{red}{\tilde{f}} \colon E_1 \to E_2$ の組であって図\ref{fig.bundlemap}
        \begin{figure}[H]
            \centering
            \begin{tikzcd}
                    E_1 \arrow[d, "\pi_1"'] \arrow[red]{r}[red]{\tilde{f}}
                        & E_2 \arrow{d}{\pi_2} \\
                    B_1 \arrow[blue]{r}[blue]{f}
                    &B_2
            \end{tikzcd}
            \caption{束写像}
            \label{fig.bundlemap}
        \end{figure}%
        を可換にし,かつ底空間 $B_1$ の各点 $b$ において,\textbf{点} $\bm{b}$ \textbf{のファイバー} $\bm{\pi_1^{-1}(\{b\})} \subset E_1$ への $\textcolor{red}{\tilde{f}}$ の制限 
        \begin{align}
            \tilde{f}|_{\pi_1^{-1}(\{b\})} \colon \pi_1^{-1}(\{b\}) \to \tilde{f} \bigl( \pi_1^{-1}(\{b\}) \bigr) \subset E_2
        \end{align}
        が微分同相写像になっているもののことを言う.
        \item ファイバー束 $\xi_1$ と $\xi_2$ が\textbf{同型} (isomorphic) であるとは,$B_1 = B_2 = B$ であってかつ
        $f \colon B \to B$ が恒等写像となるような束写像 $\tilde{f} \colon E_1 \to E_2$ が存在することを言う.記号としては $\xi_1 \bm{\simeq} \xi_2$ とかく.
        \begin{figure}[H]
            \centering
            \begin{tikzcd}[column sep=small]
                    E_1 \arrow[dr, "\pi_1"'] \arrow[red]{rr}[red]{\tilde{f}} &	& E_2 \arrow{dl}{\pi_2} \\
                    & B &
            \end{tikzcd}
            \caption{ファイバー束の同型}
            \label{fig.bundle_homo}
        \end{figure}%
        \item 積束 $(B \times F,\, \mathrm{proj}_1,\, B,\, F)$ と同型なファイバー束を\textbf{自明束} (trivial bundle) と呼ぶ.
    \end{itemize}
\end{mydef}

% \begin{mydef}[label=def.bundle_isomorphism]{ファイバー束の同型}
% 	ファイバー $F$ と底空間 $B$ を共有する二つのファイバー束 $\xi_i = (E_i,\, \pi_i,\, B,\, F)$ を与える.
% 	このとき,ファイバー束 $\xi_1$ と $\xi_2$ が\textbf{同型} (isomorphic) であるとは,
% 	$f \colon B \to B$ が恒等写像となるような束写像 $\tilde{f} \colon E_1 \to E_2$ が存在することを言う.記号として $\xi_1 \simeq \xi_2$ とかく.
% \end{mydef}

% \subsection{切断}

ファイバー束 $(E,\, \pi,\, B,\, F)$ は,射影 $\pi$ によってファイバー $F$ の情報を失う.$F$ を復元するためにも,$s \colon B \to E$ なる写像の存在が必要であろう.

\begin{mydef}[label=def.section]{\cinfty 切断}
	ファイバー束 $\xi = (E,\, \pi,\, B,\, F)$ の \cinfty \textbf{切断} (cross section) とは,\cinfty 写像 $s \colon B \to E$ であって $ \pi \circ s = \mathrm{id}_B$ となるもののことを言う.
    \tcblower 
    $\xi$ の切断全体の集合を $\bm{\Gamma (B,\, E)}$ あるいは $\bm{\Gamma(E)}$ と書く.
\end{mydef}
% 各点 $b \in B$ に対して,明らかに $s(b) \in \pi^{-1}(\{b\})$ である.

% 切断は\textbf{大域的な}対象であり,与えられたファイバー束が切断を持つとは限らない.一方,各点 $b \in B$ の開近傍 $U$ 上であれば,図\ref{fig.fiber1}の示す局所自明性から\textbf{局所切断} $s \colon U \to \pi^{-1}(U)$ が必ず存在する.
% $\mathrm{proj}_1^{-1}(\{	b \}) = \{b\} \times F$ であることを考慮すると $\pi^{-1}(\{b\}) \simeq F$ とわかるので,局所切断 $s \colon U \to \pi^{-1}(U)$ は \cinfty 写像 $\tilde{s} \colon U \to F$ と一対一に対応する.

% $B$ 上の切断全体の集合を $\Gamma(B,\, E)$ と書くことにする.

\section{変換関数によるファイバー束の構成}

$\xi = (E,\, \pi,\, B,\, F)$ を\hyperref[def.fiber-1]{ファイバー束}とする.底空間 $B$ の開被覆 $\{ U_\lambda\}_{\lambda \in \Lambda}$ をとると,定義\ref{def.fiber-1}から,どの $\alpha \in \Lambda$ に対しても局所自明性(図\ref{subfig.fiber-l})
が成り立つ.ここでもう一つの $\beta \in \Lambda$ をとり,$U_\alpha \cap U_\beta$ に関して局所自明性の図式を横に並べることで,自明束 $\mathrm{proj}_1 \colon (U_\alpha \cap U_\beta) \times F \to U_\alpha \cap U_\beta$ の\hyperref[def.bundlemap]{束の自己同型}(図\ref{subfig.fiber-lm})が得られる.
\begin{figure}[H]
	\centering
	\begin{subfigure}{0.4\columnwidth}
		\centering
		\begin{tikzcd}
			U_\alpha \times F \arrow[dr, "\mathrm{proj}_1"']  & 
				\pi^{-1}(U_\alpha) \arrow[l, "\varphi_\alpha"'] \arrow[d, "\pi"] \\
			& U_\alpha
		\end{tikzcd}
		\caption{$U_\alpha$ に関する局所自明性}
		\label{subfig.fiber-l}
	\end{subfigure}
	\hspace{5mm}
	\begin{subfigure}{0.4\columnwidth}
		\centering
		\begin{tikzcd}
			\pi^{-1}(U_\beta) \arrow[d, "\pi"'] \arrow{r}{\varphi_\beta} & 
			U_\beta \times F \arrow{dl}{\mathrm{proj}_1} \\
			U_\beta &
		\end{tikzcd}
		\caption{$U_\beta$ に関する局所自明性}
		\label{subfig.fiber-m}
	\end{subfigure}
	\vspace{5mm}
	\begin{subfigure}{0.4\columnwidth}
		\centering
		\begin{tikzcd}[column sep=small]
			(U_\alpha \cap U_\beta) \times F \arrow[dr, "\mathrm{proj}_1"'] \arrow[red]{rr}[red]{\varphi_\beta \circ \varphi_\alpha^{-1}} & &
				(U_\alpha \cap U_\beta) \times F \arrow{dl}{\mathrm{proj}_1}  \\
				& U_\alpha \cap U_\beta &
		\end{tikzcd}
		\caption{自明束 $(U_\alpha \cap U_\beta) \times F$ の自己同型}
		\label{subfig.fiber-lm}
	\end{subfigure}
	\caption{局所自明性の結合}
	\label{fig.fiber2}
\end{figure}
全ての $U_\alpha \cap U_\beta$ に関する変換関数の族 $\{t_{\alpha\beta}\}$ が $\forall b \in U_\alpha \cap U_\beta \cap U_\gamma$ に対して条件
\begin{align}
	\label{eq.cocycle}
	t_{\alpha\beta}(b) t_{\beta\gamma}(b) = t_{\alpha\gamma}(b)
\end{align}
を充たすことは図式\ref{fig.fiber2}より明かである.
次の命題は,ファイバー束 $(E,\, \pi,\, B,\, F)$ を構成する「素材」には
\begin{itemize}
	\item 底空間となる\cinfty 多様体 $B$
	\item ファイバーとなる\cinfty 多様体 $F$
	\item Lie群 $G$ と,その $F$ への\hyperref[def:Lie-action]{左作用} $\btr \colon G \times F \lto F$
	\item $B$ の開被覆 $\{ U_\lambda \}$
	\item \eqref{eq.cocycle}を充たす\cinfty 写像の族 $\{t_{\alpha\beta} \colon U_\beta \cap U_\alpha \to G\}_{\alpha,\, \beta \in \Lambda}$
\end{itemize}
があれば十分であることを主張する:

\begin{myprop}[label=prop.cocycle]{ファイバー束の構成}
	\begin{itemize}
        \item \cinfty 多様体 $B,\, F$
        \item Lie群 $G$ と,$G$ の $F$ への\hyperref[def:Lie-action]{左作用} $\btr \colon G \times F \lto G$
        \item $B$ の開被覆 $\{ U_\lambda \}_{\lambda \in \Lambda}$
        \item \textbf{コサイクル条件}\eqref{eq.cocycle}を充たす\cinfty 関数の族 $\{t_{\alpha\beta} \colon U_\beta \cap U_\alpha \to G\}$
    \end{itemize}
    を与える.
    このとき,構造群 $G$ と変換関数 $\{t_{\alpha\beta}\}_{\alpha,\, \beta \in \Lambda}$ を持つファイバー束 $\xi = (E,\, \pi,\, B,\, F)$ が存在する.
\end{myprop}
\begin{proof}
	まず手始めに,cocycle条件\eqref{eq.cocycle}より
	\begin{align}
		t_{\alpha\alpha}(b) t_{\alpha\alpha} (b) = t_{\alpha\alpha}(b),\quad \forall b \in U_\alpha
	\end{align}
	だから $t_{\alpha\alpha}(b) = 1_G$ であり,また
	\begin{align}
		t_{\alpha\beta}(b) t_{\beta\alpha} (b) = t_{\alpha\alpha}(b) = 1_G,\quad \forall b \in U_\alpha \cap U_\beta
	\end{align}
	だから $t_{\beta\alpha}(b) = t_{\alpha\beta}(b)^{-1}$ である.

	開被覆 $\{U_\lambda\}$ の添字集合を $\Lambda$ とする.このとき $\forall \lambda \in \Lambda$ に対して,$U_\lambda \subset B$ には底空間 $B$ からの\hyperref[def.reltopo]{相対位相}を入れ,$U_\lambda \times F$ にはそれと $F$ の位相との\hyperref[def.prodtopo]{積位相}を入れることで,\hyperref[def.disjoint_topo]{直和位相空間}
	\begin{align}
	\mathcal{E} \coloneqq \coprod_{\lambda \in \Lambda} U_\lambda \times F
	\end{align}
	を作ることができる\footnote{$\mathcal{E}$ はいわば,「貼り合わせる前の互いにバラバラな素材(局所自明束 $U_\alpha \times F$)」である.証明の以降の部分では,これらの「素材」を $U_\alpha \cap U_\beta \neq \emptyset$ の部分に関して「良い性質\eqref{eq.cocycle}を持った接着剤 $\{ t_{\alpha\beta} \}$」を用いて「貼り合わせる」操作を,位相を気にしながら行う.}.
	$\mathcal{E}$ の任意の元は $(\textcolor{red}{\lambda},\, b,\, f) \in  \textcolor{red}{\Lambda} \times  U_\lambda \times F$ と書かれる.

	さて,$\mathcal{E}$ 上の二項関係 $\sim$ を以下のように定める:
	\begin{align}
		\label{eq.prop9-1_equiv}
		(\alpha,\, b,\, f) \bm{\sim} \bigl(\beta,\, b,\, t_{\alpha\beta}(b) \btr f\bigr) \quad \forall b \in U_\alpha \cap U_\beta,\; \forall f \in F
	\end{align}
	$\sim$ が同値関係の公理を充たすことを確認する:
	\begin{description}
		\item[\textbf{反射律}] 冒頭の議論から $t_{\alpha\alpha}(b) = 1_G$ なので良い.
		\item[\textbf{対称律}] 冒頭の議論から $t_{\beta\alpha}(b) = t_{\alpha\beta}(b)^{-1}$  なので,
		\begin{align}
			&(\alpha,\, b,\, f) \sim (\beta,\, c,\, h) \\
            &\Longrightarrow \quad b=c \in U_\alpha \cap U_\beta \AND f = t_{\alpha\beta}(b) \btr h \\
			&\Longrightarrow \quad c=b \in U_\alpha \cap U_\beta \AND h = t_{\alpha\beta}(b)^{-1} \btr f = t_{\beta\alpha}(b) \btr f \\
			&\Longrightarrow \quad (\beta,\, c,\, h) \sim (\alpha,\, b,\, f).
		\end{align}
		\item[\textbf{推移律}] cocycle条件\eqref{eq.cocycle}より
		\begin{align}
			&(\alpha,\, b,\, f) \sim (\beta,\, c,\, h) \AND (\beta,\, c,\, h) \sim (\gamma,\, d,\, k) \\
			&\Longrightarrow \quad b=c \in U_\alpha \cap U_\beta \AND c=d \in U_\beta \cap U_\gamma\AND f = t_{\alpha\beta}(b) \btr h,\, h = t_{\beta\gamma}(c) \btr k \\
			&\Longrightarrow \quad b=d \in U_\alpha \cap U_\beta \cap U_\gamma \AND f = \bigl(t_{\alpha\beta}(b) t_{\beta\gamma}(b)\bigr) \btr k = t_{\alpha\gamma}(b)\btr k  \\
			&\Longrightarrow \quad (\alpha,\, b,\, f) \sim (\gamma,\, d,\, k).
		\end{align}
	\end{description}
	したがって $\sim$ は同値関係である.
	$\sim$ による $\mathcal{E}$ の商集合を $E$ と書き,商写像を $\mathrm{pr} \colon \mathcal{E} \to E,\; (\alpha,\, b,\, f)  \mapsto [ (\alpha,\, b,\, f)]$ と書くことにする.

	集合 $E$ に商位相を入れて $E$ を位相空間にする.このとき商位相の定義から開集合 $\{\alpha\} \times U_\alpha \times F \subset \mathcal{E}$ は $\mathrm{pr}$ によって $E$ の開集合 $\mathrm{pr}(\{\alpha\} \times U_\alpha \times F) \subset{E}$ に移される.ゆえに $E$ は $\bigl\{\, \mathrm{pr}(\{\alpha\} \times U_\alpha \times V_\beta)\, \bigr\}$ を座標近傍にもつ\cinfty 多様体である(ここに $\{ V_\beta \}$ は,\cinfty 多様体 $F$ の座標近傍である).
	
	次に\cinfty の全射 $\pi \colon E \lto B$ を
	\begin{align}
		\pi \bigl(\, [(\alpha,\, b,\, f)]\, \bigr) \coloneqq b
	\end{align}
	と定義すると,これは $\forall \alpha \in \Lambda$ に対して微分同相写像
    \footnote{
        逆写像は $\varphi_\alpha^{-1} \colon U_\alpha \times F \lto \pi^{-1}(U_\alpha),\; (b,\, f) \lmto [(\alpha,\, b,\, f)]$ である.
        $\varphi_\alpha$ も $\varphi_\alpha^{-1}$ も \cinfty 写像の合成で書けるので \cinfty 写像である.
    }
	\begin{align}
		\varphi_\alpha \colon \pi^{-1}(U_\alpha) \lto U_\alpha \times F,\; [(\alpha,\, b,\, f)] \lmto (b,\, f)
	\end{align}
	による\hyperref[fig.bundle_homo]{局所自明性}を持つ.
	従って組 $\xi \coloneqq (E,\, \pi,\, B,\, F)$ は構造群 $G$,局所自明化 $\{\varphi_\alpha\}_{\alpha \in \Lambda}$,変換関数 $\{t_{\alpha\beta}\}_{\alpha,\, \beta \in \Lambda}$ を持つ\hyperref[def.fiber-1]{ファイバー束}になり,証明が終わる.
\end{proof}

\section{主束とその同伴束}

この節で導入する\hyperref[def:associated-vect]{主束の同伴ベクトル束}は,次章でゲージ場を導入する舞台となる.

\begin{mydef}[label=def.PFD]{主束}
    構造群を $G$ に持つ\hyperref[def.fiber-1]{ファイバー束} $\xi = (P,\, \pi,\, M,\, G)$ が\textbf{主束} (principal bundle) であるとは,
    $G$ の $G$ 自身への左作用が自然な\hyperref[def:Lie-action]{左作用}\footnote{つまり,$g \btr x \coloneqq gx$(Lie群の積)である.}であることを言う.
\end{mydef}

% 主束 $(P,\, \pi,\, M,\, G)$ は $(P,\, \pi,\, M,\, G)$ とか $P(M,\, G)$ と書かれることもある.
次の命題は証明の構成が極めて重要である:
\begin{myprop}[label=prop.PFD_right]{主束の全空間への右作用}
	$\xi = (P,\, \pi,\, M,\, G)$ を\hyperref[def.PFD]{主束}とする.
    このとき,$G$ の全空間 $P$ への\hyperref[def:Lie-action]{自由な右作用}が自然に定義され,
	その軌道空間 (orbit space) $P/G$ が $M$ になる.
\end{myprop}

\begin{proof}
    $\xi$ の\hyperref[def.fiber-1]{局所自明化}を $\{\, \varphi_\lambda \,\}_{\lambda \in \Lambda}$,変換関数を $\{t_{\alpha\beta} \colon U_\alpha \cap U_\beta \to G\}_{\alpha,\, \beta \in \Lambda}$ と書く.
	$\forall u \in P,\, \forall g \in G$ をとる.$\pi(u) \in U_\alpha$ となる $\alpha \in \Lambda$ を選び,対応する\hyperref[def.fiber-1]{局所自明化} $\varphi_\alpha$ による $u$ の像を $\varphi_\alpha (u) \eqqcolon (p,\, h) \in U_\alpha \times G$ とおく\footnote{つまり,$p \coloneqq \pi(u),\; h \coloneqq \mathrm{proj}_2 \circ \varphi_\alpha (u)$ と言うことである.}.
    このとき $G$ の $P$ への右作用 $\btl \colon P \times G \lto P$ を次のように定義する\footnote{右辺の $h \textcolor{red}{g}$ はLie群の乗法である.}:
    % \footnote{$G$ の $G$ 自身への右作用は,$G$ の右からの積演算を選ぶ.この作用は推移的かつ効果的である.}:
	\begin{align}
		\label{def:ractionP}
		u \btl \textcolor{red}{g} \coloneqq \varphi_\alpha^{-1}(p,\, h \textcolor{red}{g})
	\end{align}
	
	\begin{description}
		\item[\textbf{$\btl$ のwell-definedness}] 

			$\beta \neq \alpha$ に対しても $\pi(u) \in U_\beta$ であるとする.このとき $\varphi_\beta(u) = (p,\, h') \in (U_\alpha \cap U_\beta) \times G$ と書けて,また変換関数の定義から
			\begin{align}
				h' = t_{\alpha\beta}(p) h \quad \bigl(\, t_{\alpha\beta}(p) \in G\, \bigr)
			\end{align}
			である.したがって
			\begin{align}
				\varphi_\beta^{-1}(p,\, h'g) = \varphi_\beta^{-1}\Bigl(p,\, \bigl(t_{\alpha\beta}(p) h\bigr)g\Bigr) = \varphi_\beta^{-1}\bigl(p,\, t_{\alpha\beta}(p)  hg \bigr) = \varphi_\beta^{-1} \circ (\varphi_\beta \circ \varphi_\alpha^{-1})(p,\, h g) = \varphi_\alpha^{-1}(p,\, h g)
			\end{align}
			が分かり,式\eqref{def:ractionP}の右辺は局所自明化の取り方によらない.
		\item[\textbf{$\btl$ は右作用}] 
			写像 $\rho \colon G^{\text{op}} \lto \Diff P,\; g \lmto (u \lmto u \btl g)$ が群準同型であることを示す.
            \begin{enumerate}
				\item $u \btl 1_G = \varphi_\alpha^{-1}(p,\, h1_G) = \varphi_\alpha^{-1}(p,\, h) = u$
				\item $\forall g_1,\, g_2 \in G$ をとる.
				\begin{align}
					u \btl (g_1g_2) = \varphi_\alpha^{-1}\bigl(p,\, (hg_1)g_2 \bigr) = \varphi_\alpha^{-1}(p,\, hg_1) \btl g_2 = (u \btl g_1) \btl g_2
				\end{align} 
			\end{enumerate}
		\item[\textbf{$\btl$ は自由}] 
		
            $\forall \alpha \in \Lambda$ に対して
            $\forall u = (p,\, g) \in \pi^{-1}(U_\alpha)$ をとる.$u \btl g' = u$ ならば
            \begin{align}
                u \btl g' = \varphi_\alpha^{-1} (p,\, gg') = u = \varphi_\alpha^{-1}(p,\, g 1_G)
            \end{align}
            が成り立つが,局所自明化は全単射なので $gg' = g$ が言える.$g$ は任意なので $g' = 1_G$ が分かった.
        \item[\textbf{軌道空間が $\bm{M}$}]  
        
            $\forall \alpha \in \Lambda$ に対して,$G$ の右作用\eqref{def:ractionP}による $U \times G$ の軌道空間は $(U \times G) / G = U \times \{1_G\} = U$ となる.故に $P$ 全域に対しては $P/G = B$ となる.
	\end{description}
\end{proof}

\begin{mytheo}[label=thm:principal]{}
    コンパクトHausdorff空間 $P$ と,$P$ に\hyperref[def:Lie-action]{自由に作用}しているコンパクトLie群 $G$ を与える.この時,軌道空間への商写像
    \begin{align}
        \pi \colon P \lto P/G
    \end{align}
    は\hyperref[def.PFD]{主束}である.
\end{mytheo}

\begin{proof}
    
\end{proof}

構造群を $G$ とする\hyperref[def.fiber-1]{ファイバー束} $F \hookrightarrow E \xrightarrow{\pi} M$ が与えられたとき,命題\ref{prop.cocycle}を使うと,変換関数が共通の\hyperref[def.PFD]{主束} $G \hookrightarrow P \xrightarrow{p} M$ が存在することがわかる.
このようにして得られる主束をファイバー束 $F \hookrightarrow E \xrightarrow{\pi} M$ に\textbf{同伴する} (associated) 主束と呼ぶ.

\begin{myexample}[label=def:framebundle]{フレーム束}
    変換関数 $\{t_{\alpha\beta}\colon M \lto \gGL{N}{\mathbb{K}}\}$ を持つ\hyperref[def:vect]{階数 $N$ のベクトル束} $\mathbb{K}^N \hookrightarrow E \xrightarrow{\pi} M$ に同伴する主束は,例えば次のようにして構成できる:
    $\forall x \in M$ に対して
    \begin{align}
        P_x \coloneqq \bigl\{\, f \in \Hom{}(\mathbb{K}^N,\, E_x) \bigm| \text{同型写像} \,\bigr\} 
    \end{align}
    とし,
    \begin{align}
        P \coloneqq \coprod_{x \in M} P_x, \quad
        \varpi \colon P \lto M,\; (x,\, f) \lmto x
    \end{align}
    と定める.$\gGL{N}{\mathbb{K}} \hookrightarrow P \xrightarrow{\varpi} M$ に適切な局所自明化を入れて,変換関数が $\{t_{\alpha\beta}\colon M \lto \gGL{N}{\mathbb{K}}\}$ となるような主束を構成する.

     $\forall (x,\, f) \in P_x$ をとる. 
    このとき $\mathbb{K}^N$ の標準基底を $e_1,\, \dots,\, e_N$ とすると,$f \in \Hom{}(\mathbb{K}^n,\, E_x)$ は $E_x$ の基底 $f(e_1),\, \dots,\, f(e_N)$ と同一視される
    \footnote{実際 $\forall v = v^\mu e_\mu \in \mathbb{K}^n$ に対して $f(v) = v^\mu f(e_\mu)$ が成り立つので,$f(e_1),\, \dots,\, f(e_N) \in E_x$ が指定されれば $f$ が一意的に決まる.}
    ことに注意しよう.
    このことに由来して,$f_\mu \coloneqq f(e_\mu)$ とおいて $f = (f_1,\, \dots,\, f_N) \in P_x$ と表すことにする.
    % $\gGL{n}{\mathbb{K}}$ の右作用は基底の取り替えに相当する.

     $E$ の局所自明化 $\{\varphi_\alpha \colon \pi^{-1}(U_\alpha) \lto U_\alpha \times \mathbb{K}^n \}$ を与える.このとき,$n$ 個の \underline{$E$ の}\hyperref[def.section]{局所切断} $s_\alpha{}_1,\, \dots s_\alpha{}_N \in \Gamma(E|_{U_\alpha})$ を
    \begin{align}
        s_\alpha{}_\mu(x) \coloneqq \varphi_\alpha^{-1} (x,\, e_\mu)
    \end{align}
    と定義すると,$\forall x \in U_\alpha$ に対して $s_\alpha{}_1(x),\, \dots ,\, s_{\alpha}{}_N(x)$ が $E_x$ の基底となる
    \footnote{\hyperref[def:vect]{ベクトル束の定義}から $\mathrm{proj}_2 \circ \varphi_\alpha|_{E_x} \colon E_x \lto \mathbb{K}^N$ が $\mathbb{K}$-ベクトル空間の同型写像であるため.}.
    故に,\underline{$P$ の}局所切断 $p_\alpha \in \Gamma(P|_{U_\alpha})$ を
    \begin{align}
        p_\alpha (x) \coloneqq \Bigl(x,\, \bigl( s_\alpha{}_1(x),\, \dots,\, s_\alpha{}_N(x) \bigr)\Bigr) \in P_x
    \end{align}
    により定義できる.
    このとき,$\forall (x,\, f) = \bigl(x,\, (f_1,\, \dots,\, f_N)\bigr) \in \varpi^{-1}(U_\alpha)$ に対してある $g \in \gGL{N}{\mathbb{K}}$ が存在して $f = p_\alpha(x) g$ と書ける.ただし $g$ は基底の取り替え行列で,ただ単に行列の積として右から作用している.
    
    ここで,目当ての \underline{$P$ の}局所自明化を
    \begin{align}
        \psi_\alpha \colon \varpi^{-1}(U_\alpha) \lto U_\alpha \times \gGL{n}{\mathbb{K}},\; (x,\, f) = \bigl(x,\, p_\alpha (x) g\bigr)\lmto (x,\, g)
    \end{align}
    と定義する.変換関数を計算すると
    \begin{align}
        \psi_\beta^{-1}(x,\, g) 
        &= \bigl(x,\, p_\beta (x) g\bigr) \\
        &= \Bigl(x,\, \bigl(s_\beta{}_1(x),\, \dots,\, s_\beta{}_N(x)\bigr) g\Bigr) \\
        &= \Bigl(x,\, \bigl(\varphi_\beta^{-1}(x,\, e_1),\, \dots,\, \varphi_\beta^{-1}(x,\, e_N)\bigr) g\Bigr) \\
        &= \biggl(x,\, \Bigl(\varphi_\alpha^{-1}\bigl(x,\, t_{\alpha\beta}(x) e_1\bigr),\, \dots,\, \varphi_\alpha^{-1}\bigl(x,\, t_{\alpha\beta}(x) e_N\bigr)\Bigr) g\biggr) 
        % \\
        % &= \biggl(x,\, \Bigl(\varphi_\alpha^{-1}\bigl(x,\, e_1\bigr),\, \dots,\, \varphi_\alpha^{-1}\bigl(x,\, e_N\bigr)\Bigr) t_{\alpha\beta}(x) g\biggr) \\
        % &= \biggl(x,\, \bigl(s_\alpha{}_{1}(x),\, \dots,\, s_\alpha{}_N(x)\bigr) t_{\alpha\beta}(x)g\biggr) \\
        % &= \bigl(x,\, p_{\alpha} (x) t_{\alpha\beta}(x) g\bigr) \\
        % &= \psi_\alpha^{-1} (x,\, t_{\alpha\beta}(x) g)
    \end{align}
    となるが,$e_\mu$ が標準基底なので
    \begin{align}
        t_{\alpha\beta}(x) e_\mu = \mqty[t_{\alpha\beta}(x)^1{}_\mu \\ t_{\alpha\beta}(x)^2{}_\mu \\ \vdots \\ t_{\alpha\beta}(x)^n{}_\mu] = e_\nu t_{\alpha\beta}(x)^\nu{}_\mu
    \end{align}
    が成り立つこと,および\hyperref[def:vect]{ベクトル束の定義}から $\mathrm{proj}_2 \circ \varphi_\alpha|_{E_x} \colon E_x \lto \mathbb{K}^N$ が $\mathbb{K}$-ベクトル空間の同型写像であることに注意すると
    \begin{align}
        &\biggl(x,\, \Bigl(\varphi_\alpha^{-1}\bigl(x,\, t_{\alpha\beta}(x) e_1\bigr),\, \dots,\, \varphi_\alpha^{-1}\bigl(x,\, t_{\alpha\beta}(x) e_N\bigr)\Bigr) g\biggr) \\
        &= \biggl(x,\, \Bigl(\varphi_\alpha^{-1}\bigl(x,\, e_\nu\bigr) t_{\alpha\beta}(x)^\nu{}_1,\, \dots,\, \varphi_\alpha^{-1}\bigl(x,\, e_\nu\bigr)t_{\alpha\beta}(x)^\nu{}_N\Bigr) g\biggr) \\
        &= \biggl(x,\, \Bigl(\varphi_\alpha^{-1}\bigl(x,\, e_1\bigr) ,\, \dots,\, \varphi_\alpha^{-1}\bigl(x,\, e_N\bigr)\Bigr) t_{\alpha\beta}(x)g\biggr) \\
        &= \biggl(x,\, \bigl(s_\alpha{}_{1}(x),\, \dots,\, s_\alpha{}_N(x)\bigr) t_{\alpha\beta}(x)g\biggr) \\
        &= \bigl(x,\, p_{\alpha} (x) t_{\alpha\beta}(x) g\bigr) \\
        &= \psi_\alpha^{-1} (x,\, t_{\alpha\beta}(x) g)
    \end{align}
    だとわかり,目標が達成された.この $\gGL{N}{\mathbb{K}} \hookrightarrow P \xrightarrow{\pi} M$ のことを\textbf{フレーム束}と呼ぶ.
\end{myexample}

逆に,与えられた主束を素材にして,変換関数を共有するファイバー束を構成することができる.

\begin{myprop}[label=prop:Borelconst,breakable]{Borel構成}
    $G \hookrightarrow P \xrightarrow{\pi} M$ を\hyperref[def.PFD]{主束}とし,Lie群 $G$ の \cinfty 多様体への\hyperref[def:Lie-action]{左作用} $\btr \colon G \times F \lto F$ を与える.
    \eqref{def:ractionP}で定義された $G$ の \underline{$P$ への右作用}を $\btl \colon P \times G \lto P$ と書く.
    \begin{itemize}
        \item 積多様体 $P \times F$ への $G$ の新しい右作用 $\textcolor{red}{\btl} \colon (P \times F) \times G \lto P \times F$ を
        \begin{align}
            (u,\, f) \textcolor{red}{\btl}\, g \coloneqq (u \btl g,\, g^{-1} \btr f)
        \end{align}
        と定義し,この右作用による $P \times F$ の軌道空間を $\bm{P \times_G F} \coloneqq (P \times F) / G$ と書く.
        \item 商写像 $\varpi\colon P \times F \lto P \times _G F,\; (u,\, f) \lmto (u,\, f) \textcolor{red}{\btl}\, G$ による $(u,\, f) \in P \times F$ の像を $u \times_G f \in P \times_G F$ と書く.このとき写像
        \begin{align}
            q \colon P \times_G F \lto M,\; u \times_G f \lmto \pi (u)
        \end{align}
        がwell-definedになる.
    \end{itemize}
    このとき,$F \hookrightarrow P \times_G F \xrightarrow{q} M$ は構造群 $G$ をもち,変換関数が $G \hookrightarrow P \xrightarrow{\pi} M$ のそれと同じであるような\hyperref[def.fiber-1]{ファイバー束}である.
\end{myprop}

\begin{proof}
    $q$ のwell-definednessは,\eqref{def:ractionP}で定義した右作用 $\btl$ が $\pi(u)$ を不変に保つので明らか.

    主束 $G \hookrightarrow P \xrightarrow{\pi} M$ の開被覆,局所自明化,変換関数をそれぞれ $\{U_\lambda\}_{\lambda \in \Lambda},\, \{\varphi_\lambda \colon \pi^{-1}(U_\lambda) \lto U_\lambda \times G\}_{\lambda \in \Lambda},\, \bigl\{ t_{\alpha\beta} \colon M \lto G \bigr\}_{\alpha,\, \beta \in \Lambda}$ と書く.
    また,$\forall \lambda \in \Lambda$ に対して\hyperref[def.section]{局所切断} $s_\lambda \in \Gamma(P|_{U_\alpha})$ を
    \begin{align}
        s_\lambda \colon M \lto \pi^{-1}(U_\alpha),\; x \lmto \varphi_\lambda^{-1}(x,\, 1_G)
    \end{align}
    と定義する.

    このとき,$\forall \lambda \in \Lambda$ に対して \cinfty 写像
    \begin{align}
        \label{eq:loctriv-Borel}
        \psi_\lambda \colon q^{-1}(U_\lambda) \lto U_\lambda \times F,\; s_\lambda(x) \times_G f \lmto \bigl( x,\, f \bigr) 
    \end{align}
    がwell-definedな
    \footnote{$\forall u \times_G f \in q^{-1}(U_\lambda)$ をとる.このとき $q(u \times_G f) = \pi(u) \in U_\lambda$ なので $u \in P$ に\underline{主束 $G \hookrightarrow P \xrightarrow{\pi} M$ の}局所自明化 $\varphi_\lambda \colon \pi^{-1}(U_\lambda) \lto U_\lambda \times F$ を作用させることができる.
        従って $g(u) \coloneqq  \mathrm{proj}_2 \circ \varphi_\lambda (u) \in G$ とおけば,
        $G$ の $P$ への右作用の定義\eqref{def:ractionP}から $u = \varphi_\lambda^{-1} \bigl(\pi(u),\, g(u)\bigr) = \varphi_\lambda^{-1} \bigl( \pi(u),\, 1_G \bigr)  \btl g(u) = s_\lambda \bigl( \pi(u) \bigr)  \btl g(u)$ が成り立ち,$u \times_G f = \Bigl( s_\lambda \bigl( \pi(u) \bigr) \btl g(u) \Bigr) \times_G f =  s_\lambda \bigl( \pi(u) \bigr) \times_G \bigl(g(u) \btr f\bigr)$ と書くことができる.
        よって \textbf{$\psi_\lambda$ の定義\eqref{eq:loctriv-Borel}において} $\bm{\psi_\lambda (u \times_G f) = \bigl( \pi(u),\, g(u) \btr f \bigr)}$ \textbf{であり},全ての $q^{-1}(U_\lambda)$ の元の行き先が定義されていることがわかった.
        次に $u \times_G f = u' \times_G f' \in q^{-1}(U_\lambda)$ であるとする.このとき右作用 $\textcolor{red}{\btl}$ の定義からある $h \in G$ が存在して $u' = \varphi_\lambda^{-1} \bigl( \pi(u'),\, g(u') \bigr) = u \btl h = \varphi_\lambda^{-1} \bigl( \pi(u),\, g(u) h \bigr),\; f' = h^{-1} \btr f$ が成り立つので,$\pi(u') = \pi(u),\; g(u') = g(u)h,\, f' = h^{-1} \btr f$ が言える.
        従って $\psi_\lambda(u' \times_G f') = \bigl( \pi(u'),\, g(u') \btr f' \bigr) = \Bigl( \pi(u),\, \bigl(g(u) h\bigr) \btr \bigl( h^{-1} \btr f \bigr) \Bigr) = \bigl( \pi(u),\, g(u) \btr h \btr h^{-1} \btr f \bigr) = \bigl( \pi(u),\, g(u) \btr f \bigr) = \psi_\lambda (u \times_G f)$ が成り立ち,$\psi_\lambda$ がwell-definedであることが示された.
    }
    微分同相写像になる
    \footnote{
        $\pi \colon P \lto M,\; g \coloneqq \mathrm{proj}_2 \circ \varphi_\lambda \colon q^{-1}(U_\lambda) \lto G,\; \btr \colon G \times F \lto F$ は全て \cinfty 写像の合成の形をしているので \cinfty 写像であり,$\psi_\lambda \coloneqq \bigl(\pi \times (\btr \circ (g \times \mathrm{id}_F))\bigr)$ もこれらの合成として書けている(写像 $\times,\, \mathrm{id}_F$ はもちろん \cinfty 級である)ので \cinfty 写像である.
        well-definednessの証明と同じ議論で $\psi_\lambda$ の単射性がわかる.全射性は定義\eqref{eq:loctriv-Borel}より明らか.
        逆写像 $(x,\, f) \lmto s_\lambda(x) \times_G f$ も,\cinfty 写像たちの合成 $q \circ (s_\lambda \times \mathrm{id}_F)$ なので \cinfty 写像である.
    }
    ので,族 
    \begin{align}
        \bigl\{ \psi_\lambda \colon q^{-1}(U_\lambda) \lto U_\lambda \times F \bigr\}_{\lambda \in \Lambda}
    \end{align}
    を $F \hookrightarrow P \times_G F \xrightarrow{q} M$ の局所自明化にとる.
    すると $\forall \alpha,\, \beta \in \Lambda,\; \forall (x,\, f) \in (U_\alpha \cap U_\beta) \times F$ に対して
    \begin{align}
        \psi_\beta^{-1} (x,\, f) &= s_\beta(x) \times_G f \\
        &= \varphi_\beta^{-1}(x,\, 1_G) \times_G f \\
        &= \varphi_\alpha^{-1}(x,\, t_{\alpha\beta}(x)1_G) \times_G f \\
        &= \varphi_\alpha^{-1}(x,\, 1_G t_{\alpha\beta}(x)) \times_G f \\
        &= \bigl(\varphi_\alpha^{-1}(x,\, 1_G) \btl t_{\alpha\beta}(x)\bigr) \times_G f \\
        &= \bigl(s_\alpha (x) \btl t_{\alpha\beta}(x)\bigr) \times_G f \\
        % &= \varpi\bigl(x  t_{\alpha\beta}(x),\, f\bigr) \\
        &= \Bigl( \bigl(s_\alpha (x) \btl t_{\alpha\beta}(x)\bigr) \btl t_{\alpha\beta}(x)^{-1} \Bigr) \times_G \bigl(t_{\alpha\beta}(x) \btr f\bigr) \\
        &= s_\alpha (x) \times_G \bigl( t_{\alpha\beta}(x) \btr f \bigr) \\
        &= \psi_\alpha^{-1} (x,\, t_{\alpha\beta}(x) \btr f)
    \end{align}
    が成り立つので $F \hookrightarrow P \times_G F \xrightarrow{q} M$ の変換関数は
    \begin{align}
        \bigl\{\, t_{\alpha\beta} \colon M \lto G  \,\bigr\}_{\alpha,\, \beta \in \Lambda}
    \end{align}
    である.
\end{proof}

\begin{myexample}[label=def:associated-vect]{同伴ベクトル束}
    \hyperref[def.fiber-1]{主束} $G \hookrightarrow P \xrightarrow{\pi} \mathcal{M}$ を任意に与える.
    Lie群 $G$ の,$N$ 次元 $\mathbb{K}$ ベクトル空間 $V$ への\hyperref[def:Lie-action]{左作用}とは,Lie群 $G$ の $N$ 次元表現 $\rho \colon G \lto \LGL (V)$ のことに他ならない\footnote{$\End V$ に標準的な \cinfty 構造を入れてLie群と見做したものを $\LGL (V)$ と書いた.}.
    このとき,命題\ref{prop:Borelconst}の方法によって構成される階数 $N$ の\hyperref[def:vect]{ベクトル束}のことを $\bm{P \times_\rho V}$ と書き,\textbf{同伴ベクトル束} (associated vector bundle) と呼ぶ.
\end{myexample}


\end{document}
