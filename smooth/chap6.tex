\documentclass[geometry_main]{subfiles}

\begin{document}

\setcounter{chapter}{5}

\chapter{ホモロジー・de Rhamコホモロジーの紹介}

\section{多様体のホモロジー}

$X$ を位相空間とする.$X$ の $l$ 次元ホモロジー群 $H_l(X)$ とは,$X$ 中の「$l$ 次元のサイクル」と呼ばれる量が本質的に何個あるかを示すものである.
ホモロジーの定義は何通りか知られているが,一般に $l$ 単体と呼ばれる単位に分割し,組み合わせ的な構造を利用して定義する.
本章ではごく部分的に多様体上のホモロジーと de Rham コホモロジーを紹介する.
詳細は~\cite[第3章]{Morita}や~\cite[Chapter 17, 18]{Lee12}を参照されたし.

\subsection{単体・三角形分割}

手始めに,まず$l$ 単体を定義しよう.
\begin{mydef}[]{$l$-単体}
	$\mathbb{R}^N$ の $l+1$ 個の点 $v_0,\, v_1,\, \dots ,\, v_{l}$ は,$l$ 個のベクトル $v_i - v_0 \; (i = 1,\, \dots l)$ が線型独立のとき,\textbf{一般の位置にある}という.

	一般の位置にある $l+1$ 個の点の集合 $\sigma = \{v_0,\, \dots ,\, v_l\}$ に対して,それらの点を含む最小の凸集合
	\begin{align} 
		\abs{\sigma} \coloneqq \bigl\{ a_0 v_0 + \cdots + a_l v_l \bigm| a_i \ge 0,\, a_0 + \cdots + a_l = 1 \bigr\} 
	\end{align}
	を\textbf{$l$-単体} ($l$-simplex) と呼ぶ.$\sigma$ の空でない部分集合 $\tau \subset \sigma$ に対して,単体 $\abs{\tau}$ のことを $\abs{\sigma}$ の\textbf{辺} (face) と呼ぶ.
\end{mydef}

\begin{mydef}[label=def.unit]{単体複体} 
	$\mathbb{R}^N$ の中の単体の集合 $K$ は,次の条件を充たすとき (Euclid) \textbf{単体複体} (Euclidean simplicial complex) と呼ぶ:
	\begin{enumerate} 
		\item $\abs{\sigma} \in K$ ならば $\abs{\sigma}$ の任意の辺はまた $K$ に属する.
		\item 二つの単体 $\abs{\sigma},\, \abs{\tau} \in K$ が空でない共通部分を持つならば $\abs{\sigma} \cap \abs{\tau}$ は $\abs{\sigma}$ と $\abs{\tau}$ の共通の辺である.
		\item $\forall \abs{\sigma} \in K$ の任意の点 $x \in \abs{\sigma}$ に対して,$x$ 開近傍 $U$ を適切に取れば $U$ と交わる $K$ の単体は有限個しか存在しないようにできる.
	\end{enumerate}
\end{mydef}

\begin{mydef}[label=triangle]{多面体・三角形分割}
	単体複体 $K$ に対して,集合
	\begin{align} 
		\abs{K} \coloneqq \bigcup_{\abs{\sigma} \in K} \abs{\sigma}
	\end{align}
	を定める.$\abs{K} \subset \mathbb{R}^N$ を\textbf{多面体} (polyhedron) と呼ぶ.

	位相空間 $X$ に対して適当な単体複体 $K$ を選び,同相写像 $t \colon \abs{K} \xrightarrow{\approx} X$ が与えられたとき,同相写像 $t$ を $X$ の\textbf{三角形分割} (triangulation) と呼ぶ.
\end{mydef}

\subsection{ホモロジー群}

\begin{mydef}[label=def.orient]{単体の向き} 
	$l$ 単体 $\abs{\sigma}$ の頂点 $\{v_0,\, \dots ,\, v_l\}$ の順序付き添字 $I = (i_0,\, \dots ,\, i_l)$ 全体の集合 $\mathcal{I}$ に以下の同値関係を定める:
	\begin{align} 
		\sim\; \coloneqq \bigl\{ (I,\, J) \in \mathcal{I} \times \mathcal{I} \bigm| \exists \tau \in \mathfrak{S}_{l+1}\, \mathrm{s.t.}\, \textbf{偶置換},\; I = \tau J \bigr\} 
	\end{align}
	このとき,$I \in \mathcal{I}$ の $\sim$ による同値類 $[I]$ のことを単体 $\abs{\sigma}$ の\textbf{向き} (orientation) と呼ぶ.

	単体 $\abs{\sigma}$ に向きが指定されているとき,$\sigma$ の同値類を\textbf{向き付けられた単体}と呼び,$\expval{\sigma}$ と表す.頂点が $I = (i_0,\, \dots ,\, i_l)$ によって向き付けられているとき,対応する向き付けられた単体を $\expval{v_{i_0} \cdots v_{i_l}}$ と書く.
\end{mydef}

\begin{mydef}[label=def.chain]{$l$-chain}
	単体複体 $K =\{\abs{\sigma}_i\}$ の各単体に向きを指定し,それぞれ $\expval{\sigma_i}$ とする.$K$ の $l$-単体 $\expval{\sigma_{i}}_l$ 全体によって生成される自由加群を $K$ の \textbf{$l$次元鎖群 $C_l(K)$} と呼び,$C_l(K)$ の元を \textbf{$l$-チェイン} と呼ぶ.
\end{mydef}

$\forall  c \in C_l(K)$ は形式和として
\begin{align} 
	c = \sum_{i \in I_l} c_i \expval{\sigma_i}_l,\quad c_i \in \mathbb{Z}
\end{align}
と書かれる.群 $C_l(K)$ の二項演算 $+$,単位元 $0$,逆元 $-c$はそれぞれ
\begin{align} 
	c + c' &\coloneqq \sum_i (c_i + c'{}_i) \expval{\sigma_i}_l, \\
	0 &\coloneqq \sum_i 0 \expval{\sigma_i}_l, \\
	-c &\coloneqq \sum_i (-c_i) \expval{\sigma_i}_l
\end{align}
である.ただし,$\expval{\sigma_i}_l$ と反対に向き付けられた $l$ 単体は $(-1)\expval{\sigma_i}_l \in C_l(K)$ と同一視する.このとき,自然に
\begin{align} 
	C_l(K) \cong \bigoplus_{I_l} \mathbb{Z}
\end{align}
である.

\begin{mydef}[label=op_boundary]{境界作用素}
	準同型写像
	\begin{align} 
		\partial_l \colon C_l(K) \to C_{l-1}(K)
	\end{align}
	を向き付けられた各 $l$-単体上
	\begin{align} 
		\partial_l \expval{v_0 v_1 \cdots v_l} \coloneqq \sum_{i=0}^l (-1)^i \expval{v_0 \cdots \hat{v_i} \cdots v_l}
	\end{align}
	と定義する.ただし,$\hat{v_i}$ は $v_i$ を省くことを意味する.
\end{mydef}

\begin{myprop}[label=prop.op_boundary]{境界の境界} 
	$\partial_l \circ \partial_{l+1} = 0$
\end{myprop}

\begin{proof} 
	$\partial_l$ は $C_l(K)$ 上の線型作用素なので生成元 $\sigma \coloneqq \expval{v_0v_1 \cdots v_{l+1}} \in C_{l+1}(K)$ に対して示せば十分.$l = 0$ のときは自明なので $l > 0$ とする.
	\begin{align} 
		&\partial_l \circ \partial_{l+1} \sigma \\
		&= \sum_{i=0}^{l+1} (-1)^i \partial_l \expval{v_0 \cdots \hat{v_i} \cdots v_{l+1}} \\
		&= \sum_{i=0}^{l+1}(-1)^i \left(\sum_{j=0}^{i-1} (-1)^j \expval{v_0 \cdots \hat{v_j} \cdots \hat{v_i} \cdots v_{l+1}} + \sum_{j = i+1}^{l+1} (-1)^{j-1} \expval{v_0 \cdots \hat{v_i} \cdots \hat{v_j} \cdots v_{l+1}} \right) \\
		&= \sum_{i > j} (-1)^{i+j} \expval{v_0 \cdots \hat{v_j} \cdots \hat{v_i} \cdots v_{l+1}} - \sum_{i < j} (-1)^{i+j}\expval{v_0 \cdots \hat{v_i} \cdots \hat{v_j} \cdots v_{l+1}} = 0.
	\end{align}
\end{proof}

命題\ref{prop.op_boundary}より,
\begin{align} 
	Z_l (K) &\coloneqq \bigl\{ c \in C_l(K) \bigm| \partial_l c = 0 \bigr\} = \Ker \partial_l \\
	B_l (K) &\coloneqq \bigl\{ \partial_{l+1} c \in C_l(K) \bigm| c \in C_{l+1}(K) \bigr\} = \Im \partial_{l+1}
\end{align}
とおくと
\begin{align} 
	B_l(K) \subset Z_l(K)
\end{align}
となる.
\begin{marker} 
	$Z_l(K)$ を\textbf{$l$-輪体群}もしくは\textbf{サイクル},$B_l(K)$ を\textbf{$l$-境界輪体群}もしくは\textbf{バウンダリー}と呼ぶ.
\end{marker}


\begin{mydef}[label=HomologyGroup]{ホモロジー群}
	上で定義した $Z_l(K),\, B_l(K)$ に対して,部分群の剰余類を考えることにより
	\begin{align} 
		H_l(K) \coloneqq Z_l(K) / B_l(K)
	\end{align}
	は商群を作る.これを $K$ の\textbf{ $l$ 次元ホモロジー群}と呼ぶ.
\end{mydef}

% サイクル $c \in Z_l(K)$ を代表元にもつホモロジー類 $[c] \in H_l(K)$ に対して,別のサイクル $d \in Z_l(K)$ が $d \in [c]$ であるとき,i.e. $c - d \in B_l(K)$ であるとき,$c,\, d$ は\textbf{ホモローグ} (homologue) であるという.

\begin{mytheo}[label=thm:simplicial-topo]{ホモロジー群は位相不変量}
	ホモロジー群は位相不変量である.i.e. 位相空間 $X,\, Y$ が互いに同相であるとし,それぞれの三角形分割 $f \colon \abs{K} \xrightarrow{\approx} X,\; g \colon \abs{L} \xrightarrow{\approx} Y$ を与える.このとき
	\begin{align}
		H_l (K) \cong H_l(L)\quad (l = 0,\, 1,\, \dots)
	\end{align}
	が成り立つ.
\end{mytheo}


\section{de Rhamコホモロジー}

\subsection{特異ホモロジー}

\begin{mydef}[label=stdsimplex]{標準$k$-単体}
	$\mathbb{R}^k$ の部分集合
	\begin{align} 
		\varDelta^k \coloneqq \bigl\{ (x^1,\, \dots ,\, x^k) \in \mathbb{R}^k \bigm| x^i \ge 0, \; x^1 + \cdots + x^k \le 1\, \bigr\} 
	\end{align}
	は\textbf{標準 $k$-単体} (standard $k$-simplex) と呼ばれる.
\end{mydef}

\begin{mydef}[label=sg_k-simplez]{\cinfty 特異 $k$-単体} 
	\cinfty 多様体 $M$ に対して,任意の\cinfty 写像
	\begin{align} 
		\sigma \colon \varDelta^k \to X
	\end{align}
	を $X$ の\textbf{\cinfty 特異 $k$ 単体} (singular $k$-simplex) と呼ぶ.$M$ の\cinfty 特異 $k$ 単体全体によって生成される自由加群を $S_k(X)$ と書き,その元を $M$ の\textbf{\cinfty 特異 $k$-チェインと呼ぶ}.
\end{mydef}

\begin{mydef}[label=def.partial]{境界作用素} 
	$i = 0,\, \dots ,\, k$ に対して連続写像 $\varepsilon_i \colon \varDelta^{k-1} \to \varDelta^k $ を
	\begin{align} 
		\varepsilon_0(x_1,\, \dots ,\, x_{k-1}) &\coloneqq \left(1 - \sum_{i=1}^{k-1} x_i,\; x_1,\, \dots ,\, x_{k-1}\right), \\
		\varepsilon_i(x_1,\, \dots ,\, x_{k-1}) &\coloneqq (x_1,\, \dots ,\, x_{i-1},\, 0 ,\, x_i,\, x_{k-1})
	\end{align}
	と定義する.このとき,\textbf{境界作用素}
	\begin{align} 
		\partial \colon S_k(M) \to S_{k-1}(M)
	\end{align}
	を次のように定義する:
	\begin{align} 
		\partial \sigma \coloneqq \sum_{i=0}^k (-1)^i \sigma \circ \varepsilon_i
	\end{align}
\end{mydef}

サイクル $Z_k(M)$ および $k$-境界輪体群 $B_k(M)$ を
\begin{align} 
	Z_k (M) &\coloneqq \Ker \partial_k \\
	B_k(M) & \coloneqq \Im \partial_{k+1}
\end{align}
と定めると,相変わらず $\partial \circ \partial =0$ であるから $B_{k} (M) \subset Z_k(M)$ が従う.故に部分群の剰余類を考えることができる:
\begin{mydef}[label=sg_HomologyGroup]{特異ホモロジー群}
	$B_{k} (M),\; Z_k(M)$ に対して,商群
	\begin{align} 
		H_k(M) \coloneqq Z_k(M) / B_k(M)
	\end{align}
	を $M$ の\textbf{特異ホモロジー群}と呼ぶ.
\end{mydef}

\subsection{微分形式のチェイン積分とStokesの定理}

$M$ を \cinfty 多様体,$S_\bullet(M) \coloneqq \{ S_k(M),\, \partial \}$ を $M$ の\cinfty 特異チェイン複体とする.

$M$ の特異 $k$ 単体
\begin{align} 
	\sigma \colon \varDelta^k \to M
\end{align}
は\cinfty 写像であるから,$k$-形式 $\omega \in \Omega^k(M)$ の引き戻し(命題\ref{prop.pullback}付近を参照) $\sigma^* \omega \in \Omega^k(\varDelta^k)$ が定義される.

\begin{mydef}[label=def.int_k-form_sg]{特異 $k$ 単体上の積分}
	$\omega \in \Omega^k(M)$ の $\sigma$ 上の積分を
	\begin{align} 
		\int_{\sigma} \omega \coloneqq \int_{\varDelta^k} \sigma^* \omega
	\end{align}
	により定義する.右辺はただの $k$-中積分である.

	一般の\cinfty 特異 $k$-チェイン $c \in S_k(M)$ が $c = \sum_{i} a_i s_i$ と表示されているときは
	\begin{align} 
		\int_c \omega \coloneqq \sum_i a_i \int_{\sigma_i} \omega
	\end{align}
	と定義する.
\end{mydef}


\begin{mytheo}[label=Stokes]{チェイン上のStokesの定理}
	\cinfty 多様体 $M$ の特異 $k$-チェイン $c \in S_k(M)$ と $k-1$-形式 $\omega \in \Omega^{k-1}(M)$ に対し,以下の等式が成立する:
	\begin{align} 
		\int_c \dd{\omega} = \int_{\partial c} \omega.
	\end{align}
\end{mytheo}

\section{de Rhamの定理}

\subsection{de Rhamコホモロジー}

\begin{mydef}[label=def.close]{閉形式・完全形式} 
	$k$-形式 $\omega \in \Omega^k(M)$ は
	\begin{itemize} 
		\item $\dd{\omega} = 0$ のとき\textbf{閉形式} (closed form)
		\item $\exists \eta \in \Omega^{k-1}(M),\; \omega = \dd{\eta}$ のとき\textbf{完全形式} (exact form)
	\end{itemize}
	と呼ばれる.
\end{mydef}

$M$ 上の閉じた $k$ -形式全体を $Z^k(M)$ ,完全な $k$-形式全体を $B^k(M)$ と書く:
\begin{align} 
	&Z^k(M) \coloneqq \Ker \bigl( \dd{}\colon \Omega^k(M) \to \Omega^{k+1}(M) \bigr) , \\
	&B^k(M) \coloneqq \Im \bigl( \dd{}\colon \Omega^{k-1}(M) \to \Omega^{k}(M) \bigr).
\end{align}
$\dd{}\circ \dd{} = 0$ なので,$B^k(M) \subset Z^k(M)$ である.

\begin{mydef}[label=def.deRham]{de Rhamコホモロジー群} 
	$\Omega^k(M)$ の部分ベクトル空間 $B^k(M),\, Z^k(M)$ に対して,商空間
	\begin{align} 
		\irm{H}{DR}^k(M) \coloneqq Z^k(M) / B^k(M)
	\end{align}
	は$M$ の $k$次\textbf{de Rhamコホモロジー群}と呼ばれる.

	$k$ -形式 $\omega \in \Omega^k(M)$ に対し,それを代表元に持つ剰余類 $[\omega] \in \irm{H}{DR}^k (M)$ を $\omega$ の表す\textbf{de Rhamコホモロジー類}と呼ぶ.
	\begin{align} 
		\irm{H}{DR}^\bullet(M) \coloneqq \bigoplus_{k=0}^n \irm{H}{DR}^k(M)
	\end{align}
	を $M$ の\textbf{de Rhamコホモロジー群}と呼ぶ.
\end{mydef}

$x \in \irm{H}{DR}^k(M),\; y \in \irm{H}{DR}^l(M)$ が $\omega \in \Omega^k(M),\; \eta \in \Omega^l(M)$ によって $x = [\omega],\; y = [\eta]$ と書かれるとき,$\irm{H}{DR}^\bullet(M)$ 上の積 $\cdot\mathrel{}\colon \irm{H}{DR}^\bullet(M) \times \irm{H}{DR}^\bullet(M) \to \irm{H}{DR}^\bullet(M)$ を以下のように定義する:
\begin{align} 
	x \cdot y \coloneqq [\omega \wedge \eta] \in \irm{H}{DR}^{k+l}(M)
\end{align}
このとき二項演算 $\cdot$ はwell-definedである,i.e. $\omega,\, \eta$ の取り方によらない.

上で定義した積構造の入った $\bigl( \irm{H}{DR}^\bullet(M),\, \cdot \bigr)$ のことを $M$ の\textbf{de Rhamコホモロジー代数}と呼ぶ.

\subsection{de Rhamの定理}

% \cinfty 多様体のde Rhamコホモロジーは $M$ の微分構造に依存せず,$M$ の位相空間としての構造のみによって定まってしまう.

コホモロジー群とホモロジー群は,Stokesの定理によって双対性を持つ.

\cinfty 多様体 $M$ および $M$ の \cinfty 特異$r$-チェイン $S_k(M)$ を与える.$\forall c \in S_k(M),\; \forall \omega \in \Omega^k(M)\; (1 \le k \le n)$ をとる.ここで双対内積 (duality pairing) を
\begin{align} 
	&\dualip{\;}{} \colon S_k(M) \times \Omega^k(M) \to \mathbb{R},\; (c,\, \omega) \mapsto \int_c \omega
\end{align}
と定義する.このとき $\dualip{c}{\omega}$ は双線型であり,$\dualip{\;}{\omega} \colon  S_k(M) \to \mathbb{R},\;  \dualip{c}{} \colon  \Omega^k(M) \to \mathbb{R}$ はどちらも線型写像である:
\begin{align} 
	\dualip{c_1 + c_2}{\textcolor{red}{\omega}} &= \int_{c_1+c_2} \omega = \int_{c_1} \omega + \int_{c_2} \omega = \dualip{c_1}{\textcolor{red}{\omega}} + \dualip{c_2}{\textcolor{red}{\omega}} \\
	\dualip{\textcolor{red}{c}}{\omega_1 + \omega_2} &= \int_{c} (\omega_1 + \omega_2) = \int_{c} \omega_1 + \int_{c} \omega_2 = \dualip{\textcolor{red}{c}}{\omega_1} + \dualip{\textcolor{red}{c}}{\omega_2} \\
\end{align}
Stokesの定理は
\begin{align} 
	\dualip{c}{\dd{\omega}} = \dualip{\partial c}{\omega}
\end{align}
と書かれ,この意味で $\dd{}$ と $\partial$ は互いに随伴写像である.

duality pairing $\dualip{\;}{}$ は内積 $\Lambda\colon H_k(M) \times \irm{H}{DR}^k(M) \to \mathbb{R}$ を誘導する.それは以下のように定義される:
\begin{align} 
	\label{eq.dualpair}
	\Lambda\bigl([c],\, [\omega]\bigr) \coloneqq \dualip{c}{\dd{\omega}}
\end{align}
定義\ref{eq.dualpair}はwell-definedである.

\begin{mytheo}[label=thm.deRham]{Poincar\'{e}双対}
	$M$ がコンパクトな\cinfty 多様体ならば $H_k(M),\, \irm{H}{DR}^k(M)$ はともに有限次元である.さらに写像
	\begin{align} 
		\Lambda \colon H_k(M) \times \irm{H}{DR}^k(M) \to \mathbb{R}
	\end{align}
	は双線型かつ非退化である.i.e. $H_r(M) = \bigl(\irm{H}{DR}^k(M) \bigr)^*$(双対ベクトル空間)である.
\end{mytheo}


\begin{mylem}[label=Poincare]{Poincar\'{e}の補題}
	$\mathbb{R}^n$ のde Rhamコホモロジーは自明である:
	\begin{align} 
		\irm{H}{DR}^k(\mathbb{R}^n) = \irm{H}{DR}^k(\text{一点}\; p_0 \in \mathbb{R}^n) = 
		\begin{cases} 
			\mathbb{R} & \colon k = 0 \\
			0 & \colon k > 0
		\end{cases}
	\end{align}
	i.e. $\omega \in \Omega^k(\mathbb{R})$ を任意の閉形式とすると,ある $k-1$ 形式 $\eta$ が存在して $ \omega = \dd{\eta}$ を充たす.
\end{mylem}


\end{document}
