\documentclass[geometry_main]{subfiles}

\begin{document}

\setcounter{chapter}{1}

\chapter{多様体のあれこれ}

この章では,主に多様体に関する内容を雑多にまとめる.

% \section{極大アトラスからの多様体の位相の構成}

% 位相多様体は第2可算なHausdorff空間 $(M,\, \mathscr{O})$ だが,集合 $M$ にいきなり\hyperref[ax.topo]{位相}を与えてしまうと微分構造の全貌が良くわからない気がする.
% 実は,極大アトラスを先に与えておくと自然にHausdorff空間を構成することができるのである.

% $M$ を集合とする.\textbf{部分集合}($M$ の位相が現時点では未定義なので,開集合ではない) $U \subset M$ を勝手にとる.
% このとき,$U$ から $\mathbb{R}^d$ の開集合 $V$ への写像 $\varphi \colon U \to V$ が\textbf{全単射}(現時点ではまだ同相写像でない)ならば,組 $(U,\, \varphi)$ を集合 $M$ の $d$ 次元の\textbf{チャート} (chart) と呼ぶことにする.
% チャート $(U,\, \varphi)$ および点 $p \in U$ が与えられたとき,$d$ 個の要素
% \begin{align}
% 	\varphi(p) = \bigl( x^1(p),\, \dots ,\, x^d(p) \bigr) \in \mathbb{R}^d
% \end{align}
% のことを点 $p$ のチャート $(U,\, \varphi)$ における\textbf{座標} (coordinates) と呼ぶ.

\section{位相多様体の性質}

まず,\hyperref[def.compact]{コンパクト性}に類似する概念をいくつか紹介する:
\begin{mydef}[label=def:cover, breakable]{被覆}
    \begin{itemize}
        \item 集合族 $\mathcal{U} \coloneqq \Familyset[\big]{U_\lambda}{\lambda \in \Lambda}$ が\underline{集合} $X$ の\textbf{被覆} (cover) であるとは,
        \begin{align}
            X \subset \bigcup_{\lambda \in \Lambda} U_\lambda
        \end{align}
        が成り立つこと.
        \item \underline{位相空間} $X$ の被覆 $\mathcal{U} \coloneqq \Familyset[\big]{U_\lambda}{\lambda \in \Lambda}$ が\textbf{開} (open) であるとは,$\forall \lambda \in \Lambda$ に対して $U_\lambda$ が $X$ の開集合であること.
        \item 位相空間 $X$ の被覆 $\mathcal{V} \coloneqq \Familyset[\big]{V_\alpha}{\alpha \in A}$ が,別の $X$ の被覆 $\mathcal{U} \coloneqq \Familyset[\big]{U_\lambda}{\lambda \in \Lambda}$ の\textbf{細分} (refinement) であるとは,
        $\forall V_\alpha \in \mathcal{V}$ に対してある $U_\lambda \in \mathcal{U}$ が存在して $V_\alpha \subset U_\lambda$ が成り立つこと.
        \item 位相空間 $X$ の\underline{開}被覆 $\mathcal{U} \coloneqq \Familyset[\big]{U_\lambda}{\lambda \in \Lambda}$ が\textbf{局所有限} (locally finite) であるとは,$\forall x \in X$ に対して以下の条件が成り立つこと:
        \begin{description}
            \item[\textbf{(locally finiteness)}] $x$ のある近傍 $V \subset X$ が存在して集合
            \begin{align}
                \bigl\{\, \lambda \in \Lambda \bigm| U_\lambda \cap V \neq \emptyset \,\bigr\} 
            \end{align}
            が有限集合になる.
        \end{description}
    \end{itemize}
\end{mydef}

\begin{mydef}[label=def:compacts]{パラコンパクト・コンパクト・局所コンパクト}
    位相空間 $X$ を与える.
	\begin{itemize}
		\item \textbf{パラコンパクト} (paracompact) であるとは,任意の\hyperref[def:cover]{開被覆}が\hyperref[def:cover]{局所有限}かつ開な細分を持つこと.
		\item 位相空間 $X$ の部分集合 $A \subset X$ は,以下の条件を充たすとき\textbf{コンパクト} (compact) であると言われる:
		\begin{description}
			\item[\textbf{(Heine-Boralの性質)}] $A$ の任意の\hyperref[def:cover]{開被覆} $\mathcal{U} \coloneqq \Familyset[\big]{U_\lambda}{\lambda \in \Lambda}$ に対して,
			ある\underline{有限}部分集合 $I \subset \Lambda$ が存在して $\Familyset[\big]{U_i}{i \in I} \subset \mathcal{U}$ が $A$ の開被覆となる\footnote{このことを「任意の開被覆は有限部分被覆を持つ」と表現する.}.
		\end{description}
		\item 位相空間 $X$ が\textbf{局所コンパクト} (locally compact) であるとは,$\forall x \in X$ が少なくとも1つのコンパクトな\hyperref[def:neighborhood]{近傍}を持つこと.
	\end{itemize}
\end{mydef}

\section{微分構造の構成}

\hyperref[maxatlas]{微分構造}を定義通りに構成するならば,
まず\hyperref[def.topomani]{位相多様体}であることを確認してから座標変換が $C^\infty$ 級であることを確認しなくてはならず,若干面倒である.
しかし,幸いにしてこの確認の工程をまとめた便利な補題がある~\cite[p.21, Lemma 1.35]{Lee12}.
\begin{mylem}[label=lem:cinfty-chart]{微分構造の構成}
	\begin{itemize}
		\item \underline{集合} $M$
		\item $M$ の部分集合族 $\Familyset[\big]{U_\lambda}{\lambda \in \Lambda}$
		\item 写像の族 $\Familyset[\big]{\varphi_\lambda \colon U_\lambda \lto \mathbb{R}^n}{\lambda \in \Lambda}$
	\end{itemize}
	の3つ組であって以下の条件を充たすものを与える:
	\begin{description}
		\item[\textbf{(DS-1)}]  $\forall \lambda \in \Lambda$ に対して 
		$\varphi_\lambda (U_\lambda) \subset \mathbb{R}^n$ は $\mathbb{R}^n$ の開集合であり,
		$\varphi_\lambda \colon U_\lambda \lto \varphi_\lambda (U_\lambda)$
		は全単射である.
		\item[\textbf{(DS-2)}]  $\forall \alpha,\, \beta \in \Lambda$ に対して $\varphi_\alpha (U_\alpha \cap U_\beta),\, \varphi_\beta (U_\alpha \cap U_\beta) \subset \mathbb{R}^n$ は $\mathbb{R}^n$ の開集合である.
		\item[\textbf{(DS-3)}]  $\forall \alpha,\, \beta \in \Lambda$ に対して,$U_\alpha \cap U_\beta \neq \emptyset$ ならば $\varphi_\beta \circ \varphi_\alpha^{-1} \colon \varphi_\alpha (U_\alpha \cap U_\beta) \lto \varphi_\beta (U_\alpha \cap U_\beta)$ は $C^\infty$ 級である.
		\item[\textbf{(DS-4)}]  添字集合 $\Lambda$ の\underline{可算濃度の}部分集合 $I \subset \Lambda$ が存在して
		$\Familyset[\big]{U_i}{i \in I}$ が $M$ の\hyperref[def:cover]{被覆}になる.
		\item[\textbf{(DS-5)}]  $p,\, q \in M$ が $p \neq q$ ならば,
		ある $\lambda \in \Lambda$ が存在して $p,\, q \in U_\lambda$ を充たすか,
		またはある $\alpha,\, \beta \in \Lambda$ が存在して $U_\alpha \cap U_\beta = \emptyset$ かつ $p \in U_\alpha,\; q \in U_\beta$ を充たす.
	\end{description}
	このとき,$M$ の\hyperref[maxatlas]{微分構造}であって,$\forall \lambda \in \Lambda$ に対して $(U_\lambda,\, \varphi_\lambda)$ を\hyperref[diffmani]{$C^\infty$ チャート}として持つものが一意的に存在する.
\end{mylem}

\begin{proof}
	\begin{description}
		\item[\textbf{位相の構成}] 
		
		 $\mathbb{R}^n$ の\hyperref[thm.metrictopo]{Euclid位相}を $\mathscr{O}_{\mathbb{R}^n}$ と表記する.集合
		\begin{align}
			\mathscr{B} \coloneqq \bigl\{\, \varphi_\lambda^{-1}(U) \bigm| \lambda \in \Lambda,\; U \in \mathscr{O}_{\mathbb{R}^n} \,\bigr\} 
		\end{align}
		が\hyperref[ax.opbase]{開基の公理} \textsf{\textbf{(B1)}}, \textsf{\textbf{(B2)}} を充たすことを確認する.
		\begin{description}
			\item[\textbf{(B1)}] \textsf{\textbf{(DS-4)}}より明らか.
			\item[\textbf{(B2)}] $B_1,\, B_2 \in \mathscr{B}$ を任意にとる.
			このとき $\mathscr{B}$ の定義から,ある $\alpha,\, \beta \in \Lambda$ および $U,\, V \in \mathscr{O}_{\mathbb{R}^n}$ が存在して $B_1 = \varphi_\alpha^{-1}(U),\, B_2 = \varphi_\beta^{-1}(V)$ と書ける.
			補題\ref{lem:sets}-(4) より
			\begin{align}
				B_1 \cap B_2 &= \varphi_\alpha^{-1}(U) \cap \varphi_\beta^{-1}(V) \\
				&= \varphi_\alpha^{-1} \bigl( U \cap (\varphi_\alpha \circ \varphi_\beta^{-1})(V) \bigr) \\
				&= \varphi_\alpha^{-1} \bigl( U \cap (\varphi_\beta \circ \varphi_\alpha^{-1})^{-1}(V) \bigr)
			\end{align}
			が成り立つが,\textsf{\textbf{(DS-3)}}より $\varphi_\beta \circ \varphi_\alpha^{-1}$ は連続なので $(\varphi_\beta \circ \varphi_\alpha^{-1})^{-1}(V) \in \mathscr{O}_{\mathbb{R}^n}$ である.よって
			\begin{align}
				B_1 \cap B_2 \in \mathscr{B}
			\end{align}
			であり,\textsf{\textbf{(B2)}}が示された.
		\end{description}
		従って定理\ref{thm.optotopo}より,$\mathscr{B}$ を\hyperref[def.opbase]{開基}とする $M$ の位相 $\mathscr{O}_M$ が存在する.
		\item[$\bm{\varphi_\lambda}$ \textbf{が同相写像であること}] 
		
		 $\forall \lambda \in \Lambda$ を1つ固定する.$\mathscr{O}_M$ の構成と補題\ref{lem:sets}-(4) より,$\forall V \in \mathscr{O}_{\mathbb{R}^n}$ に対して
		$\varphi_\lambda^{-1}\bigl( V \cap \varphi_\lambda(U_\lambda) \bigr) = \varphi_\lambda^{-1}(V) \cap U_\lambda$ は $U_\lambda$ の開集合である\footnote{$U_\lambda$ には $\phase{M}$ からの\hyperref[def.reltopo]{相対位相}が,$\varphi_\lambda(U_\lambda)$ には $\phase{\mathbb{R}^n}$ からの相対位相を入れている.}.i.e. $\varphi_\lambda \colon U_\lambda \lto \varphi_\lambda(U_\lambda)$ は連続である.
		
		 $\forall B \in \mathscr{B}$ をとる.このとき補題\ref{lem:sets}-(9) より $\varphi_\lambda(B \cap U_\lambda) = \varphi_\lambda(B) \cap \varphi_\lambda(U_\lambda)$ が成り立つが,$\mathscr{O}_M$ の定義より $\varphi_\lambda(B) \in \mathscr{O}_{\mathbb{R}^n}$ なので $\varphi_\lambda(B \cap U_\lambda)$ は $\varphi_\lambda(U_\lambda)$ の開集合である.
		相対位相の定義とde Morgan則より,$U_\lambda$ の任意の開集合は $B \cap U_\lambda$ の形をした部分集合の和集合で書けるので,補題\ref{lem:sets}-(1) と\hyperref[ax.topo]{位相空間の公理}から $\varphi_\lambda$ は $U_\lambda$ の開集合を $\varphi_\lambda(U_\lambda)$ の開集合に移す.
		i.e. $\varphi_\lambda \colon U_\lambda \lto \varphi_\lambda(U_\lambda)$ は連続な全単射でかつ開写像であるから同相写像である.

		\item[\textbf{Hausdorff性}] 
		
		 位相空間 $\phase{M}$ がHausdorff空間であることを示す.
		$M$ の異なる2点 $p,\, q$ を勝手にとる.このとき\textsf{\textbf{(DS-5)}}より,
		\begin{itemize}
			\item ある $\lambda \in \Lambda$ が存在して $p,\, q \in U_\lambda$ を充たす
			\item ある $\alpha,\, \beta \in \Lambda$ が存在して $U_\alpha \cap U_\beta = \emptyset$ かつ $p \in U_\alpha,\; q \in U_\beta$ を充たす
		\end{itemize}
		のいずれかである.後者ならば証明することは何もない.

		 前者の場合を考える.このとき $\varphi_\lambda(U_\lambda)$ は $\mathbb{R}^n$ の開集合だから,$\mathbb{R}^n$ のHausdorff性から $\varphi_\lambda(U_\lambda)$ もHausdorff空間であり,従って $\varphi_\lambda(U_\lambda)$ の開集合 $U,\, V \subset \varphi\lambda(U_\lambda)$ であって $\varphi_\lambda(p) \in U \AND \varphi_\lambda(q) \in V \AND U \cap V = \emptyset$ を充たすものが存在する.
		このとき補題\ref{lem:sets}-(4) より $\varphi_\lambda^{-1}(U) \cap \varphi_\lambda^{-1}(V) = \varphi^{-1}_\lambda(U\cap V) = \emptyset$ で,かつ $\mathscr{O}_M$ の構成から $\varphi_\lambda^{-1}(U),\, \varphi_\lambda^{-1}(V) \subset M$ はどちらも $M$ の開集合である.
		そのうえ $p \in \varphi_\lambda^{-1}(U) \AND q \in \varphi_\lambda^{-1}(V)$ が成り立つので $M$ はHausdorff空間である.

		\item[\textbf{第2可算性}] 
		
		 $\mathbb{R}^n$ は\hyperref[def:second-countable]{第2可算}なので,$\forall \lambda \in \Lambda$ に対して $\varphi_\lambda(U_\lambda)$ も第2可算である.$\varphi_\lambda \colon U_\lambda \lto \varphi_\lambda (U_\lambda)$ は同相写像なので,$U_\lambda$ も第2可算である.
		従って\textsf{\textbf{(DS-4)}}から $M$ も第2可算である.
	\end{description}
	以上の考察から,位相空間 $\phase{M}$ が\hyperref[def.topomani]{位相多様体}であることが示された.
	さらに\textsf{\textbf{(DS-3)}}より $\mathcal{A} \coloneqq \Familyset[\big]{(U_\lambda,\, \varphi_\lambda)}{\lambda \in \Lambda}$ は $\phase{M}$ の\hyperref[diffmani]{$C^\infty$ アトラス}であることもわかる.

	最後に,$\mathcal{A}$ の\hyperref[maxatlas]{極大アトラス} $\mathcal{A}^+$ が,\underline{集合} $M$ 上の,与えられた全ての $(U_\lambda,\, \varphi_\lambda)$ を\hyperref[diffmani]{$C^\infty$ チャート}とする唯一の微分構造であることを示す.

	\item[\textbf{位相の一意性}] 
	
	 与えられた集合 $M$ の上の\hyperref[ax.topo]{位相} $\mathscr{T}$ であって,位相空間 $(M,\, \mathscr{T})$ が\hyperref[def:second-countable]{第2可算}な\hyperref[def:separation]{Hausdorff空間}となるようなものを任意にとる.
	$\forall \lambda \in \Lambda$ に対して
	与えられた全単射 $\varphi_\lambda \colon U_\lambda \lto \varphi_\lambda(U_\lambda)$ が同相写像であるためには,$\forall V \in 2^{U_\lambda}$ に対して
	\begin{align}
		V \in \mathscr{T} \IFF \varphi_\lambda(V) \in \mathscr{O}_{\mathbb{R}^n}
	\end{align}
	が成り立つことが必要十分である.
	そしてこのとき 
\end{proof}


\section{部分多様体}

\begin{mydef}[label=def.submani]{部分多様体}
	$n$ 次元\cinfty 多様体 $(M,\, \mathscr{O}_M)$ を与える.部分集合 $N \subset M$ は以下の条件を充たすとき\textbf{部分多様体} (submanifold) と呼ばれる:
	\begin{description} 
		\item[\textbf{(sub)}] $\forall p \in N$ に対してある開近傍 $U \in \mathscr{O}_M$ と $U$ 上定義された座標関数 $x^\mu \colon U \to \mathbb{R}$ が存在して,
		\begin{align} 
			\exists k \ge 0,\; N \cap U = \bigl\{ q \in U \bigm| x^{k+1}(q) = \cdots = x^n(q) = 0 \bigr\}.
		\end{align}
	\end{description}
	
	$N$ が $M$ の閉集合であるときは\textbf{閉部分多様体}と呼ぶ.
\end{mydef}

\begin{mydef}[label=def.embedding]{はめ込みと埋め込み}
	\cinfty 多様体 $M,\, N$ と \cinfty 写像 $f \colon M \to N$ を与える.
	\begin{enumerate} 
		\item $\forall p \in M$ において $f$ の微分写像 $f_* \colon T_p M \to T_{f(p)}N$ が\underline{単射}のとき,$f$ を\textbf{はめ込み} (immersion) と呼ぶ.
		\item $f \colon M \to N$ が\underline{はめ込み}であって,かつ全射 $f\colon M \twoheadrightarrow f(M)$ が同相写像であるとき,$f$ を\textbf{埋め込み} (embedding) と呼ぶ.
		\item $f \colon M \to N$ が\underline{全射}であって,かつ $\forall p \in M$ において $f_* \colon T_p M \to T_{f(p)}N$ が\underline{全射}であるとき, $f$ を\textbf{沈め込み} (submersion) と呼ぶ.
	\end{enumerate}
\end{mydef}

\begin{mytheo}[]{埋め込みと部分多様体}
	$f \colon M \to N$ を埋め込みとする.このとき $f(M) \subset N$ は $N$ の部分多様体であり,$f \colon M \twoheadrightarrow f(M)$ は微分同相写像である.
	
	逆に $M$ が $N$ の部分多様体であるとき,包含写像\footnote{$M \subset N$ のとき,$p \in M$ を $N$ の元として扱う写像.$\iota(p) = p$ である.\textbf{標準単射} (canonical injection) と呼ばれることもある.} $\iota \colon M \hookrightarrow N$ は埋め込みである.
\end{mytheo}

\begin{mytheo}[label=thm.Whitney]{Whitney の埋め込み定理}
	任意の $n$ 次元 \cinfty 多様体は $\mathbb{R}^{2n+1}$ の中に閉部分多様体として埋め込むことができる.
\end{mytheo}

\subsection{誘導計量}

\begin{mydef}[label=def.induced_metric]{誘導計量}
	$(N,\, h)$ をRiemann多様体,\cinfty 写像 $f \colon M \to N$ をはめ込みとする.このとき,2-形式 $h \in \Omega^2(N)$ の引き戻し(\ref{def.k-form_pullback})$f^*h$ は $M$ 上のRiemann計量 $g \in \Omega^2(M)$ を定める:
	\begin{align}
		g_p (u,\, v) \coloneqq h_{f(p)} \bigl(f_*(u),\, f_*(v)\bigr), \quad \forall p \in M,\, \forall u,\, v \in T_p M
	\end{align}
	これを $f$ による $M$ の\textbf{誘導計量}と呼ぶ.
\end{mydef}

誘導計量を $M$ のチャート $(U;\, x^\mu)$ および $N$ のチャート $(V;\, y^\nu)$ に関して成分表示すると
\begin{align}
	g_p (u,\, v) &= g_{\mu\nu}(p) u^\mu v^\nu\\
	&= h_{\alpha \beta}\bigl( f(p) \bigr)\, \pdv{y^\alpha}{x^\mu}\bigl( f(p) \bigr)\pdv{y^\beta}{x^\nu}\bigl( f(p) \bigr) u^\mu v^\nu
\end{align}
だから,
\begin{align}
	g_{\mu\nu}(p) = h_{\alpha \beta}\bigl( f(p) \bigr)\, \pdv{y^\alpha}{x^\mu}\bigl( f(p) \bigr)\pdv{y^\beta}{x^\nu}\bigl( f(p) \bigr)
\end{align}
である.
特に\cinfty 多様体 $M$ のEuclid空間 $\mathbb{R}^{n}$ へのはめ込み $\vb*{r} \colon M \to \mathbb{R}^{n},\, (x^\mu) \mapsto \vb*{r}(x^\mu)$ が与えられたとき,
$M$ のRiemann計量がしばしば
\begin{align}
	g_{\mu\nu} =  \pdv{\vb*{r}}{x^\mu} \vdot \pdv{\vb*{r}}{x^\nu}
\end{align}
と書かれるのはこのためである.

\begin{marker}
	多様体 $N$ が擬Riemann多様体のときは,多様体 $M$ が誘導計量を持つとは限らない.
\end{marker}

例えばEuclid空間 $\mathbb{R}^3$ に埋め込まれた単位球面 $S^2$ を考える.はめ込みを
\begin{align}
	\vb*{r} \colon (\theta,\, \phi) \mapsto \mqty[\sin \theta \cos\phi \\ \sin\theta \sin \phi \\ \cos \theta]
\end{align}
として与えると,$S^2$ の誘導計量は
\begin{align}
	g_{\mu\nu} \dd{x^\mu} \otimes \dd{x^\nu} &= \pdv{\vb*{r}}{x^\mu} \vdot \pdv{\vb*{r}}{x^\nu} \dd{x^\mu} \otimes \dd{x^\nu} \\
	&= \dd{\theta} \otimes \dd{\theta} + \sin^2 \theta\, \dd{\phi} \otimes \dd{\phi}
\end{align}
と求まる.

\section{隅付き多様体}


\section{力学系としての多様体}



\end{document}