\documentclass[geometry_main]{subfiles}

\begin{document}

\setcounter{chapter}{7}

\chapter{複素多様体の紹介}

この章の内容は~\cite[3.4]{EGH}によるところが大きい.

複素関数 $f \colon \mathbb{C}^m \to \mathbb{C}$ が\textbf{正則} (holomorphic) であるとは,$f(z^1,\, \dots ,\, z^m) = f_1(x^1,\, \dots ,\, x^m;\; y^1,\, \dots ,\, y^m) + \iunit f_2(x^1,\, \dots ,\, x^m;\; y^1,\, \dots ,\, y^m)$ が各変数 $z^\mu = x^\mu + \iunit y^\mu$ に関して\textbf{Cauchy-Riemann の関係式}
\begin{align} 
	\pdv{f_1}{x^\mu} = \pdv{f_2}{y^\mu},\quad \pdv{f_2}{x^\mu} = -\pdv{f_1}{y^\mu}
\end{align}
を充たすことを言う.

写像 $(f^1,\, \dots,\, f^n) \colon \mathbb{C}^m \to \mathbb{C}^n$ は,各関数 $f^\lambda$ が正則であるとき正則であると言う.

複素多様体の定義を再掲する:

\begin{mydef}[label=def.complexmani]{複素多様体(再掲)}
	$M$ を位相空間とする.
	集合族 $\mathcal{S} = \{(U_\lambda,\, \varphi_\lambda)\mid \varphi_\lambda \colon U_\lambda \xrightarrow{\approx} \mathbb{C}^m \}_{\lambda \in \Lambda}$ が与えられ,$\{\, U_\lambda \, \}_{\lambda \in \Lambda}$ が $M$ の開被覆になっており,かつその全ての座標変換 $f_{\beta\alpha} = \varphi_{\beta} \circ \varphi_{\alpha}^{-1}$ が\underline{正則}であるとき,$M$ は\textbf{複素多様体}と呼ばれる.

	座標近傍が $\mathbb{C}^m$ と同相のとき,$m$ を\textbf{複素次元}と呼んで $\dim_{\mathbb{C}} M = m$ と書く.実次元 $2m$ は単に $\dim M = 2m$ と書く.
\end{mydef}

$M$ のアトラスの上には,定義\ref{manieq}で定めた同値関係が定まる.

\begin{mydef}[label=complex_structure]{複素構造}
	複素多様体 $M$ のアトラス $\mathcal{S}$ を与える.定義\ref{manieq}で定めた同値関係による $\mathcal{S}$ の同値類 $[\mathcal{S}]$ を $M$ の\textbf{複素構造} (complex structure) と呼ぶ.
\end{mydef}

\section{複素化の概要}

複素次元 $m$ の複素多様体 $M$ のチャート $(U;\, z^\mu)$ をとる.$2m$ 個の接ベクトルを
\begin{align} 
	\pdv{}{z^\mu} &\coloneqq \frac{1}{2} \left(\pdv{}{x^\mu} - \iunit \pdv{}{y^\mu}\right) \\
	\pdv{}{\bar{z}^\mu} &\coloneqq \frac{1}{2} \left(\pdv{}{x^\mu} + \iunit \pdv{}{y^\mu}\right) \\
\end{align}
と定義し,対応する双対ベクトルを
\begin{align} 
	\dd{z_\nu} &\coloneqq \dd{x_\nu} + \iunit \dd{y_\nu} \\
	\dd{\bar{z}_\nu} &\coloneqq \dd{x_\nu} - \iunit \dd{y_\nu} \\
\end{align}
と定義する.このとき,\underline{\cinfty 関数} $f$ の外微分は
\begin{align} 
	\dd{f} &= \sum_\mu \left(\pdv{f_1}{x^\mu} \dd{x^\mu} + \pdv{f_1}{y^\mu} \dd{y^\mu} \right) + \iunit \sum_\mu \left(\pdv{f_2}{x^\mu} \dd{x^\mu} + \pdv{f_2}{y^\mu} \dd{y^\mu} \right) \\
	&= \sum_\mu \pdv{f}{z^\mu} \dd{z^\mu} + \sum_\mu \pdv{f}{\bar{z}^\mu} \dd{\bar{z}^\mu}
\end{align}
と書かれる.
\begin{align} 
	\partial f &\coloneqq \sum_\mu \pdv{f}{z^\mu} \dd{z^\mu} \\
	\overline{\partial} f &\coloneqq \sum_\mu \pdv{f}{\bar{z}^\mu} \dd{\bar{z}^\mu}
\end{align}
とおけば $\dd{f} = \partial f + \overline{\partial} f$ である.1変数正則関数 $f(z)$ の場合,Cauchy-Riemannの関係式から
\begin{align} 
	\overline{\partial} f = \pdv{f}{\bar{z}} \dd{\bar{z}} = 0
\end{align}
であった.これは $m$ 変数\underline{正則関数} $f$ の場合に一般化され,
\begin{align} 
	\pdv{f}{\bar{z}^\mu} = 0\quad (1\le \forall \mu\le m)
\end{align}
あるいは $\overline{\partial} f = 0$ となる.

別のチャート $(V;\, w^\nu)$ をとると,座標変換は正則である.故に $U \cap V$ 上で
\begin{align} 
	\dd{w^\mu} &= \partial w^\mu + \overline{\partial} w^\mu = \partial w^\mu = \sum_\nu \pdv{w^\mu}{z^\nu} \dd{z^\nu}, \\
	\dd{\bar{w}^\mu} &= \partial \bar{w}^\mu + \overline{\partial} \bar{w}^\mu = \overline{\partial} \bar{w}^\mu = \sum_\nu \pdv{\bar{w}^\mu}{\bar{z}^\nu} \dd{\bar{z}^\nu}
\end{align}
が成立する.
従って,
\begin{align} 
	T^{1,\, 0}M &\coloneqq \Span \left\{ \pdv{}{z^\mu} \right\},\quad  T^{0,\, 1}M \coloneqq \Span \left\{ \pdv{}{\bar{z}^\mu} \right\} \\
	T^{1,\, 0}{}^*M &\coloneqq \Span \left\{ \dd{z^\mu} \right\},\quad  T^{0,\, 1}{}^*M \coloneqq \Span \left\{ \dd{\bar{z}^\mu} \right\} \\
\end{align}
と定義すると直和分解
\begin{align} 
	TM \otimes \mathbb{C} &= T^{1,\, 0}M \oplus T^{0,\, 1}M \\
	TM^* \otimes \mathbb{C} &= T^{1,\, 0}{}^*M \oplus T^{0,\, 1}{}^*M \\
\end{align}
が成立する.

\subsection{Dolbeault作用素}

\textbf{複素 $r$-形式}の基底は $p$ 個の $\dd{z^\mu}$ と $q$ 個の $\dd{\bar{z}^\mu}$ がWedge積で結合したものである(ただし $r = p+q$).対 $(p,\, q)$ のことを\textbf{双次数}と呼ぶ.
双次数 $(p,\, q)$ の複素 $r$ 形式全体の集合を $\Omega^{p,\, q}(M)$ と書くと,直和分解
\begin{align} 
	\Omega^r(M){}^\mathbb{C} = \bigoplus_{p+q = r} \Omega^{p,\, q}(M)
\end{align}
が成り立つ.

先ほど定義した $\partial \colon \Omega^{0,\, 0}(M) \to \Omega^{1,\, 0}(M),\; \bar{\partial} \colon \Omega^{0,\, 0}(M) \to \Omega^{0,\, 1}(M)$ は,次のように一般化される:
\begin{align} 
	\partial \colon \Omega^{p,\, q}(M) \to \Omega^{p+1,\, q}(M),\quad \bar{\partial} \colon \Omega^{p,\, q}(M) \to \Omega^{p,\, q+1}(M)
\end{align}
このような $\partial,\, \bar{\partial}$ は\textbf{Dolbeault作用素}と呼ばれる.
Dolbeault作用素によって,複素$p+q$-形式の外微分 $\dd{} \colon \Omega^{p+q}(M){}^\mathbb{C} \to \Omega^{p+q+1}(M){}^\mathbb{C}$ は
\begin{align} 
	\dd{} = \partial + \bar{\partial}
\end{align}
と分解される.

Hodge作用素は
\begin{align} 
	\star \colon \Omega^{p,\, q}(M) \to \Omega^{m-p,\, m-q}
\end{align}
と作用する.定義\ref{def.adj_extdiff}を参照すると,複素Riemann多様体 $M$ の外微分 $\dd{}$ の随伴作用素は
\begin{align} 
	\delta = - \star \dd{} \star \colon \Omega^{p+q}(M){}^\mathbb{C} \to \Omega^{p+q-1}(M){}^\mathbb{C}
\end{align}
と定義される.Dolbeaut作用素の随伴
\begin{align} 
	\partial^\dagger \colon \Omega^{p,\, q}(M) \to \Omega^{p-1,\, q}(M),\quad \bar{\partial}^\dagger \colon \Omega^{p,\, q}(M) \to \Omega^{p,\, q-1}(M)
\end{align}
によって
\begin{align} 
	\delta = \partial^\dagger + \bar{\partial}^\dagger
\end{align}
と書かれる.このとき,3通りのLaplacianが定義できる:
\begin{align} 
	\Delta &\coloneqq (\dd{} + \delta)^2 \\
	\Delta_{\partial} &\coloneqq (\partial + \partial^\dagger)^2 \\
	\Delta_{\bar{\partial}} &\coloneqq (\bar{\partial} + \bar{\partial}^\dagger)^2 
\end{align}
一般に,これらの間に特別な関係はない.

\subsection{概複素構造}

多様体 $M$ は,線型写像 $J \colon TM \to TM$ であって $J^2 = -1$ を充たす $J$ が\underline{大域的に}存在するとき,\textbf{概複素構造} (almost complex structure) を持つという.奇数次元の多様体が概複素構造を持つことはない.

任意の複素多様体 $M$ が概複素構造を持つことを確認する.$M$ の各点 $p$ と,それを含むチャート $(U;\, z^\mu) = (U;\, x^\mu;\, y^\mu)$ をとる.において,線型写像 $J_p \colon T_p M \to T_p M$ を
\begin{align} 
	J_p \left(\pdv{}{x^\mu}\right) \coloneqq \pdv{}{y^\mu},\quad J_p \left(\pdv{}{y^\mu}\right) = -\pdv{}{x^\mu}
\end{align}
で定義する.定義から $(J_p)^2 = -1_p$ であることは明らか.

別のチャート $(V;\, w^\mu) = (V;\, u^\mu,\, v^\mu)$ をとり $w^\mu = u^\mu + \iunit v^\mu$ とすると,Cauchy-Riemannの関係式から
\begin{align} 
	J_p \left(\pdv{}{u^\mu}\right) &= J_p \left(\pdv{x^\nu}{u^\mu} \pdv{}{x^\nu} + \pdv{y^\nu}{u^\mu} \pdv{}{y^\nu} \right) = \pdv{x^\nu}{v^\mu} \pdv{}{y^\nu} + \left(-\pdv{y^\nu}{v^\mu}\right) \left(-\pdv{}{x^\nu}\right) = \pdv{}{v^\mu} \\
	J_p \left(\pdv{}{v^\mu}\right) &= J_p \left(\pdv{x^\nu}{v^\mu} \pdv{}{x^\nu} + \pdv{y^\nu}{v^\mu} \pdv{}{y^\nu} \right) = -\pdv{x^\nu}{u^\mu} \pdv{}{y^\nu} + \pdv{y^\nu}{u^\mu} \left(-\pdv{}{x^\nu}\right) = -\pdv{}{u^\mu}
\end{align}
が成立するため,$J_p$ の作用はチャートの取り方によらない.ゆえに,$J_p$ の表現行列はチャートの取り方によらずに
\begin{align} 
	\mqty(0 & I_m \\ -I_m & 0)\label{eq.rep1}
\end{align}
である.$J_p$ は,その構成からわかるように $\forall p \in M$ で定義でき,それらを集めた $J = \{\, J_p \mid p \in M\, \}$ は $M$ 上至る所\cinfty 級である.故に $J \in \mathfrak{T}^2_0(M)$ であり,$M$ の外複素構造を定める.

\subsection{K\"ahler多様体}

複素多様体 $M$ のRiemann計量 $g$ が,$\forall p \in M$ と $\forall X,\, Y \in T_pM$ に対して
\begin{align} 
	g_p(J_p X,\, J_p Y) = g_p(X,\, Y)
\end{align}
を充たすとき,$g$ は\textbf{Hermite計量}と呼ばれる.このとき,組 $(M,\, g)$ を\textbf{Hermite多様体}と呼ぶ.
任意の複素多様体 $(M,\, g)$ は,
\begin{align} 
	\hat{g}_p(X,\, Y) \coloneqq \frac{1}{2} \bigl( g_p(X,\, Y) + g_p(J_p X,\, J_p Y) \bigr) 
\end{align}
として新しい計量 $\hat{g}$ を定義することでHermite多様体になる.

\begin{tcolorbox} 
	複素多様体 $\; \Longrightarrow \;$ Hermite多様体
\end{tcolorbox}


Hermite多様体 $(M,\, g)$ 上には自然に
\begin{align} 
	K \coloneqq \frac{\iunit}{2} g_{\mu \bar{\nu}} \dd{z^\mu} \wedge \dd{\bar{z}^\nu}
\end{align}
なる $2$-形式が定義される.$K$ を\textbf{K\"ahker形式}と呼ぶ.$K$ は実形式である.

\begin{mydef}[label=Kahler]{K\"ahker多様体}
	Hermite多様体 $(M,\, g)$ であって,そのK\"ahler形式 $K$ が閉形式であるものを言う.i.e. $\dd{K} = 0.$ このとき,計量 $g$ は $M$ のK\"ahler計量と呼ばれる.
\end{mydef}

例えば,$\cdim M = 1$ の任意の複素多様体 $(M,\, g)$ はK\"ahler多様体である.$M$ のK\"ahler形式 $K$ は実 $2$-形式だから,その外微分 $\dd{K}$ は$3$-形式であり,$\dim M = 2$ であることから $0$ になるのである.
1次元コンパクト複素多様体は\textbf{Riemann面}と呼ばれる.

\end{document}
